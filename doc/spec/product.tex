% ==============================================================================
%  $RCSfile: product.tex,v $, $Revision: 1.33 $
%  $Date: 2003/03/31 16:29:39 $
%  $Author: squig $
%
%  Description: Beschreibung des Produktes. �bersicht �ber die Funktionen
%               und allgemeine Bedienkonzepte.
%
%  Last-Ispelled-Revision: 1.32
%
% ==============================================================================


% ==============================================================================
\section{GIANT Projekte} \index{Projekte} \index{Persistenz}

Innerhalb eines Projektes fasst GIANT Informationen, 
wie erzeugte Anzeigefenster und
erzeugte IML-Teilgraphen f�r einen vorgegebenen IML-Graphen, zusammen
und speichert diese persistent. Die Zusammenfassung zu Projekten soll der 
�bersicht dienen und den Austausch von Teilergebnissen, 
wie z.B. einzelner Anzeigefenster, erleichtern.\\
F�r weitere Informationen zu Projekten siehe 
Kapitel \ref{GIANT Projektverwaltung}.


% ==============================================================================
\section{IML-Teilgraphen und Selektionen}
Hinsichtlich der logischen Zusammenfassung von Knoten und Kanten unterscheidet
GIANT IML-Teilgraphen und Selektionen.

  \subsection{IML-Teilgraphen} \index{IML-Teilgraphen}
  \index{Graph-Knoten} \index{Graph-Kanten} \index{hervorheben}
  Ein IML-Teilgraph beschreibt eine Knoten- und Kantenmenge aus 
  dem IML-Graphen  bei der jede Kante einen Start- und einen Zielknoten hat 
  -- die sogenannten Graph-Knoten und Graph-Kanten. 
  Die Graph-Knoten und Graph-Kanten von IML-Teilgraphen k�nnen 
  in Anzeigefenstern
  hervorgehoben werden. Des weiteren k�nnen die Graph-Knoten und Graph-
  Kanten von IML-Teilgraphen als Fenster-Knoten und Fenster-Kanten 
  in Anzeigefenster eingef�gt und layoutet werden.
  Die IML-Teilgraphen sind aber v�llig unabh�ngig von den Anzeigefenstern und
  enthalten insbesondere keine Layoutinformationen f�r ihre 
  Graph-Knoten und Graph-Kanten.

  \subsection{Selektionen} \index{Selektionen}
  Selektionen stellen die lokale Variante von Knoten- und Kantenmengen dar.
  Sie sind immer einem festen Anzeigefenster zugeordnet.
  Selektionen k�nnen beliebige Knoten und Kanten umfassen, ohne dass diese 
  einen Teilgraphen bilden m�ssen.
  Jeder Knoten und jede Kante einer Selektion muss auch im Anzeigefenster
  als Fenster-Knoten oder Fenster-Kante vorhanden sein.

% ==============================================================================
\section{Anzeigefenster} \index{Anzeigefenster}
Anzeigefenster sind die Fenster von GIANT, in denen eine benutzerdefinierte
Auswahl von Knoten und Kanten visualisiert wird. Es kann beliebig viele
Anzeigefenster geben und jedes Anzeigefenster kann beliebig viele Selektionen
haben.

  \subsection{Pins} \index{Anzeigefenster!Pins} \index{Pins}
  Da bei gro�en Graphen selten alle zu einem Anzeigeinhalt 
  geh�renden Fenster-Knoten und Fenster-Kanten gemeinsam auf dem Bildschirm
  sichtbar dargestellt werden k�nnen, kann sich der
  Benutzer zu jedem Anzeigefenster eine Liste von Pins anlegen. In den Pins wird
  jeweils die Position des sichtbaren Anzeigeinhaltes und die Zoomstufe
  \index{Zoomstufe}
  gespeichert, so dass zu beliebigen Zeitpunkten die Position des sichtbaren
  Anzeigeinhalts rekonstruiert werden kann.

  \subsection{Visualisierungsstile} \index{Visualisierungsstile}
  Mittels sogenannter Visualisierungsstile kann der Benutzer die Darstellung
  von Fenster-Kanten und Fenster-Knoten (z.B. welche Attribute dargestellt
  werden sollen) auch w�hrend der Laufzeit von GIANT beeinflussen.
  Da �ber entsprechende XML-Dateien verschiedene 
  Visualisierungsstile definiert werden k�nnen, kann GIANT so
  an spezifische Problemstellungen angepasst werden.\\
  Weitere Informationen hierzu sind unter Abschnitt
  \ref{Config Visualisierungsstile} verf�gbar.

\section{Knoten-Annotationen} \index{Knoten-Annotationen}
Jeder Knoten kann mit einer textuellen Annotation versehen werden. Diese
Annotation kann in einem Fenster au�erhalb des Anzeigefensters
zur Anzeige gebracht und bearbeitet werden.\\
F�r weitere Informationen hierzu siehe Abschnitt
\ref{Project Persistenz von Knoten-Annotationen}.


% ==============================================================================
\section{Anfragen} \index{GQSL} \index{Anfragesprache}
Eine vielseitige Anfragesprache -- die GIANT Query and Scripting Language GQSL --
stellt nahezu die gesamte Funktionalit�t von GIANT zur Verf�gung und kann 
insbesondere auch zum Aufruf via Kommandozeile
genutzt werden. Die drei Schritte des anschlie�end beschriebenen 
\gq{Drei-Stufen-Konzeptes} k�nnen �ber diese Anfragesprache auch \gq{auf 
einen Schlag} erledigt werden.\\
Die GQSL ist unter Kapitel \ref {GIANT Query Skripting Language} im
Detail spezifiziert.


% ==============================================================================
\section{Drei-Stufen-Konzept} \index{Drei-Stufen-Konzept}

Die im Folgenden beschriebenen drei Schritte sollen dem Benutzer die M�glichkeit
bieten, von den Knoten und Kanten des IML-Graphen ausgehend geeignete
Teilgraphen in Anzeigefenstern zu visualisieren. Von der
Gestaltung der UseCases her, werden diese drei Schritte sequentiell nacheinander
ausgef�hrt, dies ist aber nicht zwingend n�tig.


\begin {enumerate}
  \item Erzeugen geeigneter IML-Teilgraphen, d.h.\ Auswahl geeigneter Knoten 
        und Kanten aus dem IML-Graphen mittels der Anfragesprache.
  
  \item Einf�gen dieser IML-Teilgraphen in ein Anzeigefenster unter Anwendung
  eines Layoutalgorithmus \index{Layoutalgorithmen}. 
  
  \item Weitere Bearbeitung der Fenster-Knoten und Fenster-Kanten (wie z.B.
  Verschieben, Annotieren und Erzeugen von Selektionen).
  
\end {enumerate}






