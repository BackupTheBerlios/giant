% ==============================================================================
%  $RCSfile: product.tex,v $, $Revision: 1.1 $
%  $Date: 2003/02/13 03:15:14 $
%  $Author: schulzgt $
%
%  Description:
%
% ==============================================================================

\section{IML-Teilgraphen}
\section{Anzeigefenster und Selektionen}
\section{Pins}
\section{Drei-Stufen-Konzept}

\chapter{Visualisierung des IML-Graphen}
Genaue Beschreibung der Visualisierung des Graphen in einem Anzeigefenster.
\section{Visualisierung von Knoten}
\section{Visualisierung von Kanten}
\section{Kantenknickpunkte}
\section{Minimap}

\chapter{Anfragesprache und Kommandozeilenaufruf}
\section{Beschreibung der Anfragesprache}
\section{Kommandozeilenaufruf}

\section{Das Projektverzeichnis}
\subsection{Verwaltungsdateien f�r Anzeigefenster}
\subsection{Verwaltungsdateien f�r IML-Teilgraphen}
\subsection{Verwaltungsdatei f�r Anfragen}
Kann er auch selber mittels Paste und Copy im Emacs machen.
\subsection{Identifikation der IML-Graph Datei}
\subsection{Grundlegendes Verhalten beim Speichern}
