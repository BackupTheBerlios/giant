% ==============================================================================
%  $RCSfile: product.tex,v $, $Revision: 1.4 $
%  $Date: 2003/02/14 14:34:49 $
%  $Author: schwiemn $
%
%  Description:
%
% ==============================================================================


\section{Allgemeine Bedienkonzepte}

\subsection{GIANT Projekt}

\section{IML-Teilgraphen und Selektionen}
Hinsichtlich der logischen Zusammenfassung von Knoten und Kanten
unterscheidet GIANT mit den IML-Teilgraphen und den Selektionen
einmal ein globales Modell (unabh�ngig von Anzeigefenstern) und ein 
lokales Modell, welches an ein bestimtes Anzeigefenster gebunden ist.\\ 
Diese IML-Teilgraphen und Selektionen k�nnen nat�rlich ineinander umgewandelt 
werden.


  \subsection{IML-Teilgraphen}
  Ein IML-Teilgraph beschreibt eine Knoten und Kantenmenge aus dem IML-Graph, welche 
  einen Teilgraphen des IML-Graphen darstellt, d.h. jede Kante hat einen Start- und 
  einen Zielknoten. Die Knoten und Kanten von IML-Teilgraphen k�nnen in Anzeigefenstern
  hervorgehoben werden, des weiteren k�nnen die Knoten und Kanten von
  IML-Teilgraphen in Anzeigefenstern layouted und dargestellt werden.\\
  Die IML-Teilgraphen sind aber v�llig unabh�ngig von den Anzeigefenstern und enthalten
  insbesondere keine Layoutinformationen f�r ihre Knoten und Kanten.

  \subsection{Selektionen}
  Selektionen stellen die lokale Variante von Knoten- und Kantenmengen dar. Sie sind
  immer einem festen Anzeigefenster zugeordnet. 

\section{Anzeigefenster}

Anzeigefenster sind die Fenster von GIANT in denen eine benutzerdefinierte Auswahl
von Knoten und Kanten visualisiert wird. Es kann beliebig viele Anzeigefenster geben
und jedes Anzeigefenster kann beliebig viele Selektionen haben.

  \subsection{Pins}
  Da selten alle zu einem Azeigeinhalt geh�renden Knoten und Kanten insgesamt sichtbar
  dargestellt werden k�nnen (da es zu viele sind) kann sich der Benutzer zu jedem
  Anzeigefenster eine Liste von Pins anlegen. In den Pins wird jeweils
  die Position des sichtbaren Anzeigeinhaltes und die Zoomstufe gespeichert, so
  dass zu beliebigen Zeitpunkten der jeweilige sichtbare Anzeigeinahlt rekonstruiert
  werden kann.


  \subsection{Visualisierungsstiele}
  Mittels sogenannter Visualisierungsstiele kann der Benutzer zur Laufzeit von GIANT
  die Darstellung von Knoten und Kanten jeder Zeit flexibel beeinflussen.


\section{Anfragen}

  Eine vielseitige Anfragesprache stellt nahezu die gesamte Funktionalit�t
  von GIANT zur Verf�gung und kann insbesondere auch zum Aufruf via Kommando-
  zeile genutzt werden.



\section{Drei-Stufen-Konzept}

Die im Folgenden Beschriebenen drei Schritte sollen dem Benutzer die M�glichkeit,
von den Knoten und Kanten des IML-Graphen ausgehend geeignete Teilgraphen zur
in Anzeigefenstern zu visualisieren. Rein logisch und von der Gestaltung der
Gestaltung der UseCases her, werden diese drei Schritte sequentiell nacheinander
ausgef�hrt. Die komplexe Anfragesprache erlaubt es aber, diese Schritte 
quasi \gq{auf einen Schlag} durchzuf�hren.

\begin {enumerate}

  \item Erzeugen geeigneter IML-Teilgraphen, d.h Auswahl geeignter Knoten und
        Kanten aus dem IML-Graphen mittels der Anfragesprache.
  
  \item Ableiten weiterer IML-Teilgraphen aus bereits bestehenden IML-Teilgraphen 
        (mittels der Anfragesprache, z.B. Mengenschnitt).

  \item Einf�gen dieser IML-Teilgraphen in ein Anzeigefenster unter Anwendung eines
        Layoutalgorithmus. Innerhalb des Anzeigefensters ist der eingef�gte IML-Teilgraph
        dann als Selektion sichtbar.

\end {enumerate}




\subsection{Fehlerverhalten}
EVENTEULL EIGENES KAPITEL, WELCHES ALLES DAZU BEHANDELT; SO DASS MAN DIES NICHT BEI
DEN USE CASES KL�REN MUSS.





SOLLTE UNEBDINGT UNMITTELBAR VOR DER USECASE BESCHREIBUNG STEHEN


\section{Starten von GIANT}
NORMALER STARTVORGANG ohne �ber gabe ienes

\subsection{Starten von GIANT mit Kommandozeilenparametern}





\subsection{Starten von GIANT mit Kommandozeilenparametern}
Erl�uterung der Kommandozeilenparameter

\subsection{Beenden von GIANT}
Beenden durch Mausklick oder Alt+F4.