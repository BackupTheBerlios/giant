% ==============================================================================
%  $RCSfile: development.tex,v $, $Revision: 1.6 $
%  $Date: 2003/03/31 16:29:39 $
%  $Author: squig $
%
%  Description:
%
%  Last-Ispelled-Revision: 1.5
%
% ==============================================================================

\section {Compiler und Bibliotheken} 
\index{Compiler}
\index{Bibliotheken}
\index{GTK/ADA}
\index{GNAT}
Das System soll in der Sprache Ada95 mit GNAT 3.14 entwickelt werden, 
wobei GTK/ADA 1.2.12 als grafische
GUI-Bibliothek eingesetzte werden soll. 
Das System baut auf der vom Kunden bereit gestellten IML-Graph-Bibliothek auf 
und soll des weiteren zur Unterst�tzung der Wartbarkeit m�glichst auch die 
vom Kunden zur Verf�gung gestellten 
Datenstrukturen (wie z.B. Hashtables aus Bauhaus/reuse/src) nutzen.

  
  \subsection {Lizenzrechtliches zu den Paketen des Kunden}
  \index{Pakete der Bauhaus-Reengineering GmbH}
  Folgendes gilt nicht f�r die IML-Graph-Bibliothek.\\
  Die vom Kunden zur Verf�gung gestellten Datenstrukturen 
  werden den Entwicklern von GIANT ohne lizenzrechtliche Bedingungen 
  �berlassen. Die Nutzungsrechte der Entwickler am Produkt GIANT werden 
  durch Einsatz dieser Datenstrukturen in keinster Weise ber�hrt.
 


\section {Einlesen und Schreiben von XML Dateien}
\index{XML Dateien}
Auf XML-Dateien soll mittels des DOM (Document Object Model) Parsers
aus XML/Ada 0.7.1 zugegriffen werden. XML/Ada 0.7.1 unterliegt lizenzrechtlich der \gq{GNAT Modified GNU Public
License} (GMGPL).\\
Als Alternative ist der XML-Parser aus GTK/Ada vorgesehen -- Paket Glib.XML.



\section {Sprache}
\index{Sprache der Dokumente}
Hier wird beschrieben, in welcher Sprache die einzelnen Dokumente
des Systems GIANT verfasst werden.

\section {Spezifikation}
Die Spezifikation -- dieses Dokument -- wird in deutscher Sprache verfasst.

\section {Benutzerhandbuch}
Das Benutzerhandbuch wird in deutscher Sprache verfasst.

\section {Entwurf}
Der Entwurf wird in Englisch verfasst.

\section {Sprache der internen Dokumentation}
Die interne Dokumentation von GIANT -- Kommentare im Quellcode -- 
erfolgt in englischer Sprache.

\section {Interaktionssprache mit dem Benutzer}
\index{Interaktionssprache}
Die GUI von GIANT interagiert mit dem Benutzer ausschlie�lich in
englischer Sprache.

\section {Sprache der Konfigurationsdateien}
Die Knoten und Attribute der XML-Konfigurationsdateien
werden mit englischen Begriffen benannt.



