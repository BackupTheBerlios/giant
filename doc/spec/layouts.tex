% ==============================================================================
%  $RCSfile: layouts.tex,v $, $Revision: 1.7 $
%  $Date: 2003/04/08 12:31:13 $
%  $Author: koppor $
%
%  Description: Beschreibung der Layouts
%
%  Last-Ispelled-Revision: 1.2
%
% ==============================================================================
%chapter-remark included in spec.tex

Hier werden die Layoutalgorithmen von \product beschrieben, mittels
derer das Layout von Fenster-Knoten innerhalb von Anzeigefenstern
automatisch berechnet werden kann.

\product bietet zwei verschiedene Layoutalgorithmen an:
\begin{enumerate}
  \item Treelayout
  \item Matrixlayout
\end{enumerate}

\section{Treelayout}
\subsection{Parameter}
\begin{enumerate}
  \item \textbf{IML-Teilgraph}\\
Der IML-Teilgraph, der layoutet werden soll.
  \item \textbf{Kantenklassen}\\
Zu beachtete Kantenklassen k\"onnen angegeben werden. Ist dies nicht der Fall, werden alle ber\"ucksichtigt.
  \item \textbf{Zielposition}\\
Die Position, an die der Mittelpunkt des Wurzelknotens des Baums positioniert werden soll.
\end{enumerate}

\subsection{Beschreibung}
Das Treelayout basiert auf Walkers Algorithmus. Dabei werden folgende Eigenschaften erreicht:
\begin{enumerate}
   \item Alle Kinder eines Knotens sind auf der gleichen Ebene
   \item Alle Kinder eines Knotens haben den gleichen horizontalen Abstand zueinander
\end{enumerate}

Als Wurzelknoten wird der erste Knoten des \"ubergebenen IML-Teilgraphen gew\"ahlt. Die Reihenfolge der Kinder von der \texttt{IML\_Reflection}\xspace\"ubernommen. In der Grundfunktionalit\"at werden alle Kanten betrachtet.

Der vertikale Abstand von Ebene zu Ebene richtet sich nach der H\"ohe des h\"ochsten Knotens (siehe \ref{Visualization-Knoten-Rechteck}).

Falls der \"ubergebene IML-Teilgraph keinen Baum darstellt, werden die Knoten in die Ebene eingegliedert, die bei einer Tiefensuche als erstes erreicht wird.

Zerf\"allt der IML-Teilgraph in nicht-zusammenh\"angede Komponenten, so wird das Layout f\"ur jede dieser Komponente durchgef\"uhrt.

Als Erweiterung kann der Algorithmus nur bestimmte Kanten betrachten. Dabei werden im ersten Lauf nur die Knoten betrachtet, die durch die gegebenen Kanten erreicht werden k\"onnen.
Im zweiten Lauf werden Knoten, die durch die gegebenen Kanten nicht erreicht werden konnten, im Matrixlayout (siehe \ref{Matrixlayout}) rechts von dem letzten Baum angeordnet.

\section{Matrixlayout}
\label{Matrixlayout}
\subsection{Parameter}
\begin{enumerate}
  \item \textbf{IML-Teilgraph}\\
Der IML-Teilgraph, der layoutet werden soll
  \item \textbf{Zielposition}\\
Die Position, an der die linke obere Ecke des Quadrats liegen soll.
\end{enumerate}

\subsection{Beschreibung}
Im Matrixlayout werden die Knoten in einem Rechteck ausgerichtet.
Dabei ist die Zahl der Knoten pro Zeile und Spalte gleich.  Die Breite
des Rechtecks wird mittels der abgerundeten Wurzel der Anzahl der zu
layouteten Knoten bestimmt; die H\"ohe ergibt sich automatisch.  Die
Knoten werden mittels einer Breitensuche besucht und in dieser
Besuchsreihenfolge in das Rechteck zeilenweise eingef\"ugt. Dabei hat
die obere Begrenzung jedes Knotens einer Zeile die selbe Y-Koordinate.
Der vertikale Abstand von Zeile zu Zeile richtet sich nach der H\"ohe
des h\"ochsten Knotens (siehe \ref{Visualization-Knoten-Rechteck}).

