% ==============================================================================
%  $RCSfile: layouts.tex,v $, $Revision: 1.4 $
%  $Date: 2003/04/02 11:23:40 $
%  $Author: koppor $
%
%  Description: Beschreibung der Layouts
%
%  Last-Ispelled-Revision: 1.2
%
% ==============================================================================
%chapter-remark included in spec.tex

Hier werden die Layoutalgorithmen von \product beschrieben, mittels
derer das Layout von Fenster-Knoten innerhalb von Anzeigefenstern
automatisch berechnet werden kann.

\product bietet zwei verschiedene Layoutalgorithmen an:
\begin{itemize}
  \item Treelayout
  \item Matrixlayout
\end{itemize}.

Alle Knoten werden beim Layouten als gleich gro\"s betrachtet. Diese Gr\"osse ist konstant und betr\"agt 120x100 Pixel.

\section{Treelayout}
Das Treelayout basiert auf Walkers Algorithmus. Dabei werden folgende Eigenschaften erreicht:
\begin{itemize}
   \item Alle Kinder eines Knotens sind auf der gleichen Ebene
   \item Alle Kinder eines Knotens haben den gleichen horizontalen Abstand zueinander
   \item \ldots
\end{itemize}.

Beim Layouten wird die Reihenfolge der Kinder von der \verb$IML_Reflection$ \"ubernommen. In der Grundfunktionalit\"at werden alle Kanten betrachtet. 

Als Erweiterung kann der Algorithmus nur bestimmte Kanten betrachten.
Werden mehr Knoten layoutet als durch die gegebenen Kanten erreicht werden k\"onnen, werden in einem zweiten Lauf diese Knoten im Matrixlayout (siehe \ref{Matrixlayout} rechts von dem Baum angeordnet.

\section{Matrixlayout}
\label{Matrixlayout}
Im Matrixlayout werden die Knoten in einem Quadrat ausgerichtet.
Dabei werden die Knoten mittels einer Breitensuche besucht und in dieser Besuchsreihenfolge in das Quadrat eingef\"ugt.


