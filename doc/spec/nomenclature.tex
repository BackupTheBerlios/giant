% ==============================================================================
%  $RCSfile: nomenclature.tex,v $, $Revision: 1.18 $
%  $Date: 2003/04/17 17:53:57 $
%  $Author: schwiemn $
%
%  Description: Begriffslexikon
%
%  Last-Ispelled-Revision: 1.14
%
% ==============================================================================

\begin{nomenclature}
  
  \term{Anfrage}{Query}{Eine Anfrage beschreibt einen Vorgang, bei dem
    �ber geeignete Kriterien Konten und Kanten aus dem IML-Graphen
    oder aus IML-Teilgraphen ausgew�hlt werden.}
  
  \term{Anzeigefenster}{Visualization Window}{Ein Fenster in dem ein
    Teilgraph des IML-Graphen nach bestimmten Kriterien visualisiert
    ist. Jedem Anzeigefenster ist ein Anzeigeinhalt zugeordnet.}
  
  \term{Anzeigeinhalt}{Window Content}{ Eine \gq{virtuelle} Oberfl�che
    auf der die Objekte des visualisierten Teilgraphen (also
    Fenster-Knoten und Fenster-Kanten) angeordnet sind, d.h. r�umliche
    Layoutinformation zu allen Objekten des entsprechenden
    Anzeigefensters. Abh�ngig von der Zoomstufe ist jeweils nur ein
    bestimmter Teil des Anzeigeinhaltes sichtbar - der sichtbare
    Anzeigeinhalt.  Die Gr��e des Anzeigeinhaltes ist theoretisch
    unbegrenzt.}
  
  \term{Fenster-Kante}{Window Edge}{Die graphische Repr�sentation einer
    IML-Kante innerhalb eines Anzeigefensters.}
  
  \term{Fenster-Knoten}{Window Node}{Die graphische Repr�sentation
    eines IML-Knoten innerhalb eines Anzeigefensters.}
  
  \term{Graph-Kante}{Graph Edge}{Eine Kante des IML-Graphen welche
    Bestandteil eines IML-Teilgraphen ist.}
  
  \term{Graph-Knoten}{Graph Node}{Ein Knoten des IML-Graphen welcher
    Bestandteil eines IML-Teilgraphen ist.}
  
  \term{IML-Graph}{IML Graph}{Der IML-Graph, wie er von der Bauhaus
    Reengineering GmbH gestellt wird.  Auf diesen Graphen wird �ber
    das sogenannte Reflection Model zugegriffen.}
  
  \term{IML-Kante}{IML Edge}{Eine Kante des IML-Graphen.}
  
  \term{IML-Knoten}{IML Node}{Ein Knoten des IML-Graphen.}
  
  \term{IML-Teilgraph}{IML Subgraph}{Eine Menge �ber Knoten und Kanten
    des IML-Graphen, die so gestaltet ist, dass sie einen Teilgraphen
    des IML-Graphen darstellt.}
  
  \term{Kantenklasse}{Edge Class}{Die Einteilung der Kanten des
    IML-Graphen in verschiedene Klassen, wie sie sich aus der
    IML-Graph-Bibliothek von Bauhaus ergibt.
    Innerhalb dieser Spezifikation ist der Begriff Kantenklasse so
    zu verstehen, dass jede Kante eindeutig zu genau einer Kantenklasse
    geh�rt, eine Vererbungshierarchie existiert nicht. Die Zuordnung
    einer Kante zu einer Kantenklasse wird durch die Knotenklasse
    des Start-Knotens und den Namen des Attributes
    (aus dem Bauhaus-IML-Graphen), welches die Kante
    beschreibt, festgelegt. Jede vorkommende Kombination aus der
    Knoten-Klasse eines Start-Knotens und dem Namen eines Attributes,
    welches eine Kante beschreibt, ist somit eine eigene Kantenklasse}

  \term{Klassenmenge}{Class Set}{Eine durch die IML-Graph Bibliothek
    vorgegebene Zusammenfassung von Kantenklassen und Knotenklassen.
    Es kann mehrere Klassenmengen geben. Knotenklassen und Kantenklassen
    k�nnen gleichzeitig zu mehreren Klassenmengen geh�ren.}

  \term{Knoten-Annotationen}{Node Annotation}{Eine textuelle
    Beschreibung zu einem bestimmten Knoten des IML-Graphen.}
  
  \term{Knotenklasse}{Node Class}{Die Einteilung der Knoten des
    IML-Graphen in verschiedene Klassen, wie sie sich aus der
    IML-Graph-Bibliothek von Bauhaus ergibt.
    Im Sinne der Verwendung dieses Begriffes innerhalb der Spezifikation
    liegt den Knotenklassen keine Vererbungshierarchie zu Grunde, jeder
    Knoten geh�rt also eindeutig zu genau einer Knotenklasse}
  
  \term{Layout}{Layout}{Die zweidimensionale r�umliche Anordnung von
    Fenster-Knoten und Fenster-Kanten innerhalb eines Anzeigefensters
    auf dem sogenannten Anzeigeinhalt.}
 
  \term{Reflektion}{Reflection Model}{Die Schnittstelle zum Zugriff
    auf den IML-Graphen.}
  
  \term{Schleife}{Loop}{Eine Kante mit identischem Start- und Zielknoten.
    Wird oft auch als Selbstkante bezeichnet.}
  
  \term{Selektion}{Selection}{Eine Auswahl von Knoten und Kanten eines
    visualisierten Teilgraphen des IML-Graphen innerhalb eines
    Anzeigefensters.}
  
  \term{Sichtbarer Anzeigeinhalt}{Visible Window Content}{ Der Bereich
    des Anzeigeinhaltes eines Anzeigefensters, der zur Zeit sichtbar
    dargestellt wird.}
  
  \term{Zoomstufe}{Zoom Level}{ Dieser Faktor beschreibt die Gr��e des
    sichtbaren Anzeigeinhaltes.  Bei einer sehr niedrigen Zoomstufe
    (auch: weit weg gezoomt) ist ein gr��erer Teil des im
    Anzeigefenster visualisierten IML-Graphen sichtbar als bei einer
    hohen Zoomstufe (auch: sehr nach heran gezoomt).}
  
  \term{hervorheben}{to highlight}{Hervorheben bedeutet, dass in einem
    Anzeigefenster visualisierte Fenster-Knoten oder Fenster-Kanten
    z.B.\ durch eine farbige Umrahmung von anderen Fenster-Knoten oder
    Fenster-Kanten unterscheidbar gemacht werden.}
    
  \term{selektieren}{to select}{Selektieren beschreibt einen Vorgang
    �ber den der Benutzer, z.B.\ durch Anklicken von Fenster-Knoten
    oder Fenster-Kanten mit der Maus, eine Selektion aufbaut.}

\end{nomenclature}

%%% Local Variables: 
%%% TeX-master: "spec"
%%% End: 
