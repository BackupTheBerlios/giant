\begin{nomenclature}

\term{Begriff}{Englische �bersetzung}{Erkl�rung}


\term{Anzeigefenster}{XXXXXXX Window}{
Ein Fenster in dem ein Teilgraph des IML-Graphen nach bestimmten Kriterien 
visualisiert ist. Jedem Anzeigefenster ist ein Anzeigeinhalt zugeordnet.
}

\term{Anzeigeinhalt}{XXXXXXX}{
Eine "virtuelle" Oberfl�che auf der die Objekte des visualisierten Teilgraphen
angeordnet sind, d.h. r�umliche Layoutinformation zu allen Objekten des 
entsprechenden Anzeigefensters. Abh�ngig von der Zoomstufe ist jeweils nur ein 
bestimmter Teil des Anzeigeinhaltes sichtbar - der scihtbare Anzeigeinhalt. 
Die Gr��e des Anzeigeinhaltes ist theoretisch unbegrenzt. 
}

\term{Sichtbarer Anzeigeinhalt}{XXXXXXXXXXXX}{
Der Bereich des Inhaltes eines Anzeigefensters, der zur Zeit sichtbar 
dargestellt wird.
}

\term{IML-Graph}{IML Graph}{
Der IML-Graph wie er von der Bauhaus Reengineering GmbH gestetellt wird.
Auf diesen Graphen wird nur �ber das sogenannte Reflection-Model zugegriffen.
}

\term{Graph-Selektion ANDERER NAME DA KEINE SELEKTION}{Graph Selection}{
Eine Menge �ber Knoten und Kanten des IML-Graphen, die so gestaltet ist,
dass sie einen Teilgraphen des IML-Graphen darstellt.
}

\term{Fenster-Selektion}{Window Selection}{
Eine Auswahl von Knoten und Kanten eines visualisierten Teilgraphen des 
IML-Graphen innerhalb eines Anzeigefensters
}

\term{IML-Knoten}{IML Node}{
Erkl�rung
}

\term{IML-Kante}{IML Edge}{
Erkl�rung
}

\term{Fenster-Knoten}{Window Node}{Erkl�rung}

\term{Fenster-Kante}{Window Edge}{Erkl�rung}

\term{Graph-Knoten}{Graph Node}{Erkl�rung}

\term{Graph-Kante}{Graph Edge}{Erkl�rung}

\term{Knotenklasse}{XXXXXXX}{Erkl�rung}

\term{Kantenklasse}{XXXXXXX}{Erkl�rung}

\term{Layout}{XXXXXXX}{Erkl�rung}

\term{Selektieren}{XXXXXXX}{Erkl�rung}



\end{nomenclature}

%%% Local Variables: 
%%% TeX-master: "spec"
%%% End: 
