% ==============================================================================
%  $RCSfile: nomenclature.tex,v $, $Revision: 1.11 $
%  $Date: 2003/02/18 15:21:04 $
%  $Author: schwiemn $
%
%  Description:
%
% ==============================================================================

\begin{nomenclature}

\term{Anzeigefenster}{Visualisation Window}{
Ein Fenster in dem ein Teilgraph des IML-Graphen nach bestimmten Kriterien 
visualisiert ist. Jedem Anzeigefenster ist ein Anzeigeinhalt zugeordnet.
}

\term{Anzeigeinhalt}{Window Content}{
Eine \gq{virtuelle} Oberfl�che auf der die Objekte des visualisierten Teilgraphen
(also Fenster-Knoten und Fenster-Kanten) angeordnet sind, 
d.h. r�umliche Layoutinformation zu allen Objekten des 
entsprechenden Anzeigefensters. Abh�ngig von der Zoomstufe ist jeweils nur ein 
bestimmter Teil des Anzeigeinhaltes sichtbar - der sichtbare Anzeigeinhalt. 
Die Gr��e des Anzeigeinhaltes ist theoretisch unbegrenzt. 
}

\term{Sichtbarer Anzeigeinhalt}{Visible Window Content}{
Der Bereich des Anzeigeinhaltes eines Anzeigefensters, der zur Zeit sichtbar 
dargestellt wird.
}

\term{Zoomstufe}{Zoom Level}{
Dieser Faktor beschreibt die Gr��e des sichtbaren Anzeigeinhaltes.
Bei einer sehr niedrigen Zoomstufe (auch: weit weg gezoomt) ist ein gr��erer 
Teil des im Anzeigefenster visualisierten IML-Graphen sichtbar als bei einer
hohen Zoomstufe (auch: sehr nach heran gezoomt).
}

\term{IML-Graph}{IML Graph}{
Der IML-Graph, wie er von der Bauhaus Reengineering GmbH gestetellt wird.
Auf diesen Graphen wird �ber das sogenannte Reflection Model zugegriffen.
}

\term{Reflektion}{Reflection Model}{
Die Schnittstelle zum Zugriff auf den IML-Graphen.
}

\term{IML-Teilgraph}{IML Subgraph}{
Eine Menge �ber Knoten und Kanten des IML-Graphen, die so gestaltet ist,
dass sie einen Teilgraphen des IML-Graphen darstellt.
}

\term{Anfrage}{Query}{
Eine Anfrage beschreibt einen Vorgang, bei dem �ber geeignete Kriterien
Konten und Kanten aus dem IML-Graphen oder aus IML-Teilgraphen ausgew�hlt
werden.
}

\term{Selektion}{Selection}{
Eine Auswahl von Knoten und Kanten eines visualisierten Teilgraphen des 
IML-Graphen innerhalb eines Anzeigefensters.
}

\term{IML-Knoten}{IML Node}{
Ein Knoten des IML-Graphen.
}

\term{IML-Kante}{IML Edge}{
Eine Kante des IML-Graphen.
}

\term{Graph-Knoten}{Graph Node}{
Ein Knoten des IML-Graphen welcher Bestandteil eines IML-Teilgraphen ist.
}

\term{Graph-Kante}{Graph Edge}{
Eine Kante des IML-Graphen welche Bestandteil eines IML-Teilgraphen ist.
}

\term{Fenster-Knoten}{Window Node}{
Die grafische Repr�sentation eines IML-Knoten innerhalb eines Anzeigefensters.
}

\term{Fenster-Kante}{Window Edge}{
Die grafische Repr�sentation einer IML-Kante innerhalb eines Anzeigefensters.
}

\term{Knoten-Annotationen}{Node Annotation}{
Eine textuelle Beschreibung zu einem bestimmten Knoten des IML-Graphen.
}


\term{Knotenklasse}{Node Class}{
Die Einteilung der Knoten des IML-Graphen in verschiedene Klassen, 
wie sie sich aus der IML-Graph-Bibliothek von Bauhaus ergibt.
}

\term{Kantenklasse}{Edge Class}{
Die Einteilung der Kanten des IML-Graphen in verschiedene Klassen, 
wie sie sich aus der IML-Graph-Bibliothek von Bauhaus ergibt.
}

\term{Layout}{Layout}{
Die zweidimensionale r�umliche Anordnung von Fenster-Knoten und Fenster-Kanten innerhalb
eines Anzeigefensters auf dem sogenannten Anzeigeinhalt.
}


\term{selektieren}{to select}{
Selektieren beschreibt einen Vorgang �ber den der Benutzer, z.B.\ durch Anklicken 
von Fenster-Knoten oder Fenster-Kanten mit der Maus, eine Selektion aufbaut.
}


\term{hervorheben}{to highlight}{
Hervorheben bedeutet, dass in einem Anzeigefenster visualisierte Fenster-Knoten oder Fenster-Kanten 
z.B.\ durch  eine farbige Umrahmung von anderen Fenster-Knoten oder Fenster-Kanten unterscheidbar
gemacht werden
}

\end{nomenclature}

%%% Local Variables: 
%%% TeX-master: "spec"
%%% End: 
