% ==============================================================================
%  $RCSfile: todo_spec.tex,v $ 
%  $Date: 2003/02/06 16:06:58 $
%  $Author: schwiemn $ 
%
%  Description: ToDo File f�r die Spezifikation
%
% ==============================================================================

%=======================================================================
\section {Konfigurationsdatei}
Konfigurationsdatei in extra Kapitel komplett beschreiben
- was kann konfiguriert werden ?


%===============================================================================
%
% Anforderungen des Kunden
%
\section {Anforderungen des Kunden}
Anbei die Anforderungen des Kunden gem�� der Email vom 30.01.2003

\begin{itemize}

\item Form: DIN A4, gen�gend Platz f�r Korrektur und handschriftliche 
Anmerkungen


\item Gedruckte Dokumentation + PDF

\item Versionierung: Jedes Dokument erh�lt eine Revisionsnummer, 
Erstellungsdatum wird vermerkt

\item Inhaltliche Form: Jedes Dokument verf�gt �ber ein Inhaltsverzeichnis,
Index ist erw�nscht, Seitenzahlen


\item Referenzen: Konsistente globale Nummerierung f�r alle Dokumente
(lebenslang und f�r immer und ewig)

\item Querverweise (nat�rlich innerhalb des selben Dokuments), 
Referenzen auf weitere Dokumente �ber Seite + Abschnitt

\item Inhalt: Begriffslexikon, Konzepte der GUI, 
Erkl�rungen der GUI zum Beispiel anhand abstrakter Bilder,
Usecases mit Zielsetzung spezifizieren

\item vollst�ndige Abdeckung der Anforderungen an das Produkt

\item pr�zise, knapp

\item Traceability: \\
Spezifikation ist der erste Schritt auf dem langen
(und beschwerlichen) Weg zum fertigen Software-Produkt und
dient als Grundlage f�r das weitere Vorgehen.\\
Entwurf, Implementierung und Test referenzieren sie.
Testf�lle f�r den Systemtest sollen aus der Spezifikation
ableitbar sein.


\item Sprache:\\
Wenn Deutsch als Sprache gew�hlt wird, sollten englische Fach-Begriffe
verwendet werden. (Die selben Begriffe werden in Spezifikation
und Implementierung verwendet.)\\
Keine Eindeutschung.\\
Einheitliche Rechtschreibung.

\end{itemize}


%=============================================================================
%
% Mit Kunden zu kl�ren
%
\section {Mit Kunden zu kl�rende Punkte}
Alles was noch offen ist.

\begin{itemize}

\item Interaktionssprache von GIANT (insbeondere der GUI und gegebenfalls innerhalb der Konfigurationsdateien) mit Kunden kl�ren (Deutsch oder Englisch).


\end{itemize}