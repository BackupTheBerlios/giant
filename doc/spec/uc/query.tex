% ==============================================================================
%  $RCSfile: query.tex,v $, $Revision: 1.2 $
%  $Date: 2003/02/21 17:08:30 $
%  $Author: schwiemn $
%
%  Description: Use-Cases f�r die Anfragen
%
% ==============================================================================

% ==============================================================================
\begin{uc}[Anfrage ausf�hren]{UC: Neue Anfrage ausf�hren}
Mit diesem UseCase kann eine Anfrage �ber den Anfragedialog eingegeben werden.
Die M�glichkeiten der Anfragesprache sind unter <<<<<<<<<<VERWEIS >>>>>>>>>><
beschrieben.

  \begin{precond}
    \cond Ein Projekt ist geladen.
  \end{precond}

  \begin{postsuccess}
    \cond Die Anfrage wurde ausgef�hrt. Alle Ergebnisse liegen vor.

  \end{postsuccess}

  \begin{postfail}
    \cond Das System bleibt im bisherigen Zustand.
  \end{postfail}
  
  \begin{proc}    
    \step[1]
    Der Benutzer startet den UseCase durch Auswahl des Eintrages \gq{Anfrage
    ausf�hren} aus dem Hauptmen� 
    <<<<<<<<<<<<<<<<<<<<<<VERWEIS GUI>>>>>>>>>>>>>>>>>>>>>
    
    \step[2] 
    GIANT �ffnet den Anfragedialog <<<<VERWEIS GUI>>>>.
      
    \step[3] 
    Der Benutzer gibt dort im daf�r vorgesehenen Textfeld die Anfrage ein
    und best�tigt mit OK.
        
    \step[4]
    GIANT f�hrt die entsprechende Anfrage aus.\\
    W�hrend der Abarbeitung der Anfrage ist das System nicht bedienbar.\\
    Die Abarbeitung kann aber zu jeder Zeit abgebrochen werden.
  
  \end{proc}

  \begin{aproc}
    \astep{2} Der Benutzer bricht mit Cancel ab.
  \end{aproc}

\end{uc}


% ==============================================================================
\begin{uc}[Label]{UC: Anfrageverwaltung (History)}
\end{uc}

% ==============================================================================
\begin{uc}[Label]{UC: Anfrage erstellen}
\end{uc}

% ==============================================================================
\begin{uc}[Label]{UC: Anfrage l�schen}
\end{uc}
