% ==============================================================================
%  $RCSfile: query.tex,v $, $Revision: 1.4 $
%  $Date: 2003/03/28 13:21:47 $
%  $Author: schwiemn $
%
%  Description: Use-Cases f�r die Anfragen
%
% ==============================================================================

% ==============================================================================
\begin{uc}[Anfrage ausf�hren]{UC: Neue Anfrage ausf�hren}
Mit diesem UseCase kann eine Anfrage �ber den Anfragedialog
(siehe \ref{GUI Anfragedialog}) eingegeben werden.
Die M�glichkeiten der GQSL sind im Detail unter Kapitel
\ref {GIANT Query Skripting Language} beschrieben.


  \begin{precond}
    \cond Ein Projekt ist geladen.
  \end{precond}

  \begin{postsuccess}
    \cond Die Anfrage wurde ausgef�hrt. Alle Ergebnisse liegen vor.
      
  \end{postsuccess}

  \begin{postfail}
    \cond Das System bleibt im bisherigen Zustand.

    \cond Wurde der UseCase w�hrend der Berechnung des
          Anfrageergebnisses durch GIANT abgebrochen, so
          gehen s�mtliche bereits fertig gestellten Teilergebnisse
          verloren. 
  \end{postfail}
  
  \begin{proc}    
    \step[1]
    Der Benutzer startet den UseCase durch Auswahl des Eintrages 
    \gq{Tools -- Execute GQSL Query} im Hauptmen�  
    (siehe hierzu \ref{Main-Window-Tools}).
      
    \step[2] 
    GIANT �ffnet den Anfragedialog (siehe \ref{GUI Anfragedialog}).
      
    \step[3]
    Der Benutzer gibt dort im daf�r vorgesehenen Textfeld die GQSL Anfrage 
    (siehe auch \ref {GIANT Query Skripting Language}) ein und best�tigt 
    mit \gq {Start Query}.

    \step[4]
    GIANT pr�ft das eingegebene GQSL Skript. Sollte das Skript
    nicht den Vorgaben der Grammatik 
    (siehe \ref {GIANT Query Skripting Language} entsprechen, erscheint
    eine Fehlermeldung (siehe \ref {afa Fehlerverhalten})
    und das System kehrt zu Schritt 3 des UseCase zur�ck.
        
    \step[5]    
    GIANT berechnet die Anfrage und teilt dem Benutzer
    den Fortschritt mittels eines Progressbars mit.
    !!!!!!!!!!!VERWEIS GUI!!!!!!!!!!!!!!!!!!!!!!!!!!!!!!!!!!!
    W�hrend der Abarbeitung der Anfrage ist das System nicht bedienbar.\\
    
  
  \end{proc}

  \begin{aproc}
    \astep{3} Der Benutzer bricht mit Cancel ab.

    \astep{5} Die laufende Berechnung des Anfrageergebnisses 
    kann vom Benutzer jeder Zeit durch Bet�tigen des
    Buttons \gq{Stop Calculation} abgebrochen werden.
     !!!!!!!!!!!!VERWEIS GUI!!!!!!!!!!!!!!!!!!!!!!!.

  \end{aproc}

\end{uc}


% ==============================================================================
\begin{uc}[Label]{UC: Anfrage laden}
Der Benutzer kann zus�tzlich zur manuellen Eingabe von GQSL Anfragen
(siehe \ref{Anfrage ausf�hren}) auch gespeicherte Anfragen aus 
einer Anfragedatei (siehe \ref {Config Anfrage-Dateien}) laden.

  \begin{precond}
    \cond Ein Projekt ist geladen.
  \end{precond}

  \begin{postsuccess}
    \cond Die Anfrage wurde ausgef�hrt. Alle Ergebnisse liegen vor.
      
  \end{postsuccess}

  \begin{postfail}
    \cond Das System bleibt im bisherigen Zustand.

    \cond Wurde der UseCase w�hrend der Berechnung des
          Anfrageergebnisses durch GIANT abgebrochen, so
          gehen s�mtliche bereits fertig gestellten Teilergebnisse
          verloren. 
  \end{postfail}
  
  \begin{proc}    
    \step[1]
    Der Benutzer startet den UseCase durch Auswahl des Eintrages 
    \gq{Tools -- Execute GQSL Query} im Hauptmen�  
    (siehe hierzu \ref{Main-Window-Tools}).
      
    \step[2] 
    GIANT �ffnet den Anfragedialog (siehe \ref{GUI Anfragedialog}).
      
    \step[3] 
    Der Benutzer bet�tigt im Dialog den Button \gq{Load Query}.

    \step[4]
    Daraufhin zeigt GIANT den Standard-Filechooser-Dialog 
    (siehe \ref {Standard-Filechooser-Dialog}).

    \step[5]
    Der Benutzer w�hlt die gew�nschte Anfragedatei (siehe 
    \ref {Config Anfrage-Dateien}) aus.
        
    \step[6]
    GIANT zeigt das aus der Anfragedatei geladene GQSL Skript 
    (siehe \ref {GIANT Query Skripting Language}) im Textfeld
    des Anfragedialoges an.

    \step[7]
    Falls gew�nscht kann der Benutzer das GQSL Skript im Textfeld
    noch manuell weiter modifizieren.
    
    \step[8]
    Der Benutzer startet die Berechnung der Anfrage durch Bet�tigung
    des \gq{Start Query} im Anfragedialog (siehe \ref{GUI Anfragedialog}).

    \step[9]
    GIANT pr�ft das geladene und eventuell modifizierte GQSL Skript. 
    Sollte das Skript nicht den Vorgaben der Grammatik 
    (siehe \ref {GIANT Query Skripting Language} entsprechen, erscheint
    eine Fehlermeldung (siehe \ref {afa Fehlerverhalten})
    und das System kehrt zu Schritt 7 des UseCase zur�ck.

    \step[10]
    GIANT berechnet die Anfrage und teilt dem Benutzer
    den Fortschritt mittels eines Progressbars mit.
    !!!!!!!!!!!VERWEIS GUI!!!!!!!!!!!!!!!!!!!!!!!!!!!!!!!!!!!
    W�hrend der Abarbeitung der Anfrage ist das System nicht bedienbar.\\

  \end{proc}

  \begin{aproc}
    \astep{3} Der Benutzer bricht den UseCase mit Cancel ab.
    \astep{4} Der Benutzer bricht die Auswahl der Anfragedatei mit Cancel ab.
              Das System kehrt dann zu Schritt 2 bei der
              Abarbeitung des UseCase zur�ck.
    \astep{6} Der Benutzer bricht den UseCase mit Cancel ab.

    \astep{10} Die laufende Berechnung des Anfrageergebnisses 
    kann vom Benutzer jeder Zeit durch Bet�tigen des
    Buttons \gq{Stop Calculation} abgebrochen werden.
     !!!!!!!!!!!!VERWEIS GUI!!!!!!!!!!!!!!!!!!!!!!!.

  \end{aproc}

\end{uc}

% ==============================================================================
\begin{uc}[Label]{UC: Anfrage speichern}
Mit diesem UseCase kann der Benutzer GQSL Skripte aus dem 
Anfragedialog (siehe \ref{GUI Anfragedialog}) in Anfragedateien 
(siehe \ref {Config Anfrage-Dateien}) speichern.

  \begin{precond}
    \cond Der Anfragedialog (siehe \ref{GUI Anfragedialog}) ist
          ge�ffnet und enth�lt in dem daf�r vorgesehenen Textfeld
          entweder ein manuell eingegebenes oder ein aus einer
          Datei geladenes und eventuell modifiziertes GQSL Skript.

  \end{precond}

  \begin{postsuccess}
    \cond Eine Anfragedatei, welche das GQSL Skript enth�lt, wurde
          angelegt.
    \cond GIANT zeigt den Anfragedialog, das gespeicherte GQSL Skript
          ist weiterhin in dem Textfeld vorhanden.
      
  \end{postsuccess}

  \begin{postfail}
    \cond Das System bleibt im bisherigen Zustand.
    \cond Es wurd keine Anfragedatei erzeugt
    \cond GIANT zeigt weiterhin den  Anfragedialog 
          (siehe \ref{GUI Anfragedialog}) und alle dort get�tigten
          Eingaben (insbesondere das GQSL Skript im Textfeld des
          Dialoges) bleiben erhalten.
   
  \end{postfail}
  
  \begin{proc}    

    \step[1]
    Der Benutzer bet�tigt im Anfragedialog (siehe \ref{GUI Anfragedialog})
    den Button \gq{Save Query}.
    
    \step[2]
    GIANT pr�ft das GQSL Skript im Textfeld des Anfragedialoges den
    Vorgaben der Grammatik (siehe \ref {GIANT Query Skripting Language} 
    entspricht. Falls nein erscheint eine Fehlermeldung 
    (siehe \ref {afa Fehlerverhalten})
    und das System kehrt zu Schritt 1 des UseCase zur�ck.

    \step[3]
    GIANT �ffnet den Standard-Filechooser-Dialog (siehe 
    \ref {Standard-Filechooser-Dialog}).
  
    \step[4]
    Der Benutzer gibt den Pfad und die Datei, in der das GQSL Skript
    gespeichert werden soll, vor und best�tigt mit OK.


    \step[5]
    GIANT speichert das GQSL Skript in der vorgegebenen Anfragedatei.

  \end{proc}

  \begin{aproc}
    \astep{4} Der Benutzer bricht den UseCase mit Cancel ab.

  \end{aproc}

\end{uc}

