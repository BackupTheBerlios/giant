% ==============================================================================
%  $RCSfile: selection.tex,v $, $Revision: 1.2 $
%  $Date: 2003/02/19 18:02:50 $
%  $Author: schwiemn $
%
%  Description: Use-Cases f�r die Selektionen
%
%
% ==============================================================================


\begin{uc}[Selektion zur aktuellen Selektion machen]
          {UC: Selektion zur aktuellen Selektion machen}
	  
Dieser UseCase dient dazu, eine Selektion zur aktuellen Selektion zu machen.
Dies ist n�tig, da nur die aktuelle Selektion mittels der Maus
modifiziert werden kann. 
Siehe auch \ref{Aktuelle Selektion vs Selektionen}. 
	  
	 
  \begin{precond}
    \cond Es gibt ein Anzeigefenster mit mindestens einer Selektion.
  \end{precond}

  \begin{postsuccess}
    \cond Die vorherige aktuelle Selektion ist nicht mehr aktuell.
    \cond Die entsprechende Selektion ist nun die aktuelle Selektion und
          als solche erkennbar angezeigt und hervorgehoben.
    
  \end{postsuccess}
 
  \begin{proc}    
    \step[1]
    Der Benutzer f�hrt mit der linken Maustaste einen Doppelklick 
    auf eine nicht aktuelle Selektion durch.

    \step[2]
    GIANT macht die entsprechende Selektion zur aktuellen Selektion.  
  \end{proc}
	  
\end{uc}


% ==============================================================================
\begin{uc}[Aktuelle Selektion zur�ck stufen]
          {UC: Aktuelle Selektion zur�ck stufen}
	  
Siehe auch \ref{Aktuelle Selektion vs Selektionen}. Mit diesem UseCase kann
eine aktuelle Selektion auf den Status einer \gq{normalen} Selektion
zur�ckgestuft werden.
	  
	 
  \begin{precond}
    \cond Es gibt ein Anzeigefenster mit einer aktuellen Selektion.
  \end{precond}

  \begin{postsuccess}
    \cond Es gibt keine aktuelle Selektion mehr.
    
  \end{postsuccess}
 
  \begin{proc}    
    \step[1]
    Der Benutzer f�hrt mit der linken Maustaste einen Doppelklick 
    auf die aktuelle Selektion aus.

    \step[2]
    GIANT stuft die aktuelle Selektion auf den Status einer \gq{normalen}
    Selektion zur�ck.
  \end{proc}
    
\end{uc}


% ==============================================================================
\begin{uc}[Selektion graphisch hervorheben]
          {UC: Selektion graphisch hervorheben}
	  
Dieser UseCase dient zum Hervorheben von Selektionen innerhalb eines
Anzeigefensters.

  \begin{precond}
    \cond Es gibt ein Anzeigefenster mit mindestens einer Selektion.
   
  \end{precond}

  \begin{postsuccess}
    \cond 
    Die Selektion ist im entsprechenden Anzeigefenster hervorgehoben.
    
    \cond
    Die Selektion, welche vorher mit der gleichen Farbe hervorgehoben war,
    ist nicht mehr hervorgehoben.
 
    
  \end{postsuccess}
 
  \begin{proc}    
    \step[1]
    Der Benutzer startet den UseCase �ber das PopUp Men�
    >>>    Dort gibt es drei Eint�ge f�r jede definierbare Farbe
        
      
    \step[2]
    GIANT hebt die Selektion mit der entsprechenden Farbe hervor.
    
  \end{proc}



\end{uc}


% ==============================================================================
\begin{uc}[Graphische Hervorhebung einer Selektion aufheben]
      {UC: Graphische Hervorhebung einer Selektion aufheben}
      
Dieser UseCase dient dazu, die Hervorhebung von Selektionen innerhalb eines 
Anzeigefensters aufzuheben.
      
  \begin{precond}
    \cond 
    Es gibt ein Anzeigefenster mit mindestens einer hervorgehobenen
    Selektion.
   
  \end{precond}

  \begin{postsuccess}
    \cond 
    Die Selektion ist im entsprechenden Anzeigefenster nicht mehr 
    hervorgehoben.
 
  \end{postsuccess}

  
  \begin{proc}    
    \step[1]
    Der Benutzer startet den UseCase �ber das PopUp Men�:
    \\VERWEIS   
    

    \step[2]
    GIANT setzt die Hervorhebung der Selektion zur�ck.
    
  \end{proc} 
      
      
\end{uc}

%===============================================================================
\begin{uc}[Neue Selektion anlegen]{UC: Neue Selektion anlegen}

Mit diesem UseCase k�nnen neue, leere Selektionen angelegt werden.

  \begin{precond}
    \cond Es gibt ein ge�ffnetes Anzeigefenster.
  \end{precond}

  \begin{postsuccess}
    \cond Eine neue Selektion mit entsprechendem Namen
         ist angelegt und erscheint in der Liste der Selektionen 
	 (SELECTION\_LIST).
    \cond Diese neue Selektion hat keinen Inhalt (selektierte Knoten und
          Kanten).

  \end{postsuccess}

  \begin{postfail}
    \cond Das System bleibt im bisherigen Zustand.
  \end{postfail}
  
  \begin{proc}    
    \step[1]
    Der Benutzer startet den UseCase �ber das PopUp Men�:
    !!!!!ENTSPRECHENDEN MEN�EINTAG NENNEN.
    
    \step[2] 
    GIANT �ffnet den allgemeinen Texteingabedialog <<<<VERWEIS GUI>>>>.
      
    \step[3] 
    Der Benutzer gibt dort einen zul�ssigen Namen f�r die neue Selektion 
    ein und best�tigt mit OK.\\
    Hat bereits eine andere Selektion innerhalb des Anzeigefensters den selben
    Namen erscheint eine Fehlermeldung.
    
    \step[4]
    GIANT erzeugt die neue Selektion.
  
  \end{proc}

  \begin{aproc}
    \astep{3} Der Benutzer bricht die Verarbeitung mit Cancel ab.
  \end{aproc}

\end{uc}


%===============================================================================
\begin{uc}[Selektion kopieren]{UC: Selektion kopieren}
Dieser UseCase dient zum Kopieren von Selektionen innerhalb eines 
Anzeigefensters (nicht zum Kopieren in ein anderes Anzeigefenster).

  \begin{precond}
    \cond Es gibt ein ge�ffnetes Anzeigefenster mit einer Selektion.
  \end{precond}

  \begin{postsuccess}
    \cond 
    Eine neue Selektion mit entsprechendem Namen ist angelegt 
    und erscheint in der Liste der Selektionen (SELECTION\_LIST).
    \cond
    Die neue Selektion umfasst die gleichen Knoten und Kanten wie die
    Selektion, von der kopiert wurde.

  \end{postsuccess}

  \begin{postfail}
    \cond Das System bleibt im bisherigen Zustand.
  \end{postfail}
  
  \begin{proc}    
    \step[1]
    Der Benutzer startet den UseCase durch Rechtsklick auf die zu kopierende
    Selektion (\gq{Quellselektion}) und w�hlt aus dem PopUp Men�:
    !!!!!ENTSPRECHENDEN MEN�EINTAG NENNEN.
    
    \step[2] 
    GIANT �ffnet den allgemeinen Texteingabedialog. <<<<VERWEIS GUI>>>>.
      
    \step[3] 
    Der Benutzer gibt dort einen zul�ssigen Namen f�r die neue Selektion 
    ein und best�tigt mit OK.\\
    Hat bereits eine andere Selektion innerhalb des Anzeigefensters den selben
    Namen erscheint eine Fehlermeldung.
    
    \step[4]
    GIANT kopiert die Quellselektion und legt eine neue Selektion an.
  
  \end{proc}

  \begin{aproc}
    \astep{3} Der Benutzer bricht die Verarbeitung mit Cancel ab.
  \end{aproc}

\end{uc}
%===============================================================================
\begin{uc}[Selektion umbenennen]{UC: Selektion umbenennen}
>>>> WEG LASSEN - KANN AUCH DURCH KOPIEREN UND L�SCHEN ERLEDIGT WERDEN

\end{uc}
%===============================================================================
\begin{uc}[Label]{UC: Selektion l�schen}

Dieser UseCase dient zum L�schen von Selektionen innerhalb eines 
Anzeigefensters.

  \begin{precond}
     \cond Es gibt ein ge�ffnetes Anzeigefenster mit einer Selektion.
   \end{precond}


  \begin{postsuccess}
    \cond 
    Die entsprechende Selektion ist gel�scht.
    
    \cond 
    Die Fenster-Knoten und Fenster-Kanten, 
    die zu dieser Selektion geh�rten, werden nicht gel�scht.
	  
    \cond War die Selektion hervorgehoben, so wird die entsprechende
    Hervorhebung der Fenster-Knoten und Fenster-Kanten aufgehoben.
    
  \end{postsuccess}

  \begin{postfail}
    \cond Das System bleibt im bisherigen Zustand.
  \end{postfail}
  
  \begin{proc}    
    \step[1]
    Der Benutzer startet den UseCase durch Rechtsklick auf die zu l�schende
    Selektion (\gq{Quellselektion}) und w�hlt aus dem PopUp Men�:
    !!!!!ENTSPRECHENDEN MEN�EINTAG NENNEN.
    
    \step[2]
    GIANT l�scht die entsprechende Selektion.
  
  \end{proc}

\end{uc}
%===============================================================================
\begin{uc}[Selektion manuell modifizieren]
         {UC: Selektionen manuell modifizieren}
	 
Jeweils die aktuelle Selektion kann mittels der Maus modifiziert 
werden. 


  \begin{precond}
     \cond 
     Es gibt ein ge�ffnetes Anzeigefenster mit mindestens
     einer Selektion.
   \end{precond}


  \begin{postsuccess}
    \cond 
    Die entsprechenden �nderungen an der Selektion werden von Giant
    sofort durchgef�hrt und �bernommen.
    
  \end{postsuccess}

  
  \begin{proc}    
    \step[1]
    Falls noch nicht der Fall, macht der Benutzer die zu modifizierende 
    Selektion zur aktuellen Selektion
    (siehe \ref{Selektion zur aktuellen Selektion machen}).
    
    \step[2]
    Mittels der unter 
    \ref{Selektieren von Fenster-Knoten und Fenster-Kanten in Anzeigefenstern}
    beschriebenen M�glichkeiten f�gt der Benutzer der Selektion neue
    Fenster-Knoten und Fenster-Kanten hinzu oder entfernt bestehende
    Fenster-Knoten und Fenster-Kanten aus der Selektion.
    
  \end{proc}


\end{uc}
%===============================================================================
\begin{uc}[Label]{UC: Mengenvereinigung von 2 Selektionen}
\end{uc}
%===============================================================================
\begin{uc}[Label]{UC: Mengendifferenz von 2 Selektionen}
\end{uc}
%===============================================================================
\begin{uc}[Label]{UC: Mengenschnitt von 2 Selektionen}
\end{uc}
