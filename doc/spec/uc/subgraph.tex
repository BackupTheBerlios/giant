% ==============================================================================
%  $RCSfile: subgraph.tex,v $, $Revision: 1.6 $
%  $Date: 2003/03/19 19:54:17 $
%  $Author: stupro $
%
%  Description: Use-Cases f�r IML-Teilgraphen
%
% ==============================================================================

\begin{uc}[IML-Teilgraph graphisch hervorheben]
      {UC: IML-Teilgraph graphisch hervorheben}
	  
Dieser UseCase dient zum Hervorheben von IML-Teilgraphen innerhalb der
Anzeigefenster.

  \begin{precond}
    \cond Es gibt mindestens einen IML-Teilgraphen.
   
  \end{precond}

  \begin{postsuccess}
    \cond 
    Der IML-Teilgraph ist in jedem ge�ffneten Anzeigefenster entsprechend
    hervorgehoben.
    
    \cond
    Der IML-Teilgraph, welcher vorher mit der gleichen Farbe 
    hervorgehoben war, ist nicht mehr hervorgehoben.
 
    
  \end{postsuccess}
 
  \begin{proc}    
    \step[1]
    Der Benutzer startet den UseCase �ber das PopUp Men� der Selektionsauswahlliste
    \ref{Selektionsauswahlliste} durch Auswahl von \gq{Highlight Selection Color 1 (2,3)}
            
    \step[2]
    GIANT hebt den IML-Teilgraphen in allen ge�ffneten Anzeigefenstern
    mit der entsprechenden Farbe hervor.
    
  \end{proc}


\end{uc}



% ==============================================================================
\begin{uc}[Graphische Hervorhebung von IML-Teilgraphen aufheben]
      {UC: Graphische Hervorhebung von IML-Teilgraphen aufheben}

Mit diesem UseCase kann die grafische Hervorhebung von IML-Teilgraphen 
aufgehoben werden.

  \begin{precond}
    \cond 
    Es gibt einen hervorgehobenen IML-Teilgraphen.
   
  \end{precond}

  \begin{postsuccess}
    \cond 
    Der IML-Teilgraph wird wieder in der normalen Darstellung angezeigt.
 
  \end{postsuccess}

  
  \begin{proc}    
    \step[1]
    Der Benutzer startet den UseCase durch Rechtklick auf den
    entsprechenden IML-Teilgraphen in der Selektionsauwahlliste
    \ref{Selektionsauswahlliste} und Auswahl des Men�punktes
    \gq{Unhighlight Selection}.
    

    \step[2]
    GIANT setzt die Hervorhebung des IML-Teilgraphen zur�ck.
    
  \end{proc} 


\end{uc}


% ==============================================================================
\begin{uc}[IML-Teilgraph aus einer Selektion erzeugen]
      {UC: IML-Teilgraph aus einer Selektion erzeugen}

Leitet einen neuen IML-Teilgraphen aus einer Quell-Selektion ab.
Siehe hierzu \ref{IML-Teilgraphen aus Selektion ableiten}.


  \begin{precond}
     \cond 
     Es gibt ein ge�ffnetes Anzeigefenster.
     
     \cond
     Es gibt mindestens eine Selektion.
     
   \end{precond}


  \begin{postsuccess}
    \cond 
    Es wurde ein neuer IML-Teilgraph mit entsprechendem Namen erzeugt.
    
  \end{postsuccess}
  
  \begin{postfail}
    \cond Das System bleibt im bisherigen Zustand.
  \end{postfail}
  
  \begin{proc} 
     
    \step[1]
    Der Benutzer f�hrt einen Rechtsklick auf die Quell-Selektion im
    der Selektionsauswahlliste \ref{Selektionsauswahlliste} aus
    und w�hlt im PopUp Men� den Eintrag New IML Subgraph from Selection
     
    \step[2]
    GIANT zeigt den allgemeinen Texteingabedialog \gq{Please enter Name for
    new IML Subgraph from Selection X in Window Y}
     
    \step[3]
    Der Benutzer gibt einen Namen f�r den neu zu erstellenden IML-Teilgraphen
    ein und best�tigt mit OK.\\
    Gibt der Benutzer hier keinen Namen ein, so vergibt GIANT automatisch 
    einen Namen.
     
    \step[4]
    GIANT erzeugt gem�� der unter IML-Teilgraphen aus Selektion ableiten
    \ref{IML-Teilgraphen aus Selektion ableiten}
    beschriebenen Konvention eine neuen IML-Teilgraphen aus der Selektion im
    Anzeigefenster, in welchem der Men�punkt ausgew�hlt wurde.

  \end{proc}
  
  \begin{aproc}
    \astep{3} Der Benutzer bricht den UseCase mit Cancel ab.
  \end{aproc}

\end{uc}



% ==============================================================================
\begin{uc}[IML-Teilgraph kopieren]
      {UC: IML-Teilgraph kopieren}
      
 Kopiert einen Quell-IML-Teilgraphen in einen neuen IML-Teilgraphen.
 Bestehende IML-Teilgraphen k�nnen nicht �berschrieben werden.

  \begin{precond}
     
     \cond
     Es gibt mindestens einen IML-Teilgraphen.
     
   \end{precond}


  \begin{postsuccess}
    \cond 
    Es wurde ein neuer IML-Teilgraph mit entsprechendem Namen erzeugt.
    
    \cond
    Der neue IML-Teilgraph hat alle Graph-Knoten und Graph-Kanten des
    Quell-IML-Teilgraphen.
    
  \end{postsuccess}
  
  \begin{postfail}
    \cond Das System bleibt im bisherigen Zustand.
  \end{postfail}
  
  \begin{proc} 
     
    \step[1]
    Der Benutzer f�hrt einen Rechtsklick auf den zu kopierenden 
    Quell-IML-Teilgraphen in der SUBGRAPH\_LIST im Hauptfenster aus und
    w�hlt im PopUp Men� \ref{SUBGRAPH-LIST-POPUP} den Eintrag
    Copy IML Subgraph aus.
     
    \step[2]
    GIANT zeigt den allgemeinen Texteingabediolog <<<<<<<<<VERWEIS GUI>>>>>
      	    
    \step[3]
    Der Benutzer gibt einen Namen f�r den neu zu erstellenden IML-Teilgraphen
    ein und best�tigt mit OK.\\
    Gibt der Benutzer hier keinen Namen ein, so vergibt GIANT automatisch 
    einen Namen.\\
    Gibt der Benutzer den Namen eines bereits vorhandenen IML-Teilgraphen
    ein, so erscheint eine Fehlermeldung.
       
    \step[4]
    GIANT kopiert den Quell-IML-Teilgraphen in einen neuen Teilgraphen.

  \end{proc}
  
  \begin{aproc}
    \astep{3} Der Benutzer bricht den UseCase mit Cancel ab.
  \end{aproc}

\end{uc}


% ==============================================================================
\begin{uc}[Label]{UC: IML-Teilgraph umbenennen}
>>>> WEG LASSEN, KANN ER AUCH �ber Kopieren und L�schen machen

\end{uc}

% ==============================================================================
\begin{uc}[Label]{UC: IML-Teilgraph l�schen}
Dieser UseCase l�scht den entsprechenden IML-Teilgraphen.

  \begin{precond}
     
     \cond
     Es gibt mindestens einen IML-Teilgraphen.
     
   \end{precond}


  \begin{postsuccess}
    \cond 
    Es wurde ein neuer IML-Teilgraph mit entsprechendem Namen erzeugt.
    
    \cond
    Der neue IML-Teilgraph hat alle Graph-Knoten und Graph-Kanten des
    Quell-IML-Teilgraphen.
    
  \end{postsuccess}
  
  \begin{postfail}
    \cond Das System bleibt im bisherigen Zustand.
  \end{postfail}
  
  \begin{proc} 
     
    \step[1]
    Der Benutzer f�hrt einen Rechtsklick auf den zu kopierenden 
    Quell-IML-Teilgraphen aus und w�hlt im PopUp Men� den Eintrag
    <<<<<<<<<<< VERWEIS GUI - Eintag: kopieren >>>>>> aus.
     
    \step[2]
    GIANT zeigt den allgemeinen Texteingabediolog <<<<<<<<<VERWEIS GUI>>>>>
      	    
    \step[3]
    Der Benutzer gibt einen Namen f�r den neu zu erstellenden IML-Teilgraphen
    ein und best�tigt mit OK.\\
    Gibt der Benutzer hier keinen Namen ein, so vergibt GIANT automatisch 
    einen Namen.\\
    Gibt der Benutzer den Namen eines bereits vorhandenen IML-Teilgraphen
    ein, so erscheint eine Fehlermeldung.
       
    \step[4]
    GIANT kopiert den Quell-IML-Teilgraphen in einen neuen Teilgraphen.

  \end{proc}
  
  \begin{aproc}
    \astep{3} Der Benutzer bricht den UseCase mit Cancel ab.
  \end{aproc}

\end{uc}


% ==============================================================================
\begin{uc}[Mengenoperationen auf 2 IML-Teilgraphen]
      {UC: Mengenoperationen auf 2 IML-Teilgraphen}

Erg�nzend zu den M�glichkeiten der Anfragesprache kann der Benutzer
die g�ngigen Mengenoperationen, wie Mengenvereinigung, 
Schnitt und Differenz, auch direkt �ber einen entsprechenden Dialog 
ausf�hren.
Beschreibung des Dialoges siehe \ref{Common-Set-Operation-Dialog}.
Vorgehen analog zu \ref{Mengenoperationen auf 2 Selektionen}



  \begin{precond}
    \cond Es gibt mindestens zwei IML-Teilgraphen.
  \end{precond}

  \begin{postsuccess}
    \cond 
    Eine neuer IML-Teilgraph mit entsprechendem Namen 
    (im Dialog eingegeben unter TARGET) ist angelegt 
    und erscheint in der Liste �ber alle IML-Teilgraphen des Projektes.
    <<<<<<<<<VERWEIS GUI >>>>>>>>>>>>>>>>>>>>>>>>>>>>><
       
    \cond
    Im Falle einer Mengenvereinigung umfasst, der neue IML-Teilgraph TARGET alle
    Knoten und Kanten aus dem LEFT\_SOURCE IML-Teilgraphen und dem 
    RIGHT\_SOURCE IML-Teilgraphen.
    
    \cond
    Im Falle einer Mengendifferenz umfasst, der neue IML-Teilgraph TARGET alle
    Knoten und Kanten aus dem LEFT\_SOURCE IML-Teilgraphen, die nicht 
    Bestandteil des RIGHT\_SOURCE IML-Teilgraphen sind.

    \cond
    Im Falle eines Mengenschnitts umfasst, der neue IML-Teilgraph TARGET alle
    Knoten und Kanten, die sowohl dem LEFT\_SOURCE IML-Teilgraphen als auch 
    dem RIGHT\_SOURCE IML-Teilgraphen angeh�ren.

  \end{postsuccess}

  \begin{postfail}
    \cond Das System bleibt im bisherigen Zustand.
  \end{postfail}
  
  \begin{proc}    
    \step[1]
    Der Benutzer startet den UseCase �ber das PopUp Men� 
    >>>>VERWEIS GUI - PopUp unter Liste der IML-Teilgraphen<<<<< 
    
    \step[2] 
    GIANT �ffnet den Common\_Set\_Operation\_Dialog 
    (siehe \ref{Common-Set-Operation-Dialog}).
    
    
    \step[3] 
    Der Benutzer w�hlt dort die beiden Quell-IML-Teilgraphen 
    (LEFT\_SOURCE und RIGHT\_SOURCE) aus, bestimmt die auszuf�hrende
    Mengenoperation und gibt unter TARGET den Namen 
    des neu zu erzeugenden IML-Teilgraphen ein.\\
    Er best�tigt mit OK.\\
    Existiert bereits ein IML-Teilgraph mit dem unter Target eingegebenen 
    Namen, so erscheint eine entsprechende Fehlermeldung.
       
    \step[4]
    GIANT f�hrt die Mengenoperation aus.
  
  \end{proc}

  \begin{aproc}
    \astep[3] Der Benutzer bricht die Eingabe der Daten mit Cancel ab.
  \end{aproc}


\end{uc}



% ==============================================================================
\begin{uc}[Label]{UC: Teilgraph exportieren}
>>>> W�rd' ich nicht anbieten, kann man auch manuell machen.

\end{uc}


% ==============================================================================
\begin{uc}[Label]{UC: Teilgraph importieren}
>>>> W�rd' ich nicht anbieten, kann man auch manuelle machen.


\end{uc}
