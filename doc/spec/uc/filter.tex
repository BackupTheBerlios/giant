% ==============================================================================
%  $RCSfile: filter.tex,v $, $Revision: 1.11 $
%  $Date: 2003/04/05 21:08:37 $
%  $Author: squig $
%
%  Description: UseCases f�r Filter
%
%  Last-Ispelled-Revision: 1.5
%
% ==============================================================================

\begin{uc}[Selektionen ausblenden]{UC: Selektionen ausblenden}
Mit diesem UseCase k�nnen Selektionen innerhalb
eines Anzeigefensters ausgeblendet werden.

  \begin{precond}
    \cond Es gibt ein ge�ffnetes Anzeigefenster mit einer Selektion.
  \end{precond}

  \begin{postsuccess}

    \cond Alle zu der Selektion geh�rende Fenster-Knoten und
          Fenster-Kanten sind ausgeblendet, d.h. sie sind
          im Anzeigefenster nicht mehr sichtbar. Dies trifft
          auch f�r Fenster-Knoten und Fenster-Kanten, die
          noch zu weiteren Selektionen, geh�ren zu.
    
    \cond Die ausgeblendeten Selektionen k�nnen nicht 
          zur aktuellen Selektion gemacht werden (siehe 
          \ref {Selektion zur aktuellen Selektion machen}) und damit
          nicht mehr direkt bearbeitet werden, 
          d.h. die Menge der selektierten 
          Fenster-Knoten und Fenster-Kanten kann nicht
          mehr abge�ndert werden (siehe \ref {Selektieren von 
          Fenster-Knoten und Fenster-Kanten in Anzeigefenstern}).


    \cond Die Fenster-Knoten und Fenster-Kanten sind aber immer noch 
          Bestandteil des Anzeigefensters und k�nnen �ber den folgenden
          UseCase (siehe \ref{Selektionen einblenden}) wieder zur Anzeige
          gebracht werden.
    
  \end{postsuccess}

   
  \begin{proc}

    \step[1] 
    Der Benutzer f�hrt einen Rechtsklick mit der Maus auf
    die auszublendende Selektion in der Selektionsauswahlliste 
    (siehe \ref{Selektionsauswahlliste}) durch 
    und w�hlt im Popup-Men� den Eintrag \gq{Hide Selection}
    aus. Diese Funktionalit�t kann nicht auf die aktuelle
    Selektion angewendet werden 
    (siehe \ref{Aktuelle Selektion vs Selektionen}).

    \step[2]
    GIANT blendet die Selektion aus.
 
  \end{proc}

\end{uc}



\begin{uc}[Selektionen einblenden]{UC: Selektionen einblenden}
Mit diesem UseCase k�nnen ausgeblendete Selektionen
wieder eingeblendet werden.

  \begin{precond}
    \cond Es gibt ein ge�ffnetes Anzeigefenster mit einer 
          ausgeblendeten Selektion.
  \end{precond}

  \begin{postsuccess}

    \cond 
    Die Selektion ist wieder eingeblendet, alle zu ihr geh�renden 
    Fenster-Knoten und Fenster-Kanten sind wieder
    im Anzeigeinhalt sichtbar dargestellt.
    
  \end{postsuccess}

   
  \begin{proc}

    \step[1] 
    Der Benutzer f�hrt einen Rechtsklick mit der Maus auf
    eine ausgeblendete Selektion in der Selektionsauswahlliste 
    (siehe \ref{Selektionsauswahlliste}) durch 
    und w�hlt im Popup-Men� den Eintrag \gq{Show Selection}
    aus. 

    \step[2]
    GIANT blendet die Selektion ein.
 
  \end{proc}

\end{uc}


%\begin{uc}[Detailfilter f�r ein Anzeigefenster]{UC: Detailfilter f�r ein Anzeigefenster}
%<<<<<<<Martin: BIN MIR NICHT GANZ SICHER
%ICH GLAUBE DIESE ANFORDERUNG WURDE GECANCELT>>>>>>>>>>>>>>>
%
%
%
%\end{uc}
%
%\begin{uc}[Detailfilter f�r eine Selektion]{UC: Detailfilter f�r eine Selektion}
%<<<<<<<Martin: BIN MIR NICHT GANZ SICHER - MARTIN
%ICH GLAUBE DIESE ANFORDERUNG WURDE GECANCELT>>>>>>>>>>>>>>>
%
%\end{uc}
