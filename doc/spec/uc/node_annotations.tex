% ==============================================================================
%  $RCSfile: node_annotations.tex,v $
%  $Date: 2003/02/18 18:20:48 $
%  $Author: schwiemn $
%
%  Description: Use-Cases zum Annotieren von Knoten
%
% ==============================================================================

\begin{uc}[Fenster-Knoten annotieren]{UC: Fenster-Knoten annotieren}
  Dieser UseCase dient zum Erzeugen von Knoten-Annotationen.
  
  \begin{precond}
    \cond Ein Projekt ist geladen.
    \cond Es gibt mindestens ein Anzeigefenster mit Fenster-Knoten.
  \end{precond}

  \begin{postsuccess}
    \cond Die neue Annotation ist noch nicht
          in der Verwaltungsdatei f�r Knoten-Annotationen eingetragen.
    \cond Der annotierte Knoten wird in allen Anzeigefenstern als annotiert
          hervorgehoben (hat nun das entsprechende Icon).
	  
  \end{postsuccess}

  \begin{postfail}
    \cond Das System bleibt im bisherigen Zustand.
  \end{postfail}
  
  \begin{proc}
    \step[1] 
    Der Benutzer w�hlt den Fenster-Knoten aus, der annotiert werden soll und
    �ffnet das entsprechende POPUPMEN� -> VERWEIS ZU GUI.
    \step[2]
    GIANT zeigt nun den EINGABEDIALOG F�R ANNOTATIONEN.
    
    \step[3]
    Der Benutzer gibt dort den entsprechenden Text f�r die Knoten-Annotation 
    ein.
  
    \step[4]
    Nach Abschluss der Eingabe bet�tigt der Benutzer den \gq{OK Button}.
            
    \step[5]
    GIANT �bernimmt die neu Annotation. Die eingebegbene Annotation 
    muss allerdings mindestens ein Zeichen haben, ansonsten erscheint eine
    Fehlermeldung.

  \end{proc}

  \begin{aproc}
    \ageneral 
    Der Benutzer kann die Eingabe der neuen Knoten-Annotation jeder Zeit 
    mittels des \gq{Cancel Buttons} abbrechen.
    

  \end{aproc}
\end{uc}

% ==============================================================================

\begin{uc}[Fenster-Knoten annotieren]{UC: Knoten-Annotation �ndern}
  Dient zum �ndern bestehender Knoten-Annotationen.
  
  \begin{precond}
    \cond Ein Projekt ist geladen.
    \cond Es gibt mindestens einen annotierten Fenster-Knoten.
  \end{precond}

  \begin{postsuccess}
    \cond Die �nderung an der Annotation ist noch nicht
          in der Verwaltungsdatei f�r Knoten-Annotationen eingetragen.
    \cond Die �nderung der Annotation ist dem System bekannt und wird
          entsprechend angezeigt.
	  
  \end{postsuccess}

  \begin{postfail}
    \cond Das System bleibt im bisherigen Zustand.
  \end{postfail}
  
  \begin{proc}
    \step[1] 
    Der Benutzer w�hlt den Fenster-Knoten aus, dessen Annotation ge�ndert
    werden soll und �ffnet das entsprechende POPUPMEN� -> VERWEIS ZU GUI.
    
    \step[2]
    GIANT zeigt nun den EINGABEDIALOG F�R ANNOTATIONEN. Hier wird der Text
    f�r die bisherige Annotation dargestellt.
    
    \step[3]
    Der Benutzer �ndert den Text f�r die Annotation entsprechend ab.
  
    \step[4]
    Nach Abschluss der Eingabe bet�tigt der Benutzer den \gq{OK Button}.
            
    \step[5]
    GIANT �bernimmt die vorgenommenen �nderungen an der Annotation. 
    Die neue Annotation muss allerdings mindestens ein Zeichen haben, 
    ansonsten erscheint eine Fehlermeldung.

  \end{proc}

  \begin{aproc}
    \ageneral 
    Der Benutzer kann das �ndern der Annotation jeder Zeit 
    mittels des \gq{Cancel Buttons} abbrechen.
    

  \end{aproc}
\end{uc}

% ==============================================================================

\begin{uc}[Fenster-Knoten annotieren]{UC: Knoten-Annotation l�schen}
  L�scht eine bestehende Knoten-Annotationen.
  
  \begin{precond}
    \cond Ein Projekt ist geladen.
    \cond Es gibt mindestens einen annotierten Fenster-Knoten.
  \end{precond}

  \begin{postsuccess}
    \cond Die Annotation ist noch nicht aus
          der Verwaltungsdatei f�r Knoten-Annotationen entfernt.
    \cond Das Entfernen der Annotation ist dem System bekannt und wird
          entsprechend angezeigt.
	  
  \end{postsuccess}

  \begin{postfail}
    \cond Das System bleibt im bisherigen Zustand.
  \end{postfail}
  
  \begin{proc}
    \step[1] 
    Der Benutzer w�hlt den Fenster-Knoten aus, dessen Annotation gel�scht 
    werden soll POPUPMEN� -> VERWEIS ZU GUI.
          
    \step[2] 
    GIANT l�scht die entsprechende Annotation.

  \end{proc}


\end{uc}

% ==============================================================================

\begin{uc}[Label]{UC: Knoten-Annotationen filtern}
  L�scht alle bestehenden Knoten-Annotationen des Projektes
  f�r die es in den dem Projekt bekannten Anzeigefenstern keine
  Fenster-Knoten und auch keine Graph-Knoten gibt.\\
  Dies dieser Filter soll es erm�glichen, nicht mehr ben�tigte 
  Annotationen automatisch zu entfernen.
  
  \begin{precond}
    \cond Ein Projekt ist geladen.
  \end{precond}

  \begin{postsuccess}
    \cond Die Annotationen sind noch nicht aus
          der Verwaltungsdatei f�r Knoten-Annotationen entfernt.
    \cond Das Entfernen der Annotationen ist dem System bekannt und wird
          entsprechend angezeigt.
	  
  \end{postsuccess}

  \begin{postfail}
    \cond Das System bleibt im bisherigen Zustand.
  \end{postfail}
  
  \begin{proc}
    \step[1] 
    Der Benutzer w�hlt im HAUPTEMNB� den entsprechenden Eintrag aus.
    
    \step[2]
    GIANT Zeigt eine Sicherheitsabfrage und fordert den Benutzer zur 
    Best�tigung auf.
    
    \step[3]
    Der Benutzer best�tigt Sicherheitsabfrage.
           
    \step[4]
    GIANT l�scht alle Annotationen f�r die es keine Fenster-Knoten und
    Graph-Knoten gibt, d.h. jede Knoten-Annotation deren Knoten weder in 
    einem Anzeigefenster visualisiert noch Bestandteil eines IML-Teilgraphen
    ist, wird entfernt.

  \end{proc}

  \begin{aproc}
    \astep[2] Der Vorgang kann nat�rlich Abgebrochen werden, indem die
    Sicherheitsabfrage nicht best�tigt wird.

  \end{aproc}
\end{uc}
