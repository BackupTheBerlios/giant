% ==============================================================================
%  $RCSfile: node_annotations.tex,v $
%  $Date: 2003/02/18 15:21:04 $
%  $Author: schwiemn $
%
%  Description: Use-Cases zum Annotieren von Knoten
%
% ==============================================================================

\begin{uc}[Fenster-Knoten annotieren]{UC: Fenster-Knoten annotieren}
  Dieser UseCase dient zum Erzeugen, �ndern und Anschauen 
  von Knoten-Annotationen.
  Er erstellt eine neue Annotation f�r einen Fenster-Knoten, falls dieser
  noch nicht annotiert ist.\\
  Ist der Fenster-Knoten schon annotiert, wird diese Annotation angezeigt 
  und kann vom Benutzer ge�ndert werden.
  
  
  \begin{precond}
    \cond Das Programm ist gestartet.
    \cond Ein Projekt ist geladen.
    \cond Es gibt mindestens ein Anzeigefenster mit Fenster-Knoten.
  \end{precond}

  \begin{postsuccess}
    \cond Die neue Annotation oder die �nderungen sind noch nicht
          in der Verwaltungsdatei f�r Knoten-Annotationen eingetragen.
    \cond Der annotierte Knoten wird in allen Anzeigefenstern als annotiert
          hervorgehoben (hat nun das entsprechende Icon).
	  
  \end{postsuccess}

  \begin{postfail}
    \cond Das System und die Verwaltungsdatei f�r Knoten-Annotationen
          bleiben im bisherigen Zustand.
  \end{postfail}
  
  \begin{proc}
    \step[1] 
    Der Benutzer w�hlt den Fenster-Knoten aus, der annotiert werden soll und
    �ffnet das entsprechende POPUPMEN� -> VERWEIS ZU GUI.
    \step[2]
    GIANT zeigt nun den EINGABEDIALOG F�R ANNOTATIONEN. Existiert bereits eine
    Annotation so wird der entsprechende Text angezeigt, anderenfalls wird
    nichts angezeigt.
    
    
    \step[3a]
    Falls ein Knoten neu annotiert werden soll oder falls die Annotation zu
    einem bestehenden Knoten ge�ndert werden soll, gibt der Benutzer
    Im EINGABEDIALOG F�R ANNOTATIONEN GIBT den entsprechenden Text ein.
    \step[3b]
    Will sich der Benutzer eine bestehende Knoten-Annotation nur auschauen,
    so tut er dies jetzt, �ndert aber den Text nicht. 

    \step[4]
    Nach Abschluss der Aufgabe bet�tigt der Benutzer den \gq{OK Button}
            
    \step[5]
    GIANT �bernimmt die vorgenommenen �nderungen. Hat der Benutzer allen Text
    zu einer bestehenden Annotation entfernt, so wird die Knoten-Annotation
    gel�scht. 
    


  \end{proc}

  \begin{aproc}
    \astep{2} Der Benutzer bricht die Eingabe der Annotation mit Cancel ab.

  \end{aproc}
\end{uc}

% ==============================================================================


% ==============================================================================

\begin{uc}[Label]{UC: Knoten-Annotation l�schen}
L�scht eine bestehende Knotenannotation.
\end{uc}

% ==============================================================================

\begin{uc}[Label]{UC: Knoten-Annotationen filtern}
>>>> EVENTUELL
Entfernt alle Knoten-Annotationen, deren zugehl�rige Knoten nicht mindestens
in einem Anzeigefenster visualisiert sind.
\end{uc}
