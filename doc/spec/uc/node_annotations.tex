% ==============================================================================
%  $RCSfile: node_annotations.tex,v $
%  $Date: 2003/02/18 17:05:37 $
%  $Author: schwiemn $
%
%  Description: Use-Cases zum Annotieren von Knoten
%
% ==============================================================================

\begin{uc}[Fenster-Knoten annotieren]{UC: Fenster-Knoten annotieren}
  Dieser UseCase dient zum Erzeugen von Knoten-Annotationen.
  
  \begin{precond}
    \cond Das Programm ist gestartet.
    \cond Ein Projekt ist geladen.
    \cond Es gibt mindestens ein Anzeigefenster mit Fenster-Knoten.
  \end{precond}

  \begin{postsuccess}
    \cond Die neue Annotation ist noch nicht
          in der Verwaltungsdatei f�r Knoten-Annotationen eingetragen.
    \cond Der annotierte Knoten wird in allen Anzeigefenstern als annotiert
          hervorgehoben (hat nun das entsprechende Icon).
	  
  \end{postsuccess}

  \begin{postfail}
    \cond Das System und die Verwaltungsdatei f�r Knoten-Annotationen
          bleiben im bisherigen Zustand.
  \end{postfail}
  
  \begin{proc}
    \step[1] 
    Der Benutzer w�hlt den Fenster-Knoten aus, der annotiert werden soll und
    �ffnet das entsprechende POPUPMEN� -> VERWEIS ZU GUI.
    \step[2]
    GIANT zeigt nun den EINGABEDIALOG F�R ANNOTATIONEN.
    
    \step[3]
    Der Benutzer gibt dort den entsprechenden Text f�r die Knoten-Annotation 
    ein.
  
    \step[4]
    Nach Abschluss der Eingabe bet�tigt der Benutzer den \gq{OK Button}.
            
    \step[5]
    GIANT �bernimmt die vorgenommenen �nderungen. Die eingebegbene Annotation 
    muss allerdings mindestens ein Zeichen haben, ansonsten erscheint eine
    Fehlermeldung.

  \end{proc}

  \begin{aproc}
    \ageneral 
    Der Benutzer kann die Eingabe der neuen Knoten-Annotation jeder Zeit 
    mittels des \gq{Cancel Buttons} abbrechen.
    

  \end{aproc}
\end{uc}

% ==============================================================================

\begin{uc}[Fenster-Knoten annotieren]{UC: Knoten-Annotation �ndern}
  Dieser UseCase dient zum Erzeugen von Knoten-Annotationen.
  
  \begin{precond}
    \cond Das Programm ist gestartet.
    \cond Ein Projekt ist geladen.
    \cond Es gibt mindestens ein Anzeigefenster mit Fenster-Knoten der
          annotiert ist.
  \end{precond}

  \begin{postsuccess}
    \cond Die �nderung an der Annotation ist noch nicht
          in der Verwaltungsdatei f�r Knoten-Annotationen eingetragen.
    \cond Die �nderung der Annotation ist dem System bekannt und wird
          entsprechend angezeigt.
	  
  \end{postsuccess}

  \begin{postfail}
    \cond Das System und die Verwaltungsdatei f�r Knoten-Annotationen
          bleiben im bisherigen Zustand.
  \end{postfail}
  
  \begin{proc}
    \step[1] 
    Der Benutzer w�hlt den Fenster-Knoten aus, der annotiert werden soll und
    �ffnet das entsprechende POPUPMEN� -> VERWEIS ZU GUI.
    \step[2]
    GIANT zeigt nun den EINGABEDIALOG F�R ANNOTATIONEN.
    
    \step[3]
    Der Benutzer gibt dort den entsprechenden Text f�r die Knoten-Annotation 
    ein.
  
    \step[4]
    Nach Abschluss der Eingabe bet�tigt der Benutzer den \gq{OK Button}.
            
    \step[5]
    GIANT �bernimmt die vorgenommenen �nderungen. Die eingebegbene Annotation 
    muss allerdings mindestens ein Zeichen haben, ansonsten erscheint eine
    Fehlermeldung.

  \end{proc}

  \begin{aproc}
    \ageneral 
    Der Benutzer kann die Eingabe der neuen Knoten-Annotation jeder Zeit 
    mittels des \gq{Cancel Buttons} abbrechen.
    

  \end{aproc}
\end{uc}

% ==============================================================================

\begin{uc}[Label]{UC: Knoten-Annotation l�schen}
L�scht eine bestehende Knotenannotation.
\end{uc}

% ==============================================================================

\begin{uc}[Label]{UC: Knoten-Annotationen filtern}
>>>> EVENTUELL
Entfernt alle Knoten-Annotationen, deren zugehl�rige Knoten nicht mindestens
in einem Anzeigefenster visualisiert sind.
\end{uc}
