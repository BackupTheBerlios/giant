% ==============================================================================
%  $RCSfile: load_store.tex,v $, $Revision: 1.5 $
%  $Date: 2003/02/13 21:57:01 $
%  $Author: schulzgt $
%
%  Description: Use-Cases f�r Lade- und Speicherfunktionalit�t
%
% ==============================================================================

\begin{uc}[Neues Projekt]{UC: Neues Projekt}
  Erstellt eines neuen GIANT Projekt. Ein eventuell vorhandenes Projekt wird 
  dabei �berladen, wobei �nderungen am bisherigen Projekt auf Nachfrage 
  gespeichert werden.
  
  \begin{precond}
    \cond Das Programm ist gestartet
  \end{precond}

  \begin{postsuccess}
    \cond Ein neues GIANT Projekt ist erstellt und geladen.
    \cond Eine IML-Datei ist geladen.
    \cond �nderungen an einem bisherigen Projekt sind gespeichert
    oder verworfen.
  \end{postsuccess}

  \begin{postfail}
    \cond Das System bleibt im bisherigen Zustand.
  \end{postfail}
  
  \begin{proc}
    \step[1] Der Benutzer w�hlt in einem Dialog ein Verzeichnis und eine
    IML-Datei aus und vergibt einen Projektnamen.
    \step[2] Der Benutzer best�tigt seine Eingaben mit OK.
    \step[3] Falls bereits ein Projekt ge�ffnet ist erscheint eine Abfrage
    ob dieses gespeichert werden soll. Der Benutzer best�tigt mit SAVE.
  \end{proc}

  \begin{aproc}
    \astep{2} Der Benutzer bricht die Verarbeitung mit Cancel ab.
    \astep{3} Der Benutzer verwirft eventuelle �nderungen mit DISCARD.
  \end{aproc}
\end{uc}

% ==============================================================================

\begin{uc}[Projekt �ffnen]{UC: Projekt �ffnen}
  �ffnet ein GIANT Projekt. Ein eventuell vorhandenes Projekt wird dabei 
  �berladen, wobei �nderungen am bisherigen Projekt auf Nachfrage gespeichert 
  werden.

  \begin{precond}
    \cond Das Programm ist gestartet
  \end{precond}

  \begin{postsuccess}
    \cond Ein vorhandenes GIANT Projekt ist geladen.
    \cond Eine IML-Datei ist geladen.
    \cond �nderungen an einem bisherigen Projekt sind gespeichert
    oder verworfen.
  \end{postsuccess}

  \begin{postfail}
    \cond Das System bleibt im bisherigen Zustand.
  \end{postfail}
  
  \begin{proc}
    \step[1] Der Benutzer w�hlt in einem Dialog eine GIANT Projektdatei aus.
    \step[2] Der Benutzer best�tigt seine Eingaben mit OK.
    \step[3] Falls die dem Projekt zugeordnete IML-Datei nicht gefunden wird
    wird der Benutzer in einem Dialog aufgefordert eine IML-Datei anzugeben.
    Er best�tigt mit OK.
  \end{proc}

  \begin{aproc}
    \astep{2} Der Benutzer bricht die Verarbeitung mit Cancel ab.
    \astep{3} Der Benutzer bricht die Verarbeitung mit Cancel ab.
  \end{aproc}
\end{uc}

% ==============================================================================

\begin{uc}[Label]{UC: Projekt speichern}
\end{uc}

% ==============================================================================

\begin{uc}[Label]{UC: Anzeigefenster importieren}
\end{uc}

% ==============================================================================

\begin{uc}[Label]{UC: Anzeigefenster exportieren}
\end{uc}
