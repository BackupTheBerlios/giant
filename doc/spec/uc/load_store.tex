% ==============================================================================
%  $RCSfile: load_store.tex,v $, $Revision: 1.3 $
%  $Date: 2003/02/10 00:00:33 $
%  $Author: schulzgt $
%
%  Description: Use-Cases f�r Lade- und Speicherfunktionalit�t
%
% ==============================================================================

\begin{uc}[Label]{UC: Programm starten}
�bergabe eines Skripts oder von Kommandozeilenparametern zur automatischen
Durchf�hrung einer Suche, mit Layout etc.
\end{uc}

\begin{uc}[Label]{UC: Programm beenden}
\end{uc}

\begin{uc}[Label]{UC: Neues Projekt}
  Erstellen eines neuen GIANT Projekts.

  \begin{precond}
    \cond Das Programm ist gestartet
  \end{precond}

  \begin{postcond}
    \cond Ein neues GIANT Projekt ist erstellt und ge�ffnet.
  \end{postcond}

  \begin{actors}
    \actor{Benutzer}
  \end{actors}
  
  \begin{proc}
    \step[1] Der Benutzer w�hlt in einem Dialog ein Verzeichnis und eine
    IML-Datei aus und vergibt einen Projektnamen.
    \step[2] Der Benutzer best�tigt seine Eingaben mit OK.
  \end{proc}

  \begin{eproc}
    \estep{1} Umbennen eines Verzeichnis
  \end{eproc}

  \begin{aproc}
    \astep{2} Der Benutzer bricht die Verarbeitung mit Cancel ab.
  \end{aproc}

\end{uc}

\begin{uc}[Label]{UC: Projekt laden}
\end{uc}

\begin{uc}[Label]{UC: Projekt speichern}
\end{uc}

\begin{uc}[Label]{UC: Anzeigefenster importieren}
\end{uc}

\begin{uc}[Label]{UC: Anzeigefenster exportieren}
\end{uc}

