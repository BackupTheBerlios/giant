% ==============================================================================
%  $RCSfile: gui_window.tex,v $, $Revision: 1.2 $
%  $Date: 2003/02/03 01:17:27 $
%  $Author: schulzgt $
%
%  Description: Use-Cases f�r die Fensterfunktionalit�t der GUI
%
% ==============================================================================

\begin{uc}[Label]{UC: Leeres Anzeigefenster erzeugen}
\end{uc}

\begin{uc}[Label]{UC: Einf�gen eines IML-Teilgraph}
\end{uc}

\begin{uc}[Label]{UC: Einf�gen einer Selektion}
Layout wird behalten
\end{uc}

\begin{uc}[Label]{UC: L�schen einer Selektion}
Objekte und Selektion werden gel�scht, andere Selektionen werden angepa�t
\end{uc}

\begin{uc}[Label]{UC: Verschiebefunktionalit�t}
Knotenknickpunkte, Knoten, Fensterselektionen (jeweils die aktuelle FS),
Drag and Drop, Cut and Paste
\end{uc}

\begin{uc}[Label]{UC: Platz schaffen}
\end{uc}

\begin{uc}[Label]{UC: Knoten beschriften}
verhandeln
\end{uc}

\begin{uc}[Label]{UC: Anzeigefensters scrollen}
unbegrenzt, Scrollbar, Tastatur, Maus (Adobe-Hand)
\end{uc}

\begin{uc}[Label]{UC: Anzeigefenster zoomen}
begranzter Bereich, anpassen der Detaillierung
\end{uc}

\begin{uc}[Label]{UC: Zoomen einer Selektion}
\end{uc}

\begin{uc}[Label]{UC: Zoomen auf gesamten Fensterinhalt}
\end{uc}

\begin{uc}[Label]{UC: Zoomen auf eine Kante}
\end{uc}

\begin{uc}[Label]{UC: Pin anlegen}
wird benannt
\end{uc}

\begin{uc}[Label]{UC: Pin anspringen}
\end{uc}

\begin{uc}[Label]{UC: Pin l�schen}
\end{uc}
