% ==============================================================================
%  $RCSfile: gui_window.tex,v $, $Revision: 1.3 $
%  $Date: 2003/02/05 14:01:12 $
%  $Author: schulzgt $
%
%  Description: Use-Cases f�r die Fensterfunktionalit�t der GUI
%
% ==============================================================================

\begin{uc}[Label]{UC: Leeres Anzeigefenster erzeugen}
\end{uc}

\begin{uc}[Label]{UC: IML-Teilgraph in Anzeigefenster einf�gen}
\end{uc}

\begin{uc}[Label]{UC: Selektion in Anzeigefenster einf�gen}
Layout wird beibehalten
\end{uc}

\begin{uc}[Label]{UC: Selektion aus Anzeigefenster l�schen}
Objekte und Selektion werden gel�scht, andere Selektionen werden angepa�t
\end{uc}

\begin{uc}[Label]{UC: Anzeigefensters scrollen}
unbegrenzt, Scrollbar, Tastatur, Maus (Adobe-Hand)
\end{uc}

\begin{uc}[Label]{UC: Anzeigefenster zoomen}
begranzter Bereich, anpassen der Detaillierung
\end{uc}

\begin{uc}[Label]{UC: Zoomen einer Selektion}
\end{uc}

\begin{uc}[Label]{UC: Zoomen auf gesamten Inhalt eines Anzeigefensters}
\end{uc}

\begin{uc}[Label]{UC: Zoomen auf eine Kante}
\end{uc}

\begin{uc}[Label]{UC: Verschieben von Knoten, Selektionen, Kantenknickpunkten}
Kantenknickpunkte, Knoten, Selektionen (jeweils die aktuelle Selektion),
Drag and Drop, Cut and Paste
\end{uc}

\begin{uc}[Label]{UC: Platz schaffen}
\end{uc}

\begin{uc}[Label]{UC: Knoten beschriften}
verhandeln
\end{uc}

\begin{uc}[Label]{UC: Pin anlegen}
wird benannt
\end{uc}

\begin{uc}[Label]{UC: Pin anspringen}
\end{uc}

\begin{uc}[Label]{UC: Pin l�schen}
\end{uc}
