% =============================================================================
%  $RCSfile: fa.tex,v $, $Revision: 1.6 $
%  $Date: 2003/02/19 09:41:20 $
%  $Author: schwiemn $
%
%  Description:
%
% =============================================================================

\section{Der Aktor \gq{Benutzer}}
Aktor im Sinne der anschlie�end spezifizierten UseCases ist immmer die 
menschliche Person, die den IML-Browser GIANT gerade benutzt, also der 
\gq{Benutzer}. Da GIANT keinen Mehrbenutzerbetrieb vorsieht, ist dies immer 
eine einzige Person. 
Neben dem \gq{Benutzer} sind keine weiteren Aktoren vorgesehen.\\


  \subsection{Anforderungen an den Aktor \gq{Benutzer}}
  GIANT ist eine Profi-Werkzeug. Weder das System noch das Handbuch richten 
  sich an unerfahrene Benutzer. Zur Bedienung muss der Benutzer daher 
  zwingend �ber die folgenden Kenntnisse verf�gen:

  \begin{itemize}
    \item Erfahrung im Umgang mit grafischen GUIs und 
          dem entsprechenden Betriebssystem.
  
    \item Grundkenntnisse in XML 
          (nur f�r das Editieren der Konfigurationsdateien).

    \item Erfahrung im Bereich Reengineering.

    \item Kenntnisse �ber Struktur und Aufbau des IML-Graphen. 

  \end{itemize}


% =============================================================================
\section{Starten von GIANT ohne Kommandozeilenparameter}
\begin {enumerate}

  \item
  Der Benutzer startet das Programm GIANT durch ausf�hren der entsprechenden
  Programmdatei.
  
  \item
  GIANT startet und ziegt das Main Window an. Es ist noch kein Projekt geladen.

\end {enumerate}


\section{Starten von GIANT mit Kommandozeilenparametern}
\begin {enumerate}

  \item
  Der Benutzer startet das Programm GIANT durch aufruf der entsprechenden
  Programmdatei aus der Kommandozeile. Als Kommandozeilenparameter �bergibt
  er hierbei entweder:
    
    \subitem
    Einen Pfad zu einer existierenden Anfrage-Datei.
    
    \subitem
    Oder 
    
    
  \item
  GIANT arbeitet die entsprechende Anfrage ab und zeigt alle gew�nschten
  Informationen an.


\end {enumerate}

Erl�uterung der Kommandozeilenparameter.


% ==============================================================================
\section{Beenden von GIANT}
Giant kann durch Schlie�en des HAUPTFENSTERS oder durch Auswahl 
des entsprechenden Eintages im Hauptmen� oder durch ALT+F4 beendet werden.\\
Ablauf:

\begin {enumerate}
  \item Der Benutzer gibt das Kommando zum Beenden von GIANT.
  \item GIANT fragt nach, ob das Projekt gespeichert werden soll.
  (PROJEKT SPEICHERN) (ENTSPRCHENDER DIALOG MUSS IN DIE GUI)
\end {enumerate}
  



% =============================================================================
% Use Cases
  
% ==============================================================================
%  $RCSfile: load_store.tex,v $, $Revision: 1.24 $
%  $Date: 2003/04/19 17:40:56 $
%  $Author: birdy $
%
%  Description: UseCases f�r Lade- und Speicherfunktionalit�t
%
%  Last-Ispelled-Revision: 1.12
%
% ==============================================================================

\begin{uc}[Neues Projekt]{UC: Neues Projekt}
\index{Projekte!neues Projekt erstellen}
  Erstellt ein neues GIANT Projekt. Ein eventuell bereits ge�ffnetes
  Projekt wird dabei geschlossen, wobei �nderungen auf Nachfrage 
  vorher gespeichert werden.
  Weitere wichtige Informationen, die in engem Zusammenhang mit 
  diesem UseCase stehen, sind unter dem Abschnitt 
  \ref{Project Persistenz der Projekte} zu finden.\\
  Zul�ssige Namen f�r Projekte sind unter Abschnitt 
  \ref{afa Zulaessige Namen} spezifiziert.
  
  
  \begin{precond}
    \cond Das Programm ist gestartet.
  \end{precond}

  \begin{postsuccess}
    \cond Ein neues GIANT Projekt mit dem eingegebenen Namen 
          ist erstellt und geladen.   

    \cond Eine IML-Datei ist geladen.

    \cond Das angegebene Projektverzeichnis ist gegebenenfalls 
          (falls es noch nicht vorhanden war) erstellt 
          worden (siehe \ref {Project Das Projektverzeichnis}).

    \cond Die neu erstellte Projektdatei 
          (siehe \ref{Project Die Projektdatei})
          liegt im Projektverzeichnis.
 
    \cond Ein eventuell zuvor ge�ffnetes Projekt ist geschlossen.
          �nderungen an dem eventuell zuvor ge�ffneten Projekt sind je
          nach Entscheidung des Benutzers bei der entsprechenden 
          Sicherheitsabfrage gespeichert oder verworfen.

  \end{postsuccess}

  \begin{postfail}
    \cond Das System bleibt im bisherigen Zustand, falls der
          Benutzer den UseCase mit Cancel abgebrochen hat.

    \cond Tritt w�hrend des Erstellens des neuen Projektes,
          nachdem das eventuell zuvor ge�ffnete Projekt
          geschlossen wurde, ein Fehler auf, so bleibt das Verhalten
          des Systems unspezifiziert.

  \end{postfail}
  
  \begin{proc}
    \step[1]
    Der Benutzer startet den UseCase �ber das Men� Project
    (siehe \ref{Main-Window-Project}) durch Auswahl von \gq{New Project}.

    \step[2]
    GIANT zeigt den Standard-Filechooser-Dialog und fordert den
    Benutzer auf eine vorhandene IML-Graph Datei auszuw�hlen.
  
    \step[3] Der Benutzer w�hlt im Standard-Filechooser-Dialog
    eine IML-Datei aus und best�tigt
    seine Eingabe mit OK (siehe \ref{Standard-Filechooser-Dialog}).
    
    \step[4]
    GIANT zeigt erneut den Standard-Filechooser-Dialog und fordert den
    Benutzer zur Eingabe des Namens der Projektdatei auf.

    \step[5] Der Benutzer gibt im Standard-Filechooser-Dialog
    den Pfad und den Namen der Projektdatei (\gq{Please select Project
    Path and Project File Name}) ein, die Dateiendung wird sp�ter
    von GIANT automatisch gesetzt. Der Name der Projektdatei ist
    automatisch auch der Name f�r das Projekt. Das Verzeichnis der
    Projektdatei wird automatisch zum Projektverzeichnis.

    \step[6]
    Der Benutzer best�tigt er seine Eingabe mit OK.\\
    Existiert die eingegebene Projektdatei bereits, 
    erscheint eine Fehlermeldung gem�� den unter Abschnitt 
    \ref{afa Fehlerverhalten} beschriebenen Konventionen.\\
    Existiert in dem Projektverzeichnis bereits eine andere Projektdatei, so 
    erscheint ebenfalls eine Fehlermeldung.\\
    
    \step[7] Falls noch ein Projekt ge�ffnet ist, erscheint eine 
    Sicherheitsabfrage (siehe \ref{Sicherheitsabfrage})
    ob dieses gespeichert werden soll.    
    Entscheidet der Benutzer sich f�r Speichern, so wird
    die unter \ref{Alles Speichern} beschriebene Funktionalit�t 
    ausgef�hrt. Lehnt der Benutzer dies ab, gehen
    alle nicht gespeicherten Informationen verloren.\\
     
    \step[8]
    GIANT schlie�t das aktuell ge�ffnete Projekt (falls eines ge�ffnet war), 
    legt, falls noch nicht vorhanden, ein neues Projektverzeichnis an,
    erzeugt ein neues Projekt und �ffnet dieses.
       
  \end{proc}

  \begin{aproc}
    \astep{3} Der Benutzer bricht die Verarbeitung mit Cancel ab.  
    \astep{5} Der Benutzer bricht die Verarbeitung mit Cancel ab.
  \end{aproc}
\end{uc}

% ==============================================================================

\begin{uc}[Projekt �ffnen]{UC: Projekt �ffnen}
\index{Projekte!�ffnen}
  �ffnet ein GIANT Projekt. Ein eventuell bereits ge�ffnetes
  Projekt wird dabei geschlossen, wobei �nderungen auf Nachfrage 
  vorher gespeichert werden. \\ 
  Sollte der Benutzer die XML-Dateien 
  innerhalb des Projektverzeichnisses (siehe 
  \ref{Project Das Projektverzeichnis}) manuell modifiziert haben, so
  dass diese von den durch GIANT automatisch erstellten Dateien
  abweichen, wird keinerlei Garantie f�r das korrekte �ffnen
  des Projektes �bernommen. Das Verhalten
  bez�glich eventuell auftretender Fehler ist undefiniert.

  \begin{precond}
    \cond Das Programm ist gestartet.
  \end{precond}

  \begin{postsuccess}
    \cond Das gew�nschte GIANT Projekt ist geladen.
    \cond Die zugeh�rige IML-Datei ist geladen.
    \cond �nderungen an einem eventuell zuvor ge�ffneten Projekt sind 
          je nach Wahl des Benutzers bei der Sicherheitsabfrage gespeichert
          oder verworfen.

  \end{postsuccess}

  \begin{postfail}
    \cond Hat der Benutzer den UseCase mit Cancel abgebrochen, 
          bleibt das System im bisherigen Zustand.

    \cond Muss der UseCase w�hrend des Ladens des neuen Projektes
          aufgrund eines Fehlers abgebrochen werden, so 
          kehrt das System falls m�glich in einen Zustand zur�ck,
          der dem Start des Systems ohne �ffnen eines Projektes
          entspricht.

    
  \end{postfail}
  
  \begin{proc}    
    \step[1]
    Der Benutzer startet den UseCase �ber das Men� Project
    (siehe \ref{Main-Window-Project} Men�) durch Auswahl von \gq{Load Project}.
    GIANT zeigt daraufhin den Standard-Filechooser-Dialog und fordert
    den Benutzer zur Auswahl eines zu �ffnenden Projektes auf.
    
    \step[2] 
    Der Benutzer w�hlt aus dem Standard-Filechooser-Dialog (siehe 
    \ref {Standard-Filechooser-Dialog})
    eine vorhandene GIANT Projektdatei (siehe \ref{Project Die Projektdatei}) 
    aus und best�tigt mit OK.
    
    \step[3] Falls bereits ein Projekt ge�ffnet ist, erscheint eine 
    Sicherheitsabfrage (siehe \ref{Sicherheitsabfrage})
    ob dieses gespeichert werden soll.     
    Entscheidet der Benutzer sich f�r Speichern, so wird
    die unter \ref{Alles Speichern} beschriebene Funktionalit�t 
    ausgef�hrt. Lehnt der Benutzer dies ab, gehen
    alle nicht gespeicherten Informationen verloren.\\

    
    \step[4]
    GIANT schlie�t das alte Projekt (falls eines ge�ffnet war)
    und l�dt das angegebene neue Projekt.

  \end{proc}

  \begin{aproc}
    \astep{2} Der Benutzer bricht die Verarbeitung mit Cancel ab.
  \end{aproc}
\end{uc}

% ==============================================================================

\begin{uc}[Projekt speichern]{UC: Projekt speichern}  
\index{Projekte!speichern}
  Speichert alle �nderungen an einem Projekt. Der Zustand der
  entsprechenden Verwaltungsdateien im Projektverzeichnis entspricht
  nach erfolgreicher Ausf�hrung dieses UseCases exakt dem aktuellen
  Zustand des ge�ffneten Projektes. Alle Konventionen zur
  Persistenz von Projekten sind im Abschnitt 
  \ref{Project Persistenz der Projekte} exakt spezifiziert.

  \begin{precond}
    \cond Das Programm ist gestartet.

    \cond Ein Projekt ist ge�ffnet. 

  \end{precond}

  \begin{postsuccess}
    \cond Die Informationen des Projekts 
    (einschlie�lich aller m�glichen �nderungen) sind persistent in die
    Verwaltungsdateien geschrieben.
 
  \end{postsuccess}

  \begin{postfail}
    \cond Das System bleibt im bisherigen Zustand.
  \end{postfail}
   
  \begin{proc}    
    \step[1]
    Der Benutzer startet den UseCase �ber das Men� Project
    (siehe \ref{Main-Window-Project}) durch Auswahl von \gq{Save Project}.
      
    \step[2]
    GIANT f�hrt die unter \ref{Alles Speichern} beschriebene Funktionalit�t 
    aus und speichert alle Informationen zu dem Projekt in der zugeh�rigen
    Projektdatei.
  \end{proc}

\end{uc}



% ==============================================================================

\begin{uc}[Projekt speichern unter]{UC: Projekt speichern unter}
\index{Projekte!unter neuem Namen speichern}
Speichert alle Informationen zu einem Projekt in eine neue Projektdatei
(entsprechende Verwaltungsdateien werden ebenfalls dupliziert).


  \begin{precond}
    \cond Das Programm ist gestartet.
    \cond Ein Projekt ist ge�ffnet (entweder ein neu erzeugtes oder ein
          geladenes).
  \end{precond}

  \begin{postsuccess}
    \cond Eine neue Projektdatei ist erzeugt worden.

    \cond Das angegebene Projektverzeichnis ist, falls es noch
          nicht vorhanden war, erstellt worden.

    \cond Die Daten des Projekts sind persistent in die 
          neuen Verwaltungsdateien im Projektverzeichnis 
          der neuen Projektdatei geschrieben (siehe auch
          \ref{Project Persistenz der Projekte}).
        
    \cond Das aktuell ge�ffnete Projekt bleibt in GIANT ge�ffnet, 
          zuk�nftiges Speichern
          (siehe \ref{Projekt speichern}) betrifft nur die 
          Verwaltungsdateien im neu erzeugten Projekt.

    \cond Die alte Projektdatei und alle zugeh�rigen Verwaltungsdateien 
          bleiben unver�ndert.
    
 
  \end{postsuccess}

  \begin{postfail}
    \cond Das System bleibt im bisherigen Zustand.
  \end{postfail}
  
  \begin{proc}    
    \step[1]
    Der Benutzer startet den UseCase �ber das Men� Project
    (siehe \ref{Main-Window-Project}) durch Auswahl von \gq{Save Project As...}.
    Daraufhin zeigt GIANT den Standard-Filechooser-Dialog und fordert
    den Benutzer auf den Namen und den Ort (Pfad) der neuen Projektdatei
    einzugeben.

     
    \step[2] Der Benutzer gibt im Standard-Filechooser-Dialog 
             (siehe \ref {Standard-Filechooser-Dialog}) 
             das neue Projektverzeichnis 
             (siehe \ref{Project Das Projektverzeichnis})
             und den Namen f�r die neue Projektdatei
             (siehe \ref{Project Die Projektdatei}) ein, die Dateiendung 
             wird sp�ter von GIANT automatisch gesetzt. \\
    Der Name der Projektdatei ist automatisch auch der Name f�r das Projekt.
    Zul�ssige Namen sind unter Abschnitt \ref{afa Zulaessige Namen}
    spezifiziert. \\ Das Verzeichnis, welches f�r die Projektdatei 
    angebenen wurde, wird automatisch zum Projektverzeichnis. 
  
    \step[3]  
    Der Benutzer best�tigt seine Eingabe mit OK.\\
    Existiert die eingegebene Datei schon, erscheint eine Fehlermeldung.\\
    Existiert in dem angegebenen Projektverzeichnis 
    bereits eine andere Projektdatei, so erscheint eine entsprechende 
    Fehlermeldung.\\

    \step[4]
    GIANT speichert das alte Projekt in der neuen Projektdatei,
    falls noch nicht vorhanden wird auch das neue Projektverzeichnis
    angelegt. Alle Verwaltungsdateien werden ebenfalls im neuen
    Projektverzeichnis gespeichert.
  
  \end{proc}


 \begin{aproc}
    \astep{2} Der Benutzer bricht die Verarbeitung mit Cancel ab.  
 \end{aproc}
\end{uc}







% ==============================================================================
%  $RCSfile: gui_window.tex,v $, $Revision: 1.34 $
%  $Date: 2003/04/05 21:08:37 $
%  $Author: squig $
%
%  Description: UseCases f�r die Fensterfunktionalit�t der GUI
%
%  Last-Ispelled-Revision: 1.25
%
% ==============================================================================

\begin{uc}[Leeres Anzeigefenster erzeugen]{UC: Leeres Anzeigefenster erzeugen}
\index{Anzeigefenster!erzeugen}

�ber diesen UseCase kann der Benutzer neue Anzeigefenster innerhalb eines
Projektes anlegen.

  \begin{precond}
    \cond Ein Projekt ist geladen.
  \end{precond}

  \begin{postsuccess}
    \cond Das neue, leere Anzeigefenster ist ge�ffnet.
    \cond Das neue Anzeigefenster ist Bestandteil des Projektes.
  \end{postsuccess}

  \begin{postfail}
    \cond Das System bleibt im bisherigen Zustand.
  \end{postfail}
  
  \begin{proc}    
    \step[1] 
    Der Benutzer startet den UseCase �ber das Popup Men�
    \ref{WINDOW-LIST-POPUP} durch Auswahl von \gq{New Window}.        
    
    \step[2]
    GIANT erzeugt ein neues Anzeigefenster (siehe  \ref{GUI Anzeigefenster})
    mit einem Namen nach dem Schema ,,Unknown\_??'' und �ffnet dies.
  \end{proc}

  \begin{aproc}
    \ageneral Es sind keine alternativen Abl�ufe vorgesehen.
  \end{aproc}
\end{uc}


% ==============================================================================
\begin{uc}[Anzeigefenster �ffnen]{UC: Anzeigefenster �ffnen}
\index{Anzeigefenster!�ffnen}

Dient zum �ffnen eines Anzeigefensters des Projektes.

  \begin{precond}
    \cond Ein Projekt mit mindestens einem Anzeigefenster ist geladen.
    \cond Es gibt mindestens ein nicht ge�ffnetes Anzeigefenster.
  \end{precond}

  \begin{postsuccess}
    \cond Das Anzeigefenster ist ge�ffnet.
  \end{postsuccess}

  \begin{proc}    
    \step[1]
    Der Benutzer f�hrt einen Doppelklick auf ein nicht ge�ffnetes
    Anzeigefenster in der Liste �ber die Anzeigefenster
    (siehe \ref{WINDOW-LIST}) durch, oder w�hlt im zugeh�hrigen Popup Men�
    \ref{WINDOW-LIST-POPUP} den Eintrag ,,Open'' aus.
   
    \step[2]
    GIANT �ffnet das entsprechende Anzeigefenster.
  \end{proc}  

\end{uc}

% ==============================================================================
\begin{uc}[Anzeigefenster umbenennen]{UC: Anzeigefenster umbenennen}
\index{Anzeigefenster!umbenennen}

Dient zum Umbenennen eines Anzeigefensters des Projektes.

  \begin{precond}
    \cond Ein Projekt mit mindestens einem Anzeigefenster ist geladen.
  \end{precond}

  \begin{postsuccess}
    \cond Das Anzeigefenster hat einen neuen Namen.
  \end{postsuccess}

  \begin{postfail}
    \cond Es sind keine Fehlerf�lle vorgesehen.
  \end{postfail}

  \begin{proc}    
    \step[1]
    Der Benutzer f�hrt einen Rechtsklick auf ein Anzeigefenster in der Liste
    �ber die Anzeigefenster (siehe \ref{WINDOW-LIST}) durch und w�hlt im
    Popup Men� \ref{WINDOW-LIST-POPUP} den Eintrag ,,Rename'' aus.

    \step[2] 
    GIANT �ffnet den allgemeinen Texteingabedialog 
    \gq{Enter Name For Window} (siehe \ref{DIALOG-WINDOW}).

    \step[3]
    Der Benutzer gibt dort einen zul�ssigen Namen f�r das Anzeigefenster ein
    und best�tigt seine Eingabe mit OK.
    
    \step[4]
    GIANT benennt das Anzeigefenster (siehe \ref{GUI Anzeigefenster}) um.
  \end{proc}  

  \begin{aproc}
    \ageneral Es sind keine alternativen Abl�ufe vorgesehen.
  \end{aproc}
\end{uc}


% ==============================================================================
\begin{uc}[Anzeigefenster speichern]{UC: Anzeigefenster speichern}
\index{Anzeigefenster!speichern}

Mit diesem UseCase wird ein Anzeigefenster gespeichert.
N�heres zur Persistenz von Anzeigefenstern ist unter Abschnitt
\ref{Project Persistenz der Projekte} spezifiziert.

  \begin{precond}
    \cond Ein Projekt mit mindestens einem Anzeigefenster ist geladen.
  \end{precond}

  \begin{postsuccess}  
    \cond Nach dem letzten Speichern am Anzeigefenster vorgenommene
    Modifikationen (neue Knoten eingef�gt etc.) sind in der Verwaltungsdatei
    (siehe \ref {Project Verwaltungsdateien f�r Anzeigefenster}) gespeichert.
  \end{postsuccess}  

  \begin{postfail}
    \cond Es sind keine Fehlerf�lle vorgesehen.
  \end{postfail}

  \begin{proc}
    \step[1]
    Der Benutzer f�hrt einen Rechtsklick auf ein Anzeigefenster in der Liste
    �ber die Anzeigefenster (siehe \ref{WINDOW-LIST}) durch und w�hlt im
    Popup Men� \ref{WINDOW-LIST-POPUP} den Eintrag ,,Save'' aus.
    
    \step[2]
    GIANT schreibt alle �nderungen in die Verwaltungsdatei des Anzeigefensters.
  \end{proc}  

  \begin{aproc}
    \ageneral
    Es sind keine alternativen Abl�ufe vorgesehen.
  \end{aproc}
\end{uc}


% ==============================================================================
\begin{uc}[Anzeigefenster schliessen]{UC: Anzeigefenster schlie\ss en}
\index{Anzeigefenster!schlie\ss en}

Mit diesem UseCase wird ein ge�ffnetes Anzeigefenster geschlossen.
N�heres zur Persistenz von Anzeigefenstern ist unter Abschnitt
\ref{Persistenz!Anzeigefenster} spezifiziert.

  \begin{precond}
    \cond Ein Projekt mit mindestens einem Anzeigefenster ist geladen.
    \cond Es gibt mindestens ein ge�ffnetes Anzeigefenster.
  \end{precond}

  \begin{postsuccess}  
    \cond Das Anzeigefenster ist geschlossen.    
    \cond Nach dem letzten Speichern am Anzeigefenster vorgenommene
    Modifikationen (neue Knoten eingef�gt etc.) sind in der Verwaltungsdatei
    (siehe \ref {Project Verwaltungsdateien f�r Anzeigefenster})
    gespeichert oder nicht.
  \end{postsuccess}  

  \begin{postfail}
    \cond Es sind keine Fehlerf�lle vorgesehen.
  \end{postfail}

  \begin{proc}    
    \step[1]
    Der Benutzer schlie�t das Anzeigefenster (durch klicken auf das ''X''
    Symbol rechts oben in der Titelleiste des Anzeigefensters).\\
    Alternativ kann er das Anzeigefenster �ber das entsprechende
    Popup-Men� \ref{WINDOW-LIST-POPUP} in der Liste �ber die
    Anzeigefenster des Projektes (siehe \ref{WINDOW-LIST}) schlie�en.
         
    \step[2]
    GIANT zeigt die allgemeine Sicherheitsabfrage 
    (siehe \ref{Sicherheitsabfrage}) und fragt nach, ob eventuelle
    �nderungen im Anzeigefenster gespeichert werden sollen oder nicht
    \gq{The content has changed. Do you want to save the changes?}.
    
    \step[3]
    Best�tigt der Benutzer mit YES, werden die �nderungen in die 
    Verwaltungsdatei geschrieben. Anderenfalls gehen s�mtliche 
    nicht gespeicherten �nderungen am Anzeigefenster verloren.
    
    \step[4]
    GIANT schlie�t das Anzeigefenster.
  \end{proc}  

  \begin{aproc}
    \ageneral
    Es sind keine alternativen Abl�ufe vorgesehen.
  \end{aproc}
\end{uc}


% ==============================================================================
\begin{uc}[Anzeigefenster l�schen]{UC: Anzeigefenster l�schen}
\index{Anzeigefenster!l�schen}

Mit diesem UseCase werden bestehende Anzeigefenster aus dem Projekt entfernt
und gel�scht. Alle Informationen zu dem Anzeigefenster gehen hierbei
unwiederbringlich verloren.

  \begin{precond}
    \cond Ein Projekt mit mindestens einem Anzeigefenster ist geladen.
  \end{precond}

  \begin{postsuccess}
    \cond 
    Das gel�schte Anzeigefenster ist nicht mehr Bestandteil des Projektes.
    
    \cond
    Die Verwaltungsdatei f�r das Anzeigefenster (siehe   
    \ref {Project Verwaltungsdateien f�r Anzeigefenster}) 
    wurde ebenfalls gel�scht.
 
  \end{postsuccess}

  \begin{postfail}
    \cond Das System bleibt im bisherigen Zustand.
  \end{postfail}
  
  \begin{proc}    
    \step[1]
    Der Benutzer startet den UseCase �ber das
    Popup-Men� \ref{WINDOW-LIST-POPUP}.
    
    \step[2] 
    GIANT zeigt die allgemeine Sicherheitsabfrage (siehe 
    \ref{Sicherheitsabfrage}) und fragt nach, ob das Anzeigefenster wirklich
    gel�scht werden soll. \gq{Do you really want to delete the selected window?}
      
    \step[3] 
    Der Benutzer best�tigt mit YES.
    
    \step[4]
    GIANT entfernt das Anzeigefenster aus dem Projekt und l�scht die 
    zugeh�rige Verwaltungsdatei (siehe auch   
    \ref {Project Verwaltungsdateien f�r Anzeigefenster}).
  \end{proc}

  \begin{aproc}
    \astep{3} Der Benutzer bricht die Verarbeitung mit NO ab.
  \end{aproc}
\end{uc}


% ==============================================================================
\begin{uc}[IML-Teilgraph in Anzeigefenster einf�gen]
          {UC: IML-Teilgraph in Anzeigefenster einf�gen}
\index{IML-Teilgraphen!in Anzeigefenster einf�gen}

Mit diesem UseCase k�nnen die Graph-Kanten und Graph-Knoten 
von IML-Teilgraphen in Anzeigefenster eingef�gt werden.
Siehe auch   
\ref{Verhalten beim Einf�gen von IML-Teilgraphen und Selektionen 
in Anzeigefenster} und
insbesondere \ref{Einf�gen von IML-Teilgraphen in Anzeigefenster}.
          

  \begin{precond}
    \cond Ein Projekt mit mindestens einem ge�ffneten 
          Anzeigefenster ist geladen.
    
    \cond Es gibt mindestens einen IML-Teilgraphen.
    
  \end{precond}

  \begin{postsuccess}
    
    \cond 
    Alle Knoten und Kanten des IML-Teilgraphen sind in das Anzeigefenster
    entsprechend dem gew�hlten Layout an der vorgegebenen Position
    eingef�gt.
    
    \cond
    In dem Anzeigefenster gibt es eine neue aktuelle Selektion, 
    die die neu eingef�gten Knoten und Kanten umfasst.
   
  \end{postsuccess}

  \begin{postfail}
    \cond Hat der Benutzer den UseCase an irgendeinem Punkt abgebrochen,
    kehrt das System zu dem Zustand zur�ck, in dem es vor dem Start des
    UseCase war.
  \end{postfail}
  
  \begin{proc}    
    \step[1]
    Der Benutzer startet den UseCase �ber das Popup-Men� 
    \ref{SUBGRAPH-LIST-POPUP}
    im Hauptfenster \gq{Insert IML Subgraph}.
    Hierdurch wird der einzuf�gende IML-Teilgraph bestimmt (immer
    der IML-Teilgraph, auf dem der Rechtsklick ausgef�hrt wurde).
    
    \step[2] 
    GIANT zeigt in der Statuszeile im Hauptfenster 
    \gq{Select Position In Display Window
    For Insertion Of IML Subgraph} an.
    Der Benutzer w�hlt das entsprechende Anzeigefenster aus und
    gibt �ber das Fadenkreuz (siehe \ref{Fadenkreuzcursor}) die Position vor, 
    an der die neuen
    Fenster-Knoten und Fenster-Kanten eingef�gt werden sollen.
    Die Statuszeile im Hauptfenster schaltet auf Normalmodus zur�ck.
    
    \step[3]
    GIANT zeigt den Dialog zur Auswahl von Layoutalgorithmen
    (siehe \ref{Layoutalgorithmen-Dialog}).
    
    \step[4] Der Benutzer w�hlt einen der vorgegebenen
    Layoutalgorithmen aus. Bei semantischen Layouts gibt er �ber den
    Layoutalgorithmen Dialog (siehe \ref{Layoutalgorithmen-Dialog})
    auch die Kantenklassen vor, die f�r das Layout ber�cksichtigt
    werden sollen (siehe Kapitel \ref{Layoutalgorithmen} f�r Details
    zu Layoutalgorithmen).
           
    \step[5]
    Der Benutzer best�tigt mit OK.
          
    \step[6] GIANT berechnet das entsprechende Layout und zeigt einen
    Dialog an, der den Benutzer �ber den Fortschritt der Berechnung
    informiert (siehe \ref{Progressbar-Modale}).\\
    W�hrend der Berechnung des Layouts kann das System GIANT nicht
    bedient werden. Zug�nglich ist nur der Button zum Abbruch des
    Algorithmus (siehe \ref{Progressbar-Modale-Cancel}).
    
    \step[7]
    Nach Abschluss der Berechnung f�gt GIANT die Fenster-Knoten und 
    Fenster-Kanten in das entsprechende Anzeigefenster ein.
   
   
  
  \end{proc}

  \begin{aproc}
    \astep{4} Der Benutzer bricht den UseCase mit Cancel ab.
    \astep{6} Der Benutzer bricht die Berechnung des Layouts ab.  
  \end{aproc}



\end{uc}

% ==============================================================================
\begin{uc}[Selektion in Anzeigefenster einf�gen]
          {UC: Selektion in Anzeigefenster einf�gen}
\index{Selektionen!in Anzeigefenster einf�gen}
          
Mit diesem UseCase kann eine Selektion aus einem Quell-Anzeigefenster in ein
Ziel-Anzeigefenster unter Beibehaltung des Layouts kopiert werden
(siehe \ref{Verhalten beim Einf�gen von IML-Teilgraphen und
  Selektionen in Anzeigefenster} und \ref{Einf�gen von Selektionen in
  Anzeigefenster}).

  \begin{precond}
    \cond Ein Projekt mit mindestens zwei ge�ffneten 
          Anzeigefenstern ist geladen.
    
    \cond Es gibt mindestens eine Selektion.
    
  \end{precond}

  \begin{postsuccess}
    
    \cond
    Das Position von Fenster-Knoten, die vor dem Einf�gen bereits im
    Ziel-Anzeigefenster vorhanden waren, bleibt 
    je nach Wahl des Benutzers unver�ndert oder wird ebenfalls ge�ndert.
    
    \cond
    Die kopierte Selektion existiert auch im Ziel-Anzeigefenster
    als Ziel-Selektion und tr�gt dort ebenfalls den Namen der Selektion. 

    \cond
    Im Ziel-Anzeigefenster gibt es keine Knoten mit der selben ID
    mehrfach. 

  \end{postsuccess}

 
  \begin{proc}    
    \step[1]
    Der Benutzer startet den UseCase �ber das Popup-Men�
    \ref{Popup-Men� Subgraph List} in der Subgraph Liste im Hauptfenster.
    Durch Auswahl eines der beiden Men�eintr�ge 
    \gq{Copy Selection keep existing layout} oder 
    \gq{Copy Selection change existing layout}.
    Hierdurch wird automatisch die Selektion bestimmt (immer
    die Selektion, auf der der Rechtsklick ausgef�hrt wurde).
    
    \step[2] 
    Der Benutzer w�hlt das entsprechende Ziel-Anzeigefenster aus.
    Das Ziel-Anzeigefenster darf nicht das Quell-Anzeigefenster sein,
    sonst wird eine Fehlermeldung ausgegeben.
    GIANT zeigt in der Statuszeile im Hauptfenster 
    \gq{Select Position in Display Window
    for Insertion of copied IML Subgraph}.
    Der Benutzer gibt �ber das Fadenkreuz die Position vor, 
    an der die neuen Fenster-Knoten und Fenster-Kanten eingef�gt werden 
    sollen (siehe auch \ref{Fadenkreuzcursor}).
    
    \step[3]
    GIANT kopiert die Knoten und Kanten der 
    Selektion in das Ziel-Anzeigefenster.
    Je nachdem, welchen Eintrag der Benutzer im Popup-Men� ausgew�hlt hat,
    geschieht mit den bereits im Ziel-Anzeigefenster vorhandenen 
    Fenster-Knoten der Selektion folgendes:
    \begin {enumerate}
       \item
       Falls \gq{Copy Selection keeping existing layout} gew�hlt wurde, wird
       ihre Position im Ziel-Anzeigefenster nicht ver�ndert.
       
       \item 
       Falls \gq{Copy Selection changing existing layout} gew�hlt wurde,
       ihre Position im Zielanzeigefenster gem�� dem Layout der
       Selektion ver�ndert.
        
    \end {enumerate}
      
  \end{proc}


\end{uc}

% ==============================================================================
\begin{uc}[Fenster-Knoten und Fenster-Kanten einer Selektion aus 
           Anzeigefenster l�schen]
      {UC: Fenster-Knoten und Fenster-Kanten einer Selektion aus 
           Anzeigefenster l�schen}
\index{Fenster-Knoten!l�schen}
\index{Fenster-Kanten!l�schen}
                  
Mittels dieses UseCases k�nnen alle Fenster-Knoten und Fenster-Kanten
einer Selektion aus einem Anzeigefenster gel�scht werden
(siehe auch \ref{Verhalten beim Entfernen von Fenster-Knoten und 
Fenster-Kanten}).

 
  \begin{precond}
    \cond Ein Projekt mit mindestens einem ge�ffneten 
          Anzeigefenster ist geladen.
    
    \cond Es gibt eine Selektion.
    
  \end{precond}

  \begin{postsuccess}
    
    \cond
    Die Selektion ist aus dem Anzeigefenster gel�scht.
    
    \cond 
    Alle betroffenen Fenster-Knoten und Fenster-Kanten sind gem��
    der unter 
    \ref{Verhalten beim Entfernen von Fenster-Knoten und Fenster-Kanten}
    beschriebenen Konvention aus dem Anzeigefenster entfernt.

    \cond
    Alle anderen Selektionen des Anzeigefensters wurden aktualisiert.
 
   
  \end{postsuccess}

  \begin{postfail}
    \cond Hat der Benutzer den UseCase abgebrochen,
    kehrt das System zu dem Zustand zur�ck, in dem es vor dem Start des
    UseCase war.
  \end{postfail}
  
  \begin{proc}    
  
    \step[1]
    Der Benutzer f�hrt einen Rechtsklick auf die entsprechende Selektion in der
    Selektionsauswahlliste durch (siehe \ref{Selektionsauswahlliste})
    und w�hlt im entsprechenden Popup-Men� den Eintrag \gq{Delete Selection}
    aus.
      
    
    \step[2]
    GIANT zeigt die Sicherheitsabfrage (siehe \ref{Sicherheitsabfrage}) 
    und fragt nach, ob es die Selektion samt ihrer Knoten und Kanten
    wirklich l�schen soll (\gq{Really delete Selection xy from its window 
    including Nodes and Edges?}).
    
    \step[3]
    Der Benutzer best�tigt mit \gq{Yes}.
    
    \step[4]
    GIANT l�scht die Selektion samt allen zugeh�rigen Fenster-Knoten und
    Fenster-Kanten aus dem entsprechenden Anzeigefenster.
      
  
  \end{proc}

  \begin{aproc}
    \astep{2} Der Benutzer bricht den UseCase mit \gq{No} ab.
  \end{aproc}


\end{uc}

% ==============================================================================
\begin{uc}[Den Visualisierungsstil eines Anzeigefensters �ndern]
{UC: Den Visualisierungsstil eines Anzeigefensters �ndern}
\index{Visualisierungsstile!innerhalb eines Anzeigefensters}

Mittels dieses UseCase kann der Benutzer die Visualisierung
von Fenster-Knoten und Fenster-Kanten innerhalb eines Anzeigefensters
dynamisch durch Auswahl verschiedener frei definierbarer
Visualisierungsstile �ndern (siehe auch \ref{Config Visualisierungsstile}).

  \begin{precond}

    \cond 
    Es gibt ein ge�ffnetes Anzeigefenster.    
   
    \cond
    Es gibt mindestens einen definierten Visualisierungsstil
    (siehe auch \ref{Config Visualisierungsstile}). 
       
  \end{precond}

  \begin{postsuccess}
    
    \cond
    Die Darstellung der Fenster-Knoten und Fenster-Kanten in dem
    Anzeigefenster entspricht den Vorgaben des gew�hlten 
    Visualisierungsstils.

    \cond
    Alle anderen Zust�nde und Eigenschaften des Anzeigefensters, wie
    z.B. die hervorgehobenen Selektionen, bleiben unver�ndert.
    
  \end{postsuccess}

  \begin{proc}    
  
    \step[1]
    Der Benutzer �ndert den Visualisierungsstil eines Anzeigefensters
    dadurch, dass er in der Stilauswahl-Combobox 
    des Anzeigefensters (siehe \ref{GUI Stilauswahl-Combobox}) einen 
    anderen Visualisierungsstil einstellt.
      
    \step[2]
    GIANT �ndert die Darstellung von Fenster-Knoten und Fenster-Kanten
    entsprechend ab (n�heres zur Visualisierung von Fenster-Knoten
    und Fenster-Kanten ist in Kapitel 
    \ref{Visualisierung des IML-Graphen} spezifiziert).
              
  \end{proc}

\end{uc}

% ==============================================================================
\begin{uc}[Anzeigefenster scrollen]{UC: Anzeigefenster scrollen}
\index{Anzeigefenster!Scrollen}
\index{Scrollen}

Ver�ndert die Position des sichtbaren Anzeigeinhaltes.

  \begin{precond}
    \cond 
    Ein Projekt mit mindestens einem ge�ffneten Anzeigefenster ist geladen.
       
  \end{precond}

  \begin{postsuccess}
    
    \cond
    Die Position des sichtbaren Anzeigeinhalts wurde entsprechend abge�ndert.
    
  \end{postsuccess}

  \begin{proc}    
  
    \step[1]
    \begin {enumerate}
      \item
      Der Benutzer scrollt den sichtbaren Anzeigefokus mittels der horizontalen
      oder vertikalen Bildlaufleisten des Anzeigefensters (siehe    
      \ref{Scrolleisten}). Dies geschieht mittels
      der Maus gem�� der Konventionen von GTK/Ada f�r Bildlaufleisten.
      
      \item
      Es kann auch mittels der Cursortasten gescrollt werden.\\
      Das Dr�cken der linken Cursortaste f�hrt z.B.\ dazu, dass der sichtbare
      Anzeigeinhalt des aktiven Anzeigefensters nach links verschoben wird.
      
    \end {enumerate}
        
  \end{proc}

\end{uc}


% ==============================================================================
\begin{uc}[Anzeigefenster zoomen]{UC: Anzeigefenster zoomen}
\index{Anzeigefenster!zoomen}
\index{zoomen}
Ver�ndert den Ma�stab der Darstellung von Knoten und Kanten.

  \begin{precond}
    \cond 
    Ein Projekt mit mindestens einem ge�ffneten Anzeigefenster ist geladen.
       
  \end{precond}

  \begin{postsuccess}
    
    \cond
    Der angezeigte Bereich des sichtbaren Anzeigeinhalts wurde entsprechend
    vergr��ert oder verkleinert.
    Die Detailstufe (siehe \ref{Visualization Detailstufen})
    wurde ggf. automatisch angepasst.
    
  \end{postsuccess}

  \begin{proc}    
  
    \step[1]
    \begin {enumerate}
      \item
      Der Benutzer gibt in der Zoom-Kontrolle-Combobox 
      (siehe \ref{GUI Zoom-Kontrolle})
      des Anzeigefensters einen
      neuen Zoomwert ein, w�hlt darin einen der vorgefertigten Werte aus oder 
      �ndert den Zoomwert in festgelegten Schritten mit den \gq{+} oder 
      \gq{-} Buttons. 
      
    \end {enumerate}
    
    \step[2]
    GIANT berechnet den neuen sichtbaren Anzeigeinhalt anhand der neuen
    Zoomstufe.

        
  \end{proc}
\end{uc}

% ==============================================================================
\begin{uc}[Zoomen auf eine Selektion]{UC: Zoomen auf eine gesamte Selektion}
\index{zoomen auf Selektion}

W�hlt die passende Zoomstufe \index{Zoomstufe} 
und scrollt den sichtbaren Anzeigeinhalt so, 
dass eine Selektion im Anzeigefenster vollst�ndig
sichtbar ist.

  \begin{precond}
    \cond 
    Ein Projekt mit mindestens einem ge�ffneten Anzeigefenster ist geladen.
       
  \end{precond}

  \begin{postsuccess}
    
    \cond Der sichtbare Anzeigeinhalts wurde
    mittels Zoomen und Scrollen so ver�ndert, dass die
    ausgew�hlte Selektion vollst�ndig sichtbar ist. Die Detailstufe (siehe
    \ref{Visualization Detailstufen}) wurde ggf. automatisch
    angepasst.
        
  \end{postsuccess}

  \begin{proc}    
  
    \step[1]
    Der Benutzer klickt auf \gq{Zoom to make selection fill window} 
    im Popup-Men�
    der Selektionsauswahlliste (siehe \ref{Selektionsauswahlliste}).
      
    \step[2]
    GIANT scrollt und zoomt automatisch so, dass die gesamte Selektion
    im Anzeigefenster sichtbar wird.
        
  \end{proc}
\end{uc}

% ==============================================================================
\begin{uc}[Zoomen auf den Anzeigeinhalt]{UC: Zoomen auf gesamten Inhalt eines Anzeigefensters}
\index{zoomen auf gesamten Anzeigeinhalt}
W�hlt die passende Zoomstufe und scrollt den sichtbaren Anzeigeinhalt so, 
dass der Anzeigeinhalt vollst�ndig im sichtbaren Anzeigeinhalt
dargestellt wird.

  \begin{precond}
    \cond 
    Ein Projekt mit mindestens einem ge�ffneten Anzeigefenster ist geladen.
       
  \end{precond}

  \begin{postsuccess}

    \cond Der sichtbare Anzeigeinhalts wurde
    mittels Zoomen und Scrollen so ver�ndert, dass der
    Anzeigeinhalt vollst�ndig sichtbar ist. Die Detailstufe (siehe
    \ref{Visualization Detailstufen}) wurde ggf. automatisch
    angepasst.
    
  \end{postsuccess}

  \begin{proc}    
  
    \step[1]
    Der Benutzer klickt auf \gq{Fill window} in der Zoomkontrolle des
    Fensters (siehe \ref{GUI Zoom-Kontrolle}).
      
    \step[2]
    GIANT berechnet die neue Zoomstufe f�r das Anzeigefenster.
        
  \end{proc}

\end{uc}

% ==============================================================================
\begin{uc}[Zoomen auf eine Kante]{UC: Zoomen auf eine Kante}
\index{zoomen auf eine Kante}

W�hlt die passende Zoomstufe und scrollt den sichtbaren Anzeigeinhalt
so, dass eine Kante mit ihren Start- und Ziel-Fenster-Knoten komplett
im sichtbaren Anzeigeinhalt liegt.

  \begin{precond}
    \cond 
    Ein Projekt mit mindestens einem ge�ffneten Anzeigefenster mit mindestens 
    einer Kante ist geladen.
       
  \end{precond}


  \begin{postsuccess}
    
    \cond Der sichtbare Anzeigeinhalts wurde mittels Zoomen und
    Scrollen so ver�ndert, dass die ausgew�hlte Kante und ihre Start-
    und Ziel-Fenster-Knoten vollst�ndig sichtbar ist. Die Detailstufe
    (siehe \ref{Visualization Detailstufen}) wurde ggf. automatisch
    angepasst.
    
  \end{postsuccess}


  \begin{postfail}
    \cond Hat der Benutzer den UseCase abgebrochen,
    kehrt das System zu dem Zustand zur�ck, in dem es vor dem Start des
    UseCase war.
  \end{postfail}


  \begin{proc}    
  
    \step[1]
    Der Benutzer Klickt den Button \gq{Pick Edge} in der Zoomkontrolle
    (siehe \ref{GUI Zoom-Kontrolle}).
   
    \step[2] 
    Daraufhin erscheint in der Statuszeile von GIANT im Hauptfenster
    der Text 
    \gq{Select Edge in Window (Fenstername) to be zoomed onto} und
    der Fadenkreuz-Cursor (siehe \ref{Fadenkreuzcursor}) wird
    angezeigt, wenn der Mauscursor �ber den sichtbaren Anzeigeinhalt
    eines Anzeigefensters bewegt wird.
       
    \step[3]
    Der Benutzer klickt mit der linken Maustaste auf die gew�nschte Kante
    im gew�nschten Anzeigefenster.

    \step[4]
    GIANT berechnet f�r das gew�nschte Anzeigefenster die Zoomstufe so,
    dass  die gesamte Fenster-Kante im Anzeigeinhalt sichtbar dargestellt
    wird.
      
        
  \end{proc}

  \begin{aproc}

    \astep{3} Der Benutzer kann den UseCase abbrechen, indem er
    auf eine einen Rechtsklick mit der Maus ausf�hrt.
  
  \end{aproc}

\end{uc}


% ==============================================================================
\begin{uc}[Verschieben von Fenster-Knoten und Selektionen mittels Cut and Paste]{UC: Verschieben von Fenster-Knoten und Selektionen mittels
                  Cut and Paste}
\index{verschieben!einzelne Knoten}
\index{verschieben!ganze Selektionen}

Mit diesem UseCase k�nnen Fenster-Knoten und Selektionen auf dem
Anzeigeinhalt verschoben werden. Dieses Verschieben geschieht mittels
\gq{Cut and Paste}.


  \begin{precond}
    \cond 
    Ein Projekt mit mindestens einem ge�ffneten Anzeigefenster und mit
    mindestens einem Fenster-Knoten oder mindestens einer Selektion
    ist geladen.
       
  \end{precond}

  \begin{postsuccess}
    
    \cond
    Der einzelne Fensterknoten oder alle Fenster-Knoten der Selektion
    wurden im Anzeigeinhalt verschoben.  
  \end{postsuccess}

  \begin{postfail}
    \cond Hat der Benutzer den UseCase abgebrochen,
          so werden keine Fenster-Knoten verschoben.
  \end{postfail}  


 

  \begin{proc}    
  
    \step[1]
    Der Benutzer w�hlt die zu verschiebende Selektion oder den
    zu verschiebenden Fenster-Knoten aus (\gq{Cut}), indem er:
   
    \begin{enumerate}
      \item
      Einen Rechtsklick auf eine Selektion in der Selektionsauswahlliste
      durchf�hrt (siehe \ref{Selektionsauswahlliste}) und im
      Popup-Men� den Eintrag \gq{Move Selection} ausw�hlt,

      \item
      oder einen Rechtsklick auf einen Fenster-Knoten durchf�hrt
      und im Popup Men� (siehe \ref{Node-Popup-Men�}) den Eintrag
      \gq{Move Node} ausw�hlt.

    \end{enumerate}
    
    \step[2] GIANT geht in den den \gq{Paste Modus} �ber und zeigt
    dies in der Statusleiste (siehe \ref{Statuszeile}) des
    Hauptfensters an.  Der Cursor wird, falls er �ber den sichtbaren
    Anzeigeinhalt eines Anzeigefensters bewegt wird, zum Fadenkreuz
    (siehe \ref{Fadenkreuzcursor}).  Die Funktionalit�t zum Zoomen und
    Scrollen des Anzeigefensters mittels der beiden UseCases
    \ref{Anzeigefenster scrollen} und \ref{Anzeigefenster zoomen}
    bleibt weiterhin verf�gbar, die �brige Funktionalit�t von GIANT
    wird gesperrt.
    
    \step[3] Der Benutzer klickt mit der linken Maustaste an eine
    beliebige Stelle innerhalb des sichtbaren Anzeigeinhaltes des
    Anzeigefensters.

    \step[4]
    GIANT verschiebt die ausgew�hlten Fenster-Knoten an die gew�nschte
    Stelle.
      
  \end{proc}

  \begin{aproc}

    \astep{3} Der Benutzer kann den UseCase abbrechen, indem er
    einen Rechtsklick mit der Maus im sichtbaren Anzeigeinhalt durchf�hrt.
  \end{aproc}


\end{uc}



% ==============================================================================
\begin{uc}[Verschieben einzelner Fenster-Knoten mittels Drag and Drop]{UC: Verschieben einzelner Fenster-Knoten mittels 
                      Drag and Drop}
\index{verschieben!einzelne Knoten}

Mit diesem UseCase k�nnen einzelne Fenster-Knoten mittels Drag and
Drop auf dem sichtbaren Anzeigeinhalt verschoben werden.


  \begin{precond}
    \cond 
    Ein Projekt mit mindestens einem ge�ffneten Anzeigefenster und mit
    mindestens einem Fenster-Knoten ist geladen.
       
  \end{precond}

  \begin{postsuccess}
    
    \cond
    Der einzelne Fensterknoten wurde auf dem Anzeigeinhalt entsprechend 
    verschoben.  

  \end{postsuccess}

  \begin{proc}    
  
    \step[1]
    Der Benutzer bewegt den Mauscursor �ber den zu verschiebenden
    Fenster-Knoten und dr�ckt die linke Maustaste (\gq{Drag}), dann bewegt
    er den Fenster-Knoten an eine beliebige andere Stelle innerhalb des
    Anzeigeinhaltes und l�sst die linke Maustaste los (\gq{Drop}).\\

  \end{proc}

\end{uc}



% ==============================================================================
\begin{uc}[Platz schaffen]{UC: Platz schaffen}
\index{Fenster-Knoten!auseinanderschieben}

Dieser UseCase wird ben�tigt, um Fenster-Knoten auseinander schieben zu 
k�nnen. So kann der Benutzer an einer beliebigen Stelle des Anzeigefensters
gen�gend Platz zum Einf�gen neuer Fenster-Knoten und Fenster-Kanten schaffen.
Siehe auch Abschnitt \ref{Auseinanderschieben von Fenster-Knoten}.


  \begin{precond}
    \cond Ein Projekt mit mindestens einem ge�ffneten 
          Anzeigefenster ist geladen.
      
  \end{precond}

  \begin{postsuccess}
    
    \cond 
    Alle Fenster-Knoten und Fenster-Kanten des Anzeigefensters sind
    um den entsprechenden Betrag vom vorgegebenen Punkt innerhalb
    des Anzeigeinhaltes weggeschoben worden.
    An der entsprechenden Stelle im Anzeigeinhalt ist eine freie Fl�che
    ohne Fenster-Knoten geschaffen worden. Diese Fl�che kann aber von 
    Fenster-Kanten gekreuzt werden.
    
    \cond
    Das Layout aller Fenster-Knoten des Anzeigeinhaltes bleibt ansonsten
    weitgehend unver�ndert.
   
  \end{postsuccess}

 
  \begin{postfail}
    \cond Hat der Benutzer den UseCase abgebrochen,
          so wird der Anzeigeinhalt nicht ver�ndert.
  \end{postfail}  
  
  \begin{proc}    
    \step[1]
    Der Benutzer f�hrt einen Rechtsklick auf eine beliebige Stelle
    des sichtbaren Anzeigeinhaltes durch und w�hlt im
    daraufhin erscheinenden Popup-Men� (siehe \ref{Empty Vis Pane Right click})
    dem Eintrag \gq{Make Room} aus.
    
    \step[2]
    GIANT zeigt in der Statuszeile im Hauptfenster 
    \gq{Select Position in Display Window}.
    Der Benutzer gibt den Punkt um den herum die Fenster-Knoten (und damit
    automatisch auch die Fenster-Kanten) auseinander geschoben werden
    sollen �ber das Fadenkreuz vor (siehe \ref{Fadenkreuzcursor}).
    Hierbei kann er mit dem Fadenkreuz den aktuellen sichtbaren Anzeigeinhalt
    auf dem Anzeigefenster nicht verlassen.
     
    \step[3] 
    GIANT zeigt einen Dialog an, in dem der Benutzer ausw�hlt, um welchen 
    Betrag die Fenster-Knoten auseinander geschoben werden sollen (siehe
    \ref{Platz Schaffen-Dialog}).
   
    \step[4]
    Der Benutzer w�hlt einen geeigneten Betrag aus und best�tigt mit OK.
    
    \step[5]
    GIANT schiebt die Knoten entsprechend auseinander
    (siehe auch \ref{Auseinanderschieben von Fenster-Knoten}).
    
  \end{proc}
  \begin{aproc}

    \astep{2} Der Benutzer kann den UseCase abbrechen, indem er mit der
              Maus einen Rechtsklick auf eine beliebige Stelle 
              durchf�hrt.

    \astep{3} Der Benutzer kann den UseCase durch Bet�tigen des \gq{Cancel}
              Buttons abbrechen.

  \end{aproc}


\end{uc}


% ==============================================================================
\begin{uc}[Pin anlegen]{UC: Pin anlegen}
\index{Pins!anlegen}

Mit diesem UseCase kann ein neuer Pin 
(siehe auch \ref{VIS-PANE-Pins}) erzeugt werden.

 \begin{precond}
    \cond Es gibt ein ge�ffnetes Anzeigefenster.
  \end{precond}

  \begin{postsuccess}
    \cond 
    In der Liste �ber die Pins des Anzeigefensters
    (siehe \ref{VIS-PANE-Pins}) befindet sich ein neuer Pin mit 
    dem vom Benutzer definierten Namen.

  \end{postsuccess}

  \begin{postfail}
    \cond Das System bleibt im bisherigen Zustand.
  \end{postfail}
  
  \begin{proc}    
    \step[1]
    
    Der Benutzer f�hrt einen Rechtsklick auf den Anzeigeinhalt eines
    Anzeigefensters durch und w�hlt aus dem Popup-Men� 
    (siehe \ref{Empty Vis Pane Right click}) den Eintrag \gq{New Pin} aus.
    Der sp�ter erstellte Pin verweist dann auf die Stelle im Anzeigeinhalt,
    auf die der Rechtklick durchgef�hrt wurde.
 

    \step[2] 
    GIANT �ffnet den allgemeinen Texteingabedialog (siehe 
    \ref{DIALOG-WINDOW}).
    
      
    \step[3] 
    Der Benutzer gibt dort einen zul�ssigen Namen f�r den neuen Pin
    ein und best�tigt mit OK (zul�ssige Namen f�r sind in Abschnitt 
    \ref{afa Zulaessige Namen} spezifiziert).
      
    \step[4]
    GIANT speichert die aktuelle Zoomstufe und die Position des sichtbaren
    Anzeigeinhaltes in einem neuen Pin.
  
  \end{proc}

  \begin{aproc}
    \astep{3} Der Benutzer bricht die Verarbeitung mit Cancel ab.
  \end{aproc}

\end{uc}


% ==============================================================================
\begin{uc}[Pin anspringen]{UC: Pin anspringen}
\index{Pins!anspringen}

Stellt die im Pin (siehe auch \ref{VIS-PANE-Pins}) gespeicherte 
Position des sichtbaren Anzeigeinhaltes wieder her.

 \begin{precond}
    \cond Es gibt ein ge�ffnetes Anzeigefenster mit mindestens 
    einem Pin.
  \end{precond}

  \begin{postsuccess}
    \cond 
    Der sichtbare Anzeigefokus des Anzeigefensters ist auf die entsprechenden
    Koordinaten und die entsprechende Zoomstufe, wie sie im ausgew�hlten
    Pin hinterlegt waren, gesetzt.
    
  \end{postsuccess}
  
  \begin{proc}    
    \step[1]
   
    Das Anspringen des Pins kann �ber die folgenden beiden
    M�glichkeiten geschehen:
    \begin{enumerate}
    
      \item
      Der Benutzer f�hrt einen Doppelklick auf den entsprechenden Pin
      in der Pinliste (siehe \ref{VIS-PANE-Pins}) aus.
         
      \item
      Der Benutzer �ffnet ein Popup-Men� durch
      Rechtsklick auf den anzuspringenden Pin 
      in der Pinliste (siehe \ref{VIS-PANE-Pins}) und w�hlt den Eintrag
      \gq{Focus Pin} aus.
        
    \end {enumerate}
    
          
    \step[2]
    GIANT setzt den sichtbaren Anzeigeinhalt gem�� den im Pin gespeicherten
    Informationen.
  
  \end{proc}



\end{uc}


% ==============================================================================
\begin{uc}[Pin l�schen]{UC: Pin l�schen}
\index{Pins!l�schen}

L�scht einen Pin (siehe auch \ref{VIS-PANE-Pins}).

 \begin{precond}
    \cond Es gibt ein ge�ffnetes Anzeigefenster mit mindestens 
    einem Pin.
  \end{precond}

  \begin{postsuccess}
    \cond 
    Der Pin ist gel�scht und nicht mehr in der Pinliste (siehe \ref{VIS-PANE-Pins}) 
    des entsprechenden Anzeigefensters ausw�hlbar.
    
  \end{postsuccess}
  
  \begin{proc}    
    \step[1]
    Der Benutzer �ffnet das  Popup-Men� der Pinliste (siehe \ref{VIS-PANE-Pins})
    durch Rechtsklick auf den Pin  und w�hlt den Eintrag \gq{Delete Pin} aus.
     
    \step[2]
    GIANT l�scht den Pin.
  
  \end{proc}

\end{uc}

%%% Local Variables: 
%%% TeX-master: "../spec"
%%% End: 

% ==============================================================================
%  $RCSfile: additional.tex,v $, $Revision: 1.18 $
%  $Date: 2003/04/19 19:49:06 $
%  $Author: birdy $
%
%  Description: Sonstige UseCases
%
%  Last-Ispelled-Revision: 1.6
%
% ==============================================================================


\begin{uc}[Layout auf Selektion anwenden]{UC: Layout auf Selektion anwenden}
\index{Selektionen!layouten}

Mittels dieses UseCase k�nnen Selektionen innerhalb
eines Anzeigefensters layoutet werden.
Die verf�gbaren Layoutalgorithmen sind im Kapitel 
\ref{Layoutalgorithmen} beschrieben.


  \begin{precond}
    \cond Es gibt mindestens ein ge�ffnetes Anzeigefenster mit
          mindestens einer Selektion (dies kann auch die
          immer vorhandene Standard-Selektion sein).

  \end{precond}

  \begin{postsuccess}
    \cond 
    Die Position der Fenster-Knoten der Selektion wurde gem��
    den Vorgaben des Layoutalgorithmus ge�ndert. 

    \cond
    Alle anderen Fenster-Knoten bleiben unver�ndert.

  \end{postsuccess}

  \begin{postfail}
    \cond Das System bleibt im bisherigen Zustand.

    \cond Hat der Benutzer einen Layoutalgorithmus w�hrend der
    Berechnung des Layouts abgebrochen, 
    so bleiben die Positionen aller Fenster-Knoten des Anzeigefensters
    unver�ndert.
      
  \end{postfail}
  
  \begin{proc}
    
    \step[1] Der Benutzer w�hlt aus der Selektionsauswahlliste (siehe
    \ref{Selektionsauswahlliste}) die Selektion aus, deren
    Fenster-Knoten layoutet werden sollen. Hierzu f�hrt er einen
    Rechtsklick auf die Selektion durch und w�hlt im Popup-Men� den
    Eintrag \gq{Layout Selection} aus.

    \step[2]
    GIANT zeigt einen Dialog zur Auswahl und Konfiguration des
    gew�nschten Layoutalgorithmus 
    (siehe \ref{Layoutalgorithmen-Dialog}).

    \step[3]
    Der Benutzer w�hlt in diesem Dialog den gew�nschten Layoutalgorithmus
    aus.

    \step[4]
    Falls der gew�hlte Layoutalgorithmus dies unterst�tzt, kann der
    Benutzer in diesem Dialog die Klassenmengen, die beim
    Layout ber�cksichtigt werden sollen, vorgeben.

    \step[5]
    Der Benutzer best�tigt seine Eingaben mit OK.
    
    
    \step[6]
    GIANT fordert den Anwender �ber die Statuszeile auf, die Zielposition
    des Layouts anzugeben (\gq{Please click on desired target position for
    Layout in Window}). Der Nutzer klickt mit der linken Maustaste auf
    das Zielgebiet im Fenster.

    \step[7]
    GIANT berechnet das Layout und zeigt einen Progress-Dialog 
    (siehe \ref{Progressbar-Modale}) an. 
    W�hrend der Berechnung des Layouts ist die GUI mit Ausnahme des
    Progress-Dialogs gesperrt.
         
  \end{proc}

  \begin{aproc}
    \astep{3} Der Benutzer bricht die Verarbeitung mit Cancel ab.  
    \astep{4} Der Benutzer bricht die Verarbeitung mit Cancel ab.

    \astep{6} Der Benutzer kann die Berechnung eines Layouts jederzeit mit
    Cancel (siehe \ref{Progressbar-Modale-Cancel}) abbrechen.
  \end{aproc}

\end{uc}


\begin{uc}[Details zu Knoten in einem neuen Informationsfenster
anzeigen]{UC: Details zu Knoten in einem neuen Informationsfenster
anzeigen}

Dieser UseCase erm�glicht dem Benutzer, sich alle verf�gbaren
Informationen zu einem Fenster-Knoten (Kanten und Attribute des
zugeh�rigen IML-Knoten) anzeigen zu lassen. Mit diesem UseCase
kann der Benutzer beliebig viele Knoten-Informationsfenster 
(siehe \ref{Knoten-Informationsfenster}) �ffnen.


  \begin{precond}
    \cond Es gibt mindestens ein ge�ffnetes Anzeigefenster mit
          mindestens einem Fenster-Knoten.
  \end{precond}

  \begin{postsuccess}
    \cond 
    Die Daten �ber den Fenster-Knoten werden in einem neuen
    Knoten-Informationsfenster angezeigt. 

  \end{postsuccess}
 
  \begin{proc}    

    \step[1]
    Der Benutzer w�hlt im entsprechenden Anzeigefenster den 
    gew�nschten Fenster-Knoten aus und f�hrt einen Rechtsklick
    auf diesen Fenster-Knoten durch. Im darauf angezeigten Popup-Men�
    (siehe \ref{Node-Popup-Men�}) w�hlt der Benutzer dann 
    den Eintrag \gq{Show Node Info Window} aus.

    \step[2]
    GIANT �ffnet ein neues Knoten-Informationsfenster
    (siehe \ref{Knoten-Informationsfenster}) und zeigt alle
    verf�gbaren Informationen zu dem IML-Knoten an.
    
  \end{proc}

\end{uc}



\begin{uc}[Details zu Knoten in einem bestehenden
Informationsfenster anzeigen]{UC: Details zu Knoten in einem bestehenden
Informationsfenster anzeigen}

Mittels dieses UseCase kann sich der Benutzer alle verf�gbaren
Informationen zu einem Fenster-Knoten (Kanten und Attribute des
zugeh�rigen IML-Knoten) innerhalb eines bereits ge�ffneten
Knoten-Informationsfensters (siehe \ref{Knoten-Informationsfenster})
anzeigen lassen.


  \begin{precond}
    \cond Es gibt mindestens ein ge�ffnetes Anzeigefenster mit
          mindestens einem Fenster-Knoten.

    \cond Es gibt mindestens ein ge�ffnetes Knoten-Informationsfenster.

  \end{precond}

  \begin{postsuccess}

    \cond 
    Anstatt der Daten, die zuvor in dem Knoten-Informationsfenster
    angezeigt wurden, werden dort nun die Informationen zu dem 
    gew�hlten Fenster-Knoten angezeigt.

  \end{postsuccess}


  \begin{postfail}

    \cond Hat der Benutzer den UseCase abgebrochen, bleibt
          das System im bisherigen Zustand.

  \end{postfail}



  \begin{proc}    

    \step[1]
    Der Benutzer bet�tigt in einem ge�ffneten Knoten-Informationsfenster
    den Button \gq{Pick} (siehe \ref{Knoten-Informationsfenster}).

    \step[2]
    Daraufhin fordert GIANT den Benutzer auf, einen Fenster-Knoten
    auszuw�hlen.

    \step[3]
    Der Benutzer w�hlt den gew�nschten Fenster-Knoten dadurch aus, dass
    er innerhalb des Anzeigeinhaltes eines Anzeigefensters mittels
    eines Fadenkreuzes den Fenster-Knoten als Zielposition
    vorgibt. Das Verfahren zur Vorgabe der Zielposition ist im
    Detail unter \ref{Vorgabe der Zielposition} beschrieben.

    \step[4]
    GIANT stellt die verf�gbaren Informationen zu dem ausgew�hlten
    Fenster-Knoten im Knoten-Informationsfenster dar.
    
  \end{proc}

  \begin{aproc}
    \astep{3} W�hrend der Fadenkreuzcursor aktiv ist, kann der Benutzer 
              den UseCase mit einem Rechtsklick auf eine beliebige
              Stelle abbrechen.  
  \end{aproc}

\end{uc}




\begin{uc}[Anzeige des Quellcodes eines Knotens in einem externen Editor]
{UC: Anzeige des Quellcodes eines Knotens in externem Editor}

Mittels dieses UseCases kann der zu einem IML-Knoten korrespondierende
Quellcode innerhalb eines Editors (unterst�tzt werden Emacs und vi)
zur Anzeige gebracht werden.


  \begin{precond}
    \cond Es gibt mindestens ein ge�ffnetes Anzeigefenster mit
          einem oder mehr Fenster-Knoten.
       
  \end{precond}

  \begin{postsuccess}
    \cond 
    Der zu dem IML-Knoten korrespondierende Quellcode wird in
    dem �ber die Konfigurationsdatei vorgegebenen Editor
    angezeigt (siehe auch \ref{Konfig Editor zur Anzeige des Quellcodes}).
    Der Cursor des Editors ist auf die dem IML-Knoten
    entsprechende Position innerhalb des Quelltextes gesetzt, also
    z.B.\ auf einen Bezeichnernamen, den der IML-Knoten repr�sentiert.

    \cond
    Hat der IML-Knoten keine \gq{Source Code Position}, so wird
    nur die zugeh�rige Quellcode Datei ge�ffnet.
 
  \end{postsuccess}

  
  \begin{proc}    

    \step[1]
    Der Benutzer w�hlt in einem Anzeigefenster den 
    gew�nschten Fenster-Knoten aus und f�hrt einen Rechtsklick
    auf diesen Knoten durch. Im darauf angezeigten Popup-Men�
    (siehe \ref{Node-Popup-Men�}) w�hlt der Benutzer dann 
    den Eintrag \gq{Show Corresponding Source Code} aus.
    Der UseCase kann nicht f�r IML-Knoten durchgef�hrt werden,
    f�r die die Reflektion zur Bauhaus IML-Graph-Bibliothek
    keine zugeh�rige Quellcode-Datei liefern kann.

    \step[2]
    GIANT �ffnet die Quellcode Datei f�r den IML-Knoten in
    einem Editor.
    
  \end{proc}


\end{uc}



\begin{uc}[Verschieben des sichtbaren Anzeigeinhaltes �ber die Minimap]
{UC: Verschieben des sichtbaren Anzeigeinhaltes �ber die Minimap}

Der Benutzer kann mittels der Minimap (siehe auch \ref{GUI Minimap})
den sichtbaren Anzeigeinhalt scrollen.


 \begin{precond}

    \cond Es gibt mindestens ein ge�ffnetes Anzeigefenster.

  \end{precond}

  \begin{postsuccess}
    \cond 
    Der sichtbare Anzeigeinhalt ist auf den Bereich des
    Anzeigefensters gesetzt, der dem Punkt welcher auf der
    Minimap angeklickt wurde, entspricht.

  \end{postsuccess}

  
  \begin{proc}    

    \step[1]
    Der Benutzer klickt mit der linken Maustaste auf einen
    Punkt auf der Minimap.

    \step[2]
    GIANT zentriert den sichtbaren Anzeigeinhalt des Anzeigefensters 
    auf den vorgegebenen Punkt.
    
  \end{proc}


\end{uc}


\chapter{Knoten-Annotationen}
% ==============================================================================
%  $RCSfile: node_annotations.tex,v $
%  $Date: 2003/03/31 22:02:25 $
%  $Author: squig $
%
%  Description: UseCases zum Annotieren von Knoten
%
%  Last-Ispelled-Revision: 1.6
%
% ==============================================================================

\begin{uc}[Fenster-Knoten annotieren]{UC: Fenster-Knoten annotieren}
  Dieser UseCase dient zum Erzeugen von Knoten-Annotationen.
  F�r weitere Informationen siehe auch Abschnitt
  \ref{Project Persistenz von Knoten-Annotationen}. 
  
  \begin{precond}
    \cond Ein Projekt ist geladen.
    \cond Es gibt mindestens ein Anzeigefenster mit Fenster-Knoten.
  \end{precond}

  \begin{postsuccess}
    \cond Die neue Annotation ist noch nicht
          in der Verwaltungsdatei f�r Knoten-Annotationen (siehe
          \ref {Project Verwaltungsdatei f�r Knoten-Annotationen})
          eingetragen. Dies geschieht erst beim n�chsten Speichern
          des gesamten Projektes.

    \cond Der annotierte Knoten wird in allen Anzeigefenstern mittels
          eines Icons als annotiert gekennzeichnet 
          (siehe \ref{Visualsization Icon f�r Knoten-Annotationen}).
          
  \end{postsuccess}

  \begin{postfail}
    \cond Das System bleibt im bisherigen Zustand.
  \end{postfail}
  
  \begin{proc}
    \step[1] 
    Der Benutzer f�hrt einen Rechtsklick auf den 
    Knoten, der annotiert werden soll,
    durch und w�hlt im Popup Men� f�r Knoten den Eintrag
    \gq {Annotate Node} aus (siehe \ref{Node-Popup-Men�}).
   
    \step[2]
    GIANT zeigt nun den Knoten Annotations-Dialog 
    (siehe \ref{Knoten Annotations-Dialog}).
    
    \step[3]
    Der Benutzer gibt dort den gew�nschten Text f�r die Knoten-Annotation 
    ein.
  
    \step[4]
    Nach Abschluss der Eingabe bet�tigt der Benutzer den \gq{OK Button}.
            
    \step[5]
    GIANT �bernimmt die neu Annotation. Die eingegebene Annotation 
    muss allerdings mindestens ein Zeichen haben, ansonsten erscheint eine
    Fehlermeldung.

  \end{proc}

  \begin{aproc}

    \astep{3}
    Der Benutzer kann die Eingabe der neuen Knoten-Annotation jeder Zeit 
    mittels des \gq{Cancel Buttons} abbrechen.
    

  \end{aproc}
\end{uc}

% ==============================================================================

\begin{uc}[Knoten-Annotation �ndern]{UC: Knoten-Annotation �ndern}
  Dient zum �ndern bestehender Knoten-Annotationen (weitere Informationen
  zu Knoten-Annotationen sind unter Abschnitt 
  \ref{Project Persistenz von Knoten-Annotationen} zu finden).
  Dieser UseCase ist auch f�r das Anzeigen von Knoten-Annotationen
  vorgesehen.
  
  \begin{precond}
    \cond Ein Projekt ist geladen.
    \cond Es gibt mindestens einen annotierten Fenster-Knoten
          (solch ein Fenster-Knoten ist mit einen speziellen Icon
           gekennzeichnet 
           -- siehe \ref {Visualsization Icon f�r Knoten-Annotationen}).

  \end{precond}

  \begin{postsuccess}
    \cond Die �nderung an der Annotation ist noch nicht
          in der Verwaltungsdatei f�r Knoten-Annotationen eingetragen
          (siehe \ref {Project Verwaltungsdatei f�r Knoten-Annotationen}).
          Dies geschieht erst beim n�chsten Speichern
          des gesamten Projektes.

    \cond Die �nderung der Annotation ist dem System bekannt und wird
          bei der n�chsten Ausf�hrung dieses UseCases angezeigt.
          
  \end{postsuccess}

  \begin{postfail}
    \cond Das System bleibt im bisherigen Zustand.
  \end{postfail}
  
  \begin{proc}
    \step[1] 
    Der Benutzer f�hrt einen Rechtsklick auf den 
    Knoten, dessen Annotation ge�ndert werden soll,
    durch und w�hlt im Popup Men� f�r Knoten den Eintrag
    \gq {Change Node Annotation} aus (siehe \ref{Node-Popup-Men�}).
 
    
    \step[2]
    GIANT zeigt nun den Knoten Annotations-Dialog 
    (siehe \ref{Knoten Annotations-Dialog}). Hier wird der Text
    f�r die bisherige Annotation dargestellt.
    
    \step[3]
    Der Benutzer �ndert den Text f�r die Annotation entsprechend ab.
    Der Benutzer wird aber nicht gezwungen die bestehende Knoten-Annotation 
    zu �ndern.
  
    \step[4]
    Nach Abschluss der Eingabe bet�tigt der Benutzer den \gq{OK Button}.
            
    \step[5]
    GIANT �bernimmt die vorgenommenen �nderungen an der Annotation. 
    Die neue Annotation muss allerdings mindestens ein Zeichen haben, 
    ansonsten erscheint eine Fehlermeldung.

  \end{proc}

  \begin{aproc}
    \ageneral 
    Der Benutzer kann das �ndern der Annotation jeder Zeit 
    mittels des \gq{Cancel Buttons} abbrechen.
    

  \end{aproc}
\end{uc}

% ==============================================================================

\begin{uc}[Knoten-Annotation l�schen]{UC: Knoten-Annotation l�schen}
  L�scht eine bestehende Knoten-Annotationen (f�r weitere Informationen
  zu Knoten-Annotationen siehe
  \ref{Project Persistenz von Knoten-Annotationen}). 
  
  \begin{precond}
    \cond Ein Projekt ist geladen.
    \cond Es gibt mindestens einen annotierten Fenster-Knoten (
          ist mit einen speziellen Icon gekennzeichnet 
           -- siehe \ref {Visualsization Icon f�r Knoten-Annotationen}).
  \end{precond}

  \begin{postsuccess}
    \cond Die Annotation ist noch nicht aus
          der Verwaltungsdatei f�r Knoten-Annotationen entfernt.
          Dies geschieht erst beim n�chsten Speichern
          des gesamten Projektes.

    \cond Das Entfernen der Annotation ist dem System bekannt und wird
          entsprechend angezeigt.
          
  \end{postsuccess}

  \begin{postfail}
    \cond Das System bleibt im bisherigen Zustand.
  \end{postfail}
  
  \begin{proc}
  
    \step[1] 
    Der Benutzer f�hrt einen Rechtsklick auf den Fenster-Knoten,
    dessen Annotation gel�scht werden soll, durch und w�hlt im Popup 
    Men� f�r Knoten den Eintrag
    \gq {Delete Node Annotation} aus (siehe \ref{Node-Popup-Men�}).
          
    \step[2] 
    GIANT l�scht die entsprechende Annotation.

  \end{proc}


\end{uc}

% ==============================================================================

\begin{uc}[Knoten-Annotationen filtern]{UC: Knoten-Annotationen filtern}
  L�scht alle bestehenden Knoten-Annotationen des Projektes
  f�r die es in den dem Projekt bekannten Anzeigefenstern und 
  IML-Teilgraphen keine Fenster-Knoten und auch keine Graph-Knoten gibt.\\
  Dieser Filter soll es erm�glichen, nicht mehr ben�tigte 
  Annotationen automatisch entfernen zu lassen.
  
  \begin{precond}
     \cond Ein Projekt ist geladen.
  \end{precond}

  \begin{postsuccess}
    \cond Die Annotationen sind noch nicht aus
          der Verwaltungsdatei f�r Knoten-Annotationen 
          (siehe \ref {Project Verwaltungsdatei f�r Knoten-Annotationen})
          entfernt. Dies geschieht erst beim n�chsten Speichern
          des gesamten Projektes.

    \end{postsuccess}

    \begin{postfail}
      \cond Das System bleibt im bisherigen Zustand.
    \end{postfail}
  
    \begin{proc}

     \step[1] 
     Der Benutzer w�hlt im Hauptmen� des 
     Hauptfensters den Eintag 
     \gq{Tools -- Delete Annotations having no visible Node}
     (siehe \ref{Main-Window-Tools}) aus.
    
     \step[2]
     GIANT zeigt eine Sicherheitsabfrage und fordert den Benutzer zur 
     Best�tigung auf (siehe \ref{Sicherheitsabfrage}).
    
     \step[3]
     Der Benutzer best�tigt die Sicherheitsabfrage.
           
     \step[4]
     GIANT l�scht alle Annotationen f�r die es keine Fenster-Knoten und
     Graph-Knoten gibt, d.h. jede Knoten-Annotation deren IML-Knoten weder in 
     einem Anzeigefenster als Fenster-Knoten visualisiert noch 
     als Graph-Knoten Bestandteil eines IML-Teilgraphen
     ist, wird entfernt.

   \end{proc}

   \begin{aproc}

     \astep{2} Der Vorgang kann nat�rlich abgebrochen werden, indem die
     Sicherheitsabfrage nicht best�tigt wird.

   \end{aproc} 
\end{uc}

%%% Local Variables: 
%%% TeX-master: "../spec"
%%% End: 


\chapter{Selektionen}
Hier Text zu Selektionen
% ==============================================================================
%  $RCSfile: selection.tex,v $, $Revision: 1.22 $
%  $Date: 2003/04/20 21:40:36 $
%  $Author: schwiemn $
%
%  Description: UseCases f�r die Selektionen
%
%  Last-Ispelled-Revision: 1.11
%
% ==============================================================================


\begin{uc}[Selektion zur aktuellen Selektion machen]
          {UC: Selektion zur aktuellen Selektion machen}
          
Dieser UseCase dient dazu, eine Selektion zur aktuellen Selektion zu machen.
Dies ist n�tig, da nur die aktuelle Selektion mittels der Maus
modifiziert werden kann (siehe auch \ref{Aktuelle Selektion vs Selektionen}). 
          
         
  \begin{precond}
    \cond Es gibt mindestens ein Anzeigefenster 
          mit mindestens zwei Selektionen.

  \end{precond}

  \begin{postsuccess}
    \cond Die vorherige aktuelle Selektion ist nicht mehr aktuell.
    \cond Die gew�hlte Selektion ist nun die aktuelle Selektion und
          als solche in der Selektionsauswahlliste
          (siehe \ref {Selektionsauswahlliste}) gekennzeichnet.
    \cond Die Fenster-Knoten und Fenster-Kanten der jetzt 
          aktuellen Selektion sind entsprechend hervorgehoben (siehe   
          \ref{Hervorheben von Selektionen und der aktuelle Selektion}).

    
  \end{postsuccess}
 
  \begin{proc}    
    \step[1]
    Der Benutzer f�hrt mit der linken Maustaste einen Doppelklick 
    auf eine nicht aktuelle Selektion in der Selektionsauswahlliste
    (siehe \ref{Selektionsauswahlliste}) durch.
    Ausgeblendete Selektionen (siehe UseCase \ref{Selektionen ausblenden})
    k�nnen nicht zur aktuellen Selektion gemacht werden.

    \step[2]
    GIANT macht die entsprechende Selektion zur aktuellen Selektion.  
  \end{proc}
          
\end{uc}


% ==============================================================================
% \begin{uc}[Aktuelle Selektion zur�ck stufen]
%           {UC: Aktuelle Selektion zur�ck stufen}

%
% Gestrichen nach Beschluss vom Meeting am 2003-04-04.
%
          
% Mit diesem UseCase kann
% eine aktuelle Selektion auf den Status einer \gq{normalen} Selektion
% zur�ckgestuft werden (siehe auch \ref{Aktuelle Selektion vs Selektionen}). 
         
%   \begin{precond}
%     \cond Es gibt ein Anzeigefenster mit einer aktuellen Selektion.
%   \end{precond}

%   \begin{postsuccess}
%     \cond Es gibt keine aktuelle Selektion mehr.
    
%   \end{postsuccess}
 
%   \begin{proc}    
%     \step[1]
%     Der Benutzer f�hrt mit der linken Maustaste einen Doppelklick 
%     auf die aktuelle Selektion in der Selektionsauswahlliste
%     (siehe \ref{Selektionsauswahlliste}) durch..

%     \step[2]
%     GIANT stuft die aktuelle Selektion auf den Status einer \gq{normalen}
%     Selektion zur�ck.
%   \end{proc}
    
% \end{uc}


% ==============================================================================
\begin{uc}[Selektion graphisch hervorheben]
          {UC: Selektion graphisch hervorheben}
          
Dieser UseCase dient zum Hervorheben von Selektionen innerhalb eines
Anzeigefensters (siehe auch \ref {hervorheben von Knoten und Kanten}).

  \begin{precond}
    \cond Es gibt ein Anzeigefenster mit mindestens einer Selektion.
   
  \end{precond}

  \begin{postsuccess}
    \cond 
    Die Selektion ist im entsprechenden Anzeigefenster hervorgehoben.
    
    \cond
    War bereits eine andere Selektion mit der vom Benutzer gew�hlten
    Farbe hervorgehoben (die Farbe, welche im Popup-Men� der
    Selektionsauswahlliste unter \gq{Highlight Selection} gew�hlt wurde 
    -- siehe \ref{Selektionsauswahlliste}), so ist diese Selektion nicht
    mehr hervorgehoben.
 
    
  \end{postsuccess}
 
  \begin{proc}    
    \step[1]
    Der Benutzer startet den UseCase mit einem Rechtsklick 
    auf die gew�nschte Selektion in der
    Selektionsauswahlliste (siehe \ref{Selektionsauswahlliste})
    und w�hlt im zugeh�rigen Popup-Men� das Untermen�
    \gq{Highlight Selection} und dort die gew�nschte Farbe aus.
  
    \step[2]
    GIANT hebt die Selektion mit der ausgew�hlten Farbe hervor.
    
  \end{proc}



\end{uc}


% ==============================================================================
\begin{uc}[Graphische Hervorhebung einer Selektion aufheben]
      {UC: Graphische Hervorhebung einer Selektion aufheben}
      
Dieser UseCase dient dazu, die Hervorhebung von Selektionen innerhalb eines 
Anzeigefensters aufzuheben.
      
  \begin{precond}
    \cond 
    Es gibt ein Anzeigefenster mit mindestens einer hervorgehobenen
    Selektion.
   
  \end{precond}

  \begin{postsuccess}
    \cond 
    Die Selektion ist im entsprechenden Anzeigefenster nicht mehr 
    hervorgehoben.
 
  \end{postsuccess}

  
  \begin{proc}    
    \step[1]    
    Der Benutzer startet den UseCase mit einem Rechtsklick 
    auf die gew�nschte Selektion in der
    Selektionsauswahlliste (siehe \ref{Selektionsauswahlliste})
    und w�hlt im zugeh�rigen Popup-Men� den Eintrag
    \gq{Unhighlight Selection} aus.
  
    \step[2]
    GIANT setzt die Hervorhebung der Selektion zur�ck.
    
  \end{proc} 
      
      
\end{uc}

%===============================================================================
\begin{uc}[Neue Selektion anlegen]{UC: Neue Selektion anlegen}

Mit diesem UseCase k�nnen neue, leere Selektionen angelegt werden.

  \begin{precond}
    \cond Es gibt mindestens ein ge�ffnetes Anzeigefenster.
  \end{precond}

  \begin{postsuccess}
    \cond Eine neue Selektion mit dem vorgegebenen Namen
         ist angelegt und erscheint in der Liste der Selektionen 
         (siehe \ref{Selektionsauswahlliste}).
    \cond Diese neue Selektion hat keinen Inhalt (keine selektierten 
          Fenster-Knoten und Fenster-Kanten).

  \end{postsuccess}

  \begin{postfail}
    \cond Das System bleibt im bisherigen Zustand.
  \end{postfail}
  
  \begin{proc}    
    \step[1]
    Der Benutzer startet den UseCase mit Rechtsklick auf die
    Selektionsauswahlliste (siehe \ref{Selektionsauswahlliste})
    und w�hlt im Popup-Men� den Eintrag
    \gq{New Selection} aus.
    
    \step[2] 
    GIANT �ffnet den allgemeinen Texteingabedialog 
    (siehe \ref{DIALOG-WINDOW}).
      
    \step[3] 
    Der Benutzer gibt dort einen zul�ssigen Namen f�r die neue Selektion 
    ein und best�tigt mit OK (siehe \ref{afa Zulaessige Namen}). \\
    Hat bereits eine andere Selektion innerhalb des Anzeigefensters den selben
    Namen, erscheint eine Fehlermeldung.
    
    \step[4]
    GIANT erzeugt die neue Selektion.
  
  \end{proc}

  \begin{aproc}
    \astep{3} Der Benutzer bricht die Verarbeitung mit Cancel ab.
  \end{aproc}

\end{uc}


%===============================================================================
\begin{uc}[Selektion kopieren]{UC: Selektion kopieren}
Dieser UseCase dient zum Kopieren von Selektionen innerhalb eines 
Anzeigefensters.

  \begin{precond}
    \cond Es gibt mindestens ein ge�ffnetes Anzeigefenster mit 
          mindestens einer Selektion.
  \end{precond}

  \begin{postsuccess}
    \cond 
    Eine neue Selektion mit entsprechendem Namen ist angelegt 
    und erscheint in der Liste der Selektionen
    (siehe \ref{Selektionsauswahlliste}).
    \cond
    Die neue Selektion umfasst die selben Fenster-Knoten und Fenster-Kanten 
    wie die Selektion, von der kopiert wurde.

  \end{postsuccess}

  \begin{postfail}
    \cond Das System bleibt im bisherigen Zustand.
  \end{postfail}
  
  \begin{proc}    
    \step[1]
    Der Benutzer startet den UseCase durch Rechtsklick auf die zu kopierende
    Selektion in der Selektionsauswahlliste (siehe \ref{Selektionsauswahlliste})
    und w�hlt aus dem Popup-Men� den Eintrag \gq{Copy Selection}.
    
    \step[2] 
    GIANT �ffnet den allgemeinen Texteingabedialog 
    (siehe \ref{DIALOG-WINDOW}).
      
    \step[3] 
    Der Benutzer gibt dort einen zul�ssigen Namen f�r die neue Selektion 
    ein und best�tigt mit OK (siehe \ref{afa Zulaessige Namen}).\\
    Hat bereits eine andere Selektion innerhalb des Anzeigefensters den selben
    Namen erscheint eine Fehlermeldung.
    
    \step[4]
    GIANT kopiert die Quellselektion und legt eine neue Selektion an.
  
  \end{proc}

  \begin{aproc}
    \astep{3} Der Benutzer bricht die Verarbeitung mit Cancel ab.
  \end{aproc}

\end{uc}



%===============================================================================
\begin{uc}[Selektion l�schen]{UC: Selektion l�schen}

Dieser UseCase dient zum L�schen von Selektionen innerhalb eines 
Anzeigefensters. Hierdurch bleiben die Fenster-Knoten und Fenster-Kanten
unver�ndert. Die Standard-Selektion (siehe \ref{Standard-Selektion}) kann
nicht gel�scht werden.

  \begin{precond}
     \cond Es gibt mindestens ein ge�ffnetes Anzeigefenster mit 
           mindestens zwei Selektionen.
   \end{precond}


  \begin{postsuccess}
    \cond 
    Die entsprechende Selektion ist gel�scht.

    \cond 
    Wurde die aktuelle Selektion gel�scht (siehe 
    \ref{Aktuelle Selektion vs Selektionen}), so wird die
    Standard-Selektion zur aktuellen Selektion.
    
    \cond 
    Die Fenster-Knoten und Fenster-Kanten, 
    die zur gel�schten Selektion geh�ren, werden nicht gel�scht.
          
    \cond War die Selektion hervorgehoben, so wird die
    Hervorhebung der Fenster-Knoten und Fenster-Kanten aufgehoben.

    \cond
    Waren Fenster-Knoten und Fenster-Kanten der gel�schten Selektion
    ausgeblendet (siehe UseCase \ref{Selektionen ausblenden}), so
    werden diese Fenster-Knoten und Fenster-Kanten wieder eingeblendet.
    
  \end{postsuccess}

  \begin{postfail}
    \cond Das System bleibt im bisherigen Zustand.
  \end{postfail}
  
  \begin{proc}    
    \step[1]
    Der Benutzer startet den UseCase durch Rechtsklick auf die zu l�schende
    Selektion in der Selektionsauswahlliste (siehe 
    \ref{Selektionsauswahlliste})
    und w�hlt aus dem Popup-Men� den Eintrag \gq{Delete Selection}
    aus. Falls der Benutzer die Standard-Selektion 
    (siehe \ref{Standard-Selektion}) ausgew�hlt hat, ist
    der Eintrag \gq{Delete Selection} deaktiviert.
      
    \step[2]
    GIANT l�scht die entsprechende Selektion.
  
  \end{proc}

\end{uc}

%===============================================================================
\begin{uc}[Selektion manuell modifizieren]
         {UC: Selektionen manuell modifizieren}
         
Jeweils die aktuelle Selektion kann mittels der Maus modifiziert 
werden (siehe auch  \ref{Aktuelle Selektion vs Selektionen}).  


  \begin{precond}
     \cond 
     Es gibt mindestens ein ge�ffnetes Anzeigefenster mit mindestens
     einer Selektion.
   \end{precond}


  \begin{postsuccess}
    \cond 
    Die entsprechenden �nderungen an der Selektion werden von GIANT
    sofort durchgef�hrt und �bernommen.
    
  \end{postsuccess}

  
  \begin{proc}    
    \step[1]
    Falls noch nicht der Fall, macht der Benutzer die zu modifizierende 
    Selektion zur aktuellen Selektion
    (siehe \ref{Selektion zur aktuellen Selektion machen}).
    
    \step[2]
    Mittels der unter 
    \ref{Selektieren von Fenster-Knoten und Fenster-Kanten in Anzeigefenstern}
    beschriebenen M�glichkeiten f�gt der Benutzer der Selektion neue
    Fenster-Knoten und Fenster-Kanten hinzu oder entfernt bestehende
    Fenster-Knoten und Fenster-Kanten aus der Selektion.
    
  \end{proc}


\end{uc}

%===============================================================================
\begin{uc}[Selektion aus IML-Teilgraph erzeugen]
         {UC: Selektion aus IML-Teilgraph erzeugen}
         
Leitet eine Selektion aus einem IML-Teilgraphen ab
(siehe \ref{Selektion aus IML-Teilgraphen ableiten}).


  \begin{precond}
     \cond 
     Es gibt mindestens ein ge�ffnetes Anzeigefenster.
     
     \cond
     Es gibt mindestens einen IML-Teilgraphen.
     
   \end{precond}


  \begin{postsuccess}
    \cond 
    Im Ziel-Anzeigefenster wurde eine neue Selektion
    mit dem entsprechenden Namen erzeugt.
    
  \end{postsuccess}
  
  \begin{postfail}
    \cond Das System bleibt im bisherigen Zustand.
  \end{postfail}
  
  \begin{proc} 
     
    \step[1]
    Der Benutzer f�hrt einen Rechtsklick auf den Quell-IML-Teilgraphen 
    in der entsprechenden Liste im Hauptfenster 
    (siehe \ref{GUI Subgraph List}) aus
    und w�hlt im dazugeh�rigen Popup-Men� 
    (siehe \ref{Popup-Men� Subgraph List})
    den Eintrag \gq{Create Window Selection} aus.
        
    \step[2]
    Die Statuszeile im Hauptfenster zeigt an \gq{Please select window for
    Insertion of new window selection}, der Mauszeiger verwandelt sich in
    ein Fadenkreuz.
    
    \step[3]
    Der Nutzer klickt auf den sichtbaren Anzeigeinhalt des
    Anzeigefensters, in dem er die neue
    Selektion erstellen will (Ziel-Anzeigefenster).


    \step[4]
    GIANT zeigt den allgemeinen Texteingabedialog (siehe \ref{DIALOG-WINDOW}).
            
    \step[5] Der Benutzer gibt einen Namen (siehe auch 
             \ref{afa Zulaessige Namen}) 
             f�r die neu zu erstellende Selektion ein und
             best�tigt mit OK.

    \step[6]
    Die Statuszeile im Hauptfenster wechselt wieder zur normalen Anzeige.
    GIANT erzeugt gem�� der unter Abschnitt 
    \ref{Selektion aus IML-Teilgraphen ableiten}
    beschriebenen Konvention im Ziel-Anzeigefenster eine neue Selektion als 
    Ableitung aus dem Quell-IML-Teilgraphen.

  \end{proc}
  
  \begin{aproc}    

    \astep{3} Der Benutzer bricht den UseCase durch einen Rechtsklick ab.
    \astep{5} Der Benutzer bricht den UseCase mit Cancel ab.

  \end{aproc}


\end{uc}



%===============================================================================
\begin{uc}[Mengenoperationen auf 2 Selektionen]
{UC: Mengenoperationen auf 2 Selektionen}
Zus�tzlich zu den M�glichkeiten der Anfragesprache GSL (siehe
\ref {GIANT Scripting Language}) kann der Benutzer
die g�ngigen Mengenoperationen, wie Mengenvereinigung, Schnitt und Differenz,
auch direkt �ber einen entsprechenden Dialog 
(siehe \ref{Common-Set-Operation-Dialog})  ausf�hren.
Das Vorgehen ist analog zu dem UseCase
\ref{Mengenoperationen auf 2 IML-Teilgraphen}.



  \begin{precond}
    \cond Es gibt ein ge�ffnetes Anzeigefenster mit mindestens zwei 
          Selektionen.
  \end{precond}

  \begin{postsuccess}
    \cond 
    Eine neue Selektion mit entsprechendem Namen (eingegeben unter TARGET) 
    ist angelegt 
    und erscheint in der Liste der Selektionen 
    (siehe \ref{Selektionsauswahlliste}).
       
    \cond
    Im Falle einer Mengenvereinigung umfasst, die neue Selektion TARGET alle
    Fenster-Knoten und Fenster-Kanten aus der LEFT\_SOURCE Selektion und der 
    RIGHT\_SOURCE Selektion.
    
    \cond
    Im Falle einer Mengendifferenz umfasst, die neue Selektion TARGET alle
    Fenster-Knoten und Fenster-Kanten aus der LEFT\_SOURCE Selektion, 
    die nicht Bestandteil der  RIGHT\_SOURCE Selektion sind.

    \cond
    Im Falle eines Mengenschnitts umfasst, die neue Selektion TARGET alle
    Fenster-Knoten und Fenster-Kanten, die der LEFT\_SOURCE Selektion und der 
    RIGHT\_SOURCE Selektion gemeinsam angeh�ren.

  \end{postsuccess}

  \begin{postfail}
    \cond Das System bleibt im bisherigen Zustand.
  \end{postfail}
  
  \begin{proc}    
    \step[1]
    Der Benutzer startet den UseCase �ber durch Auswahl des
    Men�punktes \gq{Selection Set Operation} 
    aus dem Popup-Men� in der Selektionsauswahlliste (siehe
    \ref{Selektionsauswahlliste}).
    
    \step[2] 
    GIANT �ffnet den Set-Operation-Dialog 
    (siehe \ref{Common-Set-Operation-Dialog}).
    
    \step[3] 
    Der Benutzer w�hlt dort die beiden Quell-Selektionen (LEFT\_SOURCE und 
    RIGHT\_SOURCE)
    aus, bestimmt die durchzuf�hrende Mengenoperation und gibt unter 
    TARGET den Namen der neu zu erzeugenden Selektion 
    ein (g�ltige Namen siehe \ref{afa Zulaessige Namen}). \\
    Dann best�tigt er die Eingabe mit OK.
       
    \step[4]
    GIANT f�hrt die Mengenoperation aus.
  
  \end{proc}

  \begin{aproc}
    \ageneral 
    Der Benutzer bricht die Eingabe der Daten mit Cancel ab.
  \end{aproc}



\end{uc}



\chapter{Filter}
Hier Text zu Filtern
% ==============================================================================
%  $RCSfile: filter.tex,v $, $Revision: 1.15 $
%  $Date: 2003/04/21 20:54:40 $
%  $Author: schwiemn $
%
%  Description: UseCases f�r Filter
%
%  Last-Ispelled-Revision: 1.5
%
% ==============================================================================

\begin{uc}[Selektionen ausblenden]{UC: Selektionen ausblenden}
Mit diesem UseCase k�nnen Selektionen innerhalb
eines Anzeigefensters ausgeblendet werden.
Die Standard-Selektion (siehe \ref{Standard-Selektion}) und die 
aktuelle Selektion (siehe \ref{Aktuelle Selektion vs Selektionen})
k�nnen nicht ausgeblendet werden.

  \begin{precond}
    \cond Es gibt mindestens ein
          ge�ffnetes Anzeigefenster mit mindestens zwei Selektionen.

  \end{precond}

  \begin{postsuccess}

    \cond Alle zu der Selektion geh�renden Fenster-Knoten und
          Fenster-Kanten sind ausgeblendet, d.h. sie sind
          im Anzeigefenster nicht mehr sichtbar. Dies trifft
          auch f�r Fenster-Knoten und Fenster-Kanten, die
          noch zu weiteren Selektionen geh�ren, zu.
    
    \cond Die ausgeblendeten Selektionen k�nnen nicht 
          zur aktuellen Selektion gemacht werden (siehe 
          \ref {Selektion zur aktuellen Selektion machen}) und damit
          nicht mehr direkt bearbeitet werden, 
          d.h. die Menge der selektierten 
          Fenster-Knoten und Fenster-Kanten kann nicht
          mehr abge�ndert werden (siehe \ref {Selektieren von 
          Fenster-Knoten und Fenster-Kanten in Anzeigefenstern}).


    \cond Die Fenster-Knoten und Fenster-Kanten sind aber immer noch 
          Bestandteil des Anzeigefensters und k�nnen �ber den folgenden
          UseCase (siehe \ref{Selektionen einblenden}) wieder zur Anzeige
          gebracht werden.
    
  \end{postsuccess}

   
  \begin{proc}

    \step[1] 
    Der Benutzer f�hrt einen Rechtsklick mit der Maus auf
    die auszublendende Selektion in der Selektionsauswahlliste 
    (siehe \ref{Selektionsauswahlliste}) durch 
    und w�hlt im Popup-Men� den Eintrag \gq{Hide Selection}
    aus. Diese Funktionalit�t kann nicht auf die aktuelle
    Selektion oder auf die Standard-Selektion angewendet werden. 


    \step[2]
    GIANT blendet die Selektion aus.
 
  \end{proc}

\end{uc}



\begin{uc}[Selektionen einblenden]{UC: Selektionen einblenden}
Mit diesem UseCase k�nnen ausgeblendete Selektionen
wieder eingeblendet werden.

  \begin{precond}
    \cond Es gibt mindestens ein ge�ffnetes Anzeigefenster mit 
          mindestens einer ausgeblendeten Selektion.
  \end{precond}

  \begin{postsuccess}

    \cond 
    Die Selektion ist wieder eingeblendet, alle zu ihr geh�renden 
    Fenster-Knoten und Fenster-Kanten sind im Anzeigeinhalt 
    sichtbar dargestellt (auch wenn sie noch zu weiteren
    Selektionen geh�ren, die ausgeblendet sind).
    
  \end{postsuccess}

   
  \begin{proc}

    \step[1] 
    Der Benutzer f�hrt einen Rechtsklick mit der Maus auf
    eine ausgeblendete Selektion in der Selektionsauswahlliste 
    (siehe \ref{Selektionsauswahlliste}) durch 
    und w�hlt im Popup-Men� den Eintrag \gq{Show Selection}
    aus. 

    \step[2]
    GIANT blendet die Selektion ein.
 
  \end{proc}

\end{uc}


\begin{uc}[Alles einblenden]{UC: Alles einblenden}
Mit diesem UseCase k�nnen alle ausgeblendeten Fenster-Knoten
und Fenster-Kanten eines Anzeigefensters wieder eingeblendet werden.

  \begin{precond}
    \cond Es gibt mindestens ein ge�ffnetes Anzeigefenster mit 
          mindestens einer ausgeblendeten Selektion.

  \end{precond}

  \begin{postsuccess}

    \cond 
    Alle Selektionen des Anzeigefensters sind wieder eingeblendet.

    \cond 
    Alle Fenster-Knoten und Fenster-Kanten sind wieder
    im Anzeigeinhalt des Anzeigefensters sichtbar dargestellt.
    
  \end{postsuccess}

   
  \begin{proc}

    \step[1] 
    Der Benutzer f�hrt einen Rechtsklick mit der Maus auf
    die Selektionsauswahlliste 
    (siehe \ref{Selektionsauswahlliste}) durch 
    und w�hlt im Popup-Men� den Eintrag \gq{Show all}
    aus. 

    \step[2]
    GIANT blendet alle ausgeblendeten Fenster-Knoten und Fenster-Kanten
    wieder ein.
 
  \end{proc}

\end{uc}



\chapter{Teilgraphen}
Hier Text zu Teilgraphen
% ==============================================================================
%  $RCSfile: subgraph.tex,v $, $Revision: 1.17 $
%  $Date: 2003/04/17 22:44:59 $
%  $Author: schwiemn $
%
%  Description: UseCases f�r IML-Teilgraphen
%
%  Last-Ispelled-Revision: 1.10
%
% ==============================================================================

\begin{uc}[IML-Teilgraph graphisch hervorheben]
      {UC: IML-Teilgraph graphisch hervorheben}
\index{IML-Teilgraph!hervorheben}         
Dieser UseCase dient zum Hervorheben von IML-Teilgraphen innerhalb der
Anzeigefenster (siehe auch \ref {hervorheben von Knoten und Kanten}).

  \begin{precond}
    \cond Es gibt mindestens einen IML-Teilgraphen.
   
  \end{precond}

  \begin{postsuccess}
    \cond 
    Der IML-Teilgraph ist in jedem ge�ffneten Anzeigefenster entsprechend
    hervorgehoben.
      
    \cond
    Der IML-Teilgraph, welcher vorher mit der gleichen Farbe 
    hervorgehoben war, ist nicht mehr hervorgehoben.
 
    \cond
    Fenster-Knoten und Fenster-Kanten, die als Graph-Knoten bzw. Graph-Kanten
    Bestandteil eines hervorgehobenen IML-Teilgraphen sind, werden,
    auch wenn sie erst nach der Hervorhebung des IML-Teilgraphen in ein
    Anzeigefenster eingef�gt werden, ebenfalls automatisch hervorgehoben.

    
  \end{postsuccess}
 
  \begin{proc}    
    \step[1]
    Der Benutzer startet den UseCase �ber das Popup-Men� der 
    Liste �ber die IML-Teilgraphen (siehe \ref{GUI Subgraph List})
    durch Rechtsklick auf den gew�nschten IML-Teilgraphen.
    In dem Popup Men� (siehe \ref {Popup-Men� Subgraph List})
    w�hlt er das Untermen� \gq {Highlight} und dort die gew�nschte
    Farbe aus.
            
    \step[2]
    GIANT hebt den IML-Teilgraphen in allen ge�ffneten Anzeigefenstern
    mit der ausgew�hlten Farbe hervor.
    
  \end{proc}


\end{uc}



% ==============================================================================
\begin{uc}[Graphische Hervorhebung von IML-Teilgraphen aufheben]
      {UC: Graphische Hervorhebung von IML-Teilgraphen aufheben}
\index{IML-Teilgraph!Hervorhebung aufheben}  
Mit diesem UseCase kann die graphische Hervorhebung von IML-Teilgraphen 
aufgehoben werden.

  \begin{precond}
    \cond 
    Es gibt mindestens einen hervorgehobenen IML-Teilgraphen.
   
  \end{precond}

  \begin{postsuccess}
    \cond 
    Der IML-Teilgraph ist nicht mehr hervorgehoben.

  \end{postsuccess}

  
  \begin{proc}    
    \step[1]
    Der Benutzer startet den UseCase durch Rechtsklick auf den
    hervorgehobenen IML-Teilgraphen in der Liste �ber
    die IML-Teilgraphen (siehe \ref{GUI Subgraph List})
    In dem zugeh�rigen Popup-Men� (siehe \ref{Popup-Men� Subgraph List})
    w�hlt er den Eintrag \gq{Unhighlight In All Windows}
    aus.

    \step[2]
    GIANT setzt die Hervorhebung des IML-Teilgraphen zur�ck.
    
  \end{proc} 


\end{uc}


% ==============================================================================
\begin{uc}[IML-Teilgraph aus einer Selektion erzeugen]
  {UC: IML-Teilgraph aus einer Selektion erzeugen}
  \index{IML-Teilgraph!aus Selektion erzeugen} 
  
  Leitet einen neuen IML-Teilgraphen aus einer Quell-Selektion ab
  (siehe auch Abschnitt \ref{IML-Teilgraphen aus Selektion ableiten}.

  \begin{precond}
     \cond 
     Es gibt mindestens ein ge�ffnetes Anzeigefenster mit 
     mindestens einer Selektion.
     
   \end{precond}


  \begin{postsuccess}
    \cond 
    Es wurde ein neuer IML-Teilgraph mit entsprechendem Namen erzeugt.
    
  \end{postsuccess}
  
  \begin{postfail}
    \cond Das System bleibt im bisherigen Zustand.
  \end{postfail}
  
  \begin{proc} 
     
    \step[1]
    Der Benutzer f�hrt einen Rechtsklick auf die Quell-Selektion in
    der Selektionsauswahlliste (siehe \ref{Selektionsauswahlliste}) aus
    und w�hlt im Popup-Men� den Eintrag 
    \gq{Create New IML Subgraph from This Selection} aus.
     
    \step[2]
    GIANT zeigt den allgemeinen Texteingabedialog (siehe \ref{DIALOG-WINDOW})
    an.
     
    \step[3]
    Der Benutzer gibt einen Namen f�r den neu zu erstellenden IML-Teilgraphen
    ein und best�tigt mit OK (g�ltige Namen f�r IML-Teilgraphen sind 
    unter Abschnitt \ref{afa Zulaessige Namen} spezifiziert).\\
    Gibt der Benutzer hier keinen Namen ein, so vergibt GIANT automatisch 
    einen Namen.
     
    \step[4]
    GIANT erzeugt gem�� der unter Abschnitt \ref{IML-Teilgraphen aus 
    Selektion ableiten}
    beschriebenen Konvention eine neuen IML-Teilgraphen aus der 
    Quell-Selektion.

  \end{proc}
  
  \begin{aproc}
    \astep{3} Der Benutzer bricht den UseCase mit Cancel ab.
  \end{aproc}

\end{uc}



% ==============================================================================
\begin{uc}[IML-Teilgraph kopieren] {UC: IML-Teilgraph kopieren}
\index{IML-Teilgraph!kopieren}      
 Kopiert einen Quell-IML-Teilgraphen in einen neuen IML-Teilgraphen.
 Bestehende IML-Teilgraphen k�nnen nicht �berschrieben werden.

  \begin{precond}
     
     \cond
     Es gibt mindestens einen IML-Teilgraphen.
     
   \end{precond}


  \begin{postsuccess}
    \cond 
    Es wurde ein neuer IML-Teilgraph mit entsprechendem Namen erzeugt.
    
    \cond
    Der neue IML-Teilgraph hat alle Graph-Knoten und Graph-Kanten des
    Quell-IML-Teilgraphen.
    
  \end{postsuccess}
  
  \begin{postfail}
    \cond Das System bleibt im bisherigen Zustand.
  \end{postfail}
  
  \begin{proc} 
     
    \step[1]
    Der Benutzer f�hrt einen Rechtsklick auf den zu kopierenden 
    Quell-IML-Teilgraphen in der Liste �ber die IML-Teilgraphen
    im Hauptfenster aus (siehe \ref{GUI Subgraph List}) und
    w�hlt im zugeh�rigen Popup-Men� (siehe \ref{SUBGRAPH-LIST-POPUP}) 
    den Eintrag \gq{Copy IML Subgraph} aus.
     
    \step[2]
    GIANT zeigt den allgemeinen Texteingabedialog (siehe \ref{DIALOG-WINDOW}).
            
    \step[3]
    Der Benutzer gibt einen Namen f�r den neu zu erstellenden IML-Teilgraphen
    ein und best�tigt mit OK 
    (g�ltige Namen siehe \ref{afa Zulaessige Namen}).\\
    Gibt der Benutzer hier keinen Namen ein, so vergibt GIANT automatisch 
    einen Namen.\\
    Gibt der Benutzer den Namen eines bereits vorhandenen IML-Teilgraphen
    ein, so erscheint eine Fehlermeldung.
       
    \step[4]
    GIANT kopiert den Quell-IML-Teilgraphen in einen neuen Teilgraphen.

  \end{proc}
  
  \begin{aproc}
    \astep{3} Der Benutzer bricht den UseCase mit Cancel ab.
  \end{aproc}

\end{uc}



% ==============================================================================
\begin{uc}[IML-Teilgraph l�schen]{UC: IML-Teilgraph l�schen}
\index{IML-Teilgraph!l�schen}
Dieser UseCase l�scht einen IML-Teilgraphen. Die zugeh�rigen
Fenster-Knoten und Fenster-Kanten in den Anzeigefenstern bleiben
davon unber�hrt.

  \begin{precond}
     
     \cond
     Es gibt mindestens einen IML-Teilgraphen.
     
   \end{precond}


  \begin{postsuccess}
    \cond 
    Der IML-Teilgraph wurde aus der Liste �ber die IML-Teilgraphen
    gel�scht (siehe \ref{GUI Subgraph List}).
    
    \cond
    War der gel�schte IML-Teilgraph hervorgehoben, so wurde die
    Hervorhebung der zugeh�rigen Fenster-Knoten und Fenster-Kanten
    aufgehoben.

  \end{postsuccess}
  

  
  \begin{proc} 
     
    \step[1]
    Der Benutzer f�hrt einen Rechtsklick auf den zu l�schenden
    IML-Teilgraphen aus und w�hlt im Popup-Men� der
    Liste �ber die IML-Teilgraphen den Eintrag 
    \gq{Delete IML Subgraph} aus (siehe \ref {Popup-Men� Subgraph List}).
      
    \step[2]
    GIANT l�scht den gew�hlten IML-Teilgraphen.

  \end{proc}

\end{uc}


% ==============================================================================
\begin{uc}[Mengenoperationen auf 2 IML-Teilgraphen]
      {UC: Mengenoperationen auf 2 IML-Teilgraphen}
\index{IML-Teilgraph!Mengenoperationen}

Erg�nzend zu den M�glichkeiten der Anfragesprache 
(siehe Kapitel \ref{GIANT Query Skripting Language}) kann der Benutzer
die g�ngigen Mengenoperationen, wie Mengenvereinigung, 
Schnitt und Differenz, auch direkt �ber einen Dialog 
ausf�hren.
F�r eine genaue Beschreibung des Dialoges siehe auch
Abschnitt \ref{Common-Set-Operation-Dialog}.
Das Vorgehen innerhalb dieses UseCases ist im Wesentlichen 
analog zu dem UseCase \ref{Mengenoperationen auf 2 Selektionen}.


  \begin{precond}
    \cond Es gibt mindestens zwei IML-Teilgraphen.
  \end{precond}

  \begin{postsuccess}
    \cond 
    Eine neuer IML-Teilgraph mit entsprechendem Namen 
    (im Dialog eingegeben unter TARGET) ist angelegt 
    und erscheint in der Liste �ber alle IML-Teilgraphen des Projektes
    (siehe \ref{GUI Subgraph List}).
       
    \cond
    Im Falle einer Mengenvereinigung umfasst
    der neue IML-Teilgraph TARGET alle
    Graph-Knoten und Garph-Kanten aus dem LEFT\_SOURCE IML-Teilgraphen und dem 
    RIGHT\_SOURCE IML-Teilgraphen.
    
    \cond
    Im Falle einer Mengendifferenz umfasst der neue IML-Teilgraph TARGET alle
    Graph-Knoten und Graph-Kanten aus dem LEFT\_SOURCE IML-Teilgraphen, 
    die nicht 
    Bestandteil des RIGHT\_SOURCE IML-Teilgraphen sind.\\

    \cond
    Im Falle eines Mengenschnitts umfasst der neue IML-Teilgraph TARGET alle
    Graph-Knoten und Graph-Kanten, 
    die sowohl dem LEFT\_SOURCE IML-Teilgraphen als auch 
    dem RIGHT\_SOURCE IML-Teilgraphen angeh�ren.

    \cond
    H�tte der IML-Teilgraph TARGET nach der Mengenoperation Graph-Kanten 
    ohne zugeh�rigen Start- und Zielknoten, so
    sind diese Graph-Kanten aus dem IML-Teilgraphen TARGET entfernt worden
    (dieser Fall kann bei Mengenschnitt und Mengendifferenz eintreten).

  \end{postsuccess}

  \begin{postfail}
    \cond Das System bleibt im bisherigen Zustand.
  \end{postfail}
  
  \begin{proc}    
    \step[1]
    Der Benutzer startet den UseCase �ber das Popup-Men� 
    der Liste �ber die IML-Teilgraphen  
    (siehe \ref {Popup-Men� Subgraph List}) indem er
    einen Rechtsklick innerhalb der Liste ausf�hrt und  
    im Popup Men� den Eintrag \gq{IML Subgraph Set Operation}
    ausw�hlt.
    
    \step[2] 
    GIANT �ffnet den Set-Operation-Dialog 
    (siehe \ref{Common-Set-Operation-Dialog}).
    
    
    \step[3] 
    Der Benutzer w�hlt dort die beiden Quell-IML-Teilgraphen 
    (LEFT\_SOURCE und RIGHT\_SOURCE) aus, bestimmt die auszuf�hrende
    Mengenoperation und gibt unter TARGET den Namen 
    des neu zu erzeugenden IML-Teilgraphen ein (g�ltige
    Namen f�r IML-Teilgraphen siehe \ref{afa Zulaessige Namen}).\\
    Er best�tigt mit OK.\\
    Existiert bereits ein IML-Teilgraph mit dem unter TARGET eingegebenen 
    Namen, so erscheint eine Fehlermeldung.
       
    \step[4]
    GIANT f�hrt die Mengenoperation aus und erzeugt den neuen
    IML-Teilgraphen.
  
  \end{proc}

  \begin{aproc}
    \astep{3} Der Benutzer bricht die Eingabe der Daten mit Cancel ab.
  \end{aproc}


\end{uc}



\chapter{Anfragen}
Hier Text zu Anfragen
% ==============================================================================
%  $RCSfile: query.tex,v $, $Revision: 1.18 $
%  $Date: 2003/04/19 18:46:33 $
%  $Author: birdy $
%
%  Description: UseCases f�r die Anfragen
%
%  Last-Ispelled-Revision: 1.7
%
% ==============================================================================

% ==============================================================================
\begin{uc}[Anfrage ausf�hren]{UC: Neues Skript ausf�hren}
\index{Skript!ausf�hren}
Mit diesem UseCase kann ein Skript �ber den Skriptdialog
(siehe \ref{GUI Anfragedialog}) eingegeben werden.
Die M�glichkeiten der GSL sind im Detail in Kapitel
\ref {GIANT Scripting Language} beschrieben.


  \begin{precond}
    \cond Ein Projekt ist geladen.
  \end{precond}

  \begin{postsuccess}
    \cond Das Skript wurde ausgef�hrt. Alle Ergebnisse liegen vor.
      
  \end{postsuccess}

  \begin{postfail}
    \cond Wurde der UseCase vor Beginn der Berechnung des Skriptes
          abgebrochen, bleibt das System im bisherigen Zustand.

    \cond Wurde der UseCase w�hrend der Ausf�hrung des Skriptes 
          abgebrochen, so werden bereits ausgef�hrte Aktionen
          (wie z.B. das Einf�gen neuer Fenster-Knoten in Anzeigefenster)
          nicht wieder r�ckg�ngig gemacht.

  \end{postfail}
  
  \begin{proc}    
    \step[1]
    Der Benutzer startet den UseCase durch Auswahl des Eintrags 
    \gq{Execute GSL Script} im Men� Tools
    (siehe \ref{Main-Window-Tools}).
      
    \step[2] 
    GIANT �ffnet den Skriptdialog (siehe \ref{GUI Anfragedialog}).
      
    \step[3]
    Der Benutzer gibt dort im daf�r vorgesehenen Textfeld das GSL Skript
    (siehe auch Kapitel \ref{GIANT Scripting Language}) ein und 
    best�tigt mit \gq {OK}.

    \step[4]
    GIANT pr�ft das eingegebene GSL Skript. Sollte das Skript
    nicht den Vorgaben der Grammatik 
    (siehe Kapitel \ref{GIANT Scripting Language}) entsprechen, erscheint
    eine Fehlermeldung (siehe \ref{afa Fehlerverhalten})
    und das System kehrt zu Schritt 3 des UseCase zur�ck.
        
    \step[5]    
    GIANT f�hrt das Skript aus und teilt dem Benutzer
    den Fortschritt mittels eines Progress-Dialogs 
    (siehe \ref{Progressbar-Modale}) mit.
    W�hrend der Ausf�hrung des Skripts ist das System mit Ausnahme des
    Progress-Dialogs gesperrt.\\
    
  
  \end{proc}

  \begin{aproc}
    \astep{3} Der Benutzer bricht mit Cancel ab.
    
    \astep{5} Die laufende Ausf�hrung des Skripts kann vom Benutzer
    mit Cancel abgebrochen werden.

  \end{aproc}

\end{uc}


% ==============================================================================
\begin{uc}[UC Anfrage laden]{UC: Skript laden}
\index{Anfragen!aus Datei laden}
Der Benutzer kann zus�tzlich zur manuellen Eingabe von GSL Skripten
(siehe \ref{Anfrage ausf�hren}) auch gespeicherte Skripte aus 
einer Datei (siehe \ref {Config Anfrage-Dateien}) laden.

  \begin{precond}
    \cond Ein Projekt ist geladen.
  \end{precond}

  \begin{postsuccess}
    \cond Das Skript wurde ausgef�hrt. Alle Ergebnisse liegen vor.
  \end{postsuccess}

  \begin{postfail}

     \cond Wurde der UseCase vor Beginn der Berechnung des Skriptes
          abgebrochen, bleibt das System im bisherigen Zustand.

    \cond Wurde der UseCase w�hrend der Ausf�hrung des Skriptes 
          abgebrochen, so werden bereits ausgef�hrte Aktionen
          (wie z.B. das Einf�gen neuer Fenster-Knoten in Anzeigefenster)
          nicht wieder r�ckg�ngig gemacht.

  \end{postfail}
  
  \begin{proc}    
    \step[1]
    Der Benutzer startet den UseCase durch Auswahl des Eintrags 
    \gq{Execute GSL Script} im Men� Tools
    (siehe \ref{Main-Window-Tools}).
      
    \step[2] 
    GIANT �ffnet den Skriptdialog (siehe \ref{GUI Anfragedialog}).
      
    \step[3] 
    Der Benutzer bet�tigt im Dialog den Button \gq{Open...}.

    \step[4]
    Daraufhin zeigt GIANT den Standard-Filechooser-Dialog 
    (siehe \ref {Standard-Filechooser-Dialog}).
    
    \step[5] Der Benutzer w�hlt die Datei (siehe \ref {Config
      Anfrage-Dateien}) aus.
        
    \step[6]
    GIANT zeigt das aus der Datei geladene GSL Skript 
    (siehe Kapitel \ref {GIANT Scripting Language}) im Textfeld
    des Skriptdialogs an.

    \step[7]
    Falls gew�nscht kann der Benutzer das GSL Skript im Textfeld
    noch manuell weiter modifizieren.
    
    \step[8]
    Der Benutzer startet die Berechnung des Skripts durch Bet�tigung
    des \gq{Start Query} im Skriptdialog (siehe \ref{GUI Anfragedialog}).

    \step[9]
    GIANT pr�ft das geladene und eventuell modifizierte GSL Skript. 
    Sollte das Skript nicht den Vorgaben der Grammatik 
    (siehe Kapitel \ref {GIANT Scripting Language}) entsprechen, erscheint
    eine Fehlermeldung (siehe \ref {afa Fehlerverhalten})
    und das System kehrt zu Schritt 7 des UseCase zur�ck.

    \step[10]    
    GIANT f�hrt das Skript aus und teilt dem Benutzer
    den Fortschritt mittels eines Progress-Dialogs 
    (siehe \ref{Progressbar-Modale}) mit.
    W�hrend der Ausf�hrung des Skripts ist das System mit Ausnahme des
    Progress-Dialogs gesperrt.

  \end{proc}

  \begin{aproc}
    \astep{3} Der Benutzer bricht den UseCase mit Cancel ab.
    \astep{4} Der Benutzer bricht die Auswahl der Datei mit Cancel ab.
              Das System kehrt dann zu Schritt 2 bei der
              Abarbeitung des UseCase zur�ck.
    \astep{6} Der Benutzer bricht den UseCase mit Cancel ab.

    \astep{10} Die laufende Ausf�hrung des Skripts kann vom Benutzer
    mit Cancel abgebrochen werden.
  \end{aproc}

\end{uc}

% ==============================================================================
\begin{uc}[Anfrage speichern]{UC: Skript speichern}
\index{Anfragen!in eine Datei speichern}
Mit diesem UseCase kann der Benutzer GSL Skripte aus dem 
Skriptdialog (siehe \ref{GUI Anfragedialog}) in Dateien 
(siehe \ref {Config Anfrage-Dateien}) speichern.

  \begin{precond}
    \cond Der Skriptdialog (siehe \ref{GUI Anfragedialog}) ist
          ge�ffnet und enth�lt in dem daf�r vorgesehenen Textfeld
          entweder ein manuell eingegebenes oder ein aus einer
          Datei geladenes und eventuell modifiziertes GSL Skript.

  \end{precond}

  \begin{postsuccess}
    \cond Eine Datei, welche das GSL Skript enth�lt, wurde
          angelegt.
    \cond GIANT zeigt den Skriptdialog, das gespeicherte GSL Skript
          ist weiterhin in dem Textfeld vorhanden.
      
  \end{postsuccess}

  \begin{postfail}
    \cond Das System bleibt im bisherigen Zustand.
    \cond Es wurde keine Datei erzeugt
    \cond GIANT zeigt weiterhin den Skriptdialog an
          (siehe \ref{GUI Anfragedialog}) und alle dort get�tigten
          Eingaben (insbesondere das GSL Skript im Textfeld des
          Dialoges) bleiben erhalten.
   
  \end{postfail}
  
  \begin{proc}    

    \step[1]
    Der Benutzer bet�tigt im Skriptdialog (siehe \ref{GUI Anfragedialog})
    den Button \gq{Save As...}.
    
    \step[2]
    GIANT pr�ft ob das GSL Skript im Textfeld des Skriptdialogs den
    Vorgaben der Grammatik (siehe Kapitel \ref {GIANT Scripting Language}) 
    entspricht. Falls nicht, erscheint eine Fehlermeldung 
    (siehe \ref {afa Fehlerverhalten})
    und das System kehrt zu Schritt 1 zur�ck.

    \step[3]
    GIANT �ffnet den Standard-Filechooser-Dialog (siehe 
    \ref {Standard-Filechooser-Dialog}).
  
    \step[4]
    Der Benutzer gibt den Pfad und die Datei, in der das GSL Skript
    gespeichert werden soll, vor und best�tigt mit OK.

    \step[5]
    GIANT speichert das GSL Skript in der vorgegebenen Datei.

  \end{proc}

  \begin{aproc}
    \astep{4} Der Benutzer bricht den UseCase mit Cancel ab.

  \end{aproc}

\end{uc}


