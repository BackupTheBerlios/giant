% ==============================================================================
%  $RCSfile: spec.tex,v $, $Revision: 1.14 $
%  $Date: 2003/02/13 04:23:25 $
%  $Author: schulzgt $
%
%  Description:
%
% ==============================================================================

\documentclass[a4paper,titlepage,11pt,german,twoside]{scrbook}
               
\usepackage{../styles/common}
\usepackage{spec}

\makeindex

\begin{document}

\fontfamily{cmss}\selectfont

% title page
\thispagestyle{empty}
\hfill
\parbox{5cm}{Universit�t Stuttgart \\ Studienprojekt A IML Browser}

\vspace{5cm}

\begin{center}
  \Huge
  \textsf{Spezifikation}
  \vspace{1cm}\Large\today
\end{center}
\newpage


%===============================================================================
%
% ToDo - Aus fertigem Dokument entferenen
%
\chapter{ToDo -- nicht in fertiger SPez.}
Hier Sachen beschreiben, die irgenwie beim Erstellen der Spez.\ beachtet werden sollen.
% ==============================================================================
%  $RCSfile: todo_spec.tex,v $ 
%  $Date: 2003/02/04 16:59:34 $
%  $Author: schwiemn $ 
%
%  Description: ToDo File f�r die Spezifikation
%
% ==============================================================================

%===============================================================================
%
% Anforderungen des Kunden
%
\section {Anforderungen des Kunden}
Anbei die Anforderungen des Kunden gem�� der Email vom 30.01.2003

\begin{itemize}

\item Form: DIN A4, gen�gend Platz f�r Korrektur und handschriftliche 
Anmerkungen


\item Gedruckte Dokumentation + PDF

\item Versionierung: Jedes Dokument erh�lt eine Revisionsnummer, 
Erstellungsdatum wird vermerkt

\item Inhaltliche Form: Jedes Dokument verf�gt �ber ein Inhaltsverzeichnis,
Index ist erw�nscht, Seitenzahlen


\item Referenzen: Konsistente globale Nummerierung f�r alle Dokumente
(lebenslang und f�r immer und ewig)

\item Querverweise (nat�rlich innerhalb des selben Dokuments), 
Referenzen auf weitere Dokumente �ber Seite + Abschnitt

\item Inhalt: Begriffslexikon, Konzepte der GUI, 
Erkl�rungen der GUI zum Beispiel anhand abstrakter Bilder,
Usecases mit Zielsetzung spezifizieren

\item vollst�ndige Abdeckung der Anforderungen an das Produkt

\item pr�zise, knapp

\item Traceability: \\
Spezifikation ist der erste Schritt auf dem langen
(und beschwerlichen) Weg zum fertigen Software-Produkt und
dient als Grundlage f�r das weitere Vorgehen.\\
Entwurf, Implementierung und Test referenzieren sie.
Testf�lle f�r den Systemtest sollen aus der Spezifikation
ableitbar sein.


\item Sprache:\\
Wenn Deutsch als Sprache gew�hlt wird, sollten englische Fach-Begriffe
verwendet werden. (Die selben Begriffe werden in Spezifikation
und Implementierung verwendet.)\\
Keine Eindeutschung.\\
Einheitliche Rechtschreibung.

\end{itemize}


%=============================================================================
%
% Mit Kunden zu kl�ren
%
\section {Mit Kunden zu kl�rende Punkte}
Alles was noch offen ist.

\begin{itemize}

\item Interaktionssprache von GIANT (insbeondere der GUI und gegebenfalls innerhalb der Konfigurationsdateien) mit Kunden kl�ren (Deutsch oder Englisch).


\end{itemize}


%===============================================================================
%
% Inhaltsverzeichnis
%
\setcounter{tocdepth}{1}
\tableofcontents


%===============================================================================
%
% Einleitung
%
\chapter{Einleitung}
% ==============================================================================
%  $RCSfile: intro.tex,v $, $Revision: 1.2 $
%  $Date: 2003/07/06 13:39:08 $
%  $Author: birdy $
%
%  Description:
%
%  Last-Ispelled-Revision:
%
% ==============================================================================


\section{�ber dieses Dokument}

Dieses Handbuch beschreibt die Installation und Benutzung des graphischen
IML-Browsers GIANT (Graphical IML Analysis Navigation Tool).

\section{�ber GIANT -- Graphical IML Analysis Navigation Tool}
\index{GIANT}
\index{IML-Browser}

GIANT ist ein Werkzeug, welches an der Universit�t Stuttgart von der 
CodeFabrik@Stuttgart im Rahmen des Studienprojektes A IML-Browser 
des Studiengangs Softwaretechnik entwickelt wurde.\\
Ziel dieser Entwicklung war es, die bereits bestehende HTML-basierte L�sung 
der Bauhaus Reengineering GmbH \index{Bauhaus-Reengineering GmbH}
f�r IML-Graphen durch ein komfortables graphisches Werkzeug zu erg�nzen. 
GIANT erm�glicht es dem Kunden, Teile von gro�en IML-Graphen zu
visualisieren. 
Durch die Unterscheidung der verschiedenen Kanten- und Knotenklassen werden
dar�ber hinaus auch im IML-Graphen enthaltene Analyseergebnisse �bersichtlich
dargestellt.\\

\section{Aufbau dieses Dokuments}

In den ersten Kapiteln dieses Handbuches wird beschrieben, wie Anwender
mit GIANT anhand einiger Beispiele Analysen durchf�hren k�nnen.
Dies wird anhand einer Art Tutorials beschrieben. Dies erm�glicht
erfahrenen Anwendern einen schnellen Start (\gq{Top-Down}) mit GIANT.

In den sp�teren Kapiteln wird GIANT Fenster f�r Fenster und Men�punkt
f�r Men�punkt genau erkl�rt (\gq{Bottom-Up}). Dies erm�glicht auch das
Nachschlagen von Details zu einzelnen Funktionen.




%===============================================================================
%
% Produkt�bersicht
%
\chapter{Produkt�bersicht}
% ==============================================================================
%  $RCSfile: product.tex,v $, $Revision: 1.34 $
%  $Date: 2003/03/31 22:02:25 $
%  $Author: squig $
%
%  Description: Beschreibung des Produktes. �bersicht �ber die Funktionen
%               und allgemeine Bedienkonzepte.
%
%  Last-Ispelled-Revision: 1.32
%
% ==============================================================================


% ==============================================================================
\section{GIANT Projekte} \index{Projekte} \index{Persistenz}

Innerhalb eines Projektes fasst GIANT Informationen, 
wie erzeugte Anzeigefenster und
erzeugte IML-Teilgraphen f�r einen vorgegebenen IML-Graphen, zusammen
und speichert diese persistent. Die Zusammenfassung zu Projekten soll der 
�bersicht dienen und den Austausch von Teilergebnissen, 
wie z.B. einzelner Anzeigefenster, erleichtern.\\
F�r weitere Informationen zu Projekten siehe 
Kapitel \ref{GIANT Projektverwaltung}.


% ==============================================================================
\section{IML-Teilgraphen und Selektionen}
Hinsichtlich der logischen Zusammenfassung von Knoten und Kanten unterscheidet
GIANT IML-Teilgraphen und Selektionen.

  \subsection{IML-Teilgraphen} \index{IML-Teilgraphen}
  \index{Graph-Knoten} \index{Graph-Kanten} \index{hervorheben} Ein
  IML-Teilgraph beschreibt eine Knoten- und Kantenmenge aus dem
  IML-Graphen bei der jede Kante einen Start- und einen Zielknoten hat
  -- die sogenannten Graph-Knoten und Graph-Kanten.  Die Graph-Knoten
  und Graph-Kanten von IML-Teilgraphen k�nnen in Anzeigefenstern
  hervorgehoben werden. Desweiteren k�nnen die Graph-Knoten und
  Graph-Kanten von IML-Teilgraphen als Fenster-Knoten und
  Fenster-Kanten in Anzeigefenster eingef�gt und layoutet werden.  Die
  IML-Teilgraphen sind aber v�llig unabh�ngig von den Anzeigefenstern
  und enthalten insbesondere keine Layoutinformationen.

  \subsection{Selektionen} \index{Selektionen}
  Selektionen stellen die lokale Variante von Knoten- und Kantenmengen dar.
  Sie sind immer einem festen Anzeigefenster zugeordnet.
  Selektionen k�nnen beliebige Knoten und Kanten umfassen, ohne dass diese 
  einen Teilgraphen bilden m�ssen.
  Jeder Knoten und jede Kante einer Selektion muss auch im Anzeigefenster
  als Fenster-Knoten oder Fenster-Kante vorhanden sein.

% ==============================================================================
\section{Anzeigefenster} \index{Anzeigefenster}
Anzeigefenster sind die Fenster von GIANT, in denen eine benutzerdefinierte
Auswahl von Knoten und Kanten visualisiert wird. Es kann beliebig viele
Anzeigefenster geben und jedes Anzeigefenster kann beliebig viele Selektionen
haben.

  \subsection{Pins} \index{Anzeigefenster!Pins} \index{Pins}
  Da bei gro�en Graphen selten alle zu einem Anzeigeinhalt 
  geh�renden Fenster-Knoten und Fenster-Kanten gemeinsam auf dem Bildschirm
  sichtbar dargestellt werden k�nnen, kann sich der
  Benutzer zu jedem Anzeigefenster eine Liste von Pins anlegen. In den Pins wird
  jeweils die Position des sichtbaren Anzeigeinhaltes und die Zoomstufe
  \index{Zoomstufe}
  gespeichert, so dass zu beliebigen Zeitpunkten die Position des sichtbaren
  Anzeigeinhalts rekonstruiert werden kann.

  \subsection{Visualisierungsstile} \index{Visualisierungsstile}
  Mittels sogenannter Visualisierungsstile kann der Benutzer die
  Darstellung von Fenster-Kanten und Fenster-Knoten auch w�hrend der
  Laufzeit von GIANT beeinflussen.  Da �ber entsprechende XML-Dateien
  verschiedene Visualisierungsstile definiert werden k�nnen, kann
  GIANT so
  an spezifische Problemstellungen angepasst werden.\\
  Weitere Informationen hierzu sind unter Abschnitt \ref{Config
    Visualisierungsstile} verf�gbar.

\section{Knoten-Annotationen} \index{Knoten-Annotationen}
Jeder Knoten kann mit einer textuellen Annotation versehen werden. Diese
Annotation kann in einem Fenster au�erhalb des Anzeigefensters
zur Anzeige gebracht und bearbeitet werden.\\
F�r weitere Informationen hierzu siehe Abschnitt
\ref{Project Persistenz von Knoten-Annotationen}.


% ==============================================================================
\section{Anfragen} \index{GQSL} \index{Anfragesprache}
Eine vielseitige Anfragesprache -- die GIANT Query and Scripting Language GQSL --
stellt nahezu die gesamte Funktionalit�t von GIANT zur Verf�gung und kann 
insbesondere auch zum Aufruf via Kommandozeile
genutzt werden. Die drei Schritte des anschlie�end beschriebenen 
\gq{Drei-Stufen-Konzeptes} k�nnen �ber diese Anfragesprache auch \gq{auf 
einen Schlag} erledigt werden.\\
Die GQSL ist unter Kapitel \ref {GIANT Query Skripting Language} im
Detail spezifiziert.


% ==============================================================================
\section{Drei-Stufen-Konzept} \index{Drei-Stufen-Konzept}

Die im Folgenden beschriebenen drei Schritte sollen dem Benutzer die M�glichkeit
bieten, von den Knoten und Kanten des IML-Graphen ausgehend geeignete
Teilgraphen in Anzeigefenstern zu visualisieren. Von der
Gestaltung der UseCases her, werden diese drei Schritte sequentiell nacheinander
ausgef�hrt, dies ist aber nicht zwingend n�tig.


\begin {enumerate}
  \item Erzeugen geeigneter IML-Teilgraphen, d.h.\ Auswahl geeigneter Knoten 
        und Kanten aus dem IML-Graphen mittels der Anfragesprache.
  
  \item Einf�gen dieser IML-Teilgraphen in ein Anzeigefenster unter Anwendung
  eines Layoutalgorithmus \index{Layoutalgorithmen}. 
  
  \item Weitere Bearbeitung der Fenster-Knoten und Fenster-Kanten (wie z.B.
  Verschieben, Annotieren und Erzeugen von Selektionen).
  
\end {enumerate}









%===============================================================================
% 
% Funktionale Anforderungen
%
\chapter{Funktionale Anforderungen}
% =============================================================================
%  $RCSfile: fa.tex,v $, $Revision: 1.7 $
%  $Date: 2003/02/19 10:55:17 $
%  $Author: schwiemn $
%
%  Description:
%
% =============================================================================

\section{Der Aktor \gq{Benutzer}}
Aktor im Sinne der anschlie�end spezifizierten UseCases ist immmer die 
menschliche Person, die den IML-Browser GIANT gerade benutzt, also der 
\gq{Benutzer}. Da GIANT keinen Mehrbenutzerbetrieb vorsieht, ist dies immer 
eine einzige Person. 
Neben dem \gq{Benutzer} sind keine weiteren Aktoren vorgesehen.\\


  \subsection{Anforderungen an den Aktor \gq{Benutzer}}
  GIANT ist eine Profi-Werkzeug. Weder das System noch das Handbuch richten 
  sich an unerfahrene Benutzer. Zur Bedienung muss der Benutzer daher 
  zwingend �ber die folgenden Kenntnisse verf�gen:

  \begin{itemize}
    \item Erfahrung im Umgang mit grafischen GUIs und 
          dem entsprechenden Betriebssystem.
  
    \item Grundkenntnisse in XML 
          (nur f�r das Editieren der Konfigurationsdateien).

    \item Erfahrung im Bereich Reengineering.

    \item Kenntnisse �ber Struktur und Aufbau des IML-Graphen. 

  \end{itemize}


% =============================================================================
\section{Starten von GIANT}

Der Benutzer startet das Programm GIANT durch ausf�hren der entsprechenden
Programmdatei. Hierbei hat er bez�glich eventueller Kommandozeilenparameter
die folgenden Optionen. 

  \subsubsection {M�gliche Parameter}
  \begin {enumerate}
    
    \item loadiml=''Pfad''\\
    Angabe des Pfades zu einer existierenden IML-Graph Datei.
      
    \item loadproject=''Pfad''\\
    Angabe des Pfades zu einer existierenden Projektdatei.
      
    \item executequery=''Anfrage''\\
    Eingabe eines Wortes der Anfragespreche als String.
     
    \item loadquery=''Pfad''\\
    Pfad zu einer Datei in der ein Wort der Anfragespreche gespeichert ist.
      
  \end {enumerate}
      
  \subsection {M�gliche Aufrufe}

  \begin {enumerate}
  
    \item Eingabe von: <giant>\\
    GIANT startet und ziegt das Main Window an. 
    Es ist noch kein Projekt geladen.
    
    \item Eingabe von: <giant -loadiml=''my\_iml\_file''>\\
    Neues Projekt (erh�lt einen Default-Namen) wird ge�ffnet.
    Dieses Projekt ist noch nicht persistent gespeichert.
    \label{Kommandozeile_neus}

    \item Eingabe von: <giant -loadproject=''my\_project\_file''>\\
    Das angegebene Projekt wird ge�ffnet.
    \label{Kommandozeile_altes}
    
    \item
    Bei \ref{Kommandozeile_neus} und \ref{Kommandozeile_altes}
    kann zus�tzlich entweder der Parameter\\
    <... -executequery''my\_query\_string''> oder der Parameter\\
    <... -loadquery''my\_query\_file''> eingegeben werden.\\
    In diesem Falle wird dann die �bergebene Anfrage ausgef�hrt.
    Die Ergebnisse werden dem neuen oder dem bestehenden Projekt 
    hinzugef�gt, sind aber noch nicht persistent gespeichert.


  \end {enumerate}
  
% ==============================================================================
\section{Beenden von GIANT}
Giant kann durch Schlie�en des HAUPTFENSTERS oder durch Auswahl 
des entsprechenden Eintages im Hauptmen� oder durch ALT+F4 beendet werden.\\
Ablauf:

\begin {enumerate}
  \item Der Benutzer gibt das Kommando zum Beenden von GIANT.
  \item GIANT fragt nach, ob das Projekt gespeichert werden soll.
  (PROJEKT SPEICHERN) (ENTSPRCHENDER DIALOG MUSS IN DIE GUI)
\end {enumerate}
  



% =============================================================================
% Use Cases
  
% ==============================================================================
%  $RCSfile: load_store.tex,v $, $Revision: 1.24 $
%  $Date: 2003/04/19 17:40:56 $
%  $Author: birdy $
%
%  Description: UseCases f�r Lade- und Speicherfunktionalit�t
%
%  Last-Ispelled-Revision: 1.12
%
% ==============================================================================

\begin{uc}[Neues Projekt]{UC: Neues Projekt}
\index{Projekte!neues Projekt erstellen}
  Erstellt ein neues GIANT Projekt. Ein eventuell bereits ge�ffnetes
  Projekt wird dabei geschlossen, wobei �nderungen auf Nachfrage 
  vorher gespeichert werden.
  Weitere wichtige Informationen, die in engem Zusammenhang mit 
  diesem UseCase stehen, sind unter dem Abschnitt 
  \ref{Project Persistenz der Projekte} zu finden.\\
  Zul�ssige Namen f�r Projekte sind unter Abschnitt 
  \ref{afa Zulaessige Namen} spezifiziert.
  
  
  \begin{precond}
    \cond Das Programm ist gestartet.
  \end{precond}

  \begin{postsuccess}
    \cond Ein neues GIANT Projekt mit dem eingegebenen Namen 
          ist erstellt und geladen.   

    \cond Eine IML-Datei ist geladen.

    \cond Das angegebene Projektverzeichnis ist gegebenenfalls 
          (falls es noch nicht vorhanden war) erstellt 
          worden (siehe \ref {Project Das Projektverzeichnis}).

    \cond Die neu erstellte Projektdatei 
          (siehe \ref{Project Die Projektdatei})
          liegt im Projektverzeichnis.
 
    \cond Ein eventuell zuvor ge�ffnetes Projekt ist geschlossen.
          �nderungen an dem eventuell zuvor ge�ffneten Projekt sind je
          nach Entscheidung des Benutzers bei der entsprechenden 
          Sicherheitsabfrage gespeichert oder verworfen.

  \end{postsuccess}

  \begin{postfail}
    \cond Das System bleibt im bisherigen Zustand, falls der
          Benutzer den UseCase mit Cancel abgebrochen hat.

    \cond Tritt w�hrend des Erstellens des neuen Projektes,
          nachdem das eventuell zuvor ge�ffnete Projekt
          geschlossen wurde, ein Fehler auf, so bleibt das Verhalten
          des Systems unspezifiziert.

  \end{postfail}
  
  \begin{proc}
    \step[1]
    Der Benutzer startet den UseCase �ber das Men� Project
    (siehe \ref{Main-Window-Project}) durch Auswahl von \gq{New Project}.

    \step[2]
    GIANT zeigt den Standard-Filechooser-Dialog und fordert den
    Benutzer auf eine vorhandene IML-Graph Datei auszuw�hlen.
  
    \step[3] Der Benutzer w�hlt im Standard-Filechooser-Dialog
    eine IML-Datei aus und best�tigt
    seine Eingabe mit OK (siehe \ref{Standard-Filechooser-Dialog}).
    
    \step[4]
    GIANT zeigt erneut den Standard-Filechooser-Dialog und fordert den
    Benutzer zur Eingabe des Namens der Projektdatei auf.

    \step[5] Der Benutzer gibt im Standard-Filechooser-Dialog
    den Pfad und den Namen der Projektdatei (\gq{Please select Project
    Path and Project File Name}) ein, die Dateiendung wird sp�ter
    von GIANT automatisch gesetzt. Der Name der Projektdatei ist
    automatisch auch der Name f�r das Projekt. Das Verzeichnis der
    Projektdatei wird automatisch zum Projektverzeichnis.

    \step[6]
    Der Benutzer best�tigt er seine Eingabe mit OK.\\
    Existiert die eingegebene Projektdatei bereits, 
    erscheint eine Fehlermeldung gem�� den unter Abschnitt 
    \ref{afa Fehlerverhalten} beschriebenen Konventionen.\\
    Existiert in dem Projektverzeichnis bereits eine andere Projektdatei, so 
    erscheint ebenfalls eine Fehlermeldung.\\
    
    \step[7] Falls noch ein Projekt ge�ffnet ist, erscheint eine 
    Sicherheitsabfrage (siehe \ref{Sicherheitsabfrage})
    ob dieses gespeichert werden soll.    
    Entscheidet der Benutzer sich f�r Speichern, so wird
    die unter \ref{Alles Speichern} beschriebene Funktionalit�t 
    ausgef�hrt. Lehnt der Benutzer dies ab, gehen
    alle nicht gespeicherten Informationen verloren.\\
     
    \step[8]
    GIANT schlie�t das aktuell ge�ffnete Projekt (falls eines ge�ffnet war), 
    legt, falls noch nicht vorhanden, ein neues Projektverzeichnis an,
    erzeugt ein neues Projekt und �ffnet dieses.
       
  \end{proc}

  \begin{aproc}
    \astep{3} Der Benutzer bricht die Verarbeitung mit Cancel ab.  
    \astep{5} Der Benutzer bricht die Verarbeitung mit Cancel ab.
  \end{aproc}
\end{uc}

% ==============================================================================

\begin{uc}[Projekt �ffnen]{UC: Projekt �ffnen}
\index{Projekte!�ffnen}
  �ffnet ein GIANT Projekt. Ein eventuell bereits ge�ffnetes
  Projekt wird dabei geschlossen, wobei �nderungen auf Nachfrage 
  vorher gespeichert werden. \\ 
  Sollte der Benutzer die XML-Dateien 
  innerhalb des Projektverzeichnisses (siehe 
  \ref{Project Das Projektverzeichnis}) manuell modifiziert haben, so
  dass diese von den durch GIANT automatisch erstellten Dateien
  abweichen, wird keinerlei Garantie f�r das korrekte �ffnen
  des Projektes �bernommen. Das Verhalten
  bez�glich eventuell auftretender Fehler ist undefiniert.

  \begin{precond}
    \cond Das Programm ist gestartet.
  \end{precond}

  \begin{postsuccess}
    \cond Das gew�nschte GIANT Projekt ist geladen.
    \cond Die zugeh�rige IML-Datei ist geladen.
    \cond �nderungen an einem eventuell zuvor ge�ffneten Projekt sind 
          je nach Wahl des Benutzers bei der Sicherheitsabfrage gespeichert
          oder verworfen.

  \end{postsuccess}

  \begin{postfail}
    \cond Hat der Benutzer den UseCase mit Cancel abgebrochen, 
          bleibt das System im bisherigen Zustand.

    \cond Muss der UseCase w�hrend des Ladens des neuen Projektes
          aufgrund eines Fehlers abgebrochen werden, so 
          kehrt das System falls m�glich in einen Zustand zur�ck,
          der dem Start des Systems ohne �ffnen eines Projektes
          entspricht.

    
  \end{postfail}
  
  \begin{proc}    
    \step[1]
    Der Benutzer startet den UseCase �ber das Men� Project
    (siehe \ref{Main-Window-Project} Men�) durch Auswahl von \gq{Load Project}.
    GIANT zeigt daraufhin den Standard-Filechooser-Dialog und fordert
    den Benutzer zur Auswahl eines zu �ffnenden Projektes auf.
    
    \step[2] 
    Der Benutzer w�hlt aus dem Standard-Filechooser-Dialog (siehe 
    \ref {Standard-Filechooser-Dialog})
    eine vorhandene GIANT Projektdatei (siehe \ref{Project Die Projektdatei}) 
    aus und best�tigt mit OK.
    
    \step[3] Falls bereits ein Projekt ge�ffnet ist, erscheint eine 
    Sicherheitsabfrage (siehe \ref{Sicherheitsabfrage})
    ob dieses gespeichert werden soll.     
    Entscheidet der Benutzer sich f�r Speichern, so wird
    die unter \ref{Alles Speichern} beschriebene Funktionalit�t 
    ausgef�hrt. Lehnt der Benutzer dies ab, gehen
    alle nicht gespeicherten Informationen verloren.\\

    
    \step[4]
    GIANT schlie�t das alte Projekt (falls eines ge�ffnet war)
    und l�dt das angegebene neue Projekt.

  \end{proc}

  \begin{aproc}
    \astep{2} Der Benutzer bricht die Verarbeitung mit Cancel ab.
  \end{aproc}
\end{uc}

% ==============================================================================

\begin{uc}[Projekt speichern]{UC: Projekt speichern}  
\index{Projekte!speichern}
  Speichert alle �nderungen an einem Projekt. Der Zustand der
  entsprechenden Verwaltungsdateien im Projektverzeichnis entspricht
  nach erfolgreicher Ausf�hrung dieses UseCases exakt dem aktuellen
  Zustand des ge�ffneten Projektes. Alle Konventionen zur
  Persistenz von Projekten sind im Abschnitt 
  \ref{Project Persistenz der Projekte} exakt spezifiziert.

  \begin{precond}
    \cond Das Programm ist gestartet.

    \cond Ein Projekt ist ge�ffnet. 

  \end{precond}

  \begin{postsuccess}
    \cond Die Informationen des Projekts 
    (einschlie�lich aller m�glichen �nderungen) sind persistent in die
    Verwaltungsdateien geschrieben.
 
  \end{postsuccess}

  \begin{postfail}
    \cond Das System bleibt im bisherigen Zustand.
  \end{postfail}
   
  \begin{proc}    
    \step[1]
    Der Benutzer startet den UseCase �ber das Men� Project
    (siehe \ref{Main-Window-Project}) durch Auswahl von \gq{Save Project}.
      
    \step[2]
    GIANT f�hrt die unter \ref{Alles Speichern} beschriebene Funktionalit�t 
    aus und speichert alle Informationen zu dem Projekt in der zugeh�rigen
    Projektdatei.
  \end{proc}

\end{uc}



% ==============================================================================

\begin{uc}[Projekt speichern unter]{UC: Projekt speichern unter}
\index{Projekte!unter neuem Namen speichern}
Speichert alle Informationen zu einem Projekt in eine neue Projektdatei
(entsprechende Verwaltungsdateien werden ebenfalls dupliziert).


  \begin{precond}
    \cond Das Programm ist gestartet.
    \cond Ein Projekt ist ge�ffnet (entweder ein neu erzeugtes oder ein
          geladenes).
  \end{precond}

  \begin{postsuccess}
    \cond Eine neue Projektdatei ist erzeugt worden.

    \cond Das angegebene Projektverzeichnis ist, falls es noch
          nicht vorhanden war, erstellt worden.

    \cond Die Daten des Projekts sind persistent in die 
          neuen Verwaltungsdateien im Projektverzeichnis 
          der neuen Projektdatei geschrieben (siehe auch
          \ref{Project Persistenz der Projekte}).
        
    \cond Das aktuell ge�ffnete Projekt bleibt in GIANT ge�ffnet, 
          zuk�nftiges Speichern
          (siehe \ref{Projekt speichern}) betrifft nur die 
          Verwaltungsdateien im neu erzeugten Projekt.

    \cond Die alte Projektdatei und alle zugeh�rigen Verwaltungsdateien 
          bleiben unver�ndert.
    
 
  \end{postsuccess}

  \begin{postfail}
    \cond Das System bleibt im bisherigen Zustand.
  \end{postfail}
  
  \begin{proc}    
    \step[1]
    Der Benutzer startet den UseCase �ber das Men� Project
    (siehe \ref{Main-Window-Project}) durch Auswahl von \gq{Save Project As...}.
    Daraufhin zeigt GIANT den Standard-Filechooser-Dialog und fordert
    den Benutzer auf den Namen und den Ort (Pfad) der neuen Projektdatei
    einzugeben.

     
    \step[2] Der Benutzer gibt im Standard-Filechooser-Dialog 
             (siehe \ref {Standard-Filechooser-Dialog}) 
             das neue Projektverzeichnis 
             (siehe \ref{Project Das Projektverzeichnis})
             und den Namen f�r die neue Projektdatei
             (siehe \ref{Project Die Projektdatei}) ein, die Dateiendung 
             wird sp�ter von GIANT automatisch gesetzt. \\
    Der Name der Projektdatei ist automatisch auch der Name f�r das Projekt.
    Zul�ssige Namen sind unter Abschnitt \ref{afa Zulaessige Namen}
    spezifiziert. \\ Das Verzeichnis, welches f�r die Projektdatei 
    angebenen wurde, wird automatisch zum Projektverzeichnis. 
  
    \step[3]  
    Der Benutzer best�tigt seine Eingabe mit OK.\\
    Existiert die eingegebene Datei schon, erscheint eine Fehlermeldung.\\
    Existiert in dem angegebenen Projektverzeichnis 
    bereits eine andere Projektdatei, so erscheint eine entsprechende 
    Fehlermeldung.\\

    \step[4]
    GIANT speichert das alte Projekt in der neuen Projektdatei,
    falls noch nicht vorhanden wird auch das neue Projektverzeichnis
    angelegt. Alle Verwaltungsdateien werden ebenfalls im neuen
    Projektverzeichnis gespeichert.
  
  \end{proc}


 \begin{aproc}
    \astep{2} Der Benutzer bricht die Verarbeitung mit Cancel ab.  
 \end{aproc}
\end{uc}







% ==============================================================================
%  $RCSfile: gui_window.tex,v $, $Revision: 1.34 $
%  $Date: 2003/04/05 21:08:37 $
%  $Author: squig $
%
%  Description: UseCases f�r die Fensterfunktionalit�t der GUI
%
%  Last-Ispelled-Revision: 1.25
%
% ==============================================================================

\begin{uc}[Leeres Anzeigefenster erzeugen]{UC: Leeres Anzeigefenster erzeugen}
\index{Anzeigefenster!erzeugen}

�ber diesen UseCase kann der Benutzer neue Anzeigefenster innerhalb eines
Projektes anlegen.

  \begin{precond}
    \cond Ein Projekt ist geladen.
  \end{precond}

  \begin{postsuccess}
    \cond Das neue, leere Anzeigefenster ist ge�ffnet.
    \cond Das neue Anzeigefenster ist Bestandteil des Projektes.
  \end{postsuccess}

  \begin{postfail}
    \cond Das System bleibt im bisherigen Zustand.
  \end{postfail}
  
  \begin{proc}    
    \step[1] 
    Der Benutzer startet den UseCase �ber das Popup Men�
    \ref{WINDOW-LIST-POPUP} durch Auswahl von \gq{New Window}.        
    
    \step[2]
    GIANT erzeugt ein neues Anzeigefenster (siehe  \ref{GUI Anzeigefenster})
    mit einem Namen nach dem Schema ,,Unknown\_??'' und �ffnet dies.
  \end{proc}

  \begin{aproc}
    \ageneral Es sind keine alternativen Abl�ufe vorgesehen.
  \end{aproc}
\end{uc}


% ==============================================================================
\begin{uc}[Anzeigefenster �ffnen]{UC: Anzeigefenster �ffnen}
\index{Anzeigefenster!�ffnen}

Dient zum �ffnen eines Anzeigefensters des Projektes.

  \begin{precond}
    \cond Ein Projekt mit mindestens einem Anzeigefenster ist geladen.
    \cond Es gibt mindestens ein nicht ge�ffnetes Anzeigefenster.
  \end{precond}

  \begin{postsuccess}
    \cond Das Anzeigefenster ist ge�ffnet.
  \end{postsuccess}

  \begin{proc}    
    \step[1]
    Der Benutzer f�hrt einen Doppelklick auf ein nicht ge�ffnetes
    Anzeigefenster in der Liste �ber die Anzeigefenster
    (siehe \ref{WINDOW-LIST}) durch, oder w�hlt im zugeh�hrigen Popup Men�
    \ref{WINDOW-LIST-POPUP} den Eintrag ,,Open'' aus.
   
    \step[2]
    GIANT �ffnet das entsprechende Anzeigefenster.
  \end{proc}  

\end{uc}

% ==============================================================================
\begin{uc}[Anzeigefenster umbenennen]{UC: Anzeigefenster umbenennen}
\index{Anzeigefenster!umbenennen}

Dient zum Umbenennen eines Anzeigefensters des Projektes.

  \begin{precond}
    \cond Ein Projekt mit mindestens einem Anzeigefenster ist geladen.
  \end{precond}

  \begin{postsuccess}
    \cond Das Anzeigefenster hat einen neuen Namen.
  \end{postsuccess}

  \begin{postfail}
    \cond Es sind keine Fehlerf�lle vorgesehen.
  \end{postfail}

  \begin{proc}    
    \step[1]
    Der Benutzer f�hrt einen Rechtsklick auf ein Anzeigefenster in der Liste
    �ber die Anzeigefenster (siehe \ref{WINDOW-LIST}) durch und w�hlt im
    Popup Men� \ref{WINDOW-LIST-POPUP} den Eintrag ,,Rename'' aus.

    \step[2] 
    GIANT �ffnet den allgemeinen Texteingabedialog 
    \gq{Enter Name For Window} (siehe \ref{DIALOG-WINDOW}).

    \step[3]
    Der Benutzer gibt dort einen zul�ssigen Namen f�r das Anzeigefenster ein
    und best�tigt seine Eingabe mit OK.
    
    \step[4]
    GIANT benennt das Anzeigefenster (siehe \ref{GUI Anzeigefenster}) um.
  \end{proc}  

  \begin{aproc}
    \ageneral Es sind keine alternativen Abl�ufe vorgesehen.
  \end{aproc}
\end{uc}


% ==============================================================================
\begin{uc}[Anzeigefenster speichern]{UC: Anzeigefenster speichern}
\index{Anzeigefenster!speichern}

Mit diesem UseCase wird ein Anzeigefenster gespeichert.
N�heres zur Persistenz von Anzeigefenstern ist unter Abschnitt
\ref{Project Persistenz der Projekte} spezifiziert.

  \begin{precond}
    \cond Ein Projekt mit mindestens einem Anzeigefenster ist geladen.
  \end{precond}

  \begin{postsuccess}  
    \cond Nach dem letzten Speichern am Anzeigefenster vorgenommene
    Modifikationen (neue Knoten eingef�gt etc.) sind in der Verwaltungsdatei
    (siehe \ref {Project Verwaltungsdateien f�r Anzeigefenster}) gespeichert.
  \end{postsuccess}  

  \begin{postfail}
    \cond Es sind keine Fehlerf�lle vorgesehen.
  \end{postfail}

  \begin{proc}
    \step[1]
    Der Benutzer f�hrt einen Rechtsklick auf ein Anzeigefenster in der Liste
    �ber die Anzeigefenster (siehe \ref{WINDOW-LIST}) durch und w�hlt im
    Popup Men� \ref{WINDOW-LIST-POPUP} den Eintrag ,,Save'' aus.
    
    \step[2]
    GIANT schreibt alle �nderungen in die Verwaltungsdatei des Anzeigefensters.
  \end{proc}  

  \begin{aproc}
    \ageneral
    Es sind keine alternativen Abl�ufe vorgesehen.
  \end{aproc}
\end{uc}


% ==============================================================================
\begin{uc}[Anzeigefenster schliessen]{UC: Anzeigefenster schlie\ss en}
\index{Anzeigefenster!schlie\ss en}

Mit diesem UseCase wird ein ge�ffnetes Anzeigefenster geschlossen.
N�heres zur Persistenz von Anzeigefenstern ist unter Abschnitt
\ref{Persistenz!Anzeigefenster} spezifiziert.

  \begin{precond}
    \cond Ein Projekt mit mindestens einem Anzeigefenster ist geladen.
    \cond Es gibt mindestens ein ge�ffnetes Anzeigefenster.
  \end{precond}

  \begin{postsuccess}  
    \cond Das Anzeigefenster ist geschlossen.    
    \cond Nach dem letzten Speichern am Anzeigefenster vorgenommene
    Modifikationen (neue Knoten eingef�gt etc.) sind in der Verwaltungsdatei
    (siehe \ref {Project Verwaltungsdateien f�r Anzeigefenster})
    gespeichert oder nicht.
  \end{postsuccess}  

  \begin{postfail}
    \cond Es sind keine Fehlerf�lle vorgesehen.
  \end{postfail}

  \begin{proc}    
    \step[1]
    Der Benutzer schlie�t das Anzeigefenster (durch klicken auf das ''X''
    Symbol rechts oben in der Titelleiste des Anzeigefensters).\\
    Alternativ kann er das Anzeigefenster �ber das entsprechende
    Popup-Men� \ref{WINDOW-LIST-POPUP} in der Liste �ber die
    Anzeigefenster des Projektes (siehe \ref{WINDOW-LIST}) schlie�en.
         
    \step[2]
    GIANT zeigt die allgemeine Sicherheitsabfrage 
    (siehe \ref{Sicherheitsabfrage}) und fragt nach, ob eventuelle
    �nderungen im Anzeigefenster gespeichert werden sollen oder nicht
    \gq{The content has changed. Do you want to save the changes?}.
    
    \step[3]
    Best�tigt der Benutzer mit YES, werden die �nderungen in die 
    Verwaltungsdatei geschrieben. Anderenfalls gehen s�mtliche 
    nicht gespeicherten �nderungen am Anzeigefenster verloren.
    
    \step[4]
    GIANT schlie�t das Anzeigefenster.
  \end{proc}  

  \begin{aproc}
    \ageneral
    Es sind keine alternativen Abl�ufe vorgesehen.
  \end{aproc}
\end{uc}


% ==============================================================================
\begin{uc}[Anzeigefenster l�schen]{UC: Anzeigefenster l�schen}
\index{Anzeigefenster!l�schen}

Mit diesem UseCase werden bestehende Anzeigefenster aus dem Projekt entfernt
und gel�scht. Alle Informationen zu dem Anzeigefenster gehen hierbei
unwiederbringlich verloren.

  \begin{precond}
    \cond Ein Projekt mit mindestens einem Anzeigefenster ist geladen.
  \end{precond}

  \begin{postsuccess}
    \cond 
    Das gel�schte Anzeigefenster ist nicht mehr Bestandteil des Projektes.
    
    \cond
    Die Verwaltungsdatei f�r das Anzeigefenster (siehe   
    \ref {Project Verwaltungsdateien f�r Anzeigefenster}) 
    wurde ebenfalls gel�scht.
 
  \end{postsuccess}

  \begin{postfail}
    \cond Das System bleibt im bisherigen Zustand.
  \end{postfail}
  
  \begin{proc}    
    \step[1]
    Der Benutzer startet den UseCase �ber das
    Popup-Men� \ref{WINDOW-LIST-POPUP}.
    
    \step[2] 
    GIANT zeigt die allgemeine Sicherheitsabfrage (siehe 
    \ref{Sicherheitsabfrage}) und fragt nach, ob das Anzeigefenster wirklich
    gel�scht werden soll. \gq{Do you really want to delete the selected window?}
      
    \step[3] 
    Der Benutzer best�tigt mit YES.
    
    \step[4]
    GIANT entfernt das Anzeigefenster aus dem Projekt und l�scht die 
    zugeh�rige Verwaltungsdatei (siehe auch   
    \ref {Project Verwaltungsdateien f�r Anzeigefenster}).
  \end{proc}

  \begin{aproc}
    \astep{3} Der Benutzer bricht die Verarbeitung mit NO ab.
  \end{aproc}
\end{uc}


% ==============================================================================
\begin{uc}[IML-Teilgraph in Anzeigefenster einf�gen]
          {UC: IML-Teilgraph in Anzeigefenster einf�gen}
\index{IML-Teilgraphen!in Anzeigefenster einf�gen}

Mit diesem UseCase k�nnen die Graph-Kanten und Graph-Knoten 
von IML-Teilgraphen in Anzeigefenster eingef�gt werden.
Siehe auch   
\ref{Verhalten beim Einf�gen von IML-Teilgraphen und Selektionen 
in Anzeigefenster} und
insbesondere \ref{Einf�gen von IML-Teilgraphen in Anzeigefenster}.
          

  \begin{precond}
    \cond Ein Projekt mit mindestens einem ge�ffneten 
          Anzeigefenster ist geladen.
    
    \cond Es gibt mindestens einen IML-Teilgraphen.
    
  \end{precond}

  \begin{postsuccess}
    
    \cond 
    Alle Knoten und Kanten des IML-Teilgraphen sind in das Anzeigefenster
    entsprechend dem gew�hlten Layout an der vorgegebenen Position
    eingef�gt.
    
    \cond
    In dem Anzeigefenster gibt es eine neue aktuelle Selektion, 
    die die neu eingef�gten Knoten und Kanten umfasst.
   
  \end{postsuccess}

  \begin{postfail}
    \cond Hat der Benutzer den UseCase an irgendeinem Punkt abgebrochen,
    kehrt das System zu dem Zustand zur�ck, in dem es vor dem Start des
    UseCase war.
  \end{postfail}
  
  \begin{proc}    
    \step[1]
    Der Benutzer startet den UseCase �ber das Popup-Men� 
    \ref{SUBGRAPH-LIST-POPUP}
    im Hauptfenster \gq{Insert IML Subgraph}.
    Hierdurch wird der einzuf�gende IML-Teilgraph bestimmt (immer
    der IML-Teilgraph, auf dem der Rechtsklick ausgef�hrt wurde).
    
    \step[2] 
    GIANT zeigt in der Statuszeile im Hauptfenster 
    \gq{Select Position In Display Window
    For Insertion Of IML Subgraph} an.
    Der Benutzer w�hlt das entsprechende Anzeigefenster aus und
    gibt �ber das Fadenkreuz (siehe \ref{Fadenkreuzcursor}) die Position vor, 
    an der die neuen
    Fenster-Knoten und Fenster-Kanten eingef�gt werden sollen.
    Die Statuszeile im Hauptfenster schaltet auf Normalmodus zur�ck.
    
    \step[3]
    GIANT zeigt den Dialog zur Auswahl von Layoutalgorithmen
    (siehe \ref{Layoutalgorithmen-Dialog}).
    
    \step[4] Der Benutzer w�hlt einen der vorgegebenen
    Layoutalgorithmen aus. Bei semantischen Layouts gibt er �ber den
    Layoutalgorithmen Dialog (siehe \ref{Layoutalgorithmen-Dialog})
    auch die Kantenklassen vor, die f�r das Layout ber�cksichtigt
    werden sollen (siehe Kapitel \ref{Layoutalgorithmen} f�r Details
    zu Layoutalgorithmen).
           
    \step[5]
    Der Benutzer best�tigt mit OK.
          
    \step[6] GIANT berechnet das entsprechende Layout und zeigt einen
    Dialog an, der den Benutzer �ber den Fortschritt der Berechnung
    informiert (siehe \ref{Progressbar-Modale}).\\
    W�hrend der Berechnung des Layouts kann das System GIANT nicht
    bedient werden. Zug�nglich ist nur der Button zum Abbruch des
    Algorithmus (siehe \ref{Progressbar-Modale-Cancel}).
    
    \step[7]
    Nach Abschluss der Berechnung f�gt GIANT die Fenster-Knoten und 
    Fenster-Kanten in das entsprechende Anzeigefenster ein.
   
   
  
  \end{proc}

  \begin{aproc}
    \astep{4} Der Benutzer bricht den UseCase mit Cancel ab.
    \astep{6} Der Benutzer bricht die Berechnung des Layouts ab.  
  \end{aproc}



\end{uc}

% ==============================================================================
\begin{uc}[Selektion in Anzeigefenster einf�gen]
          {UC: Selektion in Anzeigefenster einf�gen}
\index{Selektionen!in Anzeigefenster einf�gen}
          
Mit diesem UseCase kann eine Selektion aus einem Quell-Anzeigefenster in ein
Ziel-Anzeigefenster unter Beibehaltung des Layouts kopiert werden
(siehe \ref{Verhalten beim Einf�gen von IML-Teilgraphen und
  Selektionen in Anzeigefenster} und \ref{Einf�gen von Selektionen in
  Anzeigefenster}).

  \begin{precond}
    \cond Ein Projekt mit mindestens zwei ge�ffneten 
          Anzeigefenstern ist geladen.
    
    \cond Es gibt mindestens eine Selektion.
    
  \end{precond}

  \begin{postsuccess}
    
    \cond
    Das Position von Fenster-Knoten, die vor dem Einf�gen bereits im
    Ziel-Anzeigefenster vorhanden waren, bleibt 
    je nach Wahl des Benutzers unver�ndert oder wird ebenfalls ge�ndert.
    
    \cond
    Die kopierte Selektion existiert auch im Ziel-Anzeigefenster
    als Ziel-Selektion und tr�gt dort ebenfalls den Namen der Selektion. 

    \cond
    Im Ziel-Anzeigefenster gibt es keine Knoten mit der selben ID
    mehrfach. 

  \end{postsuccess}

 
  \begin{proc}    
    \step[1]
    Der Benutzer startet den UseCase �ber das Popup-Men�
    \ref{Popup-Men� Subgraph List} in der Subgraph Liste im Hauptfenster.
    Durch Auswahl eines der beiden Men�eintr�ge 
    \gq{Copy Selection keep existing layout} oder 
    \gq{Copy Selection change existing layout}.
    Hierdurch wird automatisch die Selektion bestimmt (immer
    die Selektion, auf der der Rechtsklick ausgef�hrt wurde).
    
    \step[2] 
    Der Benutzer w�hlt das entsprechende Ziel-Anzeigefenster aus.
    Das Ziel-Anzeigefenster darf nicht das Quell-Anzeigefenster sein,
    sonst wird eine Fehlermeldung ausgegeben.
    GIANT zeigt in der Statuszeile im Hauptfenster 
    \gq{Select Position in Display Window
    for Insertion of copied IML Subgraph}.
    Der Benutzer gibt �ber das Fadenkreuz die Position vor, 
    an der die neuen Fenster-Knoten und Fenster-Kanten eingef�gt werden 
    sollen (siehe auch \ref{Fadenkreuzcursor}).
    
    \step[3]
    GIANT kopiert die Knoten und Kanten der 
    Selektion in das Ziel-Anzeigefenster.
    Je nachdem, welchen Eintrag der Benutzer im Popup-Men� ausgew�hlt hat,
    geschieht mit den bereits im Ziel-Anzeigefenster vorhandenen 
    Fenster-Knoten der Selektion folgendes:
    \begin {enumerate}
       \item
       Falls \gq{Copy Selection keeping existing layout} gew�hlt wurde, wird
       ihre Position im Ziel-Anzeigefenster nicht ver�ndert.
       
       \item 
       Falls \gq{Copy Selection changing existing layout} gew�hlt wurde,
       ihre Position im Zielanzeigefenster gem�� dem Layout der
       Selektion ver�ndert.
        
    \end {enumerate}
      
  \end{proc}


\end{uc}

% ==============================================================================
\begin{uc}[Fenster-Knoten und Fenster-Kanten einer Selektion aus 
           Anzeigefenster l�schen]
      {UC: Fenster-Knoten und Fenster-Kanten einer Selektion aus 
           Anzeigefenster l�schen}
\index{Fenster-Knoten!l�schen}
\index{Fenster-Kanten!l�schen}
                  
Mittels dieses UseCases k�nnen alle Fenster-Knoten und Fenster-Kanten
einer Selektion aus einem Anzeigefenster gel�scht werden
(siehe auch \ref{Verhalten beim Entfernen von Fenster-Knoten und 
Fenster-Kanten}).

 
  \begin{precond}
    \cond Ein Projekt mit mindestens einem ge�ffneten 
          Anzeigefenster ist geladen.
    
    \cond Es gibt eine Selektion.
    
  \end{precond}

  \begin{postsuccess}
    
    \cond
    Die Selektion ist aus dem Anzeigefenster gel�scht.
    
    \cond 
    Alle betroffenen Fenster-Knoten und Fenster-Kanten sind gem��
    der unter 
    \ref{Verhalten beim Entfernen von Fenster-Knoten und Fenster-Kanten}
    beschriebenen Konvention aus dem Anzeigefenster entfernt.

    \cond
    Alle anderen Selektionen des Anzeigefensters wurden aktualisiert.
 
   
  \end{postsuccess}

  \begin{postfail}
    \cond Hat der Benutzer den UseCase abgebrochen,
    kehrt das System zu dem Zustand zur�ck, in dem es vor dem Start des
    UseCase war.
  \end{postfail}
  
  \begin{proc}    
  
    \step[1]
    Der Benutzer f�hrt einen Rechtsklick auf die entsprechende Selektion in der
    Selektionsauswahlliste durch (siehe \ref{Selektionsauswahlliste})
    und w�hlt im entsprechenden Popup-Men� den Eintrag \gq{Delete Selection}
    aus.
      
    
    \step[2]
    GIANT zeigt die Sicherheitsabfrage (siehe \ref{Sicherheitsabfrage}) 
    und fragt nach, ob es die Selektion samt ihrer Knoten und Kanten
    wirklich l�schen soll (\gq{Really delete Selection xy from its window 
    including Nodes and Edges?}).
    
    \step[3]
    Der Benutzer best�tigt mit \gq{Yes}.
    
    \step[4]
    GIANT l�scht die Selektion samt allen zugeh�rigen Fenster-Knoten und
    Fenster-Kanten aus dem entsprechenden Anzeigefenster.
      
  
  \end{proc}

  \begin{aproc}
    \astep{2} Der Benutzer bricht den UseCase mit \gq{No} ab.
  \end{aproc}


\end{uc}

% ==============================================================================
\begin{uc}[Den Visualisierungsstil eines Anzeigefensters �ndern]
{UC: Den Visualisierungsstil eines Anzeigefensters �ndern}
\index{Visualisierungsstile!innerhalb eines Anzeigefensters}

Mittels dieses UseCase kann der Benutzer die Visualisierung
von Fenster-Knoten und Fenster-Kanten innerhalb eines Anzeigefensters
dynamisch durch Auswahl verschiedener frei definierbarer
Visualisierungsstile �ndern (siehe auch \ref{Config Visualisierungsstile}).

  \begin{precond}

    \cond 
    Es gibt ein ge�ffnetes Anzeigefenster.    
   
    \cond
    Es gibt mindestens einen definierten Visualisierungsstil
    (siehe auch \ref{Config Visualisierungsstile}). 
       
  \end{precond}

  \begin{postsuccess}
    
    \cond
    Die Darstellung der Fenster-Knoten und Fenster-Kanten in dem
    Anzeigefenster entspricht den Vorgaben des gew�hlten 
    Visualisierungsstils.

    \cond
    Alle anderen Zust�nde und Eigenschaften des Anzeigefensters, wie
    z.B. die hervorgehobenen Selektionen, bleiben unver�ndert.
    
  \end{postsuccess}

  \begin{proc}    
  
    \step[1]
    Der Benutzer �ndert den Visualisierungsstil eines Anzeigefensters
    dadurch, dass er in der Stilauswahl-Combobox 
    des Anzeigefensters (siehe \ref{GUI Stilauswahl-Combobox}) einen 
    anderen Visualisierungsstil einstellt.
      
    \step[2]
    GIANT �ndert die Darstellung von Fenster-Knoten und Fenster-Kanten
    entsprechend ab (n�heres zur Visualisierung von Fenster-Knoten
    und Fenster-Kanten ist in Kapitel 
    \ref{Visualisierung des IML-Graphen} spezifiziert).
              
  \end{proc}

\end{uc}

% ==============================================================================
\begin{uc}[Anzeigefenster scrollen]{UC: Anzeigefenster scrollen}
\index{Anzeigefenster!Scrollen}
\index{Scrollen}

Ver�ndert die Position des sichtbaren Anzeigeinhaltes.

  \begin{precond}
    \cond 
    Ein Projekt mit mindestens einem ge�ffneten Anzeigefenster ist geladen.
       
  \end{precond}

  \begin{postsuccess}
    
    \cond
    Die Position des sichtbaren Anzeigeinhalts wurde entsprechend abge�ndert.
    
  \end{postsuccess}

  \begin{proc}    
  
    \step[1]
    \begin {enumerate}
      \item
      Der Benutzer scrollt den sichtbaren Anzeigefokus mittels der horizontalen
      oder vertikalen Bildlaufleisten des Anzeigefensters (siehe    
      \ref{Scrolleisten}). Dies geschieht mittels
      der Maus gem�� der Konventionen von GTK/Ada f�r Bildlaufleisten.
      
      \item
      Es kann auch mittels der Cursortasten gescrollt werden.\\
      Das Dr�cken der linken Cursortaste f�hrt z.B.\ dazu, dass der sichtbare
      Anzeigeinhalt des aktiven Anzeigefensters nach links verschoben wird.
      
    \end {enumerate}
        
  \end{proc}

\end{uc}


% ==============================================================================
\begin{uc}[Anzeigefenster zoomen]{UC: Anzeigefenster zoomen}
\index{Anzeigefenster!zoomen}
\index{zoomen}
Ver�ndert den Ma�stab der Darstellung von Knoten und Kanten.

  \begin{precond}
    \cond 
    Ein Projekt mit mindestens einem ge�ffneten Anzeigefenster ist geladen.
       
  \end{precond}

  \begin{postsuccess}
    
    \cond
    Der angezeigte Bereich des sichtbaren Anzeigeinhalts wurde entsprechend
    vergr��ert oder verkleinert.
    Die Detailstufe (siehe \ref{Visualization Detailstufen})
    wurde ggf. automatisch angepasst.
    
  \end{postsuccess}

  \begin{proc}    
  
    \step[1]
    \begin {enumerate}
      \item
      Der Benutzer gibt in der Zoom-Kontrolle-Combobox 
      (siehe \ref{GUI Zoom-Kontrolle})
      des Anzeigefensters einen
      neuen Zoomwert ein, w�hlt darin einen der vorgefertigten Werte aus oder 
      �ndert den Zoomwert in festgelegten Schritten mit den \gq{+} oder 
      \gq{-} Buttons. 
      
    \end {enumerate}
    
    \step[2]
    GIANT berechnet den neuen sichtbaren Anzeigeinhalt anhand der neuen
    Zoomstufe.

        
  \end{proc}
\end{uc}

% ==============================================================================
\begin{uc}[Zoomen auf eine Selektion]{UC: Zoomen auf eine gesamte Selektion}
\index{zoomen auf Selektion}

W�hlt die passende Zoomstufe \index{Zoomstufe} 
und scrollt den sichtbaren Anzeigeinhalt so, 
dass eine Selektion im Anzeigefenster vollst�ndig
sichtbar ist.

  \begin{precond}
    \cond 
    Ein Projekt mit mindestens einem ge�ffneten Anzeigefenster ist geladen.
       
  \end{precond}

  \begin{postsuccess}
    
    \cond Der sichtbare Anzeigeinhalts wurde
    mittels Zoomen und Scrollen so ver�ndert, dass die
    ausgew�hlte Selektion vollst�ndig sichtbar ist. Die Detailstufe (siehe
    \ref{Visualization Detailstufen}) wurde ggf. automatisch
    angepasst.
        
  \end{postsuccess}

  \begin{proc}    
  
    \step[1]
    Der Benutzer klickt auf \gq{Zoom to make selection fill window} 
    im Popup-Men�
    der Selektionsauswahlliste (siehe \ref{Selektionsauswahlliste}).
      
    \step[2]
    GIANT scrollt und zoomt automatisch so, dass die gesamte Selektion
    im Anzeigefenster sichtbar wird.
        
  \end{proc}
\end{uc}

% ==============================================================================
\begin{uc}[Zoomen auf den Anzeigeinhalt]{UC: Zoomen auf gesamten Inhalt eines Anzeigefensters}
\index{zoomen auf gesamten Anzeigeinhalt}
W�hlt die passende Zoomstufe und scrollt den sichtbaren Anzeigeinhalt so, 
dass der Anzeigeinhalt vollst�ndig im sichtbaren Anzeigeinhalt
dargestellt wird.

  \begin{precond}
    \cond 
    Ein Projekt mit mindestens einem ge�ffneten Anzeigefenster ist geladen.
       
  \end{precond}

  \begin{postsuccess}

    \cond Der sichtbare Anzeigeinhalts wurde
    mittels Zoomen und Scrollen so ver�ndert, dass der
    Anzeigeinhalt vollst�ndig sichtbar ist. Die Detailstufe (siehe
    \ref{Visualization Detailstufen}) wurde ggf. automatisch
    angepasst.
    
  \end{postsuccess}

  \begin{proc}    
  
    \step[1]
    Der Benutzer klickt auf \gq{Fill window} in der Zoomkontrolle des
    Fensters (siehe \ref{GUI Zoom-Kontrolle}).
      
    \step[2]
    GIANT berechnet die neue Zoomstufe f�r das Anzeigefenster.
        
  \end{proc}

\end{uc}

% ==============================================================================
\begin{uc}[Zoomen auf eine Kante]{UC: Zoomen auf eine Kante}
\index{zoomen auf eine Kante}

W�hlt die passende Zoomstufe und scrollt den sichtbaren Anzeigeinhalt
so, dass eine Kante mit ihren Start- und Ziel-Fenster-Knoten komplett
im sichtbaren Anzeigeinhalt liegt.

  \begin{precond}
    \cond 
    Ein Projekt mit mindestens einem ge�ffneten Anzeigefenster mit mindestens 
    einer Kante ist geladen.
       
  \end{precond}


  \begin{postsuccess}
    
    \cond Der sichtbare Anzeigeinhalts wurde mittels Zoomen und
    Scrollen so ver�ndert, dass die ausgew�hlte Kante und ihre Start-
    und Ziel-Fenster-Knoten vollst�ndig sichtbar ist. Die Detailstufe
    (siehe \ref{Visualization Detailstufen}) wurde ggf. automatisch
    angepasst.
    
  \end{postsuccess}


  \begin{postfail}
    \cond Hat der Benutzer den UseCase abgebrochen,
    kehrt das System zu dem Zustand zur�ck, in dem es vor dem Start des
    UseCase war.
  \end{postfail}


  \begin{proc}    
  
    \step[1]
    Der Benutzer Klickt den Button \gq{Pick Edge} in der Zoomkontrolle
    (siehe \ref{GUI Zoom-Kontrolle}).
   
    \step[2] 
    Daraufhin erscheint in der Statuszeile von GIANT im Hauptfenster
    der Text 
    \gq{Select Edge in Window (Fenstername) to be zoomed onto} und
    der Fadenkreuz-Cursor (siehe \ref{Fadenkreuzcursor}) wird
    angezeigt, wenn der Mauscursor �ber den sichtbaren Anzeigeinhalt
    eines Anzeigefensters bewegt wird.
       
    \step[3]
    Der Benutzer klickt mit der linken Maustaste auf die gew�nschte Kante
    im gew�nschten Anzeigefenster.

    \step[4]
    GIANT berechnet f�r das gew�nschte Anzeigefenster die Zoomstufe so,
    dass  die gesamte Fenster-Kante im Anzeigeinhalt sichtbar dargestellt
    wird.
      
        
  \end{proc}

  \begin{aproc}

    \astep{3} Der Benutzer kann den UseCase abbrechen, indem er
    auf eine einen Rechtsklick mit der Maus ausf�hrt.
  
  \end{aproc}

\end{uc}


% ==============================================================================
\begin{uc}[Verschieben von Fenster-Knoten und Selektionen mittels Cut and Paste]{UC: Verschieben von Fenster-Knoten und Selektionen mittels
                  Cut and Paste}
\index{verschieben!einzelne Knoten}
\index{verschieben!ganze Selektionen}

Mit diesem UseCase k�nnen Fenster-Knoten und Selektionen auf dem
Anzeigeinhalt verschoben werden. Dieses Verschieben geschieht mittels
\gq{Cut and Paste}.


  \begin{precond}
    \cond 
    Ein Projekt mit mindestens einem ge�ffneten Anzeigefenster und mit
    mindestens einem Fenster-Knoten oder mindestens einer Selektion
    ist geladen.
       
  \end{precond}

  \begin{postsuccess}
    
    \cond
    Der einzelne Fensterknoten oder alle Fenster-Knoten der Selektion
    wurden im Anzeigeinhalt verschoben.  
  \end{postsuccess}

  \begin{postfail}
    \cond Hat der Benutzer den UseCase abgebrochen,
          so werden keine Fenster-Knoten verschoben.
  \end{postfail}  


 

  \begin{proc}    
  
    \step[1]
    Der Benutzer w�hlt die zu verschiebende Selektion oder den
    zu verschiebenden Fenster-Knoten aus (\gq{Cut}), indem er:
   
    \begin{enumerate}
      \item
      Einen Rechtsklick auf eine Selektion in der Selektionsauswahlliste
      durchf�hrt (siehe \ref{Selektionsauswahlliste}) und im
      Popup-Men� den Eintrag \gq{Move Selection} ausw�hlt,

      \item
      oder einen Rechtsklick auf einen Fenster-Knoten durchf�hrt
      und im Popup Men� (siehe \ref{Node-Popup-Men�}) den Eintrag
      \gq{Move Node} ausw�hlt.

    \end{enumerate}
    
    \step[2] GIANT geht in den den \gq{Paste Modus} �ber und zeigt
    dies in der Statusleiste (siehe \ref{Statuszeile}) des
    Hauptfensters an.  Der Cursor wird, falls er �ber den sichtbaren
    Anzeigeinhalt eines Anzeigefensters bewegt wird, zum Fadenkreuz
    (siehe \ref{Fadenkreuzcursor}).  Die Funktionalit�t zum Zoomen und
    Scrollen des Anzeigefensters mittels der beiden UseCases
    \ref{Anzeigefenster scrollen} und \ref{Anzeigefenster zoomen}
    bleibt weiterhin verf�gbar, die �brige Funktionalit�t von GIANT
    wird gesperrt.
    
    \step[3] Der Benutzer klickt mit der linken Maustaste an eine
    beliebige Stelle innerhalb des sichtbaren Anzeigeinhaltes des
    Anzeigefensters.

    \step[4]
    GIANT verschiebt die ausgew�hlten Fenster-Knoten an die gew�nschte
    Stelle.
      
  \end{proc}

  \begin{aproc}

    \astep{3} Der Benutzer kann den UseCase abbrechen, indem er
    einen Rechtsklick mit der Maus im sichtbaren Anzeigeinhalt durchf�hrt.
  \end{aproc}


\end{uc}



% ==============================================================================
\begin{uc}[Verschieben einzelner Fenster-Knoten mittels Drag and Drop]{UC: Verschieben einzelner Fenster-Knoten mittels 
                      Drag and Drop}
\index{verschieben!einzelne Knoten}

Mit diesem UseCase k�nnen einzelne Fenster-Knoten mittels Drag and
Drop auf dem sichtbaren Anzeigeinhalt verschoben werden.


  \begin{precond}
    \cond 
    Ein Projekt mit mindestens einem ge�ffneten Anzeigefenster und mit
    mindestens einem Fenster-Knoten ist geladen.
       
  \end{precond}

  \begin{postsuccess}
    
    \cond
    Der einzelne Fensterknoten wurde auf dem Anzeigeinhalt entsprechend 
    verschoben.  

  \end{postsuccess}

  \begin{proc}    
  
    \step[1]
    Der Benutzer bewegt den Mauscursor �ber den zu verschiebenden
    Fenster-Knoten und dr�ckt die linke Maustaste (\gq{Drag}), dann bewegt
    er den Fenster-Knoten an eine beliebige andere Stelle innerhalb des
    Anzeigeinhaltes und l�sst die linke Maustaste los (\gq{Drop}).\\

  \end{proc}

\end{uc}



% ==============================================================================
\begin{uc}[Platz schaffen]{UC: Platz schaffen}
\index{Fenster-Knoten!auseinanderschieben}

Dieser UseCase wird ben�tigt, um Fenster-Knoten auseinander schieben zu 
k�nnen. So kann der Benutzer an einer beliebigen Stelle des Anzeigefensters
gen�gend Platz zum Einf�gen neuer Fenster-Knoten und Fenster-Kanten schaffen.
Siehe auch Abschnitt \ref{Auseinanderschieben von Fenster-Knoten}.


  \begin{precond}
    \cond Ein Projekt mit mindestens einem ge�ffneten 
          Anzeigefenster ist geladen.
      
  \end{precond}

  \begin{postsuccess}
    
    \cond 
    Alle Fenster-Knoten und Fenster-Kanten des Anzeigefensters sind
    um den entsprechenden Betrag vom vorgegebenen Punkt innerhalb
    des Anzeigeinhaltes weggeschoben worden.
    An der entsprechenden Stelle im Anzeigeinhalt ist eine freie Fl�che
    ohne Fenster-Knoten geschaffen worden. Diese Fl�che kann aber von 
    Fenster-Kanten gekreuzt werden.
    
    \cond
    Das Layout aller Fenster-Knoten des Anzeigeinhaltes bleibt ansonsten
    weitgehend unver�ndert.
   
  \end{postsuccess}

 
  \begin{postfail}
    \cond Hat der Benutzer den UseCase abgebrochen,
          so wird der Anzeigeinhalt nicht ver�ndert.
  \end{postfail}  
  
  \begin{proc}    
    \step[1]
    Der Benutzer f�hrt einen Rechtsklick auf eine beliebige Stelle
    des sichtbaren Anzeigeinhaltes durch und w�hlt im
    daraufhin erscheinenden Popup-Men� (siehe \ref{Empty Vis Pane Right click})
    dem Eintrag \gq{Make Room} aus.
    
    \step[2]
    GIANT zeigt in der Statuszeile im Hauptfenster 
    \gq{Select Position in Display Window}.
    Der Benutzer gibt den Punkt um den herum die Fenster-Knoten (und damit
    automatisch auch die Fenster-Kanten) auseinander geschoben werden
    sollen �ber das Fadenkreuz vor (siehe \ref{Fadenkreuzcursor}).
    Hierbei kann er mit dem Fadenkreuz den aktuellen sichtbaren Anzeigeinhalt
    auf dem Anzeigefenster nicht verlassen.
     
    \step[3] 
    GIANT zeigt einen Dialog an, in dem der Benutzer ausw�hlt, um welchen 
    Betrag die Fenster-Knoten auseinander geschoben werden sollen (siehe
    \ref{Platz Schaffen-Dialog}).
   
    \step[4]
    Der Benutzer w�hlt einen geeigneten Betrag aus und best�tigt mit OK.
    
    \step[5]
    GIANT schiebt die Knoten entsprechend auseinander
    (siehe auch \ref{Auseinanderschieben von Fenster-Knoten}).
    
  \end{proc}
  \begin{aproc}

    \astep{2} Der Benutzer kann den UseCase abbrechen, indem er mit der
              Maus einen Rechtsklick auf eine beliebige Stelle 
              durchf�hrt.

    \astep{3} Der Benutzer kann den UseCase durch Bet�tigen des \gq{Cancel}
              Buttons abbrechen.

  \end{aproc}


\end{uc}


% ==============================================================================
\begin{uc}[Pin anlegen]{UC: Pin anlegen}
\index{Pins!anlegen}

Mit diesem UseCase kann ein neuer Pin 
(siehe auch \ref{VIS-PANE-Pins}) erzeugt werden.

 \begin{precond}
    \cond Es gibt ein ge�ffnetes Anzeigefenster.
  \end{precond}

  \begin{postsuccess}
    \cond 
    In der Liste �ber die Pins des Anzeigefensters
    (siehe \ref{VIS-PANE-Pins}) befindet sich ein neuer Pin mit 
    dem vom Benutzer definierten Namen.

  \end{postsuccess}

  \begin{postfail}
    \cond Das System bleibt im bisherigen Zustand.
  \end{postfail}
  
  \begin{proc}    
    \step[1]
    
    Der Benutzer f�hrt einen Rechtsklick auf den Anzeigeinhalt eines
    Anzeigefensters durch und w�hlt aus dem Popup-Men� 
    (siehe \ref{Empty Vis Pane Right click}) den Eintrag \gq{New Pin} aus.
    Der sp�ter erstellte Pin verweist dann auf die Stelle im Anzeigeinhalt,
    auf die der Rechtklick durchgef�hrt wurde.
 

    \step[2] 
    GIANT �ffnet den allgemeinen Texteingabedialog (siehe 
    \ref{DIALOG-WINDOW}).
    
      
    \step[3] 
    Der Benutzer gibt dort einen zul�ssigen Namen f�r den neuen Pin
    ein und best�tigt mit OK (zul�ssige Namen f�r sind in Abschnitt 
    \ref{afa Zulaessige Namen} spezifiziert).
      
    \step[4]
    GIANT speichert die aktuelle Zoomstufe und die Position des sichtbaren
    Anzeigeinhaltes in einem neuen Pin.
  
  \end{proc}

  \begin{aproc}
    \astep{3} Der Benutzer bricht die Verarbeitung mit Cancel ab.
  \end{aproc}

\end{uc}


% ==============================================================================
\begin{uc}[Pin anspringen]{UC: Pin anspringen}
\index{Pins!anspringen}

Stellt die im Pin (siehe auch \ref{VIS-PANE-Pins}) gespeicherte 
Position des sichtbaren Anzeigeinhaltes wieder her.

 \begin{precond}
    \cond Es gibt ein ge�ffnetes Anzeigefenster mit mindestens 
    einem Pin.
  \end{precond}

  \begin{postsuccess}
    \cond 
    Der sichtbare Anzeigefokus des Anzeigefensters ist auf die entsprechenden
    Koordinaten und die entsprechende Zoomstufe, wie sie im ausgew�hlten
    Pin hinterlegt waren, gesetzt.
    
  \end{postsuccess}
  
  \begin{proc}    
    \step[1]
   
    Das Anspringen des Pins kann �ber die folgenden beiden
    M�glichkeiten geschehen:
    \begin{enumerate}
    
      \item
      Der Benutzer f�hrt einen Doppelklick auf den entsprechenden Pin
      in der Pinliste (siehe \ref{VIS-PANE-Pins}) aus.
         
      \item
      Der Benutzer �ffnet ein Popup-Men� durch
      Rechtsklick auf den anzuspringenden Pin 
      in der Pinliste (siehe \ref{VIS-PANE-Pins}) und w�hlt den Eintrag
      \gq{Focus Pin} aus.
        
    \end {enumerate}
    
          
    \step[2]
    GIANT setzt den sichtbaren Anzeigeinhalt gem�� den im Pin gespeicherten
    Informationen.
  
  \end{proc}



\end{uc}


% ==============================================================================
\begin{uc}[Pin l�schen]{UC: Pin l�schen}
\index{Pins!l�schen}

L�scht einen Pin (siehe auch \ref{VIS-PANE-Pins}).

 \begin{precond}
    \cond Es gibt ein ge�ffnetes Anzeigefenster mit mindestens 
    einem Pin.
  \end{precond}

  \begin{postsuccess}
    \cond 
    Der Pin ist gel�scht und nicht mehr in der Pinliste (siehe \ref{VIS-PANE-Pins}) 
    des entsprechenden Anzeigefensters ausw�hlbar.
    
  \end{postsuccess}
  
  \begin{proc}    
    \step[1]
    Der Benutzer �ffnet das  Popup-Men� der Pinliste (siehe \ref{VIS-PANE-Pins})
    durch Rechtsklick auf den Pin  und w�hlt den Eintrag \gq{Delete Pin} aus.
     
    \step[2]
    GIANT l�scht den Pin.
  
  \end{proc}

\end{uc}

%%% Local Variables: 
%%% TeX-master: "../spec"
%%% End: 

% ==============================================================================
%  $RCSfile: additional.tex,v $, $Revision: 1.18 $
%  $Date: 2003/04/19 19:49:06 $
%  $Author: birdy $
%
%  Description: Sonstige UseCases
%
%  Last-Ispelled-Revision: 1.6
%
% ==============================================================================


\begin{uc}[Layout auf Selektion anwenden]{UC: Layout auf Selektion anwenden}
\index{Selektionen!layouten}

Mittels dieses UseCase k�nnen Selektionen innerhalb
eines Anzeigefensters layoutet werden.
Die verf�gbaren Layoutalgorithmen sind im Kapitel 
\ref{Layoutalgorithmen} beschrieben.


  \begin{precond}
    \cond Es gibt mindestens ein ge�ffnetes Anzeigefenster mit
          mindestens einer Selektion (dies kann auch die
          immer vorhandene Standard-Selektion sein).

  \end{precond}

  \begin{postsuccess}
    \cond 
    Die Position der Fenster-Knoten der Selektion wurde gem��
    den Vorgaben des Layoutalgorithmus ge�ndert. 

    \cond
    Alle anderen Fenster-Knoten bleiben unver�ndert.

  \end{postsuccess}

  \begin{postfail}
    \cond Das System bleibt im bisherigen Zustand.

    \cond Hat der Benutzer einen Layoutalgorithmus w�hrend der
    Berechnung des Layouts abgebrochen, 
    so bleiben die Positionen aller Fenster-Knoten des Anzeigefensters
    unver�ndert.
      
  \end{postfail}
  
  \begin{proc}
    
    \step[1] Der Benutzer w�hlt aus der Selektionsauswahlliste (siehe
    \ref{Selektionsauswahlliste}) die Selektion aus, deren
    Fenster-Knoten layoutet werden sollen. Hierzu f�hrt er einen
    Rechtsklick auf die Selektion durch und w�hlt im Popup-Men� den
    Eintrag \gq{Layout Selection} aus.

    \step[2]
    GIANT zeigt einen Dialog zur Auswahl und Konfiguration des
    gew�nschten Layoutalgorithmus 
    (siehe \ref{Layoutalgorithmen-Dialog}).

    \step[3]
    Der Benutzer w�hlt in diesem Dialog den gew�nschten Layoutalgorithmus
    aus.

    \step[4]
    Falls der gew�hlte Layoutalgorithmus dies unterst�tzt, kann der
    Benutzer in diesem Dialog die Klassenmengen, die beim
    Layout ber�cksichtigt werden sollen, vorgeben.

    \step[5]
    Der Benutzer best�tigt seine Eingaben mit OK.
    
    
    \step[6]
    GIANT fordert den Anwender �ber die Statuszeile auf, die Zielposition
    des Layouts anzugeben (\gq{Please click on desired target position for
    Layout in Window}). Der Nutzer klickt mit der linken Maustaste auf
    das Zielgebiet im Fenster.

    \step[7]
    GIANT berechnet das Layout und zeigt einen Progress-Dialog 
    (siehe \ref{Progressbar-Modale}) an. 
    W�hrend der Berechnung des Layouts ist die GUI mit Ausnahme des
    Progress-Dialogs gesperrt.
         
  \end{proc}

  \begin{aproc}
    \astep{3} Der Benutzer bricht die Verarbeitung mit Cancel ab.  
    \astep{4} Der Benutzer bricht die Verarbeitung mit Cancel ab.

    \astep{6} Der Benutzer kann die Berechnung eines Layouts jederzeit mit
    Cancel (siehe \ref{Progressbar-Modale-Cancel}) abbrechen.
  \end{aproc}

\end{uc}


\begin{uc}[Details zu Knoten in einem neuen Informationsfenster
anzeigen]{UC: Details zu Knoten in einem neuen Informationsfenster
anzeigen}

Dieser UseCase erm�glicht dem Benutzer, sich alle verf�gbaren
Informationen zu einem Fenster-Knoten (Kanten und Attribute des
zugeh�rigen IML-Knoten) anzeigen zu lassen. Mit diesem UseCase
kann der Benutzer beliebig viele Knoten-Informationsfenster 
(siehe \ref{Knoten-Informationsfenster}) �ffnen.


  \begin{precond}
    \cond Es gibt mindestens ein ge�ffnetes Anzeigefenster mit
          mindestens einem Fenster-Knoten.
  \end{precond}

  \begin{postsuccess}
    \cond 
    Die Daten �ber den Fenster-Knoten werden in einem neuen
    Knoten-Informationsfenster angezeigt. 

  \end{postsuccess}
 
  \begin{proc}    

    \step[1]
    Der Benutzer w�hlt im entsprechenden Anzeigefenster den 
    gew�nschten Fenster-Knoten aus und f�hrt einen Rechtsklick
    auf diesen Fenster-Knoten durch. Im darauf angezeigten Popup-Men�
    (siehe \ref{Node-Popup-Men�}) w�hlt der Benutzer dann 
    den Eintrag \gq{Show Node Info Window} aus.

    \step[2]
    GIANT �ffnet ein neues Knoten-Informationsfenster
    (siehe \ref{Knoten-Informationsfenster}) und zeigt alle
    verf�gbaren Informationen zu dem IML-Knoten an.
    
  \end{proc}

\end{uc}



\begin{uc}[Details zu Knoten in einem bestehenden
Informationsfenster anzeigen]{UC: Details zu Knoten in einem bestehenden
Informationsfenster anzeigen}

Mittels dieses UseCase kann sich der Benutzer alle verf�gbaren
Informationen zu einem Fenster-Knoten (Kanten und Attribute des
zugeh�rigen IML-Knoten) innerhalb eines bereits ge�ffneten
Knoten-Informationsfensters (siehe \ref{Knoten-Informationsfenster})
anzeigen lassen.


  \begin{precond}
    \cond Es gibt mindestens ein ge�ffnetes Anzeigefenster mit
          mindestens einem Fenster-Knoten.

    \cond Es gibt mindestens ein ge�ffnetes Knoten-Informationsfenster.

  \end{precond}

  \begin{postsuccess}

    \cond 
    Anstatt der Daten, die zuvor in dem Knoten-Informationsfenster
    angezeigt wurden, werden dort nun die Informationen zu dem 
    gew�hlten Fenster-Knoten angezeigt.

  \end{postsuccess}


  \begin{postfail}

    \cond Hat der Benutzer den UseCase abgebrochen, bleibt
          das System im bisherigen Zustand.

  \end{postfail}



  \begin{proc}    

    \step[1]
    Der Benutzer bet�tigt in einem ge�ffneten Knoten-Informationsfenster
    den Button \gq{Pick} (siehe \ref{Knoten-Informationsfenster}).

    \step[2]
    Daraufhin fordert GIANT den Benutzer auf, einen Fenster-Knoten
    auszuw�hlen.

    \step[3]
    Der Benutzer w�hlt den gew�nschten Fenster-Knoten dadurch aus, dass
    er innerhalb des Anzeigeinhaltes eines Anzeigefensters mittels
    eines Fadenkreuzes den Fenster-Knoten als Zielposition
    vorgibt. Das Verfahren zur Vorgabe der Zielposition ist im
    Detail unter \ref{Vorgabe der Zielposition} beschrieben.

    \step[4]
    GIANT stellt die verf�gbaren Informationen zu dem ausgew�hlten
    Fenster-Knoten im Knoten-Informationsfenster dar.
    
  \end{proc}

  \begin{aproc}
    \astep{3} W�hrend der Fadenkreuzcursor aktiv ist, kann der Benutzer 
              den UseCase mit einem Rechtsklick auf eine beliebige
              Stelle abbrechen.  
  \end{aproc}

\end{uc}




\begin{uc}[Anzeige des Quellcodes eines Knotens in einem externen Editor]
{UC: Anzeige des Quellcodes eines Knotens in externem Editor}

Mittels dieses UseCases kann der zu einem IML-Knoten korrespondierende
Quellcode innerhalb eines Editors (unterst�tzt werden Emacs und vi)
zur Anzeige gebracht werden.


  \begin{precond}
    \cond Es gibt mindestens ein ge�ffnetes Anzeigefenster mit
          einem oder mehr Fenster-Knoten.
       
  \end{precond}

  \begin{postsuccess}
    \cond 
    Der zu dem IML-Knoten korrespondierende Quellcode wird in
    dem �ber die Konfigurationsdatei vorgegebenen Editor
    angezeigt (siehe auch \ref{Konfig Editor zur Anzeige des Quellcodes}).
    Der Cursor des Editors ist auf die dem IML-Knoten
    entsprechende Position innerhalb des Quelltextes gesetzt, also
    z.B.\ auf einen Bezeichnernamen, den der IML-Knoten repr�sentiert.

    \cond
    Hat der IML-Knoten keine \gq{Source Code Position}, so wird
    nur die zugeh�rige Quellcode Datei ge�ffnet.
 
  \end{postsuccess}

  
  \begin{proc}    

    \step[1]
    Der Benutzer w�hlt in einem Anzeigefenster den 
    gew�nschten Fenster-Knoten aus und f�hrt einen Rechtsklick
    auf diesen Knoten durch. Im darauf angezeigten Popup-Men�
    (siehe \ref{Node-Popup-Men�}) w�hlt der Benutzer dann 
    den Eintrag \gq{Show Corresponding Source Code} aus.
    Der UseCase kann nicht f�r IML-Knoten durchgef�hrt werden,
    f�r die die Reflektion zur Bauhaus IML-Graph-Bibliothek
    keine zugeh�rige Quellcode-Datei liefern kann.

    \step[2]
    GIANT �ffnet die Quellcode Datei f�r den IML-Knoten in
    einem Editor.
    
  \end{proc}


\end{uc}



\begin{uc}[Verschieben des sichtbaren Anzeigeinhaltes �ber die Minimap]
{UC: Verschieben des sichtbaren Anzeigeinhaltes �ber die Minimap}

Der Benutzer kann mittels der Minimap (siehe auch \ref{GUI Minimap})
den sichtbaren Anzeigeinhalt scrollen.


 \begin{precond}

    \cond Es gibt mindestens ein ge�ffnetes Anzeigefenster.

  \end{precond}

  \begin{postsuccess}
    \cond 
    Der sichtbare Anzeigeinhalt ist auf den Bereich des
    Anzeigefensters gesetzt, der dem Punkt welcher auf der
    Minimap angeklickt wurde, entspricht.

  \end{postsuccess}

  
  \begin{proc}    

    \step[1]
    Der Benutzer klickt mit der linken Maustaste auf einen
    Punkt auf der Minimap.

    \step[2]
    GIANT zentriert den sichtbaren Anzeigeinhalt des Anzeigefensters 
    auf den vorgegebenen Punkt.
    
  \end{proc}


\end{uc}


\chapter{Knoten-Annotationen}
% ==============================================================================
%  $RCSfile: node_annotations.tex,v $
%  $Date: 2003/03/31 22:02:25 $
%  $Author: squig $
%
%  Description: UseCases zum Annotieren von Knoten
%
%  Last-Ispelled-Revision: 1.6
%
% ==============================================================================

\begin{uc}[Fenster-Knoten annotieren]{UC: Fenster-Knoten annotieren}
  Dieser UseCase dient zum Erzeugen von Knoten-Annotationen.
  F�r weitere Informationen siehe auch Abschnitt
  \ref{Project Persistenz von Knoten-Annotationen}. 
  
  \begin{precond}
    \cond Ein Projekt ist geladen.
    \cond Es gibt mindestens ein Anzeigefenster mit Fenster-Knoten.
  \end{precond}

  \begin{postsuccess}
    \cond Die neue Annotation ist noch nicht
          in der Verwaltungsdatei f�r Knoten-Annotationen (siehe
          \ref {Project Verwaltungsdatei f�r Knoten-Annotationen})
          eingetragen. Dies geschieht erst beim n�chsten Speichern
          des gesamten Projektes.

    \cond Der annotierte Knoten wird in allen Anzeigefenstern mittels
          eines Icons als annotiert gekennzeichnet 
          (siehe \ref{Visualsization Icon f�r Knoten-Annotationen}).
          
  \end{postsuccess}

  \begin{postfail}
    \cond Das System bleibt im bisherigen Zustand.
  \end{postfail}
  
  \begin{proc}
    \step[1] 
    Der Benutzer f�hrt einen Rechtsklick auf den 
    Knoten, der annotiert werden soll,
    durch und w�hlt im Popup Men� f�r Knoten den Eintrag
    \gq {Annotate Node} aus (siehe \ref{Node-Popup-Men�}).
   
    \step[2]
    GIANT zeigt nun den Knoten Annotations-Dialog 
    (siehe \ref{Knoten Annotations-Dialog}).
    
    \step[3]
    Der Benutzer gibt dort den gew�nschten Text f�r die Knoten-Annotation 
    ein.
  
    \step[4]
    Nach Abschluss der Eingabe bet�tigt der Benutzer den \gq{OK Button}.
            
    \step[5]
    GIANT �bernimmt die neu Annotation. Die eingegebene Annotation 
    muss allerdings mindestens ein Zeichen haben, ansonsten erscheint eine
    Fehlermeldung.

  \end{proc}

  \begin{aproc}

    \astep{3}
    Der Benutzer kann die Eingabe der neuen Knoten-Annotation jeder Zeit 
    mittels des \gq{Cancel Buttons} abbrechen.
    

  \end{aproc}
\end{uc}

% ==============================================================================

\begin{uc}[Knoten-Annotation �ndern]{UC: Knoten-Annotation �ndern}
  Dient zum �ndern bestehender Knoten-Annotationen (weitere Informationen
  zu Knoten-Annotationen sind unter Abschnitt 
  \ref{Project Persistenz von Knoten-Annotationen} zu finden).
  Dieser UseCase ist auch f�r das Anzeigen von Knoten-Annotationen
  vorgesehen.
  
  \begin{precond}
    \cond Ein Projekt ist geladen.
    \cond Es gibt mindestens einen annotierten Fenster-Knoten
          (solch ein Fenster-Knoten ist mit einen speziellen Icon
           gekennzeichnet 
           -- siehe \ref {Visualsization Icon f�r Knoten-Annotationen}).

  \end{precond}

  \begin{postsuccess}
    \cond Die �nderung an der Annotation ist noch nicht
          in der Verwaltungsdatei f�r Knoten-Annotationen eingetragen
          (siehe \ref {Project Verwaltungsdatei f�r Knoten-Annotationen}).
          Dies geschieht erst beim n�chsten Speichern
          des gesamten Projektes.

    \cond Die �nderung der Annotation ist dem System bekannt und wird
          bei der n�chsten Ausf�hrung dieses UseCases angezeigt.
          
  \end{postsuccess}

  \begin{postfail}
    \cond Das System bleibt im bisherigen Zustand.
  \end{postfail}
  
  \begin{proc}
    \step[1] 
    Der Benutzer f�hrt einen Rechtsklick auf den 
    Knoten, dessen Annotation ge�ndert werden soll,
    durch und w�hlt im Popup Men� f�r Knoten den Eintrag
    \gq {Change Node Annotation} aus (siehe \ref{Node-Popup-Men�}).
 
    
    \step[2]
    GIANT zeigt nun den Knoten Annotations-Dialog 
    (siehe \ref{Knoten Annotations-Dialog}). Hier wird der Text
    f�r die bisherige Annotation dargestellt.
    
    \step[3]
    Der Benutzer �ndert den Text f�r die Annotation entsprechend ab.
    Der Benutzer wird aber nicht gezwungen die bestehende Knoten-Annotation 
    zu �ndern.
  
    \step[4]
    Nach Abschluss der Eingabe bet�tigt der Benutzer den \gq{OK Button}.
            
    \step[5]
    GIANT �bernimmt die vorgenommenen �nderungen an der Annotation. 
    Die neue Annotation muss allerdings mindestens ein Zeichen haben, 
    ansonsten erscheint eine Fehlermeldung.

  \end{proc}

  \begin{aproc}
    \ageneral 
    Der Benutzer kann das �ndern der Annotation jeder Zeit 
    mittels des \gq{Cancel Buttons} abbrechen.
    

  \end{aproc}
\end{uc}

% ==============================================================================

\begin{uc}[Knoten-Annotation l�schen]{UC: Knoten-Annotation l�schen}
  L�scht eine bestehende Knoten-Annotationen (f�r weitere Informationen
  zu Knoten-Annotationen siehe
  \ref{Project Persistenz von Knoten-Annotationen}). 
  
  \begin{precond}
    \cond Ein Projekt ist geladen.
    \cond Es gibt mindestens einen annotierten Fenster-Knoten (
          ist mit einen speziellen Icon gekennzeichnet 
           -- siehe \ref {Visualsization Icon f�r Knoten-Annotationen}).
  \end{precond}

  \begin{postsuccess}
    \cond Die Annotation ist noch nicht aus
          der Verwaltungsdatei f�r Knoten-Annotationen entfernt.
          Dies geschieht erst beim n�chsten Speichern
          des gesamten Projektes.

    \cond Das Entfernen der Annotation ist dem System bekannt und wird
          entsprechend angezeigt.
          
  \end{postsuccess}

  \begin{postfail}
    \cond Das System bleibt im bisherigen Zustand.
  \end{postfail}
  
  \begin{proc}
  
    \step[1] 
    Der Benutzer f�hrt einen Rechtsklick auf den Fenster-Knoten,
    dessen Annotation gel�scht werden soll, durch und w�hlt im Popup 
    Men� f�r Knoten den Eintrag
    \gq {Delete Node Annotation} aus (siehe \ref{Node-Popup-Men�}).
          
    \step[2] 
    GIANT l�scht die entsprechende Annotation.

  \end{proc}


\end{uc}

% ==============================================================================

\begin{uc}[Knoten-Annotationen filtern]{UC: Knoten-Annotationen filtern}
  L�scht alle bestehenden Knoten-Annotationen des Projektes
  f�r die es in den dem Projekt bekannten Anzeigefenstern und 
  IML-Teilgraphen keine Fenster-Knoten und auch keine Graph-Knoten gibt.\\
  Dieser Filter soll es erm�glichen, nicht mehr ben�tigte 
  Annotationen automatisch entfernen zu lassen.
  
  \begin{precond}
     \cond Ein Projekt ist geladen.
  \end{precond}

  \begin{postsuccess}
    \cond Die Annotationen sind noch nicht aus
          der Verwaltungsdatei f�r Knoten-Annotationen 
          (siehe \ref {Project Verwaltungsdatei f�r Knoten-Annotationen})
          entfernt. Dies geschieht erst beim n�chsten Speichern
          des gesamten Projektes.

    \end{postsuccess}

    \begin{postfail}
      \cond Das System bleibt im bisherigen Zustand.
    \end{postfail}
  
    \begin{proc}

     \step[1] 
     Der Benutzer w�hlt im Hauptmen� des 
     Hauptfensters den Eintag 
     \gq{Tools -- Delete Annotations having no visible Node}
     (siehe \ref{Main-Window-Tools}) aus.
    
     \step[2]
     GIANT zeigt eine Sicherheitsabfrage und fordert den Benutzer zur 
     Best�tigung auf (siehe \ref{Sicherheitsabfrage}).
    
     \step[3]
     Der Benutzer best�tigt die Sicherheitsabfrage.
           
     \step[4]
     GIANT l�scht alle Annotationen f�r die es keine Fenster-Knoten und
     Graph-Knoten gibt, d.h. jede Knoten-Annotation deren IML-Knoten weder in 
     einem Anzeigefenster als Fenster-Knoten visualisiert noch 
     als Graph-Knoten Bestandteil eines IML-Teilgraphen
     ist, wird entfernt.

   \end{proc}

   \begin{aproc}

     \astep{2} Der Vorgang kann nat�rlich abgebrochen werden, indem die
     Sicherheitsabfrage nicht best�tigt wird.

   \end{aproc} 
\end{uc}

%%% Local Variables: 
%%% TeX-master: "../spec"
%%% End: 


\chapter{Selektionen}
Hier Text zu Selektionen
% ==============================================================================
%  $RCSfile: selection.tex,v $, $Revision: 1.22 $
%  $Date: 2003/04/20 21:40:36 $
%  $Author: schwiemn $
%
%  Description: UseCases f�r die Selektionen
%
%  Last-Ispelled-Revision: 1.11
%
% ==============================================================================


\begin{uc}[Selektion zur aktuellen Selektion machen]
          {UC: Selektion zur aktuellen Selektion machen}
          
Dieser UseCase dient dazu, eine Selektion zur aktuellen Selektion zu machen.
Dies ist n�tig, da nur die aktuelle Selektion mittels der Maus
modifiziert werden kann (siehe auch \ref{Aktuelle Selektion vs Selektionen}). 
          
         
  \begin{precond}
    \cond Es gibt mindestens ein Anzeigefenster 
          mit mindestens zwei Selektionen.

  \end{precond}

  \begin{postsuccess}
    \cond Die vorherige aktuelle Selektion ist nicht mehr aktuell.
    \cond Die gew�hlte Selektion ist nun die aktuelle Selektion und
          als solche in der Selektionsauswahlliste
          (siehe \ref {Selektionsauswahlliste}) gekennzeichnet.
    \cond Die Fenster-Knoten und Fenster-Kanten der jetzt 
          aktuellen Selektion sind entsprechend hervorgehoben (siehe   
          \ref{Hervorheben von Selektionen und der aktuelle Selektion}).

    
  \end{postsuccess}
 
  \begin{proc}    
    \step[1]
    Der Benutzer f�hrt mit der linken Maustaste einen Doppelklick 
    auf eine nicht aktuelle Selektion in der Selektionsauswahlliste
    (siehe \ref{Selektionsauswahlliste}) durch.
    Ausgeblendete Selektionen (siehe UseCase \ref{Selektionen ausblenden})
    k�nnen nicht zur aktuellen Selektion gemacht werden.

    \step[2]
    GIANT macht die entsprechende Selektion zur aktuellen Selektion.  
  \end{proc}
          
\end{uc}


% ==============================================================================
% \begin{uc}[Aktuelle Selektion zur�ck stufen]
%           {UC: Aktuelle Selektion zur�ck stufen}

%
% Gestrichen nach Beschluss vom Meeting am 2003-04-04.
%
          
% Mit diesem UseCase kann
% eine aktuelle Selektion auf den Status einer \gq{normalen} Selektion
% zur�ckgestuft werden (siehe auch \ref{Aktuelle Selektion vs Selektionen}). 
         
%   \begin{precond}
%     \cond Es gibt ein Anzeigefenster mit einer aktuellen Selektion.
%   \end{precond}

%   \begin{postsuccess}
%     \cond Es gibt keine aktuelle Selektion mehr.
    
%   \end{postsuccess}
 
%   \begin{proc}    
%     \step[1]
%     Der Benutzer f�hrt mit der linken Maustaste einen Doppelklick 
%     auf die aktuelle Selektion in der Selektionsauswahlliste
%     (siehe \ref{Selektionsauswahlliste}) durch..

%     \step[2]
%     GIANT stuft die aktuelle Selektion auf den Status einer \gq{normalen}
%     Selektion zur�ck.
%   \end{proc}
    
% \end{uc}


% ==============================================================================
\begin{uc}[Selektion graphisch hervorheben]
          {UC: Selektion graphisch hervorheben}
          
Dieser UseCase dient zum Hervorheben von Selektionen innerhalb eines
Anzeigefensters (siehe auch \ref {hervorheben von Knoten und Kanten}).

  \begin{precond}
    \cond Es gibt ein Anzeigefenster mit mindestens einer Selektion.
   
  \end{precond}

  \begin{postsuccess}
    \cond 
    Die Selektion ist im entsprechenden Anzeigefenster hervorgehoben.
    
    \cond
    War bereits eine andere Selektion mit der vom Benutzer gew�hlten
    Farbe hervorgehoben (die Farbe, welche im Popup-Men� der
    Selektionsauswahlliste unter \gq{Highlight Selection} gew�hlt wurde 
    -- siehe \ref{Selektionsauswahlliste}), so ist diese Selektion nicht
    mehr hervorgehoben.
 
    
  \end{postsuccess}
 
  \begin{proc}    
    \step[1]
    Der Benutzer startet den UseCase mit einem Rechtsklick 
    auf die gew�nschte Selektion in der
    Selektionsauswahlliste (siehe \ref{Selektionsauswahlliste})
    und w�hlt im zugeh�rigen Popup-Men� das Untermen�
    \gq{Highlight Selection} und dort die gew�nschte Farbe aus.
  
    \step[2]
    GIANT hebt die Selektion mit der ausgew�hlten Farbe hervor.
    
  \end{proc}



\end{uc}


% ==============================================================================
\begin{uc}[Graphische Hervorhebung einer Selektion aufheben]
      {UC: Graphische Hervorhebung einer Selektion aufheben}
      
Dieser UseCase dient dazu, die Hervorhebung von Selektionen innerhalb eines 
Anzeigefensters aufzuheben.
      
  \begin{precond}
    \cond 
    Es gibt ein Anzeigefenster mit mindestens einer hervorgehobenen
    Selektion.
   
  \end{precond}

  \begin{postsuccess}
    \cond 
    Die Selektion ist im entsprechenden Anzeigefenster nicht mehr 
    hervorgehoben.
 
  \end{postsuccess}

  
  \begin{proc}    
    \step[1]    
    Der Benutzer startet den UseCase mit einem Rechtsklick 
    auf die gew�nschte Selektion in der
    Selektionsauswahlliste (siehe \ref{Selektionsauswahlliste})
    und w�hlt im zugeh�rigen Popup-Men� den Eintrag
    \gq{Unhighlight Selection} aus.
  
    \step[2]
    GIANT setzt die Hervorhebung der Selektion zur�ck.
    
  \end{proc} 
      
      
\end{uc}

%===============================================================================
\begin{uc}[Neue Selektion anlegen]{UC: Neue Selektion anlegen}

Mit diesem UseCase k�nnen neue, leere Selektionen angelegt werden.

  \begin{precond}
    \cond Es gibt mindestens ein ge�ffnetes Anzeigefenster.
  \end{precond}

  \begin{postsuccess}
    \cond Eine neue Selektion mit dem vorgegebenen Namen
         ist angelegt und erscheint in der Liste der Selektionen 
         (siehe \ref{Selektionsauswahlliste}).
    \cond Diese neue Selektion hat keinen Inhalt (keine selektierten 
          Fenster-Knoten und Fenster-Kanten).

  \end{postsuccess}

  \begin{postfail}
    \cond Das System bleibt im bisherigen Zustand.
  \end{postfail}
  
  \begin{proc}    
    \step[1]
    Der Benutzer startet den UseCase mit Rechtsklick auf die
    Selektionsauswahlliste (siehe \ref{Selektionsauswahlliste})
    und w�hlt im Popup-Men� den Eintrag
    \gq{New Selection} aus.
    
    \step[2] 
    GIANT �ffnet den allgemeinen Texteingabedialog 
    (siehe \ref{DIALOG-WINDOW}).
      
    \step[3] 
    Der Benutzer gibt dort einen zul�ssigen Namen f�r die neue Selektion 
    ein und best�tigt mit OK (siehe \ref{afa Zulaessige Namen}). \\
    Hat bereits eine andere Selektion innerhalb des Anzeigefensters den selben
    Namen, erscheint eine Fehlermeldung.
    
    \step[4]
    GIANT erzeugt die neue Selektion.
  
  \end{proc}

  \begin{aproc}
    \astep{3} Der Benutzer bricht die Verarbeitung mit Cancel ab.
  \end{aproc}

\end{uc}


%===============================================================================
\begin{uc}[Selektion kopieren]{UC: Selektion kopieren}
Dieser UseCase dient zum Kopieren von Selektionen innerhalb eines 
Anzeigefensters.

  \begin{precond}
    \cond Es gibt mindestens ein ge�ffnetes Anzeigefenster mit 
          mindestens einer Selektion.
  \end{precond}

  \begin{postsuccess}
    \cond 
    Eine neue Selektion mit entsprechendem Namen ist angelegt 
    und erscheint in der Liste der Selektionen
    (siehe \ref{Selektionsauswahlliste}).
    \cond
    Die neue Selektion umfasst die selben Fenster-Knoten und Fenster-Kanten 
    wie die Selektion, von der kopiert wurde.

  \end{postsuccess}

  \begin{postfail}
    \cond Das System bleibt im bisherigen Zustand.
  \end{postfail}
  
  \begin{proc}    
    \step[1]
    Der Benutzer startet den UseCase durch Rechtsklick auf die zu kopierende
    Selektion in der Selektionsauswahlliste (siehe \ref{Selektionsauswahlliste})
    und w�hlt aus dem Popup-Men� den Eintrag \gq{Copy Selection}.
    
    \step[2] 
    GIANT �ffnet den allgemeinen Texteingabedialog 
    (siehe \ref{DIALOG-WINDOW}).
      
    \step[3] 
    Der Benutzer gibt dort einen zul�ssigen Namen f�r die neue Selektion 
    ein und best�tigt mit OK (siehe \ref{afa Zulaessige Namen}).\\
    Hat bereits eine andere Selektion innerhalb des Anzeigefensters den selben
    Namen erscheint eine Fehlermeldung.
    
    \step[4]
    GIANT kopiert die Quellselektion und legt eine neue Selektion an.
  
  \end{proc}

  \begin{aproc}
    \astep{3} Der Benutzer bricht die Verarbeitung mit Cancel ab.
  \end{aproc}

\end{uc}



%===============================================================================
\begin{uc}[Selektion l�schen]{UC: Selektion l�schen}

Dieser UseCase dient zum L�schen von Selektionen innerhalb eines 
Anzeigefensters. Hierdurch bleiben die Fenster-Knoten und Fenster-Kanten
unver�ndert. Die Standard-Selektion (siehe \ref{Standard-Selektion}) kann
nicht gel�scht werden.

  \begin{precond}
     \cond Es gibt mindestens ein ge�ffnetes Anzeigefenster mit 
           mindestens zwei Selektionen.
   \end{precond}


  \begin{postsuccess}
    \cond 
    Die entsprechende Selektion ist gel�scht.

    \cond 
    Wurde die aktuelle Selektion gel�scht (siehe 
    \ref{Aktuelle Selektion vs Selektionen}), so wird die
    Standard-Selektion zur aktuellen Selektion.
    
    \cond 
    Die Fenster-Knoten und Fenster-Kanten, 
    die zur gel�schten Selektion geh�ren, werden nicht gel�scht.
          
    \cond War die Selektion hervorgehoben, so wird die
    Hervorhebung der Fenster-Knoten und Fenster-Kanten aufgehoben.

    \cond
    Waren Fenster-Knoten und Fenster-Kanten der gel�schten Selektion
    ausgeblendet (siehe UseCase \ref{Selektionen ausblenden}), so
    werden diese Fenster-Knoten und Fenster-Kanten wieder eingeblendet.
    
  \end{postsuccess}

  \begin{postfail}
    \cond Das System bleibt im bisherigen Zustand.
  \end{postfail}
  
  \begin{proc}    
    \step[1]
    Der Benutzer startet den UseCase durch Rechtsklick auf die zu l�schende
    Selektion in der Selektionsauswahlliste (siehe 
    \ref{Selektionsauswahlliste})
    und w�hlt aus dem Popup-Men� den Eintrag \gq{Delete Selection}
    aus. Falls der Benutzer die Standard-Selektion 
    (siehe \ref{Standard-Selektion}) ausgew�hlt hat, ist
    der Eintrag \gq{Delete Selection} deaktiviert.
      
    \step[2]
    GIANT l�scht die entsprechende Selektion.
  
  \end{proc}

\end{uc}

%===============================================================================
\begin{uc}[Selektion manuell modifizieren]
         {UC: Selektionen manuell modifizieren}
         
Jeweils die aktuelle Selektion kann mittels der Maus modifiziert 
werden (siehe auch  \ref{Aktuelle Selektion vs Selektionen}).  


  \begin{precond}
     \cond 
     Es gibt mindestens ein ge�ffnetes Anzeigefenster mit mindestens
     einer Selektion.
   \end{precond}


  \begin{postsuccess}
    \cond 
    Die entsprechenden �nderungen an der Selektion werden von GIANT
    sofort durchgef�hrt und �bernommen.
    
  \end{postsuccess}

  
  \begin{proc}    
    \step[1]
    Falls noch nicht der Fall, macht der Benutzer die zu modifizierende 
    Selektion zur aktuellen Selektion
    (siehe \ref{Selektion zur aktuellen Selektion machen}).
    
    \step[2]
    Mittels der unter 
    \ref{Selektieren von Fenster-Knoten und Fenster-Kanten in Anzeigefenstern}
    beschriebenen M�glichkeiten f�gt der Benutzer der Selektion neue
    Fenster-Knoten und Fenster-Kanten hinzu oder entfernt bestehende
    Fenster-Knoten und Fenster-Kanten aus der Selektion.
    
  \end{proc}


\end{uc}

%===============================================================================
\begin{uc}[Selektion aus IML-Teilgraph erzeugen]
         {UC: Selektion aus IML-Teilgraph erzeugen}
         
Leitet eine Selektion aus einem IML-Teilgraphen ab
(siehe \ref{Selektion aus IML-Teilgraphen ableiten}).


  \begin{precond}
     \cond 
     Es gibt mindestens ein ge�ffnetes Anzeigefenster.
     
     \cond
     Es gibt mindestens einen IML-Teilgraphen.
     
   \end{precond}


  \begin{postsuccess}
    \cond 
    Im Ziel-Anzeigefenster wurde eine neue Selektion
    mit dem entsprechenden Namen erzeugt.
    
  \end{postsuccess}
  
  \begin{postfail}
    \cond Das System bleibt im bisherigen Zustand.
  \end{postfail}
  
  \begin{proc} 
     
    \step[1]
    Der Benutzer f�hrt einen Rechtsklick auf den Quell-IML-Teilgraphen 
    in der entsprechenden Liste im Hauptfenster 
    (siehe \ref{GUI Subgraph List}) aus
    und w�hlt im dazugeh�rigen Popup-Men� 
    (siehe \ref{Popup-Men� Subgraph List})
    den Eintrag \gq{Create Window Selection} aus.
        
    \step[2]
    Die Statuszeile im Hauptfenster zeigt an \gq{Please select window for
    Insertion of new window selection}, der Mauszeiger verwandelt sich in
    ein Fadenkreuz.
    
    \step[3]
    Der Nutzer klickt auf den sichtbaren Anzeigeinhalt des
    Anzeigefensters, in dem er die neue
    Selektion erstellen will (Ziel-Anzeigefenster).


    \step[4]
    GIANT zeigt den allgemeinen Texteingabedialog (siehe \ref{DIALOG-WINDOW}).
            
    \step[5] Der Benutzer gibt einen Namen (siehe auch 
             \ref{afa Zulaessige Namen}) 
             f�r die neu zu erstellende Selektion ein und
             best�tigt mit OK.

    \step[6]
    Die Statuszeile im Hauptfenster wechselt wieder zur normalen Anzeige.
    GIANT erzeugt gem�� der unter Abschnitt 
    \ref{Selektion aus IML-Teilgraphen ableiten}
    beschriebenen Konvention im Ziel-Anzeigefenster eine neue Selektion als 
    Ableitung aus dem Quell-IML-Teilgraphen.

  \end{proc}
  
  \begin{aproc}    

    \astep{3} Der Benutzer bricht den UseCase durch einen Rechtsklick ab.
    \astep{5} Der Benutzer bricht den UseCase mit Cancel ab.

  \end{aproc}


\end{uc}



%===============================================================================
\begin{uc}[Mengenoperationen auf 2 Selektionen]
{UC: Mengenoperationen auf 2 Selektionen}
Zus�tzlich zu den M�glichkeiten der Anfragesprache GSL (siehe
\ref {GIANT Scripting Language}) kann der Benutzer
die g�ngigen Mengenoperationen, wie Mengenvereinigung, Schnitt und Differenz,
auch direkt �ber einen entsprechenden Dialog 
(siehe \ref{Common-Set-Operation-Dialog})  ausf�hren.
Das Vorgehen ist analog zu dem UseCase
\ref{Mengenoperationen auf 2 IML-Teilgraphen}.



  \begin{precond}
    \cond Es gibt ein ge�ffnetes Anzeigefenster mit mindestens zwei 
          Selektionen.
  \end{precond}

  \begin{postsuccess}
    \cond 
    Eine neue Selektion mit entsprechendem Namen (eingegeben unter TARGET) 
    ist angelegt 
    und erscheint in der Liste der Selektionen 
    (siehe \ref{Selektionsauswahlliste}).
       
    \cond
    Im Falle einer Mengenvereinigung umfasst, die neue Selektion TARGET alle
    Fenster-Knoten und Fenster-Kanten aus der LEFT\_SOURCE Selektion und der 
    RIGHT\_SOURCE Selektion.
    
    \cond
    Im Falle einer Mengendifferenz umfasst, die neue Selektion TARGET alle
    Fenster-Knoten und Fenster-Kanten aus der LEFT\_SOURCE Selektion, 
    die nicht Bestandteil der  RIGHT\_SOURCE Selektion sind.

    \cond
    Im Falle eines Mengenschnitts umfasst, die neue Selektion TARGET alle
    Fenster-Knoten und Fenster-Kanten, die der LEFT\_SOURCE Selektion und der 
    RIGHT\_SOURCE Selektion gemeinsam angeh�ren.

  \end{postsuccess}

  \begin{postfail}
    \cond Das System bleibt im bisherigen Zustand.
  \end{postfail}
  
  \begin{proc}    
    \step[1]
    Der Benutzer startet den UseCase �ber durch Auswahl des
    Men�punktes \gq{Selection Set Operation} 
    aus dem Popup-Men� in der Selektionsauswahlliste (siehe
    \ref{Selektionsauswahlliste}).
    
    \step[2] 
    GIANT �ffnet den Set-Operation-Dialog 
    (siehe \ref{Common-Set-Operation-Dialog}).
    
    \step[3] 
    Der Benutzer w�hlt dort die beiden Quell-Selektionen (LEFT\_SOURCE und 
    RIGHT\_SOURCE)
    aus, bestimmt die durchzuf�hrende Mengenoperation und gibt unter 
    TARGET den Namen der neu zu erzeugenden Selektion 
    ein (g�ltige Namen siehe \ref{afa Zulaessige Namen}). \\
    Dann best�tigt er die Eingabe mit OK.
       
    \step[4]
    GIANT f�hrt die Mengenoperation aus.
  
  \end{proc}

  \begin{aproc}
    \ageneral 
    Der Benutzer bricht die Eingabe der Daten mit Cancel ab.
  \end{aproc}



\end{uc}



\chapter{Filter}
Hier Text zu Filtern
% ==============================================================================
%  $RCSfile: filter.tex,v $, $Revision: 1.15 $
%  $Date: 2003/04/21 20:54:40 $
%  $Author: schwiemn $
%
%  Description: UseCases f�r Filter
%
%  Last-Ispelled-Revision: 1.5
%
% ==============================================================================

\begin{uc}[Selektionen ausblenden]{UC: Selektionen ausblenden}
Mit diesem UseCase k�nnen Selektionen innerhalb
eines Anzeigefensters ausgeblendet werden.
Die Standard-Selektion (siehe \ref{Standard-Selektion}) und die 
aktuelle Selektion (siehe \ref{Aktuelle Selektion vs Selektionen})
k�nnen nicht ausgeblendet werden.

  \begin{precond}
    \cond Es gibt mindestens ein
          ge�ffnetes Anzeigefenster mit mindestens zwei Selektionen.

  \end{precond}

  \begin{postsuccess}

    \cond Alle zu der Selektion geh�renden Fenster-Knoten und
          Fenster-Kanten sind ausgeblendet, d.h. sie sind
          im Anzeigefenster nicht mehr sichtbar. Dies trifft
          auch f�r Fenster-Knoten und Fenster-Kanten, die
          noch zu weiteren Selektionen geh�ren, zu.
    
    \cond Die ausgeblendeten Selektionen k�nnen nicht 
          zur aktuellen Selektion gemacht werden (siehe 
          \ref {Selektion zur aktuellen Selektion machen}) und damit
          nicht mehr direkt bearbeitet werden, 
          d.h. die Menge der selektierten 
          Fenster-Knoten und Fenster-Kanten kann nicht
          mehr abge�ndert werden (siehe \ref {Selektieren von 
          Fenster-Knoten und Fenster-Kanten in Anzeigefenstern}).


    \cond Die Fenster-Knoten und Fenster-Kanten sind aber immer noch 
          Bestandteil des Anzeigefensters und k�nnen �ber den folgenden
          UseCase (siehe \ref{Selektionen einblenden}) wieder zur Anzeige
          gebracht werden.
    
  \end{postsuccess}

   
  \begin{proc}

    \step[1] 
    Der Benutzer f�hrt einen Rechtsklick mit der Maus auf
    die auszublendende Selektion in der Selektionsauswahlliste 
    (siehe \ref{Selektionsauswahlliste}) durch 
    und w�hlt im Popup-Men� den Eintrag \gq{Hide Selection}
    aus. Diese Funktionalit�t kann nicht auf die aktuelle
    Selektion oder auf die Standard-Selektion angewendet werden. 


    \step[2]
    GIANT blendet die Selektion aus.
 
  \end{proc}

\end{uc}



\begin{uc}[Selektionen einblenden]{UC: Selektionen einblenden}
Mit diesem UseCase k�nnen ausgeblendete Selektionen
wieder eingeblendet werden.

  \begin{precond}
    \cond Es gibt mindestens ein ge�ffnetes Anzeigefenster mit 
          mindestens einer ausgeblendeten Selektion.
  \end{precond}

  \begin{postsuccess}

    \cond 
    Die Selektion ist wieder eingeblendet, alle zu ihr geh�renden 
    Fenster-Knoten und Fenster-Kanten sind im Anzeigeinhalt 
    sichtbar dargestellt (auch wenn sie noch zu weiteren
    Selektionen geh�ren, die ausgeblendet sind).
    
  \end{postsuccess}

   
  \begin{proc}

    \step[1] 
    Der Benutzer f�hrt einen Rechtsklick mit der Maus auf
    eine ausgeblendete Selektion in der Selektionsauswahlliste 
    (siehe \ref{Selektionsauswahlliste}) durch 
    und w�hlt im Popup-Men� den Eintrag \gq{Show Selection}
    aus. 

    \step[2]
    GIANT blendet die Selektion ein.
 
  \end{proc}

\end{uc}


\begin{uc}[Alles einblenden]{UC: Alles einblenden}
Mit diesem UseCase k�nnen alle ausgeblendeten Fenster-Knoten
und Fenster-Kanten eines Anzeigefensters wieder eingeblendet werden.

  \begin{precond}
    \cond Es gibt mindestens ein ge�ffnetes Anzeigefenster mit 
          mindestens einer ausgeblendeten Selektion.

  \end{precond}

  \begin{postsuccess}

    \cond 
    Alle Selektionen des Anzeigefensters sind wieder eingeblendet.

    \cond 
    Alle Fenster-Knoten und Fenster-Kanten sind wieder
    im Anzeigeinhalt des Anzeigefensters sichtbar dargestellt.
    
  \end{postsuccess}

   
  \begin{proc}

    \step[1] 
    Der Benutzer f�hrt einen Rechtsklick mit der Maus auf
    die Selektionsauswahlliste 
    (siehe \ref{Selektionsauswahlliste}) durch 
    und w�hlt im Popup-Men� den Eintrag \gq{Show all}
    aus. 

    \step[2]
    GIANT blendet alle ausgeblendeten Fenster-Knoten und Fenster-Kanten
    wieder ein.
 
  \end{proc}

\end{uc}



\chapter{Teilgraphen}
Hier Text zu Teilgraphen
% ==============================================================================
%  $RCSfile: subgraph.tex,v $, $Revision: 1.17 $
%  $Date: 2003/04/17 22:44:59 $
%  $Author: schwiemn $
%
%  Description: UseCases f�r IML-Teilgraphen
%
%  Last-Ispelled-Revision: 1.10
%
% ==============================================================================

\begin{uc}[IML-Teilgraph graphisch hervorheben]
      {UC: IML-Teilgraph graphisch hervorheben}
\index{IML-Teilgraph!hervorheben}         
Dieser UseCase dient zum Hervorheben von IML-Teilgraphen innerhalb der
Anzeigefenster (siehe auch \ref {hervorheben von Knoten und Kanten}).

  \begin{precond}
    \cond Es gibt mindestens einen IML-Teilgraphen.
   
  \end{precond}

  \begin{postsuccess}
    \cond 
    Der IML-Teilgraph ist in jedem ge�ffneten Anzeigefenster entsprechend
    hervorgehoben.
      
    \cond
    Der IML-Teilgraph, welcher vorher mit der gleichen Farbe 
    hervorgehoben war, ist nicht mehr hervorgehoben.
 
    \cond
    Fenster-Knoten und Fenster-Kanten, die als Graph-Knoten bzw. Graph-Kanten
    Bestandteil eines hervorgehobenen IML-Teilgraphen sind, werden,
    auch wenn sie erst nach der Hervorhebung des IML-Teilgraphen in ein
    Anzeigefenster eingef�gt werden, ebenfalls automatisch hervorgehoben.

    
  \end{postsuccess}
 
  \begin{proc}    
    \step[1]
    Der Benutzer startet den UseCase �ber das Popup-Men� der 
    Liste �ber die IML-Teilgraphen (siehe \ref{GUI Subgraph List})
    durch Rechtsklick auf den gew�nschten IML-Teilgraphen.
    In dem Popup Men� (siehe \ref {Popup-Men� Subgraph List})
    w�hlt er das Untermen� \gq {Highlight} und dort die gew�nschte
    Farbe aus.
            
    \step[2]
    GIANT hebt den IML-Teilgraphen in allen ge�ffneten Anzeigefenstern
    mit der ausgew�hlten Farbe hervor.
    
  \end{proc}


\end{uc}



% ==============================================================================
\begin{uc}[Graphische Hervorhebung von IML-Teilgraphen aufheben]
      {UC: Graphische Hervorhebung von IML-Teilgraphen aufheben}
\index{IML-Teilgraph!Hervorhebung aufheben}  
Mit diesem UseCase kann die graphische Hervorhebung von IML-Teilgraphen 
aufgehoben werden.

  \begin{precond}
    \cond 
    Es gibt mindestens einen hervorgehobenen IML-Teilgraphen.
   
  \end{precond}

  \begin{postsuccess}
    \cond 
    Der IML-Teilgraph ist nicht mehr hervorgehoben.

  \end{postsuccess}

  
  \begin{proc}    
    \step[1]
    Der Benutzer startet den UseCase durch Rechtsklick auf den
    hervorgehobenen IML-Teilgraphen in der Liste �ber
    die IML-Teilgraphen (siehe \ref{GUI Subgraph List})
    In dem zugeh�rigen Popup-Men� (siehe \ref{Popup-Men� Subgraph List})
    w�hlt er den Eintrag \gq{Unhighlight In All Windows}
    aus.

    \step[2]
    GIANT setzt die Hervorhebung des IML-Teilgraphen zur�ck.
    
  \end{proc} 


\end{uc}


% ==============================================================================
\begin{uc}[IML-Teilgraph aus einer Selektion erzeugen]
  {UC: IML-Teilgraph aus einer Selektion erzeugen}
  \index{IML-Teilgraph!aus Selektion erzeugen} 
  
  Leitet einen neuen IML-Teilgraphen aus einer Quell-Selektion ab
  (siehe auch Abschnitt \ref{IML-Teilgraphen aus Selektion ableiten}.

  \begin{precond}
     \cond 
     Es gibt mindestens ein ge�ffnetes Anzeigefenster mit 
     mindestens einer Selektion.
     
   \end{precond}


  \begin{postsuccess}
    \cond 
    Es wurde ein neuer IML-Teilgraph mit entsprechendem Namen erzeugt.
    
  \end{postsuccess}
  
  \begin{postfail}
    \cond Das System bleibt im bisherigen Zustand.
  \end{postfail}
  
  \begin{proc} 
     
    \step[1]
    Der Benutzer f�hrt einen Rechtsklick auf die Quell-Selektion in
    der Selektionsauswahlliste (siehe \ref{Selektionsauswahlliste}) aus
    und w�hlt im Popup-Men� den Eintrag 
    \gq{Create New IML Subgraph from This Selection} aus.
     
    \step[2]
    GIANT zeigt den allgemeinen Texteingabedialog (siehe \ref{DIALOG-WINDOW})
    an.
     
    \step[3]
    Der Benutzer gibt einen Namen f�r den neu zu erstellenden IML-Teilgraphen
    ein und best�tigt mit OK (g�ltige Namen f�r IML-Teilgraphen sind 
    unter Abschnitt \ref{afa Zulaessige Namen} spezifiziert).\\
    Gibt der Benutzer hier keinen Namen ein, so vergibt GIANT automatisch 
    einen Namen.
     
    \step[4]
    GIANT erzeugt gem�� der unter Abschnitt \ref{IML-Teilgraphen aus 
    Selektion ableiten}
    beschriebenen Konvention eine neuen IML-Teilgraphen aus der 
    Quell-Selektion.

  \end{proc}
  
  \begin{aproc}
    \astep{3} Der Benutzer bricht den UseCase mit Cancel ab.
  \end{aproc}

\end{uc}



% ==============================================================================
\begin{uc}[IML-Teilgraph kopieren] {UC: IML-Teilgraph kopieren}
\index{IML-Teilgraph!kopieren}      
 Kopiert einen Quell-IML-Teilgraphen in einen neuen IML-Teilgraphen.
 Bestehende IML-Teilgraphen k�nnen nicht �berschrieben werden.

  \begin{precond}
     
     \cond
     Es gibt mindestens einen IML-Teilgraphen.
     
   \end{precond}


  \begin{postsuccess}
    \cond 
    Es wurde ein neuer IML-Teilgraph mit entsprechendem Namen erzeugt.
    
    \cond
    Der neue IML-Teilgraph hat alle Graph-Knoten und Graph-Kanten des
    Quell-IML-Teilgraphen.
    
  \end{postsuccess}
  
  \begin{postfail}
    \cond Das System bleibt im bisherigen Zustand.
  \end{postfail}
  
  \begin{proc} 
     
    \step[1]
    Der Benutzer f�hrt einen Rechtsklick auf den zu kopierenden 
    Quell-IML-Teilgraphen in der Liste �ber die IML-Teilgraphen
    im Hauptfenster aus (siehe \ref{GUI Subgraph List}) und
    w�hlt im zugeh�rigen Popup-Men� (siehe \ref{SUBGRAPH-LIST-POPUP}) 
    den Eintrag \gq{Copy IML Subgraph} aus.
     
    \step[2]
    GIANT zeigt den allgemeinen Texteingabedialog (siehe \ref{DIALOG-WINDOW}).
            
    \step[3]
    Der Benutzer gibt einen Namen f�r den neu zu erstellenden IML-Teilgraphen
    ein und best�tigt mit OK 
    (g�ltige Namen siehe \ref{afa Zulaessige Namen}).\\
    Gibt der Benutzer hier keinen Namen ein, so vergibt GIANT automatisch 
    einen Namen.\\
    Gibt der Benutzer den Namen eines bereits vorhandenen IML-Teilgraphen
    ein, so erscheint eine Fehlermeldung.
       
    \step[4]
    GIANT kopiert den Quell-IML-Teilgraphen in einen neuen Teilgraphen.

  \end{proc}
  
  \begin{aproc}
    \astep{3} Der Benutzer bricht den UseCase mit Cancel ab.
  \end{aproc}

\end{uc}



% ==============================================================================
\begin{uc}[IML-Teilgraph l�schen]{UC: IML-Teilgraph l�schen}
\index{IML-Teilgraph!l�schen}
Dieser UseCase l�scht einen IML-Teilgraphen. Die zugeh�rigen
Fenster-Knoten und Fenster-Kanten in den Anzeigefenstern bleiben
davon unber�hrt.

  \begin{precond}
     
     \cond
     Es gibt mindestens einen IML-Teilgraphen.
     
   \end{precond}


  \begin{postsuccess}
    \cond 
    Der IML-Teilgraph wurde aus der Liste �ber die IML-Teilgraphen
    gel�scht (siehe \ref{GUI Subgraph List}).
    
    \cond
    War der gel�schte IML-Teilgraph hervorgehoben, so wurde die
    Hervorhebung der zugeh�rigen Fenster-Knoten und Fenster-Kanten
    aufgehoben.

  \end{postsuccess}
  

  
  \begin{proc} 
     
    \step[1]
    Der Benutzer f�hrt einen Rechtsklick auf den zu l�schenden
    IML-Teilgraphen aus und w�hlt im Popup-Men� der
    Liste �ber die IML-Teilgraphen den Eintrag 
    \gq{Delete IML Subgraph} aus (siehe \ref {Popup-Men� Subgraph List}).
      
    \step[2]
    GIANT l�scht den gew�hlten IML-Teilgraphen.

  \end{proc}

\end{uc}


% ==============================================================================
\begin{uc}[Mengenoperationen auf 2 IML-Teilgraphen]
      {UC: Mengenoperationen auf 2 IML-Teilgraphen}
\index{IML-Teilgraph!Mengenoperationen}

Erg�nzend zu den M�glichkeiten der Anfragesprache 
(siehe Kapitel \ref{GIANT Query Skripting Language}) kann der Benutzer
die g�ngigen Mengenoperationen, wie Mengenvereinigung, 
Schnitt und Differenz, auch direkt �ber einen Dialog 
ausf�hren.
F�r eine genaue Beschreibung des Dialoges siehe auch
Abschnitt \ref{Common-Set-Operation-Dialog}.
Das Vorgehen innerhalb dieses UseCases ist im Wesentlichen 
analog zu dem UseCase \ref{Mengenoperationen auf 2 Selektionen}.


  \begin{precond}
    \cond Es gibt mindestens zwei IML-Teilgraphen.
  \end{precond}

  \begin{postsuccess}
    \cond 
    Eine neuer IML-Teilgraph mit entsprechendem Namen 
    (im Dialog eingegeben unter TARGET) ist angelegt 
    und erscheint in der Liste �ber alle IML-Teilgraphen des Projektes
    (siehe \ref{GUI Subgraph List}).
       
    \cond
    Im Falle einer Mengenvereinigung umfasst
    der neue IML-Teilgraph TARGET alle
    Graph-Knoten und Garph-Kanten aus dem LEFT\_SOURCE IML-Teilgraphen und dem 
    RIGHT\_SOURCE IML-Teilgraphen.
    
    \cond
    Im Falle einer Mengendifferenz umfasst der neue IML-Teilgraph TARGET alle
    Graph-Knoten und Graph-Kanten aus dem LEFT\_SOURCE IML-Teilgraphen, 
    die nicht 
    Bestandteil des RIGHT\_SOURCE IML-Teilgraphen sind.\\

    \cond
    Im Falle eines Mengenschnitts umfasst der neue IML-Teilgraph TARGET alle
    Graph-Knoten und Graph-Kanten, 
    die sowohl dem LEFT\_SOURCE IML-Teilgraphen als auch 
    dem RIGHT\_SOURCE IML-Teilgraphen angeh�ren.

    \cond
    H�tte der IML-Teilgraph TARGET nach der Mengenoperation Graph-Kanten 
    ohne zugeh�rigen Start- und Zielknoten, so
    sind diese Graph-Kanten aus dem IML-Teilgraphen TARGET entfernt worden
    (dieser Fall kann bei Mengenschnitt und Mengendifferenz eintreten).

  \end{postsuccess}

  \begin{postfail}
    \cond Das System bleibt im bisherigen Zustand.
  \end{postfail}
  
  \begin{proc}    
    \step[1]
    Der Benutzer startet den UseCase �ber das Popup-Men� 
    der Liste �ber die IML-Teilgraphen  
    (siehe \ref {Popup-Men� Subgraph List}) indem er
    einen Rechtsklick innerhalb der Liste ausf�hrt und  
    im Popup Men� den Eintrag \gq{IML Subgraph Set Operation}
    ausw�hlt.
    
    \step[2] 
    GIANT �ffnet den Set-Operation-Dialog 
    (siehe \ref{Common-Set-Operation-Dialog}).
    
    
    \step[3] 
    Der Benutzer w�hlt dort die beiden Quell-IML-Teilgraphen 
    (LEFT\_SOURCE und RIGHT\_SOURCE) aus, bestimmt die auszuf�hrende
    Mengenoperation und gibt unter TARGET den Namen 
    des neu zu erzeugenden IML-Teilgraphen ein (g�ltige
    Namen f�r IML-Teilgraphen siehe \ref{afa Zulaessige Namen}).\\
    Er best�tigt mit OK.\\
    Existiert bereits ein IML-Teilgraph mit dem unter TARGET eingegebenen 
    Namen, so erscheint eine Fehlermeldung.
       
    \step[4]
    GIANT f�hrt die Mengenoperation aus und erzeugt den neuen
    IML-Teilgraphen.
  
  \end{proc}

  \begin{aproc}
    \astep{3} Der Benutzer bricht die Eingabe der Daten mit Cancel ab.
  \end{aproc}


\end{uc}



\chapter{Anfragen}
Hier Text zu Anfragen
% ==============================================================================
%  $RCSfile: query.tex,v $, $Revision: 1.18 $
%  $Date: 2003/04/19 18:46:33 $
%  $Author: birdy $
%
%  Description: UseCases f�r die Anfragen
%
%  Last-Ispelled-Revision: 1.7
%
% ==============================================================================

% ==============================================================================
\begin{uc}[Anfrage ausf�hren]{UC: Neues Skript ausf�hren}
\index{Skript!ausf�hren}
Mit diesem UseCase kann ein Skript �ber den Skriptdialog
(siehe \ref{GUI Anfragedialog}) eingegeben werden.
Die M�glichkeiten der GSL sind im Detail in Kapitel
\ref {GIANT Scripting Language} beschrieben.


  \begin{precond}
    \cond Ein Projekt ist geladen.
  \end{precond}

  \begin{postsuccess}
    \cond Das Skript wurde ausgef�hrt. Alle Ergebnisse liegen vor.
      
  \end{postsuccess}

  \begin{postfail}
    \cond Wurde der UseCase vor Beginn der Berechnung des Skriptes
          abgebrochen, bleibt das System im bisherigen Zustand.

    \cond Wurde der UseCase w�hrend der Ausf�hrung des Skriptes 
          abgebrochen, so werden bereits ausgef�hrte Aktionen
          (wie z.B. das Einf�gen neuer Fenster-Knoten in Anzeigefenster)
          nicht wieder r�ckg�ngig gemacht.

  \end{postfail}
  
  \begin{proc}    
    \step[1]
    Der Benutzer startet den UseCase durch Auswahl des Eintrags 
    \gq{Execute GSL Script} im Men� Tools
    (siehe \ref{Main-Window-Tools}).
      
    \step[2] 
    GIANT �ffnet den Skriptdialog (siehe \ref{GUI Anfragedialog}).
      
    \step[3]
    Der Benutzer gibt dort im daf�r vorgesehenen Textfeld das GSL Skript
    (siehe auch Kapitel \ref{GIANT Scripting Language}) ein und 
    best�tigt mit \gq {OK}.

    \step[4]
    GIANT pr�ft das eingegebene GSL Skript. Sollte das Skript
    nicht den Vorgaben der Grammatik 
    (siehe Kapitel \ref{GIANT Scripting Language}) entsprechen, erscheint
    eine Fehlermeldung (siehe \ref{afa Fehlerverhalten})
    und das System kehrt zu Schritt 3 des UseCase zur�ck.
        
    \step[5]    
    GIANT f�hrt das Skript aus und teilt dem Benutzer
    den Fortschritt mittels eines Progress-Dialogs 
    (siehe \ref{Progressbar-Modale}) mit.
    W�hrend der Ausf�hrung des Skripts ist das System mit Ausnahme des
    Progress-Dialogs gesperrt.\\
    
  
  \end{proc}

  \begin{aproc}
    \astep{3} Der Benutzer bricht mit Cancel ab.
    
    \astep{5} Die laufende Ausf�hrung des Skripts kann vom Benutzer
    mit Cancel abgebrochen werden.

  \end{aproc}

\end{uc}


% ==============================================================================
\begin{uc}[UC Anfrage laden]{UC: Skript laden}
\index{Anfragen!aus Datei laden}
Der Benutzer kann zus�tzlich zur manuellen Eingabe von GSL Skripten
(siehe \ref{Anfrage ausf�hren}) auch gespeicherte Skripte aus 
einer Datei (siehe \ref {Config Anfrage-Dateien}) laden.

  \begin{precond}
    \cond Ein Projekt ist geladen.
  \end{precond}

  \begin{postsuccess}
    \cond Das Skript wurde ausgef�hrt. Alle Ergebnisse liegen vor.
  \end{postsuccess}

  \begin{postfail}

     \cond Wurde der UseCase vor Beginn der Berechnung des Skriptes
          abgebrochen, bleibt das System im bisherigen Zustand.

    \cond Wurde der UseCase w�hrend der Ausf�hrung des Skriptes 
          abgebrochen, so werden bereits ausgef�hrte Aktionen
          (wie z.B. das Einf�gen neuer Fenster-Knoten in Anzeigefenster)
          nicht wieder r�ckg�ngig gemacht.

  \end{postfail}
  
  \begin{proc}    
    \step[1]
    Der Benutzer startet den UseCase durch Auswahl des Eintrags 
    \gq{Execute GSL Script} im Men� Tools
    (siehe \ref{Main-Window-Tools}).
      
    \step[2] 
    GIANT �ffnet den Skriptdialog (siehe \ref{GUI Anfragedialog}).
      
    \step[3] 
    Der Benutzer bet�tigt im Dialog den Button \gq{Open...}.

    \step[4]
    Daraufhin zeigt GIANT den Standard-Filechooser-Dialog 
    (siehe \ref {Standard-Filechooser-Dialog}).
    
    \step[5] Der Benutzer w�hlt die Datei (siehe \ref {Config
      Anfrage-Dateien}) aus.
        
    \step[6]
    GIANT zeigt das aus der Datei geladene GSL Skript 
    (siehe Kapitel \ref {GIANT Scripting Language}) im Textfeld
    des Skriptdialogs an.

    \step[7]
    Falls gew�nscht kann der Benutzer das GSL Skript im Textfeld
    noch manuell weiter modifizieren.
    
    \step[8]
    Der Benutzer startet die Berechnung des Skripts durch Bet�tigung
    des \gq{Start Query} im Skriptdialog (siehe \ref{GUI Anfragedialog}).

    \step[9]
    GIANT pr�ft das geladene und eventuell modifizierte GSL Skript. 
    Sollte das Skript nicht den Vorgaben der Grammatik 
    (siehe Kapitel \ref {GIANT Scripting Language}) entsprechen, erscheint
    eine Fehlermeldung (siehe \ref {afa Fehlerverhalten})
    und das System kehrt zu Schritt 7 des UseCase zur�ck.

    \step[10]    
    GIANT f�hrt das Skript aus und teilt dem Benutzer
    den Fortschritt mittels eines Progress-Dialogs 
    (siehe \ref{Progressbar-Modale}) mit.
    W�hrend der Ausf�hrung des Skripts ist das System mit Ausnahme des
    Progress-Dialogs gesperrt.

  \end{proc}

  \begin{aproc}
    \astep{3} Der Benutzer bricht den UseCase mit Cancel ab.
    \astep{4} Der Benutzer bricht die Auswahl der Datei mit Cancel ab.
              Das System kehrt dann zu Schritt 2 bei der
              Abarbeitung des UseCase zur�ck.
    \astep{6} Der Benutzer bricht den UseCase mit Cancel ab.

    \astep{10} Die laufende Ausf�hrung des Skripts kann vom Benutzer
    mit Cancel abgebrochen werden.
  \end{aproc}

\end{uc}

% ==============================================================================
\begin{uc}[Anfrage speichern]{UC: Skript speichern}
\index{Anfragen!in eine Datei speichern}
Mit diesem UseCase kann der Benutzer GSL Skripte aus dem 
Skriptdialog (siehe \ref{GUI Anfragedialog}) in Dateien 
(siehe \ref {Config Anfrage-Dateien}) speichern.

  \begin{precond}
    \cond Der Skriptdialog (siehe \ref{GUI Anfragedialog}) ist
          ge�ffnet und enth�lt in dem daf�r vorgesehenen Textfeld
          entweder ein manuell eingegebenes oder ein aus einer
          Datei geladenes und eventuell modifiziertes GSL Skript.

  \end{precond}

  \begin{postsuccess}
    \cond Eine Datei, welche das GSL Skript enth�lt, wurde
          angelegt.
    \cond GIANT zeigt den Skriptdialog, das gespeicherte GSL Skript
          ist weiterhin in dem Textfeld vorhanden.
      
  \end{postsuccess}

  \begin{postfail}
    \cond Das System bleibt im bisherigen Zustand.
    \cond Es wurde keine Datei erzeugt
    \cond GIANT zeigt weiterhin den Skriptdialog an
          (siehe \ref{GUI Anfragedialog}) und alle dort get�tigten
          Eingaben (insbesondere das GSL Skript im Textfeld des
          Dialoges) bleiben erhalten.
   
  \end{postfail}
  
  \begin{proc}    

    \step[1]
    Der Benutzer bet�tigt im Skriptdialog (siehe \ref{GUI Anfragedialog})
    den Button \gq{Save As...}.
    
    \step[2]
    GIANT pr�ft ob das GSL Skript im Textfeld des Skriptdialogs den
    Vorgaben der Grammatik (siehe Kapitel \ref {GIANT Scripting Language}) 
    entspricht. Falls nicht, erscheint eine Fehlermeldung 
    (siehe \ref {afa Fehlerverhalten})
    und das System kehrt zu Schritt 1 zur�ck.

    \step[3]
    GIANT �ffnet den Standard-Filechooser-Dialog (siehe 
    \ref {Standard-Filechooser-Dialog}).
  
    \step[4]
    Der Benutzer gibt den Pfad und die Datei, in der das GSL Skript
    gespeichert werden soll, vor und best�tigt mit OK.

    \step[5]
    GIANT speichert das GSL Skript in der vorgegebenen Datei.

  \end{proc}

  \begin{aproc}
    \astep{4} Der Benutzer bricht den UseCase mit Cancel ab.

  \end{aproc}

\end{uc}





%===============================================================================
% 
% Anfragesprache
%
\chapter{GIANT Query \& Skripting Language}
% =============================================================================
%  $RCSfile: language.tex,v $, $Revision: 1.4 $
%  $Date: 2003/04/06 04:23:11 $
%  $Author: keulsn $
%
%  Description:
%
%  Last-Ispelled-Revision: 1.1
%
% =============================================================================

%%%% Non-terminals
\nonterminal{\gqslargumentlist}{argument\_list}
\nonterminal{\gqslassignment}{assignment}
\nonterminal{\gqslcalculation}{calculation}
\nonterminal{\gqslcomplexstatement}{complex\_statement}
\nonterminal{\gqsldeclaration}{declaration}
\nonterminal{\gqsldeclarationpart}{declaration\_part}
\nonterminal{\gqsldigit}{digit}
\nonterminal{\gqsledgepredicate}{edge\_predicate}
\nonterminal{\gqsledgesetexpression}{edge\_set\_expression}
\nonterminal{\gqslexpression}{expression}
\nonterminal{\gqslidentifier}{identifier}
\nonterminal{\gqslintliteral}{int\_literal}
\nonterminal{\gqslletter}{letter}
\nonterminal{\gqslliteralexpression}{literal\_expression}
\nonterminal{\gqslnodepredicate}{node\_predicate}
\nonterminal{\gqslnodesetexpression}{node\_set\_expression}
\nonterminal{\gqslparameterlist}{parameter\_list}
\nonterminal{\gqslquery}{query}
\nonterminal{\gqslquerybody}{query\_body}
\nonterminal{\gqslquerycall}{query\_call}
\nonterminal{\gqslregexp}{regexp}
\nonterminal{\gqslregexpchar}{regexp\_char}
\nonterminal{\gqslscript}{script}
\nonterminal{\gqslscriptbody}{script\_body}
\nonterminal{\gqslscriptcall}{script\_call}
\nonterminal{\gqslselectionexpression}{selection\_expression}
\nonterminal{\gqslstatement}{statement}
\nonterminal{\gqslstringchar}{string\_char}
\nonterminal{\gqslstringliteral}{string\_literal}
\nonterminal{\gqslsubgraphexpression}{subgraph\_expression}
\nonterminal{\gqsltype}{type}
\nonterminal{\gqslvariableinspection}{variable\_inspection}

%%%% Tokens
\token{\tokaccept}{accept}
\token{\tokallnodes}{all\_nodes}
\token{\tokbased}{based}
\token{\tokbegin}{begin}
\token{\tokbegincomment}{//}
\token{\tokbeginrequote}{<}
\token{\tokbuild}{build}
\token{\tokclosebracket}{)}
\token{\tokcolon}{:}
\token{\tokcomma}{,}
\token{\tokdepth}{depth}
\token{\tokequal}{=}
\token{\tokedges}{edges}
\token{\tokedgeset}{edge\_set}
\token{\tokend}{end}
\token{\tokendrequote}{>}
\token{\tokfollow}{follow}
\token{\tokfrom}{from}
\token{\tokgraph}{graph}
\token{\tokintersect}{intersect}
\token{\tokinteger}{integer}
\token{\tokis}{is}
\token{\tokmake}{make}
\token{\tokminus}{-}
\token{\toknodes}{nodes}
\token{\toknodeset}{node\_set}
\token{\tokon}{on}
\token{\tokopenbracket}{(}
\token{\tokplus}{+}
\token{\tokquery}{query}
\token{\tokquote}{''}
\token{\tokreturn}{return}
\token{\tokrootnode}{root\_node}
\token{\tokscript}{script}
\token{\tokselect}{select}
\token{\tokselection}{selection}
\token{\toksemicolon}{;}
\token{\tokstring}{string}
\token{\toksubgraph}{subgraph}
\token{\tokto}{to}
\token{\tokuntil}{until}

\section{IML-Graphen}

Der Kunde stellt die Reflektion zur Verf�gung. GIANT verwendet diese
Bibliothek, um auf IML-Graph Dateien zuzugreifen.

\subsection{IML-Programme}
\index{IML-Graph Datei!Inhalt}

In jeder IML-Graph Datei ist ein IML-Programm enthalten. Das IML-Programm
besteht aus
\begin{enumerate}
\item IML-Knoten
\item Attributen von IML-Knoten
\item nichts sonst.
\end{enumerate}

Es gibt in jedem IML-Programm einen besonderen IML-Knoten, der Wurzelknoten
genannt wird.

\subsubsection{Attribute von IML-Knoten}

Jeder IML-Knoten verf�gt �ber eine Folge von Attributen.
Diese Folge definiert eine endliche Anzahl einzelner Attribute,
die der IML-Knoten besitzt sowie eine Reihenfolge dieser Attribute.
Weder die Folge von Attributen noch die in der Folge enthaltenen
Attribute eines IML-Knoten k�nnen sich w�hrend der Laufzeit von
GIANT ver�ndern.

Jedes Attribut besitzt einen Attribut-Namen. Kein IML-Knoten besitzt
zwei verschiedene Attribute, die einen gleichen Namen besitzen.

Einige Attribute sind Verweise. Sie verweisen auf h�chstens einen
IML-Knoten. Es werden unterschieden:
\begin{description}
\item[genutzte Verweise] Verweise, die auf genau einen
           IML-Knoten verweisen
\item[ungenutzte Verweise] Verweise, die nicht auf einen
           IML-Knoten verweisen
\end{description}

Jedes Attribut eines IML-Knoten geh�rt genau einer der folgenden Klassen an:
\begin{description}
\item[Einfache Attribute]: Haben einen bestimmten Wert.
           M�gliche Typen dieses Werts sind:
           \begin{enumerate}
           \item Source Location
                 \begin{itemize}
                 \item Zeilennummer
                 \item Spaltennummer
                 \item Filename
                 \item Pfadname
                 \end{itemize}
           \item Boolean
           \item Natural
           \item Folge von Strings
           \end{enumerate}
\item[Bezeichner-Attribut] Haben als Wert einen besonderen String
\item[Genutztes Verweis-Attribut] Ein genutzter Verweis
\item[Ungenutztes Verweis-Attribut] Ein ungenutzter Verweis
\item[Verweisfolgen-Attribut] Eine endliche, nicht leere Folge von
           genutzten Verweisen
\item[Verweismengen-Attribut] Eine endliche, nicht leere Menge
           von genutzten Verweisen
\end{description}



\subsection{IML-Graphen}

Jedes IML-Programm definiert einen zugeh�rigen IML-Graph.
Sprechweise: Das IML-Programm liegt dem IML-Graph zugrunde.
Ein IML-Graph ist ein gerichteter knoten- und kantenannotierter Graph mit
Schlingen und Mehrfachkanten. Er ist definiert durch die folgende
Vorschrift:

\begin{enumerate}
\item
Ein IML-Graph besitzt als Knotenmenge die Menge alle IML-Knoten des
zugrunde liegenden IML-Programms. Der IML-Graph besitzt keine
weiteren Knoten.
\item
Annotationen eines IML-Knoten sind alle Attribute des IML-Knoten.
\item
\label{language_edge_spec}
Eine Kante im IML-Graph hei�t IML-Kante. Eine IML-Kante $e$ von einem
IML-Knoten $v$ zu einem IML-Knoten $w$ des selben IML-Graph existiert
genau dann, wenn eine der folgenden Bedingungen erf�llt ist:
   \begin{enumerate}
   \item $v$ besitzt ein Genutztes Verweis-Attribut, das auf $w$ verweist.
         $e$ besitzt als Annotation den Attribut-Namen des Genutzten
         Verweis-Attributs.
   \item $v$ besitzt ein Verweisfolgen-Attribut $f$. Ein genutzter Verweis
         aus $f$ verweist auf $w$. $e$ besitzt als Annotation den
         Attribut-Namen von $f$ sowie die Nummer innerhalb der Folge
         $f$ des genutzten Verweises auf $w$.
   \item $v$ besitzt ein Verweismengen-Attribut $m$. Ein genutzter Verweis
         aus $m$ verweist auf $w$. $e$ besitzt als Annotation den
         Attribut-Namen von $m$
   \end{enumerate}
\item
Falls eine IML-Kante $e$ existiert, so hat $e$ keine Annotationen,
au�er den in Punkt \ref{language_edge_spec} genannten.
\end{enumerate}



\section{GQSL}

Die GIANT Query \& Skripting Language dient dazu, IML-Teilgraphen und
Selektionen aus einem IML-Graph anzufragen und auf diesen Aktionen
auszuf�hren.

Die Sprache l�sst sich in zwei zentrale Konstrukte gliedern:
\begin{description}
\item[Queries] dienen der Anfrage eines Werts.
\item[Scripts] werden verwendet, um Aktionen auszuf�hren.
\end{description}

Als Aktion wird die Ausf�hrung eines vordefinierten Scripts bezeichnet
(siehe Kapitel \ref{language_gqsl_predef_env}).
Aktionen k�nnen GIANT zu Reaktionen veranlassen, die f�r den Benutzer
direkt sichtbar sind und f�r seine Arbeit n�tzlich sein k�nnen
(z.B. Anzeigen eines IML-Teilgraphen in einem Anzeigefenster).


GQSL Queries k�nnen nicht direkt ausgef�hrt, sondern nur von GQSL Scripts
aufgerufen werden.
GQSL Scripts werden von einem GQSL Interpreter ausgef�hrt. 
Scripts k�nnen Queries aufrufen, um Daten zu gewinnen. Mit diesen Daten
sollen Aktionen nach den W�nschen des Benutzers konfiguriert werden.
Queries und Scripts k�nnen in Variablen Werte zwischenspeichern.


\subsection{GQSL Interpreter}

Ein GQSL Interpreter ist ein System, das GQSL Scripts ausf�hrt, gem��
der Sprachdefinition in Kapitel \ref{language_gqsl_spec}. Jeder GQSL
Interpreter muss die vordefinierte Sprachumgebung gem�� Kapitel
\ref{language_gqsl_predef_env} unterst�tzen.

\subsubsection{Kontext des Interpreters}

Der GQSL Interpreter kann im Kontext eines Anzeigefensters ausgef�hrt
werden. Ist dies der Fall, so sollen die ausgef�hrten GQSL Scripts und
GQSL Queries Zugriff auf das Anzeigefenster haben, ohne den Namen des
Anzeigefensters zu kennen.

Der Zugriff auf das Anzeigefenster in dessen Kontext der GQSL Interpreter
ausgef�hrt wird soll �ber spezielle vordefinierte Queries durchgef�hrt
werden.

\subsubsection{Verhalten im Fehlerfall}

Der GQSL Interpreter muss ein Script bzw. eine Query zur�ckweisen, falls eine
Forderung der Sprachdefinition von diesem Script bzw. dieser Query verletzt
wird. In diesem Fall muss der Interpreter dem Benutzer eine Fehlermeldung
anzeigen, die folgede Anforderungen erf�llt:
\begin{enumerate}
\item Der Benutzer wird informiert, dass der Text des Scripts bzw. der Query
fehlerhaft ist.
\item Dem Benutzer wird die Art des Fehlers pr�zise erl�utert.
\item Dem Benutzer wird angezeigt an welcher Stelle in welchem Script bzw.
in welcher Query der Fehler erkannt wurde.
\end{enumerate}
Andernfalls startet der GQSL Interpreter die Ausf�hrung.

Der GQSL Interpreter muss die Ausf�hrung sofort
stoppen, sobald die erste \gqslexpression zu fehlerhaft ausgewertet wird
oder sobald ein \gqslstatement oder \gqslcomplexstatement fehlerhaft
ausgef�hrt wird. Diese Ereignisse werden als Fehler angesehen.
Der GQSL Interpreter
darf die Ausf�hrung bereits nach einem fr�heren Ausf�hrungsschritt abbrechen,
wenn sichergestellt ist, dass zu nach einem sp�teren Ausf�hrungsschritt
einer der genannten F�lle eintreten wird.

Der GQSL Interpreter muss dem Benutzer unverz�glich eine
Meldung anzeigen, die folgende Anforderungen erf�llt:
\begin{enumerate}
\item Der Benutzer muss informiert werden, dass ein Fehler aufgetreten ist.
\item Dem Benutzer muss pr�zise und f�r ihn leicht verst�ndlich erl�utert
werden, welcher Fehler aufgetreten ist.
\item Dem Benutzer kann eine M�glichkeit aufgezeigt werden, wie er den
Fehler beseitigen kann um eine fehlerfreie Ausf�hrung des selben Scripts zu
erm�glichen.
\end{enumerate}

Der GQSL Interpreter muss nach einem Fehler entweder alle seit Beginn
der Ausf�hrung bereits ausgef�hrten Aktionen r�ckg�ngig machen, oder er
muss dem Benutzer anzeigen, welche Aktionen bereits durchgef�hrt wurden,
bevor der Fehler passierte. Anhand dieser Information soll es dem Benutzer
m�glich sein, die Aktionen manuell r�ckg�ngig zu machen.


%%%%%%%%%%%%%%%%%%%%%%%%%%%%%%%%%%%%%%%%%%%%%%%%%%%%%%%%%%%%%%%%%%%%%%%%%%%%%%
\subsection{GQSL Sprachdefinition}
\label{language_gqsl_spec}

Die Syntax von GQSL wird beschrieben in Anlehnung an die
Beschreibungsmethode des Ada Reference Manual. Siehe ARM Kapitel 1.1.4 f�r
Details.


%%%%%%%%%%%%%%%%%%%%
\subsubsection{White Space}

Leerzeichen, Tabulatoren, Zeilenumbr�che und Kommentare k�nnen als
Trennzeichen verwendet werden.

Kommentare werden durch \tokbegincomment eingeleitet und gehen immer
bis zum Zeilenende.


%%%%%%%%%%%%%%%%%%%%
\subsubsection{Schl�sselw�rter}


%%%%%%%%%%%%%%%%%%%%
\subsubsection{Bezeichner}

\begin{EBNF}
\item[\gqslidentifier] \gqslletter \{~\gqslletter | \gqsldigit~\}
\end{EBNF}

\begin{enumerate}
\item
\gqslletter steht f�r einen der Buchstaben a-z oder A-Z.
\item
\gqsldigit steht f�r eine der Ziffern 0-9.
\item
Der gesamte Text des \gqslidentifier darf nicht ein Schl�sselwort von GQSL
sein.
\end{enumerate}


%%%%%%%%%%%%%%%%%%%%
\subsubsection{Literale}

\begin{EBNF}
\item[\gqslstringliteral] \tokquote \{ \gqslstringchar\} \tokquote
\end{EBNF}

Wobei \gqslstringchar f�r ein von \tokquote verschiedenes Zeichen steht.
Der Typ eines \gqslstringliteral ist \tokstring.

\begin{EBNF}
\item[\gqslintliteral] [ \tokplus|\tokminus ] \gqsldigit \{ \gqsldigit\}
\end{EBNF}

Der Typ eines \gqslintliteral ist \tokinteger

\begin{EBNF}
\item[\gqslregexp] \tokbeginrequote \{ \gqslregexpchar \} \tokendrequote
\end{EBNF}

Wobei \gqslregexpchar f�r ein von \tokbeginrequote und \tokendrequote
und von White Space verschiedenes Zeichen steht. \{ \gqslregexpchar \}
muss ein g�ltiger regul�rer Ausdruck sein, nach der Definition des
Ada95-Pakets GNAT.Regpat.


%%%%%%%%%%%%%%%%%%%%
\subsubsection{Typen}
\label{language_types}

Jede \gqslexpression und jede Variable hat in GQSL einen Typ, der ihren
Wertebereich bestimmt.

\begin{EBNF}
\item[\gqsltype] \tokinteger | \tokstring | \toknodeset | \tokedgeset
                 | \toksubgraph | \tokselection
\end{EBNF}

\begin{description}
\item[\tokinteger] Ganze Zahlen in einem implemetierungsabh�ngigen
Bereich $min .. max$ mit $min < -16.000$ und $max > +16.000$.
Ein GQSL-Interpreter muss den genauen Wertebereich dokumentieren.
\item[\tokstring] Zeichenkette beliebiger L�nge
\item[\toknodeset] Endliche Menge von IML-Knoten
\item[\tokedgeset] Endliche Menge von IML-Kanten
\item[\toksubgraph] Endliche Menge von IML-Knoten zusammen mit endlicher
Menge von IML-Kanten. F�r jede enthaltene IML-Kante $e$ mit Quellknoten $v$
und Zielknoten $w$ sind stets $v$ und $w$ ebenfalls in dem
\toksubgraph enthalten. Variablen dieses Typs k�nnen IML-Teilgraphen
speichern.
\item[\tokselection] Endliche Menge von IML-Knoten zusammen mit endlicher
Menge von IML-Kanten. Variablen dieses Typs k�nnen Selektionen speichern.
\end{description}


%%%%%%%%%%%%%%%%%%%%
\subsubsection{Variablen}
\label{language_variables}

Eine \gqslquery oder ein \gqslscript kann in ihrem/seinem
\gqsldeclarationpart Variablen deklarieren. Jede dieser Variablen hat einen
definierten Typ (siehe Kapitel \ref{language_types}).

\begin{EBNF}
\item[\gqsldeclarationpart] \{ \gqsldeclaration \toksemicolon \}
\item[\gqsldeclaration] \gqslidentifier \tokcolon \gqsltype
\end{EBNF}

\begin{enumerate}
\item Wird eine \gqslquery oder ein \gqslscript aufgerufen, so werden alle
Variablen des zugeh�rigen \gqsldeclarationpart erzeugt. 
\item Alle diese Variablen werden zerst�rt wenn die Ausf�hrung der \gqslquery
bzw. des \gqslscript endet.
\item Zu jedem Zeitpunkt der Ausf�hrung hat jede
dieser Variablen einen Wert innerhalb des Wertebereichs ihres Typs. Der Wert
�ndert sich nur durch ein \gqslassignment an die Variable.
\item Jede \gqsldeclaration deklariert genau eine Variable, die �ber ihren
\gqslidentifier innerhalb des \gqslquerybody bzw. des \gqslscriptbody
verwendet wird. Sie ist vom Typ \gqsltype.
\item Keine zwei verschiedenen \gqsldeclaration d�rfen den gleichen
\gqslidentifier innerhalb des selben \gqsldeclarationpart verwenden.
\item Keine \gqsldeclaration darf
den gleichen \gqslidentifier verwenden wie eine \gqsldeclaration innerhalb
einer \gqslparameterlist, deren Parameter im selben \gqslquerybody bzw.
\gqslscriptbody sichtbar sind (siehe Kapitel \ref{language_parameter}).
\item Zum Zeitpunkt der Erzeugung nimmt eine Variable ihren Initialwert an.
Der Initialwert einer Variable kann nicht spezifiziert werden. Es d�rfen
keine Annahmen �ber den Initialwert getroffen werden.
\end{enumerate}


%%%%%%%%%%%%%%%%%%%%
\subsubsection{Parameter}
\label{language_parameter}

Eine \gqslquery oder ein \gqslscript kann Parameter deklarieren.
Jeder dieser Parameter hat einen
definierten Typ (siehe Kapitel \ref{language_types}).

\begin{EBNF}
\item[\gqslparameterlist] \gqsldeclaration
                          \{ \toksemicolon \gqsldeclaration \}
\end{EBNF}

\begin{enumerate}
\item Wird eine \gqslquery oder ein \gqslscript aufgerufen, so werden alle
Parameter erzeugt und mit einem durch den Aufruf spezifizierten Initialwert
belegt.
\item Alle Parameter werden zerst�rt, wenn die Ausf�hrung der \gqslquery
bzw. des \gqslscript endet.
\item Zu jedem Zeitpunkt der Ausf�hrung hat jeder Parameter einen Wert
innerhalb des Wertebereichs seines Typs. Der Wert eines Parameters �ndert
sich nur duch ein \gqslassignment an den Parameter.
\item Jede \gqsldeclaration deklariert genau einen Parameter, der �ber seinen
\gqslidentifier innerhalb des \gqslquerybody bzw. des \gqslscriptbody
verwendet wird.
\item Keine zwei verschiedenen \gqsldeclaration d�rfen den gleichen
\gqslidentifier innerhalb der selben \gqslparameterlist verwenden.
\item Keine \gqsldeclaration darf den gleichen \gqslidentifier verwenden wie
eine \gqsldeclaration innerhalb eines \gqsldeclarationpart, deren Variablen
in dem selben \gqslquerybody bzw. \gqslscriptbody sichtbar sind (siehe
Kapitel \ref{language_variables}).
\end{enumerate}


%%%%%%%%%%%%%%%%%%%%
\subsubsection{Queries}
\label{language_queries}

\begin{EBNF}
\item[\gqslquery] ~\\
                  \tokquery \gqslidentifier \\
                  ~~ [ \tokopenbracket \gqslparameterlist \tokclosebracket ] \\
                  ~~ \tokreturn \gqsltype \tokis \\
                  ~~~ \gqsldeclarationpart \\
                  \tokbegin \\
                  ~~~ \gqslquerybody \\
                  ~~~ \tokreturn \gqslexpression \toksemicolon \\
                  \tokend \toksemicolon
\end{EBNF}

\begin{enumerate}
\item Definiert eine GQSL Query, die aus jeder anderen GQSL Query sowie aus
jedem GQSL Script durch den \gqslidentifier aufgerufen werden kann.
\item In dem \gqslquerybody und in der \gqslexpression sind alle Variablen
aus dem \gqsldeclarationpart,
alle Parameter aus der \gqslparameterlist, sowie alle dem System bekannten
GQSL Queries sichtbar. Sonst ist in dem \gqslquerybody nichts sichtbar.
\item Nach einem Aufruf werden die Parameter der Query erzeugt und mit
ihren Initialwerten belegt (siehe Kapitel \ref{language_parameter}), die
Variablen werden erzeugt (siehe Kapitel \ref{language_variables}) und
schlie�lich wird der \gqslquerybody ausgef�hrt (siehe Kapitel
\ref{language_querybody}).
\item Falls der \gqslquerybody erfolgreich ausgef�hrt wurde, dann wird die
\gqslexpression (siehe Kapitel \ref{language_expression}) ausgewertet.
Andernfalls ist die Ausf�hrung der Query fehlerhaft.
\item Falls die \gqslexpression erfolgreich ausgewertet wurde, so wird ihr
Ergebnis zum R�ckgabewert. Die Ausf�hrung der Query ist erfolgreich.
Andernfalls ist die Ausf�hrung der Query fehlerhaft.
\item Falls die Ausf�hrung der Query erfolgreich ist, so werden ihre
Variablen und Parameter zerst�rt und die Ausf�hrung der Query endet.
\end{enumerate}


%%%%%%%%%%%%%%%%%%%%
\subsubsection{Scripts}
\label{language_scripts}

\begin{EBNF}
\item[\gqslscript] ~\\
                  \tokscript \gqslidentifier \\
                  ~~ [ \tokopenbracket \gqslparameterlist \tokclosebracket ] \\
                  ~~~ \gqsldeclarationpart \\
                  \tokbegin \\
                  ~~~ \gqslscriptbody \\
                  \tokend \toksemicolon
\end{EBNF}

\begin{enumerate}
\item Definiert ein GQSL Script, das von jedem GQSL Script durch den
\gqslidentifier aufgerufen werden kann. Das Script kann auch von einem
GQSL Interpreter direkt ausgef�hrt werden.
\item In dem \gqslquerybody und in der \gqslexpression sind alle Variablen
aus dem \gqsldeclarationpart,
alle Parameter aus der \gqslparameterlist, sowie alle dem System bekannten
GQSL Queries und GQSL Scripts sichtbar. Sonst ist in dem \gqslquerybody und
in der \gqslexpression nichts sichtbar.
\item Nach einem Aufruf werden die Parameter des Scripts erzeugt und mit
ihren Initialwerten belegt (siehe Kapitel \ref{language_parameter}), die
Variablen werden erzeugt (siehe Kapitel \ref{language_variables}) und
schlie�lich wird der \gqslscriptbody ausgef�hrt (siehe Kapitel
\ref{language_scriptbody}).
\item Falls der \gqslscriptbody erfolgreich ausgef�hrt wurde, dann ist die
Ausf�hrung des Scripts erfolgreich. Andernfalls ist die Ausf�hrung des Scripts
fehlerhaft.
\item Falls die Ausf�hrung des Scripts erfolgreich ist, so werden seine
Variablen und Parameter zerst�rt und die Ausf�hrung des Scripts endet.
\end{enumerate}



%%%%%%%%%%%%%%%%%%%%
\subsubsection{Query Body}
\label{language_querybody}


\begin{EBNF}
\item[\gqslquerybody] \{ \gqslstatement \}
\end{EBNF}

\begin{enumerate}
\item Ein \gqslquerybody hat eine bestimmte \emph{Sicht}. Die Sicht ist die
Menge aller Object (genauer Variablen, Parameter, Queries und Scripts),
die in dem \gqslquerybody sichtbar sind
(vgl. Kapitel \ref{language_queries}). Ein Objekt ist in dem \gqslquerybody
sichtbar, genau dann wenn es in der Sicht enthalten ist.
\item Alle Nicht-Terminale, die aus dem \gqslquerybody abgeleitet sind
(die in einem Ableitungsbaum unter dem Symbol \gqslquerybody angeordnet sind),
haben die selbe Sicht wie der \gqslquerybody.
\item Ein \gqslquerybody wird ausgef�hrt, indem alle \gqslstatement in der
Reihenfolge einer Linksableitung der Grammatik ausgef�hrt werden. Sobald
ein \gqslstatement fehlerhaft ausgef�hrt wird, stoppt die Ausf�hrung, es
wird keines der nachfogenden \gqslstatement ausgef�hrt und
der \gqslquerybody ist fehlerhaft ausgef�hrt.
\item Der \gqslquerybody ist erfolgreich ausgef�hrt, falls er entweder kein
\gqslstatement enth�lt oder falls das letzte \gqslstatement erfolgreich
ausgef�hrt wurde.
\item Die Ausf�hrung des \gqslquerybody endet sobald er entweder erfolgreich
ausgef�hrt ist, oder fehlerhaft ausgef�hrt ist.
\end{enumerate}


%%%%%%%%%%%%%%%%%%%%
\subsubsection{Script Body}
\label{language_scriptbody}

\begin{EBNF}
\item[\gqslscriptbody] \{ \gqslcomplexstatement \}
\end{EBNF}

\begin{enumerate}
\item Ein \gqslscriptbody hat eine bestimmte \emph{Sicht}. Die Sicht ist die
Menge aller Objekte (genauer Variablen, Parameter, Queries und Scripts),
die in dem \gqslscriptbody sichtbar sind
(vgl. Kapitel \ref{language_scripts}). Ein Objekt ist in dem \gqslscriptbody
sichtbar, genau dann wenn es in der Sicht enthalten ist.
\item Alle Nicht-Terminale, die aus dem \gqslscriptbody abgeleitet sind
(die in einem Ableitungsbaum unter dem Symbol \gqslscriptbody angeordnet sind),
haben die selbe Sicht wie der \gqslscriptbody.
\item Ein \gqslscriptbody wird ausgef�hrt, indem alle \gqslcomplexstatement
in der Reihenfolge einer Linksableitung der Grammatik ausgef�hrt werden. Sobald
ein \gqslcomplexstatement fehlerhaft ausgef�hrt wird, stoppt die Ausf�hrung,
es wird keines der nachfolgenden \gqslstatement ausgef�hrt und der
\gqslscriptbody ist fehlerhaft ausgef�hrt.
\item Der \gqslscriptbody ist erfolgreich ausgef�hrt, falls er entweder kein
\gqslcomplexstatement enth�lt oder falls das letzte \gqslcomplexstatement
erfolgreich ausgef�hrt wurde.
\item Die Ausf�hrung des \gqslscriptbody endet, sobald er entweder erfolgreich
ausgef�hrt ist, oder fehlerhaft ausgef�hrt ist.
\end{enumerate}


%%%%%%%%%%%%%%%%%%%%
\subsubsection{Statement}

\begin{EBNF}
\item[\gqslstatement] \gqslassignment \toksemicolon
\item[\gqslassignment] \gqslidentifier \tokequal \gqslexpression
\end{EBNF}

\begin{enumerate}
\item Der \gqslidentifier muss entweder eine sichtbare Variable oder einen
sichtbaren Parameter bezeichnen. Diese/r Variable/Parameter wird $lhs$ genannt.
\item Der Typ der \gqslexpression muss der selbe Typ sein wie der Typ von $lhs$
\item Zur Ausf�hrung des \gqslstatement wird die \gqslexpression ausgewertet.
Falls die \gqslexpression erfolgreich ausgewertet wurde, so wird der Wert
von $lhs$ auf das Ergebnis der \gqslexpression ge�ndert. In diesem Fall ist
die Ausf�hrung des \gqslstatement erfolgreich.
\item Falls \gqslexpression fehlerhaft ausgewertet wurde, dann ist die
Ausf�hrung des \gqslstatement fehlerhaft.
\end{enumerate}


%%%%%%%%%%%%%%%%%%%%
\subsubsection{Complex Statement}
\label{language_complexstatement}

\begin{EBNF}
\item[\gqslcomplexstatement] \gqslstatement | \gqslscriptcall \toksemicolon
\item[\gqslscriptcall] \gqslidentifier [ ( \gqslargumentlist ) ]
\item[\gqslargumentlist] \gqslexpression \{ \tokcomma \gqslexpression  \}
\end{EBNF}

\begin{description}
\item[\gqslstatement]
  \begin{enumerate}
  \item Die Sicht des \gqslstatement ist die selbe wie die Sicht des
  \gqslcomplexstatement
  \item Zum Ausf�hren des \gqslcomplexstatement wird das \gqslstatement
  ausgef�hrt.
  \item Das \gqslcomplexstatement ist erfogreich ausgef�hrt, falls das
  \gqslstatement erfolgreich ausgef�hrt ist.
  \item Das \gqslcomplexstatement ist fehlerhaft ausgef�hrt, falls das
  \gqslstatement fehlerhaft ausgef�hrt ist.
  \end{enumerate}
\item[\gqslscriptcall]
  \begin{enumerate}
  \item \gqslidentifier muss ein sichtbares GQSL Script $S$ bezeichnen.
  \item Falls $S$ keine \gqslparameterlist besitzt, so darf keine
  \gqslargumentlist angegeben sein.
  \item Falls $S$ eine \gqslparameterlist besitzt, so m�ssen in der
  \gqslargumentlist genau so viele \gqslexpression angegeben sein, wie in der
  \gqslparameterlist \gqsldeclaration angegeben sind.
  \item Der Typ der $i$-ten \gqslexpression der \gqslargumentlist muss der
  selbe sein wie der Typ der $i$-ten \gqsldeclaration der \gqslparameterlist
  f�r alle $i$.
  \item Zum Ausf�hren des \gqslcomplexstatement werden alle \gqslexpression
  in durch den Interpreter bestimmter Reihenfolge ausgewertet.
  \item Falls eine der \gqslexpression fehlerhaft ausgewertet wird,
  so ist die Ausf�hrung des \gqslcomplexstatement fehlerhaft. Andernfalls
  wird $S$ aufgerufen. Der $i$-te Parameter wird mit dem Wert der $i$-ten
  \gqslexpression initialisiert f�r alle $i$.
  \item Falls $S$ erfolgreich ausgef�hrt wurde, so ist das
  \gqslcomplexstatement erfolgreich ausgef�hrt. Falls $S$ fehlerhaft
  ausgef�hrt wurde, so ist das \gqslcomplexstatement fehlerhaft ausgef�hrt.
  \end{enumerate}
\end{description}

Die Ausf�hrung des \gqslcomplexstatement endet, sobald es entweder erfolgreich
oder fehlerhaft ausgef�hrt ist.


%%%%%%%%%%%%%%%%%%%%
\subsubsection{Expression}
\label{language_expression}

\begin{EBNF}
\item[\gqslexpression] ~\\
                       \gqslnodesetexpression  \\
                       | \gqsledgesetexpression \\
                       | \gqslsubgraphexpression \\
                       | \gqslselectionexpression \\
                       | \gqslvariableinspection \\
                       | \gqslquerycall \\
                       | \gqslcalculation \\
                       | \gqslliteralexpression
\end{EBNF}

Annahme: Aus den Alternativen sei die Regel mit der rechten Seite
$R$ gew�hlt ($R$ besteht aus genau einem Nicht-Terminal).
\begin{enumerate}
\item Der Typ der \gqslexpression ist der Typ des Nicht-Terminals $R$
(siehe die folgenden Kapitel f�r Details).
\item Die \gqslexpression wird erfolgreich ausgewertet, falls $R$ erfolgreich
ausgewertet wird. Andernfalls ist die \gqslexpression fehlerhaft ausgewertet.
\item Falls $R$ erfolgreich ausgewertet wird, so ist der Wert der
\gqslexpression gleich dem Wert der Auswertung von $R$.
\end{enumerate}


%%%%%%%%%%%%%%%%%%%%
\subsubsection{Node Set Expression}

\begin{EBNF}
\item[\gqslnodesetexpression] \tokselect \toknodes \tokfrom \gqslexpression
                              [ \gqslnodepredicate ] \\
                             | \tokbuild \toknodes \tokfrom \gqslexpression
                              [ \tokaccept \toknodes \gqslnodepredicate ] \\
                             | \tokallnodes \\
                             | \tokrootnode
\end{EBNF}

Der Typ einer \gqslnodesetexpression ist \toknodeset.

\begin{description}
\item[select]
  Trifft eine Auswahl bestimmter IML-Knoten aus einer Menge.
  \begin{enumerate}
  \item Der Typ von \gqslexpression muss \toknodeset, \toksubgraph oder
  \tokselection sein.
  \item Die Auswertung ist erfolgreich genau dann, wenn die Auswertung der
  \gqslexpression erfolgreich ist. Andernfalls ist die Auswertung fehlerhaft.
  \item Ist die Auswertung erfolgreich, dann ist das Ergebnis die gr��te
  Menge $M$ von IML-Knoten, so dass $\forall v \in M$
    \begin{enumerate}
    \item $v$ ist in \gqslexpression enthalten.
    \item $v$ erf�llt das \gqslnodepredicate, falls dieses angegeben ist
    (siehe Kapitel \ref{language_node_predicate}).
    \end{enumerate}
  \end{enumerate}
\item[build]
  Bildet eine Menge von IML-Knoten, die inzident sind zu einer gegebenen Menge
  von IML-Kanten.
  \begin{enumerate}
  \item Der Typ von \gqslexpression muss \tokedgeset sein.
  \item Die Auswertung ist erfolgreich genau dann, wenn die Auswertung von
  \gqslexpression erfolgreich ist. Andernfalls ist die Auswertung fehlerhaft.
  \item Ist die Auswertung erfolgreich, dann ist das Ergebnis die gr��te
  Menge $M$ von IML-Knoten, so dass $\forall v \in M$
    \begin{enumerate}
    \item $v$ ist inzident zu einer Kante $e$ aus \gqslexpression
    \item $v$ erf�llt \gqslnodepredicate, falls dieses angegeben ist.
    \end{enumerate}
  \end{enumerate}
\item[\tokallnodes]
  Die Menge aller IML-Knoten des IML-Graphen.
  \begin{enumerate}
  \item Die Auswertung ist stets erfolgreich.
  \item Das Ergebnis ist die Menge aller IML-Knoten des IML-Graphen.
  \end{enumerate}
\item[\tokrootnode]
  Die Wurzel des IML-Graphen.
  \begin{enumerate}
  \item Die Auswertung ist stets erfolgreich.
  \item Das Ergebnis ist die Menge, die den Wurzelknoten des dem IML-Graphen
  zugrundeliegenden IML-Programms enth�lt und sonst nichts.
  \end{enumerate}
\end{description}



%%%%%%%%%%%%%%%%%%%%
\subsubsection{Edge Set Expression}

\begin{EBNF}
\item[\gqsledgesetexpression] \tokselect \tokedges \tokfrom \gqslexpression
                              [ \gqsledgepredicate ] \\
                             | \tokbuild \tokedges [ \tokfrom \gqslexpression ]
                                                   [ \tokto \gqslexpression ]
                                                   [ \gqsledgepredicate ]
\end{EBNF}

Der Typ einer \gqsledgesetexpression ist \tokedgeset.

\begin{description}
\item[select] Trifft eine Auswahl von bestimmten IML-Kanten aus einer Menge.
  \begin{enumerate}
  \item Der Typ von \gqslexpression muss \tokedgeset, \toksubgraph oder
  \tokselection sein.
  \item Die Auswertung ist erfolgreich genau dann, wenn die Auswertung von
  \gqslexpression erfolgreich ist. Andernfalls ist die Auswertung fehlerhaft.
  \item Falls die Auswertung erfolgreich ist, dann ist das Ergebnis die
  gr��te Menge von IML-Kanten $M$ mit $\forall e \in M$
    \begin{enumerate}
    \item $e$ ist in der Menge der IML-Kanten aus \gqslexpression enthalten.
    \item $e$ erf�llt \gqsledgepredicate, falls das angegeben ist.
    (siehe Kapitel \ref{language_edge_predicate})
    \end{enumerate}
  \end{enumerate}
\item[build] Bildet eine Menge von IML-Kanten, die zwischen zwei Mengen
von IML-Knoten verlaufen.
  \begin{enumerate}
  \item Beide \gqslexpression m�ssen -- sofern sie angegeben sind -- den
  Typ \toknodeset haben.
  \item Die Auswertung ist erfolgreich, falls alle angegebenen
  \gqslexpression erfolgreich ausgewertet werden. Andernfalls ist die
  Auswertung fehlerhaft.
  \item Das Ergebnis ist die Menge $M$ aller IML-Kanten mit $\forall
  e \in M$
    \begin{enumerate}
    \item $e$ ist im IML-Graph enthalten.
    \item Der Start-Knoten von $e$ ist in der \tokfrom-\gqslexpression
    enthalten, falls \tokfrom-\gqslexpression angegeben ist.
    \item Der Ziel-Knoten von $e$ ist in der \tokto-\gqslexpression
    enthalten, falls \tokto-\gqslexpression angegeben ist.
    \item $e$ erf�llt das \gqsledgepredicate.
    \end{enumerate}
  \end{enumerate}
\end{description}


%%%%%%%%%%%%%%%%%%%%
\subsubsection{Subgraph Expression}

\begin{EBNF}
\item[\gqslsubgraphexpression] \tokmake \tokgraph \tokopenbracket
                                    \gqslexpression \tokcomma \gqslexpression
                                    \tokclosebracket \\
                               | \tokbuild \tokgraph \tokbased \tokon
                                    \gqslexpression \tokfollow \tokedges
                                    \gqsledgepredicate
                                    [ \tokaccept \toknodes \gqslnodepredicate ]
                                    [ \tokuntil \tokdepth \gqslexpression ]
\end{EBNF}

Der Typ einer \gqslsubgraphexpression ist immer \toksubgraph.

\begin{description}
\item[make] Erstellt einen IML-Teilgraph aus einer Menge von Knoten und einer
Menge von Kanten.
  \begin{enumerate}
  \item Der Typ der ersten \gqslexpression muss \toknodeset sein.
  \item Der Typ der zweiten \gqslexpression muss \tokedgeset sein.
  \item Die Auswertung ist erfolgreich genau dann, wenn die Auswertung beider
  \gqslexpression erfolgreich ist. Andernfalls ist die Auswertung fehlerhaft.
  \item Falls die Auswertung erfolgreich ist, so ist das Ergebnis ein
  IML-Teilgraph mit allen IML-Knoten der ersten
  \gqslexpression sowie denjenigen IML-Kanten die in der zweiten
  \gqslexpression enthalten sind und deren Start- und Zielknoten in der
  ersten \gqslexpression enthalten sind.
  \end{enumerate}
\item[build] Fragt einen Teilgraph an, ausgehend von einer Grundmenge von
IML-Knoten.
  \begin{enumerate}
  \item Der Typ der ersten \gqslexpression muss \toknodeset sein.
  \item Falls die zweite \gqslexpression angegeben ist, so muss ihr Typ
  \tokinteger sein.
  \item Die Auswertung ist erfolgreich falls die folgenden Bedingungen
  erf�llt sind:
    \begin{enumerate}
    \item Alle \gqslexpression werden erfolgreich ausgewertet.
    \item Falls die zweite \gqslexpression angegeben ist, so ist ihr Wert
    gr��er oder gleich 0.
    \end{enumerate}
  andernfalls ist die Auswertung fehlerhaft.
  \item Der Wert ergibt sich durch eine Breitensuche auf dem IML-Graph.
  Begonnen wird mit der IML-Knotenmenge, die durch die erste \gqslexpression
  gegeben ist (Ebene 0). Dann werden alle IML-Kanten verfolgt f�r die
    \begin{enumerate}
    \item Das \gqsledgepredicate erf�llt ist.
    \item Der �ber diese IML-Kante erreichte IML-Knoten das \gqslnodepredicate
    erf�llt, sofern dieses angegeben ist.
    \item Die Nummer der Ebene kleiner ist als die zweite \gqslexpression,
    sofern diese angegeben ist.
    \end{enumerate}
  Wurde eine IML-Kante verfolgt, so wird sie zum Ergebnis hinzugef�gt. Der
  erreichte IML-Knoten wird ebenfalls zum Ergebnis hinzugef�gt. Die Menge
  aller so erreichten IML-Knoten bildet die Ausgangsmenge f�r den n�chsten
  Schritt. Sie ist eine Ebene h�her. Die Suche bricht ab, wenn keine neuen
  Kanten mehr aufgenommen werden.
  \end{enumerate}
\end{description}


%%%%%%%%%%%%%%%%%%%%
\subsubsection{Selection Expression}

\begin{EBNF}
\item[\gqslselectionexpression] \tokmake \tokselection \tokopenbracket
                                   \gqslexpression \tokcomma \tokclosebracket
\end{EBNF}

\begin{enumerate}
\item die erste \gqslexpression muss den Typ \toknodeset haben.
\item die zweite \gqslexpression muss den Typ \tokedgeset haben.
\item Der Typ der \gqslselectionexpression ist \tokselection.
\item Die Auswertung ist erfolgreich genau dann, wenn die Auswertung beider
\gqslexpression erfolgreich ist. Andernfalls ist die Auswertung fehlerhaft.
\item Falls die Auswertung erfolgreich ist, dann ist der Wert die Selektion,
die alle IML-Knoten und alle IML-Kanten aus den beiden \gqslexpression und
sonst nichts enth�lt.
\end{enumerate}

%%%%%%%%%%%%%%%%%%%%
\subsubsection{Variable Inspection}

\begin{EBNF}
\item[\gqslvariableinspection] \gqslidentifier
\end{EBNF}

\begin{enumerate}
\item Der \gqslidentifier muss entweder eine sichtbare Variable oder einen
sichtbaren Parameter $v$ bezeichnen.
\item Der Typ der \gqslvariableinspection ist der Typ von $v$.
\item Eine \gqslvariableinspection wird stets erfolgreich ausgewertet.
\item Der Wert der \gqslvariableinspection ist der Wert, den $v$ zum Zeitpunkt
der Auswertung hat.
\end{enumerate}

\begin{description}
\item[Anmerkung] Eine \gqslvariableinspection kann nur w�hrend der Lebenszeit
einer Variable bzw. eines Parameters geschehen. Eine Variable hat zu jedem
Zeitpunkt ihrer Lebenszeit einen Wert. (vgl. Kapitel \ref{language_variables},
\ref{language_parameter})
\item[Anmerkung] Der Wert von $v$ ver�ndert sich durch die
\gqslvariableinspection nicht.
\end{description}


%%%%%%%%%%%%%%%%%%%%
\subsubsection{Query Call}

\begin{EBNF}
\item[\gqslquerycall] \gqslidentifier [ \tokopenbracket \gqslargumentlist
                                        \tokclosebracket ]
\end{EBNF}

Anmerkung: \gqslargumentlist siehe \ref{language_complexstatement}

\begin{enumerate}
\item \gqslidentifier muss eine sichtbare GQSL Query $Q$ bezeichnen.
\item Falls $Q$ keine \gqslparameterlist besitzt, so darf keine
\gqslargumentlist angegeben sein.
\item Falls $Q$ eine \gqslparameterlist besitzt, so m�ssen in der
\gqslargumentlist genau so viele \gqslexpression angegeben sein, wie in der
\gqslparameterlist \gqsldeclaration angegeben sind.
\item Der Typ der $i$-ten \gqslexpression der \gqslargumentlist muss
der selbe sein wie der Typ der $i$-ten \gqsldeclaration der
\gqslparameterlist f�r alle $i$.
\item Der Typ des \gqslquerycall ist der Wert des R�ckgabewerts von $Q$.
\item Zum Auswerten des \gqslquerycall werden alle \gqslexpression
in durch den Interpreter willk�rlich bestimmter Reihenfolge ausgewertet.
\item Falls eine der \gqslexpression fehlerhaft ausgewertet wird,
so ist die Auswertung des \gqslquerycall fehlerhaft. Andernfalls
wird $Q$ aufgerufen. Der $i$-te Parameter wird mit dem Wert der $i$-ten
\gqslexpression initialisiert f�r alle $i$.
\item Falls $Q$ erfolgreich ausgef�hrt wurde, so ist der
\gqslquerycall erfolgreich ausgewertet. Falls $W$ fehlerhaft
ausgef�hrt wurde, so ist der \gqslquerycall fehlerhaft ausgewertet.
\item Falls der \gqslquerycall erfolgreich ausgewertet wurde, so ist
sein Wert gleich dem R�ckgabewert von $Q$.
\end{enumerate}



%%%%%%%%%%%%%%%%%%%%
\subsubsection{Calculation}

\begin{EBNF}
\item[\gqslcalculation] \tokopenbracket \gqslexpression \tokclosebracket \\
                       | \gqslexpression \tokplus \gqslexpression \\
                       | \gqslexpression \tokminus \gqslexpression \\
                       | \gqslexpression \tokintersect \gqslexpression
\end{EBNF}


%%%%%%%%%%%%%%%%%%%%%
\subsubsection{Literal Expression}

\begin{EBNF}
\item[\gqslliteralexpression] \gqslintliteral \\
                       | \gqslstringliteral
\end{EBNF}

Eine \gqslliteralexpression wird stets erfolgreich ausgewertet.
\begin{description}
\item[\gqslintliteral]
   \begin{enumerate}
   \item Der Typ der \gqslliteralexpression ist \tokinteger.
   \item Der Wert der \gqslliteralexpression ist die Interpretation die
   Ableitung des \gqslintliteral im 10er Zahlensystem.
   \end{enumerate}
\item[\gqslstringliteral]
   \begin{enumerate}
   \item Der Typ der \gqslliteralexpression ist \tokstring.
   \item Der Wert der \gqslliteralexpression ist die Ableitung
   des \gqslstringliteral.
   \end{enumerate}
\end{description}


%%%%%%%%%%%%%%%%%%%%
\subsubsection{Logik}

\begin{EBNF}
\label{language_node_predicate}
\item[\gqslnodepredicate]
\label{language_edge_predicate}
\item[\gqsledgepredicate]
\end{EBNF}


%%%%%%%%%%%%%%%%%%%%%%%%%%%%%%%%%%%%%%%%%%%%%%%%%%%%%%%%%%%%%%%%%%%%%%%%%%%%%%
\subsection{GQSL vordefinierte Sprachumgebung}
\label{language_gqsl_predef_env}

\paragraph{Vordefinierte Queries}

\begin{verbatim}
Get_Current_Window
  return string is
// ... implementation defined ...
\end{verbatim}

Liefert als Ergebnis den Namen des Anzeigefensters, in dessen Kontext
der GQSL Interpreter sich befindet. Befindet sich der GQSL Interpreter
nicht im Kontext eines Anzeigefensters, so ist das Ergebnis undefiniert.

\begin{verbatim}
Get_Current_Selection
  (Window_Name : string)
  return selection is
// ... implementation defined ...
\end{verbatim}

Erh�lt als Argument den Namen eines Anzeigefensters und liefert als
Ergebnis die aktuelle Selektion in diesem Anzeigefenster.
Kein Anzeigefenster dieses Namens existiert, ist das Ergebnis undefiniert.

\begin{verbatim}
Get_Selection
  (Window_Name    : string;
   Selection_Name : string)
  return selection is
// ... implementation defined ...
\end{verbatim}

Erh�lt als erstes Argument den Namen eines Anzeigefensters, als zweites
Argument den Namen einer Selektion in diesem Anzeigefenster. Liefert als
Ergebnis die durch diese Namen gew�hlte Selektion. Existiert kein
Anzeigefenster des gew�hlten Namens oder existiert in dem Anzeigefenster
keine Selektion des gew�hlten Namens, so ist das Ergebnis undefiniert.

\begin{verbatim}
Get_Subgraph
  (Subgraph_Name : string)
  return subgraph is
// ... implementation defined ...
\end{verbatim}

Erh�lt als Argument den Namen eines IML-Teilgraphen. Liefert als Ergebnis
den IML-Teilgraphen des gew�hlten Namens. Existiert kein IML-Teilgraph
dieses Namens, so ist das Ergebnis undefiniert.

\paragraph{Vordefinierte Scripts}

\begin{verbatim}
Create_Window
  (Window_Name : string;
   Layout_Algo : string;
   Content     : subgraph) is
// ... implementation defined ...
\end{verbatim}

Erzeugt ein neues Anzeigefenster mit dem Namen \texttt{Window\_Name}.
Wendet den Layout-Algorithmus \texttt{Layout\_Algo} auf \texttt{Content}
an und f�gt die so erzeugte neue Selektion in das Anzeigefenster ein.

Falls \texttt{Window\_Name} keinen g�ltigen Namen f�r ein Anzeigefenster
enth�lt oder falls bereits ein Anzeigefenster dieses Namens existiert,
so ist die Ausf�hrung dieses Scripts fehlerhaft. Falls
\texttt{Layout\_Algo} kein g�ltiger Layout-Algorithmus ist, so ist die
Ausf�hrung dieses Scripts fehlerhaft.

\begin{verbatim}
Insert_Into_Window
  (Window_Name : string;
   Layout_Algo : string;
   New_Content : subgraph) is
// ... implementation defined ...
\end{verbatim}

Wendet den Layout-Algorithmus \texttt{Layout\_Algo} auf \texttt{Content}
an und f�gt die so erzeugte neue Selektion in das Anzeigefenster mit
dem Namen \texttt{Window\_Name} ein.

Falls  kein Anzeigefenster des Namens \texttt{Window\_Name} existiert,
so ist die Ausf�hrung dieses Scripts fehlerhaft. Falls
\texttt{Layout\_Algo} kein g�ltiger Layout-Algorithmus ist, so ist die
Ausf�hrung dieses Scripts fehlerhaft.

\begin{verbatim}
Set_Subgraph
  (Subgraph_Name : string;
   Value         : subgraph) is
// ... implementation defined ...
\end{verbatim}

Falls noch kein IML-Teilgraph mit Name \texttt{Subgraph\_Name}
Name existiert, dann erzeugt diesen IML-Teilgraphen. Setzt den Inhalt
des IML-Teilgraphen mit Name \texttt{Subgraph\_Name} auf \texttt{Value}.

Falls \texttt{Subgraph\_Name} kein g�ltiger Name f�r IML-Teilgraphen ist,
so ist die Ausf�hrung dieses Scripts fehlerhaft.



%===============================================================================
% 
% Projektverwaltung
%
\chapter{GIANT Projektverwaltung}
% ==============================================================================
%  $RCSfile: project.tex,v $, $Revision: 1.17 $
%  $Date: 2003/02/25 14:33:49 $
%  $Author: squig $
%
%  Description:
%
% ==============================================================================


%===================
\section {Persistenz �ber Projekte}

GIANT speichert persistente Informationen in so genannten Projekten.
\begin {enumerate}

  \item 
  Wird w�hrend des Betriebs von GIANT ein neues Projekt angelegt, so erh�lt 
  es automatisch eine Projektdatei.
  
  \item
  Ein Projekt besteht aus einem Verweis auf eine IML-Graph-Datei, auf die sich
  die gespeicherten Informationen beziehen, sowie aus den gespeicherten 
  Informationen f�r IML-Teilgraphen, Anzeigefenster und Knoten-Annotationen.
 
  \item
  Jedes Projekt hat einen vom Benutzer definierbaren Namen, dieser Name 
  entspricht dem Namen der Projektdatei und wird innerhalb des IML-Browsers 
  angezeigt.

  \item
  Der Name eines bereits angelegten Projektes kann mit den Mitteln von 
  GIANT nicht ge�ndert werden (au�er dadurch, dass  man das Projekt unter 
  neuem Namen neu speichert). 

  \item
  Der Benutzer kann beliebig viele Projekte anlegen.
  
  \item
  In GIANT darf immer nur ein Projekt gleichzeitig ge�ffnet sein. 
  
  \item
  W�hrend der Arbeit mit GIANT kann jederzeit ein Projekt geladen oder ein
  neues Projekt angelegt werden. 
  Voraussetzung hierf�r ist allerdings, dass dies 
  in der Reflektion zum IML-Graphen unterst�tzt wird.
  
  \item
  GIANT wei� nicht, ob ein geladenes Projekt gegen�ber den f�r das Projekt
  in der Projektdatei und in den Verwaltungssdateien gespeicherten 
  Informationen modifiziert wurde oder nicht.

\end {enumerate}


  \subsection {Das Projektverzeichnis}
  \begin {enumerate}

    \item  
    S�mtliche Dateien, die die Informationen f�r ein Projekt enthalten,
    befinden sich in diesem Verzeichnis. 

    \item
    In einem Projektverzeichnis darf nur ein Projekt abgelegt werden.
    
  \end {enumerate}    


  \subsection {Die Projektdatei}
  \begin {enumerate}

    \item
    Die Projektdatei liegt als XML-Datei vor.

    \item
    Die Projektdatei befindet sich im Projektverzeichnis und enth�lt 
    Informationen, die zur Identifikation des zu einem Projekt geh�renden 
    IML-Graphen n�tig sind. 
  
    \item
    Der Name der Projektdatei entspricht dem Namen des Projektes.

    \item
    Die Projektdatei enth�lt Referenzen zu allen Dateien, die Bestandteil
    des Projektes sind (Verwaltungsdateien f�r IML-Teilgraphen und
    Anzeigefenster und die Verwaltungsdatei f�r Knoten-Annotationen).

    \item
    Der Pfad zu der Datei, die den IML-Graphen enth�lt, ist in der Projektdatei 
    gespeichert. 
  \end {enumerate}
  
 
  \subsection {Pr�fung der IML-Graph Datei}
  Die Reflektion muss f�r jede IML-Graph Datei eine m�glichst eindeutige 
  Pr�fsumme berechnen k�nnen. Beim Laden eines Projektes wird �berpr�ft, 
  ob die in der Projektdatei gespeicherte Pr�fsumme
  der Pr�fsumme der zu ladenden IML-Graph Datei entspricht. 
  Das Verhalten von GIANT f�r den Fall, dass eine IML-Graph Datei geladen wird, 
  die zwar die passende Pr�fsumme hat, aber nicht den IML-Graphen enth�lt, 
  der dem Projekt eigentlich zu Grunde liegt, ist undefiniert.


  \subsection {Verwaltungsdateien f�r Anzeigefenster}
 
  \begin {enumerate}
  
    \item
    Zu jedem Anzeigefenster gibt es eine Verwaltungsdatei. 
    
    \item
    Diese Verwaltungsdatei enth�lt alle Informationen zur kompletten 
    Rekonstruktion eines Anzeigefensters. Alle Informationen werden
    in bin�rer Form gespeichert.
 
  \end {enumerate}
  Insbesondere werden folgende Informationen gespeichert:
 
  \begin {enumerate}
 
    \item Der komplette Anzeigeinhalt 
          (alle visualisierten Knoten und Kanten mit Position).
    \item Alle Pins (gespeicherte sichtbare Anzeigeinhalte).
    \item Alle dem Anzeigefenster bekannten Selektionen.
 
  \end {enumerate}
 
  Der f�r das Anzeigefenster gew�hlte Visualisierungsstiel wird zwar mit 
  gespeichert, da die Viualisierungsstiele v�llig unabh�ngig von den 
  Projekten �ber Konfigurationsdateien realisiert werden, 
  kann nicht gerantiert werden, dass der gew�nschte Stiel beim Laden 
  des Projektes auch wieder gefunden wird,
  in diesem Fall wird dann ein \gq{Standard-Stil} verwendet.


  \subsection {Verwaltungsdateien f�r IML-Teilgraphen}
  \begin {enumerate}

   \item
    F�r jeden IML-Teilgraphen gibt es eine eigene Verwaltungsdatei.
    
    \item
    S�mtliche erzeugten IML-Teilgraphen werden in Verwaltungsdateien in 
    bin�rer Form gespeichert. 

  \end{enumerate}
  
  
  \subsection {Die Verwaltungsdatei f�r Knoten-Annotationen}
  \begin {enumerate}

   \item
   Alle Knoten-Annotationen werden in einer Verwaltungsdatei gespeichert.
   
   \item
   Diese Datei liegt als XML Datei vor.
  \end{enumerate}
  

 
 
%===================
\section {Grundlegendes Verhalten von GIANT beim Speichern von Projekten} 
  

  \subsection{\gq{Alles Speichern}}\label{Alles Speichern}
  Diese Funktionalit�t wird von entsprechenden UseCases genutzt. 
  Hierbei werden alle Anzeigefenster, IML-Teilgraphen 
  und Knoten-Annotationen in die Verwaltungsdateien geschrieben, 
  wobei alle �nderungen an noch offenen Anzeigefenstern ber�cksichtigt 
  werden.\\
 

  %===
  \subsection {Persistenz von Anzeigefenstern}
  \begin {enumerate}

    \item
    S�mtliche dem Projekt bekannten Anzeigefenster (alle Anzeigefenster zu 
    denen es eine entsprechende Verwaltungsdatei gibt) werden auf der GUI in 
    der Liste �ber die Anzeigefenster angezeigt, egal ob sie ge�ffnet sind
    oder nicht. 

    \item 
    Zu jedem Anzeigefenster eines Projektes gibt es eine Verwaltungsdatei,
    beim �ffnen eines neuen Fensters wird diese Datei automatisch mit
    erzeugt.

    \item
    Wird ein ge�ffnetes Anzeigefenster geschlossen, 
    so fragt GIANT nach, ob es eventuelle �nderungen speichern soll
    oder nicht. Falls ja, werden eventuelle �nderungen in die f�r das 
    Anzeigefenster vorhandene Verwaltungsdatei geschrieben, 
    anderenfalls bleibt der Zustand des Anzeigefensters nach der letzten
    Speicherung vorhanden (die zugeh�rige Verwaltungsdatei wird nicht 
    ver�ndert).

    \item
    Modifikationen (z.B. das Verschieben von Knoten) 
    auf einem Anzeigefenster werden nicht automatisch
    nach deren Durchf�hrung gespeichert (so kann notfalls ein
    Undo durchgef�hrt werden).

    \item
    Ein zu einem Projekt geh�rendes Anzeigefenster kann gel�scht werden, 
    hierbei werden alle Information �ber das Anzeigefenster einschlie�lich
    der Verwaltungsdatei vernichtet.
    >>>>>>>>>>>>>>>>USE CASE - L�schen persistenter Anzeigefenster
  \end {enumerate}

  %===
  \subsection {Persistenz von IML-Teilgraphen}
  \begin {enumerate}

    \item
    Alle in einem Projekt bereits vorhandenen IML-Teilgraphen werden auf der 
    GUI angezeigt (in einer entsprechenden Liste).
  
    \item
    Neu erzeugte IML-Teilgraphen und �nderungen an bestehenden 
    IML-Teilgraphen k�nnen �ber \gq{alles Speichern} (siehe oben) 
    gespeichert werden.
       
    \item
    Wird das Programm beendet, ohne das zuvor \gq{Alles Speichern} ausgef�hrt 
    worden ist, so gehen  alle nicht gespeicherten Informationen zu den 
    IML-Teilgraphen verloren (alle zwischenzeitlich ausgef�hrten Modifikationen 
    und alle zwischenzeitlich neu erzeugten IML-Teilgraphen). 
    Der Zustand des Projektes  nach dem letzten Speichern bleibt dann erhalten.

    \item
    Modifikationen an bestehenden IML-Teilgraphen werden nicht automatisch 
    gespeichert, zu neu erzeugten IML-Teilgraphen wird nicht automatisch
    eine Verwaltungsdatei erzeugt.
    \\
    N�TIG; DA SONST ANFRAGEN EVENTUELL AUSGEBREMST.
    IM GEGENSATZ ZU ANZEIGEFENSTERN KANN MAN IML-TEILGRAPHEN NICHT �FFNEN
    IM GEGENSATZ ZU ANZEIGEFENSTERN (fliegen beim Schlie�en aus dem
    speicher raus) WERDEN ALL IML-TEILGRAPHEN pauschal im
    SPEICHER GEHALTEN.

    \item
    IML-Teilgraphen k�nnen gel�scht werden. Falls vorhanden, wird dann auch
    die entsprechende Verwaltungsdatei ebenfalls sofort gel�scht.
  \end {enumerate}
  
  
  
\subsection {Persistenz von Knoten-Annotationen}
  \begin {enumerate}
  
   \item
   �nderungen bestehender oder neu erzeugte Knoten-Annotationen werden nur
   �ber die Funktionali�t unter \gq{alles Speichern} in die 
   Verwaltungsdatei geschrieben.
   
   \item
   Einmal erzeugte Knoten-Annotationen werden jedem IML-Knoten mit der 
   entsprechenden ID zugeordnet, egal in welchem Anzeigefenster diese
   visualisiert sind. Ein Knoten kann auch annotiert sein, wenn er in keinem
   Anzeigefenster visualisiert ist. Wird ein annotierter Knoten gel�scht 
   (aus einem Anzeigefenster entfernt), so wird der dazu vorhandene 
   Eintrag f�r die Annotation in der Verwaltungsdatei nicht automatisch mit gel�scht.
   
   >>>>>>>>>>>>>>>> EVENTUELL M�SSEN WIR HIER EINEN FILTER BAUEN
   DER KNOTEN-Annotationen entfernt, die nicht mehr gebraucht werden.
   
 \end {enumerate}
  
  
  
  
  
  
  
  
  



%===============================================================================
% 
% Konfigurationsdatei
%
\chapter{Konfigurationsdatei}
% ==============================================================================
%  $RCSfile: config.tex,v $, $Revision: 1.35 $
%  $Date: 2003/04/21 01:13:30 $
%  $Author: schwiemn $
%
%  Description:
%
%  Last-Ispelled-Revision: 1.19
%
% ==============================================================================


\section {Allgemeines} \index{Konfiguration}
\begin {enumerate}

  \item
  S�mtliche konfigurierbaren Einstellungen werden in XML-Dateien vorgenommen.
  F�r die verschiedenen Dateitypen werden auch
  Dokumenttyp-Definitionen (DTDs) erstellt.
  
  \item \label{config-allgemeines-punkt2}
  \index{Konfigurationsdatei!globale}
  Es gibt genau eine \gq{globale Konfigurationsdatei}. Zudem kann es beliebig 
  viele weitere XML-Dateien zur Konfiguration der weiter unten beschriebenen 
  Visualisierungsstile geben.\\
  Diese Dateien liegen in einem von GIANT fest vorgegebenen Verzeichnis.
  
  \item \index{Konfigurationsdatei!benutzerdefinierte}
  Die unter Punkt \ref{config-allgemeines-punkt2} beschriebenen Dateien zu
  Konfiguration k�nnen vom Benutzer auch in einem separaten 
  Unterverzeichnis seines Home-Verzeichnisses abgelegt werden. Findet
  GIANT beim Start ein derartiges Verzeichnis, so werden die
  Einstellungen aus der dort vorgefundenen Konfigurationsdatei, sowie
  die dort vorgefundenen Visualisierungsstile geladen.
 
\end {enumerate}


\section {Die globale Konfigurationsdatei} \index{Konfigurationsdatei}
Diese Datei ist in XML verfasst. In Ihr k�nnen die anschlie�end beschriebenen
Einstellungen vorgenommen werden.

%EVENTUELL F�R JEDEN BENUTZER EIGENE KONFIGURATIONS-DATEI VORSEHEN.

  \subsection {Verweise auf GSL-Skripte} 
  \index{Makros}
  \index{GSL!Makros}
  Es k�nnen GSL-Skript-Makros definiert werden. Jeder Makro wird durch 
  einen eindeutigen Namen und durch einen Verweis
  auf die Datei, welche das GSL-Skript enth�lt, spezifiziert.
  Der Makro kann dann aus einem Popup-Men� heraus aufgerufen werden, wobei
  dann das in der Datei stehende GSL Skript ausgef�hrt wird.

  \subsection {Farbe von Hervorhebungen}
  \index{hervorheben!Farben konfigurieren}
  Einstellung der Farben, mittels derer Selektionen oder IML-Teilgraphen 
  hervorgehoben werden.
  \begin {enumerate}
    \item Eine beliebige Farbe f�r die aktuelle Selektion.
    \item Beliebige Farben f�r das
          Hervorheben weiterer Selektionen.
    \item Beliebige Farben f�r das 
          Hervorheben von IML-Teilgraphen.
  \end {enumerate}

  \subsection {Editor zur Anzeige des Quellcodes}
  \label{Konfig Editor zur Anzeige des Quellcodes}
  \index{Editoren!Anzeige des Quellcodes}

  Der Benutzer legt in der globalen Konfigurationsdatei fest, 
  mit welchem Editor der zu IML-Knoten korrespondierende
  Quellcode automatisch angezeigt wird (unterst�tzt werden Emacs und vi). 

\section {Visualisierungsstile} \label{Config Visualisierungsstile}
\index{Visualisierungsstile!Konfiguration}
\index{Konfiguration!Visualisierungsstile}

Der Benutzer kann beliebig viele Visualisierungsstile definieren.
Mittels dieser Visualisierungsstile kann die Darstellung von
Fenster-Knoten und Fenster-Kanten dynamisch zur Laufzeit ge�ndert werden
(siehe auch 
\ref{Den Visualisierungsstil eines Anzeigefensters �ndern}).
Weitere Informationen zur Visualisierung von Fenster-Knoten und
Fenster-Kanten sind unter den Abschnitten
\ref{Visualization Visualisierung von Kanten} und
\ref{Visualization Visualisierung von Knoten} zu finden.\\
F�r die Visualisierungsstile gelten die folgenden Konventionen:


\begin {enumerate}

  \item
  F�r jeden Visualisierungsstil muss es eine entsprechende XML-Datei geben. 

  \item
  Ein Visualisierungsstil beschreibt, wie die Knoten und Kanten des 
  IML-Graphen innerhalb eines Anzeigefensters graphisch dargestellt 
  werden k�nnen.

  \item
  In einem Visualisierungsstil k�nnen klassenspezifische Einstellungen 
  vorgenommen werden, die nur f�r Knoten und Kanten gelten, 
  die zu der entsprechenden Klasse geh�ren.
  Bei jeder klassenspezifischen Einstellung f�r Knoten und Kanten kann daher 
  eine Liste der betroffenen Knoten- und Kantenklassen angegeben werden, 
  f�r die diese Einstellungen gelten sollen.

  \item
  Wird eine Kanten- oder eine Knotenklasse innerhalb eines
  Visualisierungsstils mehreren klassenspezifischen Einstellungen zugeordnet,   
  f�hrt dies zu keinem Fehler, es bleibt aber unspezifiziert, 
  welche Einstellung tats�chlich genommen wird.

  \item
  Es gibt jeweils eine Standard-Einstellung, die f�r alle
  Knoten- und Kantenklassen genommen wird, f�r die keine
  extra klassenspezifischen Einstellungen vorgenommen wurden.

  \item \index{Visualisierungsstile!Standard-Visualisierungsstil}
  Es gibt immer einen von GIANT fest
  vorgegeben Standard-Visualisierungsstil. Dieser Visualisierungsstil
  wird f�r neu erzeugte Anzeigefenster genommen. 
  Kann GIANT den einem Anzeigefenster zugewiesenen Visualisierungsstil
  nicht finden (falls die zugeh�rige XML-Datei fehlt) wird ebenfalls
  der Standard-Visualisierungsstil genommen. 

\end {enumerate}


  \subsection {Name des Visualisierungsstils}
  Jeder Visualisierungsstil erh�lt einen Namen. Unter diesem Namen ist der 
  Visualisierungsstil f�r jedes Anzeigefenster einzeln ausw�hlbar.
 
  \subsection {Einstellungen innerhalb des Visualisierungsstils}
    \begin{enumerate}
      \item Die Hintergrundfarbe im Anzeigefenster.

     % \item Wann Kanten im sichtbaren Anzeigeinhalt angezeigt werden sollen.
 
     %  \subitem Nur falls Start- und Zielknoten ebenfalls im sichtbaren 
     %           Anzeigeinhalt.
     %  \subitem Auch anzeigen falls nur Startknoten im sichtbaren 
     %           Anzeigeinhalt.
     %  \subitem Auch anzeigen falls nur Zielknoten im sichtbaren Anzeigeinhalt.
     %  \subitem Auch anzeigen falls weder Start- und Zielknoten im sichtbaren 
     %           Anzeigeinhalt. 
    \end{enumerate}


  \subsection {Klassenspezifische Einstellungen f�r Knoten}
  \index{Visualisierungsstile!klassenspezifische Einstellungen}

  Folgende  Einstellungen k�nnen f�r Knotenklassen vorgenommen werden. 
  Es muss eine Standard-Einstellung erstellt werden, die f�r alle
  Knotenklassen angewendet wird, f�r die nichts anderes definiert ist.
  \begin{enumerate}  
  

    \item Ein Icon f�r die Knotenklasse (Verweis auf eine entsprechende 
          Bilddatei). Das Icon muss im Pixmap-Format 
          vorliegen. Die Gr��e des Icons ist beliebig, bei zu gro�en
          Icons wird abgeschnitten. Es wird garantiert, dass Icons
          bis zu einer Gr�\ss e von 32*32 Pixel komplett dargestellt werden
          k�nnen.

    \item Eine Liste der Attribute der Knotenklasse, welche direkt innerhalb 
          des Anzeigefensters in dem \gq{Rechteck} f�r den Knoten dargestellt 
          werden sollen.
    \item Die Farbe der Schrift, mit der die Informationen innerhalb
          des Knoten-Rechtecks dargestellt werden.
    \item Die F�llfarbe des \gq{Rechtecks} in welchem die Attribute zu dem 
          Knoten dargestellt werden.
    \item Die Rahmenfarbe des \gq{Rechtecks}.
    
  \end{enumerate}

  
  \subsection {Klassenspezifische Einstellungen f�r Kanten}
  Folgende Einstellungen k�nnen f�r Kantenklassen vorgenommen werden.
  Es muss eine Standard-Einstellung erstellt werden, die f�r alle
  Kantenklassen angewendet wird, f�r die nichts anderes definiert ist.
  \begin{enumerate}
    \item Die Farbe der Kante.
    \item Die Farbe der Kantenbeschriftung.
    \item Die Art der Linie der Kante (normal, gestrichelt).
    \item Ob die Kante mit ihrer Kantenklasse beschriftet werden soll oder 
          nicht.
  \end{enumerate}

  Eine Kantenklasse wird durch die Knotenklasse des Start-Knotens 
  und durch den Namen des Attributes, welches die Kante darstellt, 
  eindeutig festgelegt (siehe Begriffslexikon). 
  Will der Benutzer nun Einstellungen f�r
  Kantenklassen vornehmen, kann er sich entweder auf genau eine
  Kantenklasse \gq{Start-Knotenklasse.Attributname} beziehen oder
  mittels eines Platzhalters auch auf mehrere:

  \begin{enumerate}
    \item Der Benutzer kann mittels des Platzhalters \gq{*}
          eine klassenspezifische Einstellung f�r
          alle Kanten, deren Start-Knoten zu einer bestimmten
          Knotenklasse geh�rt, 
          vornehmen -- also \gq{Start-Knotenklasse.*}.

   \item  Der Benutzer kann mittels des Platzhalters \gq{*}
          eine klassenspezifische Einstellung f�r
          alle Kanten, bei denen das sie repr�sentierende Attribut
          den gleichen Namen hat,
          vornehmen -- also \gq{*.Attributname}.
 
  \end{enumerate}

  
\section {Skript-Dateien} 
\label {Config Anfrage-Dateien}
\index{GSL!Skriptdateien} \index{Skriptdateien}

Eine Skript-Datei ist eine Textdatei, die genau ein GSL Skript enth�lt.
Auf diese Art und Weise k�nnen komplexe Skripte gespeichert und 
wiederverwendet werden.\\
Geladen werden k�nnen solche Dateien beim Kommandozeilenaufruf 
(siehe \ref{fa Starten von GIANT})
oder im Dialog zur Eingabe eine Skripts 
(vgl.\ UseCase \ref{UC Anfrage laden}).
F�r eine exakte Spezifikation der GSL siehe Kapitel 
\ref {GIANT Scripting Language}. 



%===============================================================================
% 
% Beschreibung der GUI
%
\chapter{Beschreibung der GUI}
% ==============================================================================
%  $RCSfile: gui.tex,v $, $Revision: 1.27 $
%  $Date: 2003/02/24 20:28:13 $
%  $Author: birdy $
%
%  Description: Bis zu Revision 1.6 sind UseCase-artige Beschreibungen enthalten,
%  die f�r UseCases n�tzlich sein k�nnten. 
%
% ==============================================================================

\section{�ber die Benutzeroberfl�che}

\begin{enumerate}
  \item {GIANT besitzt eine graphische Benutzeroberfl�che, die per Maus zu
  bedienen ist. Ausgew�hlte Funktionen k�nnen auch per Tastatur ausgel�st
  werden.}
  \item {Die Interaktionssprache mit dem Benutzer ist Englisch}
\end{enumerate}


\section{�ber dieses Kapitel}    

Diese Beschreibung beschreibt alle Elemente der GUI. Sofern nicht anders
angegeben, sind hintereinander angegebene Elemente (z.B. Listeneintr�ge)
immer von links nach rechts oder von oben nach unten beschrieben.
Sofern nicht anders angegeben, sind alle Ausgaben immer linksb�ndig
formatiert.
Jedes Hauptmen� und jedes PopUpMen� bekommt in dieser Spezifikation einen
eindeutigen Namen.
Jedes Untermen� (horizontal) heist MEN�NAME-UNTERMEN�NAME
Jeder Eintag (vertikal) hei�t MEN�NAME-UNTERMEN�NAME-EINTRAGSNAME

\section {Main Window}

Im Hauptfenster (MAIN\_WINDOW), welches es nur einmal gleichzeitig
gibt, werden vorhandene Anzeigefenster und vorhandene
IML-Teilgraphen angezeigt. �ber Popup-Menues k�nnen Teilgraphen
und Anzeigefenster manipuliert werden.

Im Fenster befinden sich zwei Listen, zuoberst WINDOW\_LIST (2 Spalten),
und darunter SUBGRAPH\_LIST (4 Spalten).


  \subsection {Kopfzeile}
  In der Kopfzeile wird dargestellt \gq{GIANT <Version> : <IML-Graph-Name>}
 
  \subsection {Men�leiste (Main Window)}
  In der Men�leiste befinden sich folgende Eintr�ge:

    \subsubsection {Untermen� Datei (Main Window)}
    Im Untermen� Datei des Hauptfensters befinden sich folgende Eintr�ge.  

      \begin{enumerate}
         \item {Load Project\\
	  blah}
         \item {Save Project}
         \item {New Project}
         \item{ Quit}
      \end{enumerate}    
      
     %Eingef�gt von Martin Schwienbacher f�r Use Cases zur Annotation von 
     %Knoten
     \subsubsection {Untermen�  (Tools)}      
       \begin{enumerate}
          %Dient zum Filtern von Annotationen deren Knoten nicht mehr
	  %visualisiert sind.
	  % -- Wird nicht beim L�schen eines Knotens auch dessen Annotation gel�scht? Philipp
          \item {Delete annotations having no visible Node}
        \end{enumerate}
	

    \subsubsection {Untermen� Info (Main Window)}      
       \begin{enumerate}
          \item {About GIANT}
        \end{enumerate}
     
     
  
     

  \subsection {Statuszeile}

    In der Statuszeile befindet sich die Information:
    Projektname: x , Fenster: x , IML-Knoten: x, IML-Kanten: x

  \subsection {Window List}
  
  Im Fenster befindet sich zuoberst eine Liste WINDOW\_LIST.
   
    \subsubsection {Inhalt Window List}
    \label{WINDOW-LIST}
   
      WINDOW\_LIST hat die Spalten WINDOW\_NAME und WINDOW\_OPEN. In WINDOW\_NAME
      wird der Name der Anzeigefenster dargestellt, in WINDOW\_OPEN befindet
      sich entweder ein Strich-Symbol (Fenster nicht ge�ffnet) oder ein H�kchensymbol
      (Fenster ge�ffnet). 
    
    \subsubsection {PopUp Men� Window list}
    
    
        Beim Rechtsklick auf WINDOW\_LIST �ffnet sich ein Popup-Men� mit folgenden Men�punkten:
	\label{WINDOW-LIST-POPUP}

        \begin{enumerate}
          \item Open Window
          \item Close Window
          \item Delete Window
          \item New Window
          \item Insert IML Subgraph
          \item Export Window
          \item Import Window
          \end{enumerate}
    
  \subsection {Subgraph List}
    
    Under der WINDOW\_LIST im Fenster befindet sich eine Liste SUBGRAPH\_LIST.
   
    \subsubsection {Inhalt Subgraph List} 
    
      SUBGRAPH\_LIST hat die Spalten 
      \begin{enumerate}
        \item Subgraph Name (SUBGRAPH\_NAME)
        \item Nodes (SUBGRAPH\_NODES)
        \item Edges (SUBGRAPH\_EDGES)
        \item Highlight Color (SUBGRAPH\_HIGHLIGHT\_COL).
      \end{enumerate}
    
      In Subgraph Name stehen die (benutzerdefinierten) Namen der existierenden
      Subgraphen, in Nodes die Anzahl der Knoten des Subgraphen,
      in Edges die Anzahl der Kanten und in Highlight Color ein K�stchen mit der Farbe,
      die verwendet wird, wenn dieser Subgraph markiert wird.

    
    \subsubsection {PopUp Men� Subgraph List}
    \label {PopUp Men� Subgraph List}
      Beim Rechtsklick auf SUBGRAPH\_LIST �ffnet sich ein Popup-Men� mit folgenden Men�punkten:
	\label{SUBGRAPH-LIST-POPUP}
        \begin{enumerate}
          \item highlight (zum Hervorheben eines Teilgraphen in allen Fenstern mit Farbe SUBGRAPH\_COL)
          \item Change highlighting color
          \item unhighlight
          \item Copy IML Subgraph with existing layout \label{PopUp Menue Subgraph List-Old layout}
	  \item Copy IML Subgraph with new layout \label{PopUp Menue Subgraph List-New layout}
          \item Rename IML Subgraph
          \item Delete IML Subgraph
          \item IML Subgraph Set Operation (Union/Difference/Intersection)
          \item Export IML Subgraph
          \item Import IML Subgraph
        \end{enumerate}   
        
    
  \subsection {Zugeh�rige Dialoge}
  
  	\begin{enumerate}
		\label{SUBGRAPH-LIST-NEW-WINDOW}
		\item SUBGRAPH\_LIST/NEW\_WINDOW
			Dieser Dialog erm�glicht, ein neues Fenster anzulegen
			
	\end{enumerate}
  >>>>>>>>>>>>KURZE Beschreibung aller DIALOGE die NUR!!!! vom HAUPTFENSTER AUS erreichbar sind.


\section {Anzeigefenster}

  Im Programm kann es kein, eines oder mehrere Anzeigefenster geben.
  
  Ein Anzeigefenster besteht aus folgenden Elementen:

  \subsection {Minimap}

    In der MiniMap wird der gesamte IML-Teilgraph, der im Fenster
    visualisiert ist, durch einen grauen Kasten repr�sentiert.
    Dabei repr�sentiert der graue Kasten die Ausdehnung des gesamten
    Graphen inklusive der z.Z.\ nicht im Fenster sichtbaren Bereiche.
    Der im Fenster sichtbare Bereich wird in der MiniMap durch einen
    Rahmen angezeigt. Je nach Zoomstufe ist der sichtbare Bereich
    gr��er oder kleiner.
    Durch Klicks in die Minimap kann die linke obere Ecke des
    sichtbaren Bereiches versetzt werden, der in VIS\_PANE angezeigte
    Bereich wird dann entsprechend angepa�t.

\section{Anzeigefenster}

  In Anzeigefenstern (DISPLAY\_WINDOW) werden Graphen visualisiert,
  sie erscheinen im grossen quadratischen rechten Teil (VIS\_PANE) des Fensters.
  Im linken Viertel des Fensters, der Toolbar (DISPLAY\_WINDOW\_TOOLBAR)
  befinden sich untereinander die MiniMap (MINIMAP), Pinliste (PIN\_LIST)
  mit darunterliegenden Buttons, und Selektionsliste (SELECTION\_LIST),
  die Zoom-Kontrolle, die Stilauswahl-Combobox (STYLE\_COMBO) und
  ein Button zum Ausblenden der Toolbar (TOOLBAR\_HIDE).

  \subsection{Visualisierung der Knoten}

    In der VIS\_PANE werden die Knoten und Kanten angezeigt. Wenn sich der
    Mauszeiger �ber der VIS\_PANE befindet, verwandelt er sich in ein
    Fadenkreuz.
    
    
    
    Wenn auf einem Knoten die rechte Maustaste geklickt wird, �ffnet sich
    folgendes Kontextmen�:
    
    >>>>>>>> hier fehlen noch Punkte
    \begin{enumerate}
      \item Show Node Info Window ???
      \item Hide Node ????
      \item Show Parent ????
      \item Show Children ????
    \end{enumerate}
    Wenn in der VIS\_PANE, nicht auf einer Kante oder einem Knoten, die rechte
    Maustaste geklickt wird, �ffnet sich folgendes Kontextmen�:
    \label{Empty Vis Pane Right click}
    \begin{enumerate}
      \item Make Room
      \item New Pin
    \end{enumerate}
    

  \subsection{Pins} \label{VIS-PANE-Pins}

    In der Pinliste (PIN\_LIST) k�nnen Pins festgelegt werden.
    Auf der Pinliste sind folgende Men�punkte eines Popup-Men�s aktivierbar:
    
    \begin{enumerate}
      \item New Pin (PIN\_NEW)
      \item Focus Pin (PIN\_JUMP),
      \item Delete Pin (PIN\_DELETE).
    \end{enumerate}

  \subsection{Selektionen} \label{Selektionsauswahlliste}

    In der Selektionsauswahlliste (SELECTION\_LIST) sind folgende Men�punkte
    eines Popup-Men�s aktivierbar:
    \begin{enumerate}
      \item New Selection
      \item Add Nodes to Selection
      
      \item Highlight Selection  Color1 [gem�� Einstellung in Config Datei]   
      \item Highlight Selection  Color2 [gem�� Einstellung in Config Datei]   
      \item Highlight Selection  Color3 [gem�� Einstellung in Config Datei] 
      
      \item Unhighlight Selection
      \item Change highlighting color 
      \item Copy Selection
      \item Rename Selection
      \item Delete Selection \label{Delete Selection}
      \item Selection Set Operation (Union/Difference/Intersection)
      \item Export Selection
      \item Import Selection
    \end{enumerate}    


  \subsection{Zoom-Kontrolle}

    Die Zoom-Kontrolle besteht aus folgenden Elementen:
    
  \begin{enumerate}
      \item ComboBox ZOOM\_CONTROL mit vorgefertigten Werten und Eingabem�glichkeit
      \item Button \gq{Display} (ZOOM\_DISPLAY)
      \item Button \gq{Zoom+} (ZOOM\_PLUS)
      \item Button \gq{Zoom-} (ZOOM\_MINUS)
  \end{enumerate}    
     
  \subsection{Stilauswahl-Combobox}

    In der Stilauswahl-Combobox (STYLE\_CHOOSER) kann ein Anzeigestil
    festgelegt werden.

  \subsection{Ein- und Ausklappbutton}

    Durch Klick auf den Ein- und Ausklappbutton (VIS\_PANE\_BUTTON) kann die VIS\_PANE
    ein- und ausgeblendet werden.
    
  \subsection{Scrolleisten}
  \label{Scrolleisten}
    An den beiden Scrolleisten (links/rechts und oben/unten) in jedem Anzeigefenster
    kann der im Fenster sichtbare Bereich des Graphen verschoben werden.

\section{Knoten-Informationsfenster}

    In Knoten-Informationsfenstern (NODE\_INFO) k�nnen n�here Informationen
    zu Knoten angezeigt werden.
    Es kann beliebig viele Knoten-Informationsfenster geben, jedoch maximal
    eines pro Knoten.

    Zuoberst im Fenster wird die Knotentyp und Knoten-ID dargestellt
    (TYPE und ID), Format: "Node: TYPE, ID: ID) darunter befinden sich
    (jeweils untereinander) drei getrennt scrollbare Listen,
    ATTRIBUTES, SUCCESSOR\_NODES, PREDECESSOR\_NODES.
    
    Unter den drei Listen befinden sich nebeneinander mittig zwei Buttons
    CLOSE und PICK (zum Ausw�hlen eines anderen Knotens zur Anzeige).
    
    \subsection{Liste ATTRIBUTES}

      Die Liste ATTRIBUTES enth�lt alle Attribute des Knotens, hat zwei Spalten:
    
      \begin{enumerate}
        \item ATTRIBUTE (Attribute Name)
        \item VALUE (Attribute Value)
      \end{enumerate}
    
    \subsection{Liste SUCCESSOR\_NODES}
    
      Die Liste SUCCESSOR\_EDGES hat eine Zeile f�r jede vom Knoten abgehende
      Kante mit drei Spalten:
    
    \begin{enumerate}
      \item EDGE\_TYPE (Typ der abgehenden Kante: Edge Type)
      \item NODE\_TYPE (Typ des Knotens: Node Type)
      \item NODE\_ID (ID des Knotens, zu dem sie f�hrt: Node ID)
    \end{enumerate}    

    \subsection{Liste PREDECESSOR\_NODES}
      
      Die Liste PREDECESSOR\_EDGES hat eine Zeile f�r jede vom Knoten abgehende
      Kante mit drei Spalten:

      \begin{enumerate}
        \item EDGE\_TYPE (Typ der ankommenden Kante: Edge Type)
        \item NODE\_TYPE (Typ des Knotens: Node Type)
        \item NODE\_ID (ID des Knotens, von dem sie kommt: Node ID)
      \end{enumerate}

\section{Anfragedialog}

Der Anfragedialog QUERY\_DIALOG hat zuoberst ein Textfeld QUERY\_TEXT,
in dem der Text der Anfrage eingegeben werden kann.

Unter QUERY\_TEXT befinden sich folgende Elemente:

      \begin{enumerate}
        \item Button QUERY\_START (Start Query)
        \item Button QUERY\_CANCEL (Cancel)
        \item Button QUERY\_LOAD (Load Query)
        \item Button QUERY\_SAVE (Save Query)
      \end{enumerate}

>>>>>>>>>>>>> WIRD DER FOLGENDE TEXT BENOETIGT?
Nach Beendigung der Berechnung erscheint ein Fenster QUERY\_RESULT,
in dem \gq{Anfrage beendet. Gefundene Knoten: QUERY\_FOUND\_NODES ,
Gefundene Kanten: QUERY\_FOUND\_EDGES} steht. Darunter befindet sich
der Text �bernehmen als IML-Teilgraph und eine Combo-Box IML\_SUBGRAPHS,
in der die bisher geladenen IML-Teilgraphen und eine Eingabem�glichkeit
f�r einen neuen Namen sind.
Darunter befinden sich zwei Buttons: Anfrage verwerfen und Anfrage
�bernehmen (QUERY\_USE , QUERY\_DISCARD) zum Verwerfen oder
�bernehmen als IML-Teilgraph. 

\section{Allgemeiner Texteingabedialog}
\label{DIALOG-WINDOW}

Ein allgemeiner Texteingabedialog ist ein Fenster DIALOG\_WINDOW mit dem
Titel \gq{GIANT Dialog}. Im Fenster ist ein einzeiliges Textfeld TEXT\_FIELD
mit einem Prompt-Text TEXT\_LABEL davor, zwei Buttons OK\_BUTTON und CANCEL\_BUTTON
mit den Beschriftungen OK\_LABEL und CANCEL\_LABEL.
Defaultm��ig sind die Buttons mit \gq{OK} und \gq{Cancel} beschriftet.



\section{Common\_Set\_Operation\_Dialog}\label{Common-Set-Operation-Dialog}

Dieser Dialog dient dazu, Mengenoperationen �ber Selektionen und
IML-Teilgraphen durchzuf�hren.\\
Die Mengenoperation l�sst sich dann wie folgt beschreiben:
TARGET := LEFT\_SOURCE <op> RIGHT\_SOURCE.\\ \\


Bestandteile des Dialoges sind:
\begin {enumerate}

 \item LEFT\_SOURCE Combobox\\
 Soll eine Mengenoperation f�r Selektionen ausgef�hrt werden, so werden hier
 alle Selektionen des entsprechenden Anzeigefensters angezeigt.\\
 Bei einer Mengenoperation �ber IML-Teilgraphen, werden alle IML-Teilgraphen
 des Projektes angezeigt.
 
 \item TARGET Textfield\\
 Name der neuen Selektion oder des neuen IML-Teilgraphen als Ergebnis
 der Mengenoperation.
  
 \item Ein OK Button: OK\\
 Entsprechende Mengenoperation wird ausgef�hrt.
 
 \item Ein Cancel Button: CANCEL\\
 Bricht die Eingabe ab.
 
\end {enumerate}


\section {Layoutalgorithmen Dialog}\label{Layoutalgorithmen-Dialog}

Der Dialog Layoutalgorithmen bietet folgende M�glichkeiten:

\begin {enumerate}
\item {Auswahl des Layoutalgorithmus}
\item Eingabe der zu ber�cksichtigenden Kanten bei semantischen Layouts
\item OK Button
\item Cancel Button
\item Textfeld zur Eingabe des Namens f�r die Selektion (in dem Anzeigefenster wo sie eingef�gt wurde)

\end {enumerate}




\section{Dateneingabe}
Beschreibung von verschiedenen Textfeldtypen etc. zur Eingabe von Daten.

\subsection{Platz Schaffen-Dialog}
\label{Platz Schaffen-Dialog}

Im Dialogfenster MAKE\_ROOM wird in einem Textfeld als Zahl angegeben, um wieviel
Pixel die Knoten an der vorher markierten Stelle auseinandergeschoben werden sollen.
Mit dem Button OK kann best�tigt werden, mit Button Cancel abgebrochen werden.

\section{Ausgabe von Fehlermeldungen}

Ein allgemeiner Fehlerdialog ist ein Fenster ERROR\_WINDOW mit dem
Titel \gq{GIANT Error}. Im Fenster ist
ein Fehlertext Text ERROR\_LABEL und ein Button OK\_BUTTON mit der
Beschriftung OK\_LABEL.
Defaultm��ig ist der Button mit \gq{OK} beschriftet.

\section{Sicherheitsabfrage}\label{Sicherheitsabfrage}

Eine allgemeine Sicherheitsabfrage ist ein Fenster CONFIRMATION\_WINDOW mit dem
Titel \gq{GIANT Confirmation Prompt}. Im Fenster ist
ein Fragetext Text CONFIRMATION\_LABEL und ein Button YES\_BUTTON mit der
Beschriftung YES\_LABEL, sowie ein NO\_BUTTON mit der Beschriftung
NO\_LABEL.
Defaultm��ig ist der YES\_BUTTON mit \gq{Yes} beschriftet, NO\_BUTTON
mit \gq{No}.

\section{Auswahl von Dateien}

  \subsection {Der \gq{Standard-Filechooser-Dialog}}
  Die Auswahl von Dateien (Laden/Speichern) passiert bei GIANT mittels
  des von GTK/ADA bereit gestellten Filechooser-Dialogs.


\section{Fadenkreuzcursor}
\label{Fadenkreuzcursor}
	Der Cursor verwandelt sich nach Auswahl bestimmter
	Funktionen in ein Fadenkreuz, rechts unten im Fadenkreuz ist
	entweder NODE, EDGE oder POS zu sehen, um zu zeigen, da� entweder ein
	Knoten, eine Kante oder eine Position im Fenster auszuw�hlen ist.
	Falls beides m�glich ist, erscheint dieser Text nicht. Durch Rechtsklick
	auf einen
	Knoten/eine Kante/leere Position in einem Fenster wird dieser Knoten/diese
	Kante/Position f�r die jeweilige Funktion ausgew�hlt. Der Cursor verwandelt
	sich dann wieder in den Normalcursor.


\section{Fortschrittsanzeige}

\label{Progressbar}
Es gibt in GIANT drei M�glichkeiten zur Fortschrittsanzeige w�hrend laufender Berechnungen:

\subsection{Progressbar-1}
\label{Progressbar-1}
     Anzeige eines Dialoges mit einem sich st�ngig leicht �ndernden
     Inhalt, so dass erkennbar ist, dass das System noch arbeitet.

\subsection{Progressbar-2}  
\label{Progressbar-2}   
     Anzeige eines Progressbars, der darstellt, welcher Anteil der
     Gesamtberechnung in Prozent bereits erledigt ist.

\subsection{Progressbar-3}
\label{Progressbar-3} 
     Anzeige eines Progressbars, der �ber den aktuellen Fortschritt
     einer Berechnung informiert, ohne dabei den Gesamtaufwand zu 
     kennen. Dabei wird kontinuierlich die Anzahl der schon bearbeiteten
     Datens�tze ausgegeben.



%===============================================================================
% 
% Nichtfunktionale Anforderungen
%
\chapter{Nichtfunktionale Anforderungen}
% ==============================================================================
%  $RCSfile: nfa.tex,v $, $Revision: 1.9 $
%  $Date: 2003/04/06 21:14:54 $
%  $Author: squig $ 
%
%  Description: Nichtfunktionale Anforderungen
%
%  Last-Ispelled-Revision: 
%
% ==============================================================================

\section{Plattformunabh�ngigkeit}
\index{Plattformunabh�ngigkeit}
Das Produkt soll ein hohes Ma� an Plattformunabh�ngigkeit, wie sie sich aus der 
eingesetzten Entwicklungsumgebung ergibt, verf�gen 
(also alle Systeme auf denen GTK/ADA 1.2.12 lauff�hig ist). 
Explizit w�hrend der Entwicklung getestet und damit garantiert werden kann 
dies aber nur f�r Sun Solaris und Linux.

\section{Wartbarkeit}
\index{Wartbarkeit des Produktes}
Das System soll sich durch ein hohes Ma� an Wartbarkeit und Erweiterbarkeit 
auszeichnen. Erreicht werden soll dies durch die folgenden Grunds�tze.
\begin {enumerate}

  \item
  Wartbarkeit und Erweiterbarkeit gehen �ber Performanz.
  
  \item Strukturierung des Entwurfes nach Kriterien des
  Software-Engineerings.

  \item
  Strikte Trennung von GUI und Funktionalit�t.

  \item
  Einsatz von XML f�r alle editierbaren Konfigurationsdateien.

  \item
  Konfigurierbarkeit der Men�eintr�ge �ber eine XML Datei. Vorbehaltlich
  der Machbarkeit regelt der Entwurf hierzu n�heres.
 
\end {enumerate}


\section{Portabilit�t}
\index{Portabilit�t}

\begin {enumerate}

  \item
  Generell soll GIANT Betriebssystem unabh�ngig entworfen und Implementiert 
  werden.
  
  \item Weitere Ma�nahmen zur expliziten Unterst�tzung der Portabilit�t
  sind nicht vorgesehen.

\end {enumerate}

\section{Mengenger�st}
\index{Mengenger�st}
\index{Speicherverhalten}

\begin {enumerate}

  \item
  Ein explizites Mengenger�st, z.B. hinsichtlich der minimalen und 
  maximalen Anzahl von darstellbaren Knoten, gibt es nicht. 

  \item
  Gr��enbeschr�nkungen (z.B. Gr��e der zu ladenden IML-Graph-Datei, 
  Anzahl Knoten in einem Anzeigefenster ...) sollen sich alleine aus dem 
  verf�gbaren Arbeitsspeicher ergeben. 
   
  \item
  Falls der verf�gbare Arbeitsspeicher ersch�pft ist, soll das System nicht 
  abst�rzen, sondern vielmehr die entsprechende Aktion kontrolliert abbrechen 
  und dem Benutzer ein Weiterarbeiten erm�glichen.
\end {enumerate}

\section {Robustheit}
\index{Robustheit}
\begin {enumerate}

  \item
  GIANT soll m�glichst auf alle Fehler mit einer qualifizierenden Fehlermeldung
  reagieren und nicht unkontrolliert abst�rzen.

  \item
  Wo immer m�glich soll dem Benutzer nach Auftreten eines Fehlers ein sinnvolles
  Weiterarbeiten erm�glicht werden.

  \item
  Insbesondere bei Fehlern, die mit dem �berschreiten der verf�gbaren
  Hauptspeicherkapazit�t in Zusammenhang stehen, kann solch ein robustes 
  Verhalten nicht garantiert werden.
  
  \item 
  Speicherlecks k�nnen im Falle Abbruchs einer Aktion im
  Fehlerfalle nicht ausgeschlossen werden.

\end {enumerate}



\section{Robustheit gegen�ber �nderungen der IML-Graph-Spezifikation}
\index{IML-Graph-Bibliothek!robuste Anbindung}
Das System soll so an Bauhaus angebunden werden, dass �nderungen in der 
Spezifikation des IML-Graphen m�glichst keine Wartungsarbeiten erfordern. 
Besonders soll dies f�r die verschiedenen Attribute, 
Knoten- und Kantenklassen des Bauhaus-IML-Graphen gelten, 
da diese sich oft �ndern k�nnen.
Die Klassen und Attribute der Bauhaus-IML-Graph Bibliothek werden 
durch den IML-Browser nur so unterschieden, 
dass neu hinzukommende Attribute automatisch mit erfasst und damit 
angezeigt werden k�nnen.



\section{Leistungsanforderungen}
\index{Leistungsanforderungen}
\begin {enumerate}

  \item
  Bei GIANT handelt es sich um ein striktes Einbenutzersystem.
  
  \item
  Die selbe Installation von GIANT darf nicht gleichzeitig mehrfach
  gestartet werden.
   
  \item
  Das Systemverhalten f�r den Fall, dass zwei Benutzer mittels verschiedener
  laufender Instanzen von GIANT (als Prozess im Betriebssystem) 
  gleichzeitig auf denselben Daten (Projekte und entsprechende 
  Verwaltungsdateien) arbeiten, ist undefiniert.

\end {enumerate}
  
  
   
\section{Antwortverhalten}
\index{Antwortverhalten}
\begin{enumerate}
 
  \item
  Bei Anfragen und Layoutalgorithmen kann keine maximale Rechenzeit garantiert
  werden. 
  
  \item
  Der Benutzer hat aber zu jeder Zeit die M�glichkeit entsprechende
  Aktionen abzubrechen.
  

\end{enumerate}


\section{Sicherheit}
\index{Datensicherheit}

\begin{enumerate}

   \item
   Ein automatisches Schreiben von �nderungen etc. in die Projektdateien ist
   nicht vorgesehen, der Benutzer muss dies explizit veranlassen.\\
   Bevor Informationen verloren gehen (z.B. beim Beenden von GIANT) erscheint
   aber eine Sicherheitsabfrage.
   
   \item
   Muss GIANT aufgrund eines Fehlers beendet werden, so gehen alle bis dahin
   nicht persistent f�r das Projekt gespeicherten Informationen
   unwiederbringlich verloren.


\end{enumerate}


\section{Erweiterbarkeit}
\index{Erweiterbarkeit des Produktes}

Eine sp�tere Erweiterbarkeit von GIANT ist besonders hinsichtlich der im
Folgenden beschriebenen Funktionalit�t vorgesehen. Dies wird im Rahmen des
Entwurfes ber�cksichtigt.

\begin{enumerate}

  \item
  Erweiterung um neue Layoutalgorithmen.
  
  \item
  Erweiterung im einen Dialog (eher ein ganzes Werkzeug) zur graphisch 
  unterst�tzten Eingabe von GQSL-Skripten.
  
  \item \index{Kantenknickpunkte}
  Erweiterung um Kantenknickpunkte.


\end{enumerate}



%===============================================================================
% 
% Technische Produktumgebung
%
\chapter{Technische Produktumgebung}
% ==============================================================================
%  $RCSfile: technical.tex,v $, $Revision: 1.6 $
%  $Date: 2003/09/23 19:33:38 $
%  $Author: birdy $
%
%  Description: Hier sind die Anforderungen an die Programmumgebung 
%  spezifiziert. Dazu geh�ren die verwendete Hard- und Software, sowie
%  Schnittstellen zu anderen Produkten.
%
%  Last-Ispelled-Revision:
%
% ==============================================================================

\section{Software}

\index{Anforderungen!an die Software}

Zum Betrieb von GIANT wird folgende Hardware und Software ben�tigt:
\begin{itemize}
  \item Sun Solaris, Linux oder Windows Betriebssystem
  \item Emacs oder vi Texteditor f�r die Anzeige vom Sourcecode
  \item GTK 2.0 oder h�her
  \item GTKAda 2.2
\end{itemize}


\section{Hardware}

\index{Anforderungen!an die Hardware}
Das Programm l�uft auf SPARC Workstations und x86 kompatiblen PCs.
Im Folgenden sind die minimalen Hardwareanforderungen zur Arbeit mit kleinen 
und mittleren Projekten beschrieben. 
Bei gro�en Projekten ist ein Speicherausbau
von 2 GB und mehr empfehlenswert.

\subsection{Hardwareanforderungen SPARC}

\begin{itemize}
  \item UltraSPARC-II 300 MHz
  \item 512 MB Hauptspeicher
  \item 8 Bit Grafik mit einer min. Aufl�sung von 1024*786
  \item Maus mit mindestens zwei Tasten
\end{itemize}

\subsection{Hardwareanforderungen x86}
\begin{itemize}
  \item Pentium III 600 MHz
  \item 512 MB Hauptspeicher
  \item 8 Bit Grafik mit einer min. Aufl�sung von 1024*786
  \item Maus mit mindestens zwei Tasten
\end{itemize}

\section{Installation}

Die Giant-Installation unter Linux geht folgenderma�en vonstatten:\\

Zun�chst mu� das Installationsarchiv (welches sich im Userverzeichnis,
z.B. /home/sysop befindet), entpackt werden. Bei der GIANT-Version
1.1 ist das entpackte Distributions-Archiv ca. 450 MB gro�.\\

Entpacken des gzip-Archives:

�cd /home/sysop�\\

�gzip -d giant-1.1.0.tar.gz�\\

Entpacken des darin befindlichen Tar-Archives:

�tar xf giant-1.1.0.tar�\\

Im nun existierenden Verzeichnis giant-1.1.0 kann
GIANT mit ./giant gestartet werden.

\section{Compilieren}

GIANT kann durch Ausf�hren von �make� im Verzeichnis
src compiliert werden.




%===============================================================================
% 
% Anforderungen an die Entwicklungsumgebung
%
\chapter{Anforderungen an die Entwicklungsumgebung}
% ==============================================================================
%  $RCSfile: development.tex,v $, $Revision: 1.4 $
%  $Date: 2003/03/28 21:57:12 $
%  $Author: schwiemn $
%
%  Description:
%
% ==============================================================================

\section {Compiler und Bibliotheken} 
\index{Compiler}
\index{Bibliotheken}
\index{GTK/ADA}
\index{GNAT}
Das System soll in der Sprache Ada95 mit GNAT 3.14 entwickelt werden, 
wobei GTK/ADA 1.2.12 als grafische
GUI-Bibliothek eingesetzte werden soll. 
Das System baut auf der vom Kunden bereit gestellten IML-Graph-Bibliothek auf 
und soll des weiteren zur Unterst�tzung der Wartbarkeit m�glichst auch die 
vom Kunden zur Verf�gung gestellten 
Datenstrukturen (wie z.B. Hashtables aus Bauhaus/reuse/src) nutzen.

  
  \subsection {Lizenzrechtliches zu den Paketen des Kunden}
  \index{Pakete der Bauhaus-Reengineering GmbH}
  Folgendes gilt nicht f�r die IML-Graph-Bibliothek.\\
  Die vom Kunden zur Verf�gung gestellten Datenstrukturen 
  werden den Entwicklern von GIANT ohne lizenzrechtliche Bedingungen 
  �berlassen. Die Nutzungsrechte der Enwtickler am Produkt GIANT werden 
  durch Einsatz dieser Datenstrukturen in keinster Weise ber�hrt.
 


\section {Einlesen und Schreiben von XML Dateien}
\index{XML Dateien}
Auf XML-Dateien soll mittels des DOM (Document Object Model) Parsers
aus XML/Ada 0.7.1 zugegriffen werden. XML/Ada 0.7.1 unterliegt lizenzrechtlich der \gq{GNAT Modified GNU Public
License} (GMGPL).\\
Als Alternative ist der XML-Parser aus GTK/Ada vorgesehen -- Paket Glib.XML.




\section {Sprache}
\index{Sprache der Dokumente}
Hier wird beschrieben, in welcher Sprache die einzelnen Dokumente
des Systems GIANT verfasst werden, und in welcher SPrache die

\section {Spezifikation}
Die Spezifikation -- dieses Dokument -- wird in deutscher Sprache verfasst.

\section {Benutzerhandbuche}
Das Benutzerhandbuch wird in deutscher Sprache verfasst.

\section {Entwurfs}
Der Entwurf wird in Englisch verfasst.

\section {Sprache der internen Dokumentation}
Die interne Dokumentation von GIANT -- Kommentare im Quellcode -- 
erfolgt in englischer Sprache.

\section {Interaktionssparche mit dem Benutzer}
\index{Interaktionssprache}
Die GUI von GIANT interagiert mit dem Benutzer ausschlie�lich in
englischer Sprache.

\section {Spache der Konfigurationsdateien}
Die Knoten und Attribute der XML-Konfigurationsdateien
werden mit englischen Begriffen benannt.






%===============================================================================
% 
% Begriffslexikon
%
\chapter{Begriffslexikon}
% ==============================================================================
%  $RCSfile: nomenclature.tex,v $, $Revision: 1.11 $
%  $Date: 2003/02/18 15:21:04 $
%  $Author: schwiemn $
%
%  Description:
%
% ==============================================================================

\begin{nomenclature}

\term{Anzeigefenster}{Visualisation Window}{
Ein Fenster in dem ein Teilgraph des IML-Graphen nach bestimmten Kriterien 
visualisiert ist. Jedem Anzeigefenster ist ein Anzeigeinhalt zugeordnet.
}

\term{Anzeigeinhalt}{Window Content}{
Eine \gq{virtuelle} Oberfl�che auf der die Objekte des visualisierten Teilgraphen
(also Fenster-Knoten und Fenster-Kanten) angeordnet sind, 
d.h. r�umliche Layoutinformation zu allen Objekten des 
entsprechenden Anzeigefensters. Abh�ngig von der Zoomstufe ist jeweils nur ein 
bestimmter Teil des Anzeigeinhaltes sichtbar - der sichtbare Anzeigeinhalt. 
Die Gr��e des Anzeigeinhaltes ist theoretisch unbegrenzt. 
}

\term{Sichtbarer Anzeigeinhalt}{Visible Window Content}{
Der Bereich des Anzeigeinhaltes eines Anzeigefensters, der zur Zeit sichtbar 
dargestellt wird.
}

\term{Zoomstufe}{Zoom Level}{
Dieser Faktor beschreibt die Gr��e des sichtbaren Anzeigeinhaltes.
Bei einer sehr niedrigen Zoomstufe (auch: weit weg gezoomt) ist ein gr��erer 
Teil des im Anzeigefenster visualisierten IML-Graphen sichtbar als bei einer
hohen Zoomstufe (auch: sehr nach heran gezoomt).
}

\term{IML-Graph}{IML Graph}{
Der IML-Graph, wie er von der Bauhaus Reengineering GmbH gestetellt wird.
Auf diesen Graphen wird �ber das sogenannte Reflection Model zugegriffen.
}

\term{Reflektion}{Reflection Model}{
Die Schnittstelle zum Zugriff auf den IML-Graphen.
}

\term{IML-Teilgraph}{IML Subgraph}{
Eine Menge �ber Knoten und Kanten des IML-Graphen, die so gestaltet ist,
dass sie einen Teilgraphen des IML-Graphen darstellt.
}

\term{Anfrage}{Query}{
Eine Anfrage beschreibt einen Vorgang, bei dem �ber geeignete Kriterien
Konten und Kanten aus dem IML-Graphen oder aus IML-Teilgraphen ausgew�hlt
werden.
}

\term{Selektion}{Selection}{
Eine Auswahl von Knoten und Kanten eines visualisierten Teilgraphen des 
IML-Graphen innerhalb eines Anzeigefensters.
}

\term{IML-Knoten}{IML Node}{
Ein Knoten des IML-Graphen.
}

\term{IML-Kante}{IML Edge}{
Eine Kante des IML-Graphen.
}

\term{Graph-Knoten}{Graph Node}{
Ein Knoten des IML-Graphen welcher Bestandteil eines IML-Teilgraphen ist.
}

\term{Graph-Kante}{Graph Edge}{
Eine Kante des IML-Graphen welche Bestandteil eines IML-Teilgraphen ist.
}

\term{Fenster-Knoten}{Window Node}{
Die grafische Repr�sentation eines IML-Knoten innerhalb eines Anzeigefensters.
}

\term{Fenster-Kante}{Window Edge}{
Die grafische Repr�sentation einer IML-Kante innerhalb eines Anzeigefensters.
}

\term{Knoten-Annotationen}{Node Annotation}{
Eine textuelle Beschreibung zu einem bestimmten Knoten des IML-Graphen.
}


\term{Knotenklasse}{Node Class}{
Die Einteilung der Knoten des IML-Graphen in verschiedene Klassen, 
wie sie sich aus der IML-Graph-Bibliothek von Bauhaus ergibt.
}

\term{Kantenklasse}{Edge Class}{
Die Einteilung der Kanten des IML-Graphen in verschiedene Klassen, 
wie sie sich aus der IML-Graph-Bibliothek von Bauhaus ergibt.
}

\term{Layout}{Layout}{
Die zweidimensionale r�umliche Anordnung von Fenster-Knoten und Fenster-Kanten innerhalb
eines Anzeigefensters auf dem sogenannten Anzeigeinhalt.
}


\term{selektieren}{to select}{
Selektieren beschreibt einen Vorgang �ber den der Benutzer, z.B.\ durch Anklicken 
von Fenster-Knoten oder Fenster-Kanten mit der Maus, eine Selektion aufbaut.
}


\term{hervorheben}{to highlight}{
Hervorheben bedeutet, dass in einem Anzeigefenster visualisierte Fenster-Knoten oder Fenster-Kanten 
z.B.\ durch  eine farbige Umrahmung von anderen Fenster-Knoten oder Fenster-Kanten unterscheidbar
gemacht werden
}

\end{nomenclature}

%%% Local Variables: 
%%% TeX-master: "spec"
%%% End: 



%===============================================================================
% 
% Anhang
%
\appendix
Anhang

\end{document}
