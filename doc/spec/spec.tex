% ==============================================================================
%  $RCSfile: spec.tex,v $, $Revision: 1.23 $
%  $Date: 2003/03/27 07:14:46 $
%  $Author: stupro $
%
%  Description: Master Datei der Spezifikation
%
% ==============================================================================

\documentclass[a4paper,titlepage,11pt,german,twoside]{scrbook}
               
\usepackage{../styles/common}
\usepackage{spec}

% subsubsections nummerieren
\setcounter{secnumdepth}{3}

%\newcommand\version{Version 1.0\xspace}
\newcommand\version{\today\xspace}

% header
\fancyhead{}
\fancyhead[LE,RO]{\slshape \company}
\fancyhead[LO,RE]{\slshape \leftmark}
\fancyfoot[LE,RO]{\thepage{}}
\fancyfoot[LO,RE]{\version}

\makeindex

\begin{document}

% title page
\thispagestyle{empty}
\hfill
\parbox{5cm}
{Universit�t Stuttgart\\
Studienprojekt A - IML Browser\\
\company}

\vspace{5cm}

\begin{center}
  \Huge
  \textsf{Spezifikation \product}\\
  \vspace{1cm}\Large\today
\end{center}
\newpage


%===============================================================================
%
% ToDo - Aus fertigem Dokument entferenen
%
\chapter{ToDo -- nicht in fertiger SPez.}

Hier Sachen beschreiben, die irgenwie beim Erstellen der Spez.\ 
beachtet werden sollen.

\input{todo_spec}


%===============================================================================
%
% Inhaltsverzeichnis
%
\setcounter{tocdepth}{1}
\tableofcontents


%===============================================================================
%
% Einleitung
%
\chapter{Einleitung}
% ==============================================================================
%  $RCSfile: intro.tex,v $, $Revision: 1.1 $
%  $Date: 2003/02/13 03:15:14 $
%  $Author: schulzgt $
%
%  Description:
%
% ==============================================================================

\section{�ber dieses Dokument}
doc
\section{�ber GIANT}
giant



%===============================================================================
%
% Produkt�bersicht
%
\chapter{Produkt�bersicht}

In diesem Kapitel werden grundlegende Konzepte von GIANT 
vorgestellt und kurz beschrieben. Eine ausf�hrliche
Spezifikation dieser Konzepte erfolgt weiter hinten
in der Spezifikation.

% ==============================================================================
%  $RCSfile: product.tex,v $, $Revision: 1.1 $
%  $Date: 2003/02/13 03:15:14 $
%  $Author: schulzgt $
%
%  Description:
%
% ==============================================================================

\section{IML-Teilgraphen}
\section{Anzeigefenster und Selektionen}
\section{Pins}
\section{Drei-Stufen-Konzept}

\chapter{Visualisierung des IML-Graphen}
Genaue Beschreibung der Visualisierung des Graphen in einem Anzeigefenster.
\section{Visualisierung von Knoten}
\section{Visualisierung von Kanten}
\section{Kantenknickpunkte}
\section{Minimap}

\chapter{Anfragesprache und Kommandozeilenaufruf}
\section{Beschreibung der Anfragesprache}
\section{Kommandozeilenaufruf}

\section{Das Projektverzeichnis}
\subsection{Verwaltungsdateien f�r Anzeigefenster}
\subsection{Verwaltungsdateien f�r IML-Teilgraphen}
\subsection{Verwaltungsdatei f�r Anfragen}
Kann er auch selber mittels Paste und Copy im Emacs machen.
\subsection{Identifikation der IML-Graph Datei}
\subsection{Grundlegendes Verhalten beim Speichern}



%===============================================================================
%
% Allgemeines zu den Funktionalen ANforderungen
%
\chapter{Allgemeine Funktionale Anforderungen}

Dieses Kapitel beschreibt die funktionalen Anforderungen
an GIANT, die das System als Ganzes und nicht nur
einzelne UseCases betreffen. 
Dies sind z.B. allegemeine Bedienkonzepte zur Selektion
von Knoten und Kanten, sowie das grundlegene 
Fehlerverhalten von GIANT.
% ==============================================================================
%  $RCSfile: afa.tex,v $, $Revision: 1.3 $
%  $Date: 2003/02/26 07:50:26 $
%  $Author: schwiemn $
%
%  Description: Beschreibung allgemeiner funktionaler Anforderungen - UseCase
%  �bergreifend.
%
% ==============================================================================


\section{Fehlerverhalten}
\begin {enumerate}
  \item
  Fehlermeldungen werden grunds�tzlich �ber einen allgemeinen Fehlerdialog
  ausgegeben. Ist dies nicht m�glich, wird eine entsprechende Fehlermeldung
  �ber die Kommandozeile ausgegeben.
  \item
  GIANT verh�lt sich bez�glich der Fehlerbehandlung konsistent.
  Das im Folgenden beschriebene Fehlerverhalten gilt daher pauschal f�r
  das gesamte System. 
  \item
  Abweichungen von diesem Verhalten werden gegebenenfalls an der 
  entsprechenden Stelle explizit spezifiziert.
\end{enumerate}

  \subsection {Ung�ltige oder unvollst�ndige Eingaben bei Dialogen}
  \begin {enumerate}
    \item
    Eingaben in Dialogen werden immer dann gepr�ft, wenn der Benutzer
    alle Eingaben in dem Dialogfenster durch klicken auf einen
    \gq{OK-Button} best�tigt.

    \item
    Stellt das System dann eine unzul�ssige oder fehlende Eingabe fest, so
    erscheint der allgemeine Fehlerdialog mit einer Mitteilung
    zur Fehlerursache.
  
    \item
    Nach Schlie�en des Fehlerdialoges kehrt das System zu der Stelle, die
    im Ablauf unmittelbar vor der Fehler�berpr�fung lag, zur�ck. Also
    typischerweise zu dem Dialog, in dem die fehlerhafte Eingabe get�tigt 
    wurde. Die zuvor in den jeweiligen Dialogen get�tigten Eingaben des 
    Benutzers bleiben erhalten.
    
  \end {enumerate}

  \subsection {Ung�ltige oder unvollst�ndige Eingaben bei Kommandozeilenaufruf}
  Der Start von GIANT wird abgeborchen. Auf der Kommandozeile wird eine
  Meldung mit Hinweisen zur Fehlerursache ausgegeben.

  \subsection {Fehler beim Einlesen von Dateien}
  Aufgrund der Vielzahl der hierbei m�glichen Fehlern (technische Defekte etc.)
  kann das Verhalten grunds�tlich nicht spezifiziert werden.
  Generell gilt:
  
  \begin{enumerate}
    \item K�nnen zum Betrieb von GIANT n�tige Dateien nicht an der 
          daf�r vorgesehenen Position gefunden werden, so wird das 
          System mit einer entsprechenden Fehlermeldung beendet.

    \item Das Verhalten bei anderen m�glichen Fehlern (fehlende 
          Zugriffsrechte, defekte Sektoren etc.) ist 
          undefiniert.
  \end{enumerate} 


  \subsection {Einlesen unzul�ssiger Daten aus Dateien}

  \begin {enumerate}
  
    \item
    GIANT verf�gt �ber keine Mechanismen, die speziell darauf
    ausgelegt sind, die Korrektheit von Daten, die aus Dateien gelesen werden,
    zu pr�fen.
    
    \item
    Stellt das System dennoch fest, dass eine einzulesende Datei ung�ltige 
    Daten enth�lt, so wird das System mit einer entsprechenden Fehlermeldung
    beendet.
  \end {enumerate}



\section {Grundlegendes Feedbackverhalten}


   \subsection{\gq{I am busy Feedback}}
   W�hrend der der Berechnung durch Layoutalgorithmen und Anfragen 
   informiert GIANT den Benutzer laufend, dass es gerade mit Berechnungen 
   besch�ftigt ist. Dies kann wie folgt geschehen:
   \begin {enumerate}
   
     \item 
     Anzeige eines Dialoges mit einem sich st�ngig leicht �ndernden
     Inhalt, so dass erkennbar ist, dass das System noch l�uft.
     
     \item
     Anzeige eines Progressbars, der �ber den aktuellen Fortschritt
     einer Berechnung informiert, ohne dabei den Gesamtaufwand zu 
     kennen.     
     
   \end {enumerate}


   
\section {Zul�ssige Eingaben}
Hier wird beschrieben, welche Eingaben f�r Namen etc. innerhalb
von GIANT (z.B. bei den entsprechenden Texteingabedialogen) zul�ssig sind.
Unzul�ssige Namen werden generell abgelehnt und f�hren zu einer Fehlermeldung.

  \subsection {STANDARD NAME}
  Ein String, der nur die im Folgenden beschriebenen Zeichen enthalten darf.
  Der String darf nicht mit einem Schl�sselwort der GQSL Anfragesprache 
  �bereinstimmen.\\
  Zul�ssige Zeichen sind:
  \begin {enumerate}
  
    \item
    Gro�- und Kleibuchstaben des deutschen Alphabetes. Nicht zul�ssig sind
    Umlaute und das Zeichen ''"s''.
    
    \item
    Die Ziffern ''0'' bis ''9''.
    
    \item
    Das Zeichen ''\_''. 
  
   
  \end {enumerate}


  \subsection {Zul�ssige Namen}
  \begin {enumerate}
  
    \item Name eines Projektes: STANDARD NAME
    
    \item Name eines IML-Teilgraphen: STANDARD NAME\\
          Dieser Name muss f�r jeden IML-Teilgraphen innerhalb eines
	  Projektes eindeutig sein.
    
    \item Name eines Anzeigefensters: STANDARD NAME\\
          Dieser Name muss f�r jedes Anzeigefenster innerhalb des Projektes
	  eindeutig sein.
    
    \item Name einer Selektion: STANDARD NAME \\
          Dieser Name muss f�r jede Selektionn 
	  innerhalb eines Anzeigefensters eindeutig sein.
    
    \item Name eines Pins: STANDARD NAME \\
          Dieser Name muss f�r jeden Pin
	  innerhalb eins Anzeigfensters eindeutig sein.
 
  \end {enumerate}
 
  
\section {Allgemeine Bedienkonzepte}
Hier werden grundlegende Konzepte der Bedienung von GIANT beschrieben.

  \subsection {Selektieren von Fenster-Knoten und Fenster-Kanten in 
               Anzeigefenstern}
  \begin{enumerate}
  
   \item Selektieren durch Anklicken mit linker Maustaste.\\
   Nur der angeklickte Fenster-Knoten oder die angeklickte Fenster-Kante 
   wird selektiert. 
   Andere bis dahin selektierte Fenster-Knoten und Fenster-Kanten 
   werden deselektiert.
   
   \item Selektieren durch Anklicken mittels Left-Shift + linke Maustatste.\\
   Falls nocht nicht selektiert wird der angeklickte Fenster-Knoten oder die
   angeklickte Fenster-Kante zu der aktuellen Selektion hinzugef�gt.
   Falls bereits selektiert wird der Fenster-Knoten oder die Fenster-Kante 
   aus der aktuellen Selektion entfernt.
   
   \item Selektieren durch Aufziehen eines Rahmens mittes gedr�ckter 
   linker Maustaste.\\
   Alle Fenster-Knoten und Fenster-Kanten innerhalb des Rahmens werden 
   selektiert, alle Fenster-Knoten und Fenster-Kanten au�erhalb des 
   Rahmens werden deselektiert.
   
   \item Selektieren durch Aufziehen eines Rahmens mittes 
   Left-Shift + gedr�ckte linke Maustaste.\\
   Alle Fenster-Knoten und Kanten innerhalb des Rahmens, 
   die noch nicht selektiert sind, werden der aktuellen Selektion 
   hinzugef�gt.\\
   Alle Fenster-Knoten und Kanten innerhalb des Rahmens, 
   die bereits selektiert
   sind, werden aus der aktuellen Selektion entfernt.
  
  \end{enumerate}
  

    
  \subsection {Aktuelle Selektion vs.\ Selektionen} 
               \label{Aktuelle Selektion vs Selektionen}	       
  Abgesehen von den hier beschriebenen Punkten, verh�lt sich die aktuelle
  Selektion wie jede andere Selektion. Falls nicht expilzit anders 
  spezifiziert, ist insbesondere jeder UseCase f�r Selektionen auch auf die 
  aktuelle Selektion anwendbar.
	       
  \begin {enumerate}
  
    \item
    Neben den vom Benutzer erzeugten Selektionen
    gibt es zu jedem Anzeigefenster standardm��ig immer eine aktuelle
    Selektion, diese kann auch leer sein.
  
    \item
    Es gibt zu jedem Anzeigefenster beliebig viele Selektionen, 
    davon ist maximal immer eine die aktuelle Selektion (diese kann
    nat�rlich auch leer sein, also keine Fenster-Knoten oder 
    Fenster-Kanten enthalten).
    
    \item 
    Das oben beschriebene Selektieren mittels linker Maustaste etc. betrifft
    immer nur die aktuelle Selektion. \\
    Alle anderen Selektionen k�nnen zwar hervorgehoben werden, aber nicht 
    durch das Selektieren mittels linker Maustaste etc. ge�ndert werden.
      
    \item
    Der Benutzer kann jederzeit bestimmen, welche der Selektion die
    aktuelle Selektion ist.
    
    \item 
    Die aktuelle Selektion wird immer hervorgehoben. 
  
  \end {enumerate}
  
  
  
\section {Verhalten beim Einf�gen von IML-Teilgraphen und Selektionen
in Anzeigefenster}
  \label{Verhalten beim Einf�gen von IML-Teilgraphen und Selektionen
         in Anzeigefenster}


   \subsection {Vorgabe der Zielposition}
     \label{Vorgabe der Zielposition}
   F�r bestimmte UseCases (siehe z.B.\ 
   \ref{IML-Teilgraph in Anzeigefenster einf�gen})
   muss der Benutzer manuell Zielpositionen innerhalb eines 
   zu bestimmenden Anzeigefensters ausw�hlen. An dieser
   Zielposition werden dann z.B. neue Fenster-Knoten etc. eingef�gt.\\
   
   Ablauf:
   
   
   \begin {enumerate}
   
     \item
     Der Benutzer w�hlt das entsprechende Anzeigefenster aus
     (dadurch dass er ihm je nach Betriebssystemkonvention den
     Fokus gibt). Hierbei k�nnen nur bereits ge�ffnete Anzeigefenster
     des Projektes ausgew�hlt werden.
     
     \item
     Befindet sich die Maus �ber dem Bereich des Anzeigefensters, wo
     der IML-Graph visualisiert wird, so wird anstatt des
     Mauszeigers ein Fadenkreuz angezeigt.
     
     \item
     Durch Klicken mit der linken Maustaste wird dann �ber das Fadenkreuz die
     Zielposition vorgegeben.
     
   \end{enumerate}
   
   
   \subsection {Automatisches Erzeugen neuer Selektionen im Zielfenster}
   S�mtliche Fenster-Knoten und Fenster-Kanten, die in einem Schritt in
   ein Anzeigefenster eingef�gt werden, werden dort automatisch
   zur aktuellen Selektion und k�nnen dann vom Benutzer 
   mit einem Namen versehen werden und als Selektion gespeichert werden.
   
   
   \subsection{Einf�gen von Selektionen in Anzeigefenster}
      \label{Einf�gen von Selektionen in Anzeigefenster}
   Hier wird das grundlegende Verhalten beim Einf�gen von Selektionen
   in Anzeigefenster (Kopieren einer Selektion aus einem Anzeigefenster
   in ein anderes) beschrieben.
   
   \begin{enumerate}
   
     \item
     Das Layout der Selektion wird beim Einf�gen in ein Anzeigefenster
     �bernommen.
     
     \item
     Die Position von Knoten der einzuf�genden Selektion, 
     die bereits in dem Anzeigefenster visualisiert sind,
     bleibt falls nicht an entsprechender Stelle anders
     spezifiziert unver�ndert.
     
     \item
     Kanten der Selektion werden nur eingef�gt, falls ihr Start- und Zielknoten
     ebenfalls Bestandteil der Selektion ist oder bereits im Anzeigefenster
     visualisiert ist. 
     
     \item
     Die neu eingef�gte Selektion ist nun auch dem entsprechenden 
     Anzeigefenster (dort wo sie eingef�gt wurde) als Selektion bekannt.
   
   \end{enumerate}
     
  
   \subsection{Einf�gen von IML-Teilgraphen in Anzeigefenster}
     \label{Einf�gen von IML-Teilgraphen in Anzeigefenster}
   Hier wird das grundlegende Verhalten beim Einf�gen von IML-Teilgraphen
   in Anzeigefenster (Kopieren einer Selektion aus einem Anzeigefenster
   in ein anderes) beschrieben.
   
   \begin{enumerate}
   
     \item
     Das Layout f�r den IML-Teilgraphen wird vor dem Einf�gen gem�� eines
     vom Benutzer vorgegebenen Layoutalgorithmus berechnet.
     
     \item
     Die Graph-Knoten und Graph-Kanten des IML-Teilgraphen werden dann
     genau wie eine Selektion in das Anzeigefenster eingef�gt, dies geschieht
     dann zu den gleichen Bedingungen, wie oben unter 
     \ref{Einf�gen von Selektionen in Anzeigefenster} beschrieben.
       
   \end{enumerate} 
  
  
% ==============================================================================  
\section {Verhalten bei Umwandeln von Selektionen und IML-Teilgraphen}
  \label{Verhalten bei Umwandeln von Selektionen und IML-Teilgraphen}
  
  \subsection {Selektion aus IML-Teilgraphen ableiten}
    \label{Selektion aus IML-Teilgraphen ableiten}
  
  \begin {enumerate}
  
    \item
    Der IML-Teilgraph bleibt unver�ndert.
  
    \item
    Im zu bestimmenden Anzeigefenster wird eine neue Selektion erzeugt,
    die alle Knoten und Kanten des IML-Teilgraphen umfasst, welche bereits
    als Fenster-Knoten und Fenster-Kanten im Anzeigefenster vorhanden sind.
    
    \item
    Knoten und Kanten des IML-Teilgraphen, die nicht als Fenster-Knoten und
    Fenster-Kanten im entsprechenden Anzeigefenster vorhanden sind, werden
    irgnoriert.
 
  \end {enumerate}
  
  
  
  \subsection {IML-Teilgraph aus Selektion ableiten}
    \label{IML-Teilgraphen aus Selektion ableiten}
  
  \begin {enumerate}
  
    \item
    Die Selektion bleibt unver�ndert.
    
    \item
    Es wird ein neuer IML-Teilgraph erzeugt. Dieser umfasst alle Knoten und
    Kanten der Selektion, welche einen g�ltigen Teilgraphen bilden.
    Kanten der Selektion, deren Start- und Zielknoten nicht ebenfalls
    Bestandteil der Selektion sind, werden also ignoriert.
 
  \end {enumerate}
  

\section {Verhalten beim Entfernen von Fenster-Knoten und Fenster-Kanten}
  \label{Verhalten beim Entfernen von Fenster-Knoten und Fenster-Kanten}
  
  
  \subsection {Entfernen aller Knoten und Kanten einer Selektionen}.
    \label {Entfernen aller Knoten und Kanten einer Selektionen}
  
  \begin {enumerate}
  
    \item
    Alle Fenster-Knoten und Fenster-Kanten der Selektion werden aus dem
    Anzeigefenster entfernt.
    
    \item
    Fenster-Kanten im Anzeigefenster, deren Start- oder Zielknoten
    entfernt wird, werden ebenfalls entfernt (auch wenn sie nicht
    zur entsprechenden Selektion geh�ren).
    
        
    \item 
    Andere Selektionen des Anzeigefensters werden entsprechend aktualisiert.
      
  \end{enumerate}




\section {Auseinanderschieben von Fenster-Knoten}
  \label {Auseinanderschieben von Fenster-Knoten}
  
Sollen Fenster-Knoten auf dem Anzeigeinhalt automatisch
auseinandergeschoben werden, so geschieht dies nach dem folgenden
Modus (keine zentrische Streckung).

\begin {enumerate}

  \item
  Es gibt einen fixen Punkt von dem aus alle Fenster-Knoten weggeschoben
  werden.
  
  \item 
  Die Fenster-Knoten werden immer entlang einer Strecke von diesem
  Punkt durch den jeweiligen Fensterknoten verschoben.
  
  \item
  Jeder Fenster-Knoten wird um einen konstanten Betrag (kann vom Benutzer 
  eingegeben werden) von dem Punkt weggeschoben.
 
\end {enumerate}






















%===============================================================================
% 
% Funktionale Anforderungen
%
\chapter{Funktionale Anforderungen}

Dieses Kapitel beschreibt die funktionalen Anforderungen
aln GIANT inform von UseCases, d.h. alle Aktionsm�glichkeiten, 
die dem Benutzer �ber die GUI von GIANT zug�nglich sind, 
weden in diesem Kapitel durch einen UseCase repr�sentiert.

% =============================================================================
%  $RCSfile: fa.tex,v $, $Revision: 1.7 $
%  $Date: 2003/02/19 10:55:17 $
%  $Author: schwiemn $
%
%  Description:
%
% =============================================================================

\section{Der Aktor \gq{Benutzer}}
Aktor im Sinne der anschlie�end spezifizierten UseCases ist immmer die 
menschliche Person, die den IML-Browser GIANT gerade benutzt, also der 
\gq{Benutzer}. Da GIANT keinen Mehrbenutzerbetrieb vorsieht, ist dies immer 
eine einzige Person. 
Neben dem \gq{Benutzer} sind keine weiteren Aktoren vorgesehen.\\


  \subsection{Anforderungen an den Aktor \gq{Benutzer}}
  GIANT ist eine Profi-Werkzeug. Weder das System noch das Handbuch richten 
  sich an unerfahrene Benutzer. Zur Bedienung muss der Benutzer daher 
  zwingend �ber die folgenden Kenntnisse verf�gen:

  \begin{itemize}
    \item Erfahrung im Umgang mit grafischen GUIs und 
          dem entsprechenden Betriebssystem.
  
    \item Grundkenntnisse in XML 
          (nur f�r das Editieren der Konfigurationsdateien).

    \item Erfahrung im Bereich Reengineering.

    \item Kenntnisse �ber Struktur und Aufbau des IML-Graphen. 

  \end{itemize}


% =============================================================================
\section{Starten von GIANT}

Der Benutzer startet das Programm GIANT durch ausf�hren der entsprechenden
Programmdatei. Hierbei hat er bez�glich eventueller Kommandozeilenparameter
die folgenden Optionen. 

  \subsubsection {M�gliche Parameter}
  \begin {enumerate}
    
    \item loadiml=''Pfad''\\
    Angabe des Pfades zu einer existierenden IML-Graph Datei.
      
    \item loadproject=''Pfad''\\
    Angabe des Pfades zu einer existierenden Projektdatei.
      
    \item executequery=''Anfrage''\\
    Eingabe eines Wortes der Anfragespreche als String.
     
    \item loadquery=''Pfad''\\
    Pfad zu einer Datei in der ein Wort der Anfragespreche gespeichert ist.
      
  \end {enumerate}
      
  \subsection {M�gliche Aufrufe}

  \begin {enumerate}
  
    \item Eingabe von: <giant>\\
    GIANT startet und ziegt das Main Window an. 
    Es ist noch kein Projekt geladen.
    
    \item Eingabe von: <giant -loadiml=''my\_iml\_file''>\\
    Neues Projekt (erh�lt einen Default-Namen) wird ge�ffnet.
    Dieses Projekt ist noch nicht persistent gespeichert.
    \label{Kommandozeile_neus}

    \item Eingabe von: <giant -loadproject=''my\_project\_file''>\\
    Das angegebene Projekt wird ge�ffnet.
    \label{Kommandozeile_altes}
    
    \item
    Bei \ref{Kommandozeile_neus} und \ref{Kommandozeile_altes}
    kann zus�tzlich entweder der Parameter\\
    <... -executequery''my\_query\_string''> oder der Parameter\\
    <... -loadquery''my\_query\_file''> eingegeben werden.\\
    In diesem Falle wird dann die �bergebene Anfrage ausgef�hrt.
    Die Ergebnisse werden dem neuen oder dem bestehenden Projekt 
    hinzugef�gt, sind aber noch nicht persistent gespeichert.


  \end {enumerate}
  
% ==============================================================================
\section{Beenden von GIANT}
Giant kann durch Schlie�en des HAUPTFENSTERS oder durch Auswahl 
des entsprechenden Eintages im Hauptmen� oder durch ALT+F4 beendet werden.\\
Ablauf:

\begin {enumerate}
  \item Der Benutzer gibt das Kommando zum Beenden von GIANT.
  \item GIANT fragt nach, ob das Projekt gespeichert werden soll.
  (PROJEKT SPEICHERN) (ENTSPRCHENDER DIALOG MUSS IN DIE GUI)
\end {enumerate}
  



% =============================================================================
% Use Cases
  
% ==============================================================================
%  $RCSfile: load_store.tex,v $, $Revision: 1.16 $
%  $Date: 2003/04/05 12:41:57 $
%  $Author: squig $
%
%  Description: UseCases f�r Lade- und Speicherfunktionalit�t
%
%  Last-Ispelled-Revision: 1.12
%
% ==============================================================================

\begin{uc}[Neues Projekt]{UC: Neues Projekt}
\index{Projekte!neues Projekt erstellen}
  Erstellt ein neues GIANT Projekt. Ein eventuell bereits ge�ffnetes
  Projekt wird dabei geschlossen, wobei �nderungen auf Nachfrage 
  vorher gespeichert werden.
  Weitere wichtige Informationen, die in engem Zusammenhang mit 
  diesem UseCase stehen, sind unter dem Abschnitt 
  \ref{Project Persistenz der Projekte} zu finden.\\
  Zul�ssige Namen f�r Projekte sind unter Abschnitt 
  \ref{afa Zulaessige Namen} spezifiziert.
  
  
  \begin{precond}
    \cond Das Programm ist gestartet.
  \end{precond}

  \begin{postsuccess}
    \cond Ein neues GIANT Projekt mit dem eingegebenen Namen 
          ist erstellt und geladen.   

    \cond Eine IML-Datei ist geladen.

    \cond Das angegebene Projektverzeichnis ist gegebenenfalls erstellt 
          worden (siehe \ref {Project Das Projektverzeichnis}).

    \cond Die Projektdatei (siehe \ref{Project Die Projektdatei})
          liegt im Projektverzeichnis.
 
    \cond Ein eventuell zuvor ge�ffnetes Projekt ist geschlossen.
    
    \cond �nderungen an einem zuvor ge�ffneten Projekt sind gespeichert
          oder verworfen.
  \end{postsuccess}

  \begin{postfail}
    \cond Das System bleibt im bisherigen Zustand.
  \end{postfail}
  
  \begin{proc}
    \step[1]
    Der Benutzer startet den UseCase �ber das Hauptmen� \gq{New Project}
    (siehe \ref{Main-Window-File}) 
  
    \step[2] Der Benutzer w�hlt im Standard-Filechooser-Dialog
    (\gq{Please Select IML File}) eine IML-Datei aus und best�tigt
    seine Eingabe mit OK (siehe \ref{Standard-Filechooser-Dialog}).
    
    \step[3] Der Benutzer gibt im Standard-Filechooser-Dialog
    den Pfad und Namen der Projektdatei (\gq{Please select Project
    Path and Project File Name}) ein. Der Name der Projektdatei ist
    automatisch auch der Name f�r das Projekt.

    \step[4]
    Dann best�tigt er seine Eingabe mit OK.\\
    Existiert die eingegebene Projektdatei bereits, 
    erscheint eine Fehlermeldung gem�� den unter Abschnitt 
    \ref{afa Fehlerverhalten} beschriebenen Konventionen.\\
    Existiert in dem Projektverzeichnis bereits eine andere Projektdatei, so 
    erscheint ebenfalls eine Fehlermeldung.\\
    Das Verzeichnis in dem die Projektdatei liegt, wird automatisch zum 
    Projektverzeichnis.
    
    \step[5] Falls bereits ein Projekt ge�ffnet ist, erscheint eine 
    Sicherheitsabfrage (siehe \ref{Sicherheitsabfrage})
    ob dieses gespeichert werden soll. Lehnt der Benutzer dies ab, gehen
    alle nicht gespeicherten Informationen verloren.\\
    Entscheidet der Benutzer sich f�r Speichern, so wird
    die unter \ref{Alles Speichern} beschriebene Funktionalit�t 
    ausgef�hrt.
    
    \step[6]
    GIANT schlie�t das aktuell ge�ffnete Projekt, erstellt 
    ein neues Projekt und �ffnet dieses.
       
  \end{proc}

  \begin{aproc}
    \astep{2} Der Benutzer bricht die Verarbeitung mit Cancel ab.  
    \astep{3} Der Benutzer bricht die Verarbeitung mit Cancel ab.
  \end{aproc}
\end{uc}

% ==============================================================================

\begin{uc}[Projekt �ffnen]{UC: Projekt �ffnen}
\index{Projekte!�ffnen}
  �ffnet ein GIANT Projekt. Ein eventuell bereits vorhandenes 
  Projekt wird dabei geschlossen, wobei �nderungen auf Nachfrage 
  vorher gespeichert werden. \\ 
  Sollte der Benutzer die XML-Dateien 
  innerhalb des Projektverzeichnisses (siehe 
  \ref{Project Das Projektverzeichnis}) manuell modifiziert haben, so
  dass diese von den durch GIANT automatisch erstellten Dateien
  abweichen, so wird keinerlei Garantie f�r das korrekte �ffnen
  des Projektes �bernommen. Das Verhalten
  bez�glich eventuell auftretender Fehler ist undefiniert.

  \begin{precond}
    \cond Das Programm ist gestartet.
  \end{precond}

  \begin{postsuccess}
    \cond Das gew�nschte GIANT Projekt ist geladen.
    \cond Die zugeh�rige IML-Datei ist geladen.
    \cond �nderungen an einem zuvor ge�ffneten Projekt sind gespeichert
    oder verworfen.
  \end{postsuccess}

  \begin{postfail}
    \cond Das System bleibt im bisherigen Zustand.
  \end{postfail}
  
  \begin{proc}    
    \step[1]
    Der Benutzer startet den UseCase �ber das Hauptmen�. 
    (siehe \ref{Main-Window-File} Men�) \gq{Load Project}
    
    \step[2] 
    Der Benutzer w�hlt aus dem Standard-Filechooser-Dialog (siehe 
    \ref {Standard-Filechooser-Dialog})
    eine vorhandene GIANT Projektdatei (siehe \ref{Project Die Projektdatei}) 
    aus und best�tigt mit OK.
    
    \step[3] Falls bereits ein Projekt ge�ffnet ist, erscheint eine 
    Sicherheitsabfrage (siehe \ref{Sicherheitsabfrage})
    ob dieses gespeichert werden soll. Lehnt der Benutzer dies ab, gehen
    alle nicht gespeicherten Informationen verloren.\\
    Entscheidet der Benutzer sich f�r Speichern, so wird
    die unter \ref{Alles Speichern} beschriebene Funktionalit�t 
    ausgef�hrt.
    
    \step[4]
    GIANT schlie�t das alte Projekt und l�dt das angegebene Projekt.
  \end{proc}

  \begin{aproc}
    \astep{2} Der Benutzer bricht die Verarbeitung mit Cancel ab.
  \end{aproc}
\end{uc}

% ==============================================================================

\begin{uc}[Projekt speichern]{UC: Projekt speichern}  
\index{Projekte!speichern}
  Speichert alle �nderungen an einem Projekt. Der Zustand der
  entsprechenden Verwaltungsdateien im Projektverzeichnis entspricht
  nach erfolgreicher Ausf�hrung dieses UseCases exakt dem aktuellen
  Zustand des ge�ffneten Projektes. Alle Konventionen zur
  Persistenz von Projekten sind im Abschnitt 
  \ref{Project Persistenz der Projekte} exakt spezifiziert.

  \begin{precond}
    \cond Das Programm ist gestartet.

    \cond Ein Projekt ist ge�ffnet. 

  \end{precond}

  \begin{postsuccess}
    \cond Die Informationen des Projekts 
    (einschlie�lich aller m�glichen �nderungen) sind persistent in die
    Verwaltungsdateien geschrieben.
 
  \end{postsuccess}

  \begin{postfail}
    \cond Das System bleibt im bisherigen Zustand.
  \end{postfail}
  
  \begin{proc}    
    \step[1]
    Der Benutzer startet den UseCase �ber das Hauptmen�
    (\ref{Main-Window-File}) \gq{Save Project}.
 
     
    \step[2]
    GIANT f�hrt die unter \ref{Alles Speichern} beschriebene Funktionalit�t 
    aus und speichert alle Informationen zu dem Projekt in der zugeh�rigen
    Projektdatei.
  \end{proc}


  \begin{aproc}

    \ageneral 
    Es sind keine alternativen Abl�ufe vorgesehen.

  \end{aproc}

\end{uc}



% ==============================================================================

\begin{uc}[Projekt speichern unter]{UC: Projekt speichern unter}
\index{Projekte!unter neuem Namen speichern}
Speichert alle Informationen zu einem Projekt in eine neue Projektdatei
(entsprechende Verwaltungsdateien werden ebenfalls dupliziert).


  \begin{precond}
    \cond Das Programm ist gestartet.
    \cond Ein Projekt ist ge�ffnet (entweder ein neu erzeugtes oder ein
    geladenes).
  \end{precond}

  \begin{postsuccess}
    \cond Eine neue Projektdatei ist erzeugt.
    \cond Das angegebene Projektverzeichnis ist gegebenenfalls erstellt 
          worden.
    \cond Die Informationen des Projekts sind persistent in die 
          neuen Verwaltungsdateien im Projektverzeichnis 
          der neuen Projektdatei geschrieben (siehe auch
          \ref{Project Persistenz der Projekte}).
        
    \cond Das Projekt bleibt in GIANT ge�ffnet, zuk�nftiges Speichern
          (siehe \ref{Projekt speichern} betrifft nur die Dateien 
          im neu erzeugten Projekt.

    \cond Die alte Projektdatei und alle Verwaltungsdateien bleiben
          unver�ndert.
    
 
  \end{postsuccess}

  \begin{postfail}
    \cond Das System bleibt im bisherigen Zustand.
  \end{postfail}
  
  \begin{proc}    
    \step[1]
    Der Benutzer startet den UseCase �ber das Hauptmen�
    (siehe \ref{Main-Window-File}) \gq{Save Project As...}.

     
    \step[2] Der Benutzer gibt im Standard-Filechooser-Dialog 
             (siehe \ref {Standard-Filechooser-Dialog}) den Pfad, 
             das neue Projektverzeichnis 
             (siehe \ref{Project Das Projektverzeichnis})
             und den Namen f�r die neue Projektdatei
             (siehe \ref{Project Die Projektdatei}) vor. \\
    Der Name der Projektdatei ist automatisch auch der Name f�r das Projekt.
    Zul�ssige Namen sind unter Abschnitt \ref{afa Zulaessige Namen}
    spezifiziert. \\
    Das Verzeichnis in dem die Projektdatei liegt, wird automatisch zum 
    Projektverzeichnis. 
  
    \step[3]  
    Der Benutzer best�tigt seine Eingabe mit OK.\\
    Existiert die eingegebene Datei schon, erscheint eine Fehlermeldung.\\
    Existiert in dem Projektverzeichnis bereits eine andere Projektdatei, so 
    erscheint eine entsprechende Fehlermeldung.\\
  
  \end{proc}


 \begin{aproc}
    \astep{2} Der Benutzer bricht die Verarbeitung mit Cancel ab.  
 \end{aproc}
\end{uc}







% ==============================================================================
%  $RCSfile: gui_window.tex,v $, $Revision: 1.17 $
%  $Date: 2003/03/19 16:40:55 $
%  $Author: stupro $
%
%  Description: Use-Cases f�r die Fensterfunktionalit�t der GUI
%
% ==============================================================================

\begin{uc}[Leeres Anzeigefenster erzeugen]{UC: Leeres Anzeigefenster erzeugen}
�ber diesen UseCase kann der Benutzer neue Anzeigefenster innerhalb eines
Projektes anlegen.


  \begin{precond}
    \cond Ein Projekt ist geladen.
  \end{precond}

  \begin{postsuccess}
    \cond 
    Das neue leere Anzeigefenster ist ge�ffnet.
    
    \cond
    Das neue Anzeigefenster ist Bestandteil des Projektes. Eine
    Verwaltungsdatei f�r das Anzeigefenster ist im Projektverzeichnis
    angelegt.

  \end{postsuccess}

  \begin{postfail}
    \cond Das System bleibt im bisherigen Zustand.
  \end{postfail}
  
  \begin{proc}    
    \step[1]
    Der Benutzer startet den UseCase �ber das PopUp Men� \ref{WINDOW-LIST-POPUP}
    durch Auswahl von \gq{New Window}
    
    \step[2] 
    GIANT �ffnet den allgemeinen Texteingabedialog \gq{Enter Name for new Window}.
    \ref{DIALOG-WINDOW}.
      
    \step[3] 
    Der Benutzer gibt dort einen zul�ssigen Namen f�r das neue Anzeigefenster
    ein und best�tigt seine Eingabe mit OK.
    
    \step[4]
    GIANT erzeugt ein neues Anzeigefenster und �ffnet dies.
  
  \end{proc}

  \begin{aproc}
    \astep{3} Der Benutzer bricht die Verarbeitung mit Cancel ab.
  \end{aproc}


\end{uc}




% ==============================================================================
\begin{uc}[Anzeigefenster l�schen]{UC: Anzeigefenster l�schen}
Mit diesem UseCase werden bestehende Anzeigefenster aus dem Projekt 
entfernt und gel�scht. Alle Informationen zu dem Anzeigefenster
gehen hierbei unwiederbringlich verloren.


  \begin{precond}
    \cond Ein Projekt mit mindestens einem Anzeigefenster ist geladen.
  \end{precond}

  \begin{postsuccess}
    \cond 
    Das gel�schte Anzeigefenster ist nicht mehr Bestandteil des Projektes.
    
    \cond
    Die Verwaltungsdatei f�r das Anzeigefenster wurde ebenfalls gel�scht.
 
  \end{postsuccess}

  \begin{postfail}
    \cond Das System bleibt im bisherigen Zustand.
  \end{postfail}
  
  \begin{proc}    
    \step[1]
    Der Benutzer startet den UseCase �ber das PopUp Men� \ref{WINDOW-LIST-POPUP}
    
    \step[2] 
    Giant zeigt die allgemeine Sicherheitsabfrage und fragt nach, ob das
    Anzeigefenster wirklich gel�scht werden soll.
    \gq{Do you really want to delete Window xy ?}
      
    \step[3] 
    Der Benutzer best�tigt mit YES.
    
    \step[4]
    GIANT entfernt das Anziegefenster aus dem Projekt und l�scht die 
    zugeh�rige Verwaltungsdatei.
  
  \end{proc}

  \begin{aproc}
    \astep{3} Der Benutzer bricht die Verarbeitung mit NO ab.
  \end{aproc}
  
\end{uc}


% ==============================================================================
\begin{uc}[Anzeigefenster �ffnen]{UC: Anzeigefenster �ffnen}
Dient zum �ffnen eines Anzeigefensters des Projektes.


  \begin{precond}
    \cond Ein Projekt mit mindestens einem Anzeigefenster ist geladen.
    \cond Es gibt mindestens ein nicht ge�ffnetes Anzeigefenster.
  \end{precond}

  \begin{postsuccess}
  
    \cond Das Anzeigefenster ist ge�ffnet.
 
  \end{postsuccess}


  \begin{proc}    

    \step[1]
    Der Benutzer f�hrt einen Rechtsklick auf ein nicht ge�ffnetes
    Anzeigefenster in der \ref{WINDOW-LIST} durch und w�hlt im
    PopUp Men� \ref{WINDOW-LIST-POPUP}  den entsprechenden Eintag aus.
   
    \step[2]
    GIANT �ffnet das entsprechende Anzeigefenster.
  
  \end{proc}
 
\end{uc}

% ==============================================================================
\begin{uc}[Anzeigefenster schliesen]{UC: Anzeigefenster schliessen}
Mit diesem UseCase wird ein ge�ffnetes Anzeigefenster geschlossen.

  \begin{precond}
    \cond Ein Projekt mit mindestens einem Anzeigefenster ist geladen.
    \cond Es gibt mindestens ein ge�ffnetes Anzeigefenster.
  \end{precond}

  \begin{postsuccess}
  
    \cond Das Anzeigefenster ist geschlossen.
    
    \cond Nach dem letzten Speichern am Anzeigefenster vorgenommene
    Modifikationen (neue Knoten eingef�gt etc.) sind in der Verwaltungsdatei
    gespeichert oder nicht.
 
  \end{postsuccess}


  \begin{proc}    

    \step[1]
    Der Benutzer schlie�t das Anzeigefenster (durch klicken auf das ''X''
    Symbol rechts oben in der Titelleiste des Anzeigefensters).\\
    Alternativ kann er das Anzeigefenster �ber das entsprechende PopUp
    Men� \ref{WINDOW-LIST-POPUP} in der Liste \ref{WINDOW-LIST} schlie�en.
         
    \step[2]
    GIANT zeigt die allgemeine Sicherheitsabfrage 
    (siehe \ref{Sicherheitsabfrage}) und fragt nach, ob es eventuelle
    �nderungen im Anziegefenster speichern soll oder nicht.
    \gq{Do you want to save changes made to Window xy before closing it?}
    
    
    \step[3]
    Best�tigt der Benutzer mit YES, werden die �nderungen in die 
    Verwaltungsdatei geschrieben. Anderenfalls gehen s�mtliche 
    nicht gespeicherten �nderungen am Anzeigefenster verloren.
    
    \step[4]
    GIANT schlie�t das Anzeigefenster.
  
  \end{proc}


\end{uc}


% ==============================================================================
\begin{uc}[IML-Teilgraph in Anzeigefenster einf�gen]
          {UC: IML-Teilgraph in Anzeigefenster einf�gen}

Mit diesem UseCase k�nnen die Graph-Kanten und Graph-Knoten 
von IML-Teilgraphen in Anzeigefenster eingef�gt werden.
Siehe hierzu auch   
\ref{Verhalten beim Einf�gen von IML-Teilgraphen und Selektionen 
in Anzeigefenster},
insbesondere \ref{Einf�gen von IML-Teilgraphen in Anzeigefenster}.
	  

  \begin{precond}
    \cond Ein Projekt mit mindestens einem ge�ffneten 
          Anzeigefenster ist geladen.
    
    \cond Es gibt mindestens einen IML-Teilgraphen.
    
  \end{precond}

  \begin{postsuccess}
    
    \cond 
    Alle Knoten und Kanten des IML-Teilgraphen sind in das Anzeigefenster
    entsprechend dem gew�hlten Layout an der vorgegebenen Position
    eingef�gt.
    
    \cond
    In dem Anzeigefenster gibt es eine Selektion, die die neu eingef�gten
    Knoten und Kanten umfasst.
   
  \end{postsuccess}

  \begin{postfail}
    \cond Hat der Benutzer den UseCase an irgendeinem Punkt abgebrochen,
    kehrt das System zu dem Zustand zur�ck, in dem es vor Start des
    UseCase war.
  \end{postfail}
  
  \begin{proc}    
    \step[1]
    Der Benutzer startet den UseCase �ber das PopUp Men� \ref{SUBGRAPH-LIST-POPUP}
    im Hauptfenster \gq{Insert IML Subgraph}.
    Hierdurch wird der einzuf�gende IML-Teilgraph bestimmt (immer
    der IML-Teilgraph, auf dem der Rechtsklick ausgef�hrt wurde).
    
    \step[2] 
    GIANT zeigt in der Statuszeile im Hauptfenster \gq{Select Position in Display Window
    for Insertion of IML Subgraph}
    Der Benutzer w�hlt das entsprechende Anzeigefenster aus und
    gibt �ber das Fadenkreuz \ref{Fadenkreuzcursor} die Position vor, an der die neuen
    Fenster-Knoten und Fenster-Kanten eingef�gt werden sollen.
    Die Statuszeile im Hauptfenster schaltet auf Normalmodus zur�ck.
    
    \step[3]
    GIANT zeigt den Dialog zur Auswahl des entsprechenden Layoutalgorithmus
    \ref{Layoutalgorithmen-Dialog}.
    
    \step[4]
    Der Benutzer w�hlt einen der vorgegebenen Layoutalgorithmen aus. Bei 
    semantischen Layouts gibt er auch die Kantenklassen vor,
    die ber�cksichtigt werden sollen.
    
    \step[5]
    Der Benutzer gibt den Namen f�r die Selektion ein, die im Zielfenster
    f�r die neu eingef�gten Fenster-Knoten und Fenster-Kanten erzeugt wird.\\
    Gibt der Benutzer keinen Namen an, vergibt Giant automatisch einen Namen.
        
    \step[6]
    Der Benutzer best�tigt mit OK.
          
    \step[7] 
    GIANT berechnet das entsprechende Layout und zeigt derweil einen
    DIALOG an, �ber den der Benutzer �ber den Fortschritt der Berechnung
    informiert wird \ref{Progressbar}\\
    W�hrend der Berechnung des Layouts kann das System GIANT nicht bedient
    werden (die gesamte GUI ist gesperrt). Zug�nglich ist nur der
    Button zum Abbruch des Algorithmus.
    
    \step[8]
    Nach Abschluss der Berechnung f�gt GIANT die Fenster-Knoten und 
    Fenster-Kanten in das entsprechende Anzeigefenster ein.
   
   
  
  \end{proc}

  \begin{aproc}
    \astep{6} Der Benutzer bricht die Eingabe mit Cancel ab.
    \astep{7} Der Benutzer bricht die Berechnung des Layouts ab.  
  \end{aproc}



\end{uc}

% ==============================================================================
\begin{uc}[Selektion in Anzeigefenster einf�gen]
          {UC: Selektion in Anzeigefenster einf�gen}
	  
Mit diesem UseCase kann eine Selektion aus einem Anzeigefenster
in ein anderes Anzeigefenster unter Beibehaltung des Layouts
der Quell-Selektion kopiert werden. 
Siehe hierzu  
\ref{Verhalten beim Einf�gen von IML-Teilgraphen und Selektionen in Anzeigefenster},
insbesondere \ref{Einf�gen von Selektionen in Anzeigefenster}.  

  \begin{precond}
    \cond Ein Projekt mit mindestens zwei ge�ffneten 
          Anzeigefenstern ist geladen.
    
    \cond Es gibt mindestens eine Selektion.
    
  \end{precond}

  \begin{postsuccess}
    
    \cond 
    Alle Fenster-Knoten und Fenster-Kanten der \gq{Quell-Selektion} 
    sind auch in der \gq{Ziel-Anzeigefenster} unter Beibehaltung des
    Layouts der Quell-Selektion vorhanden.
    
    \cond
    Das Layout von Fenster-Knoten, die vor dem Einf�gen bereits im
    Ziel-Anzeigefenster vorhanden waren, bleibt 
    je nach Wahl des Benutzers unver�ndert oder wird ebenfalls ge�ndert.
    
    \cond
    Die kopierte Quell-Selektion existiert auch im \gq{Ziel-Anzeigefenster}
    als Selektion und tr�gt dort ebenfalls den Namen der Quell-Selektion. 

   
  \end{postsuccess}

 
  \begin{proc}    
    \step[1]
    Der Benutzer startet den UseCase �ber das PopUp Men�
    \ref{PopUp Men� Subgraph List} im Hauptfenster.
    Durch Auswahl einer der Men�eintr�ge 
    Copy IML Subgraph with new layout oder Copy IML Subgraph with existing layout.

    Hierdurch wird automatisch die Quell-Selektion bestimmt (immer
    die Selektion, auf der der Rechtsklick ausgef�hrt wurde).
    
    \step[2] 
    Der Benutzer w�hlt das entsprechende Ziel-Anzeigefenster aus.
    GIANT zeigt in der Statuszeile im Hauptfenster \gq{Select Position in Display Window
    for Insertion of copied IML Subgraph}
    Der Nutzer gibt gibt �ber das Fadenkreuz die Position vor, an der die neuen
    Fenster-Knoten und Fenster-Kanten eingef�gt werden sollen.
    \ref{Fadenkreuzcursor}
    Die Statuszeile im Hauptfenster schaltet auf Normalmodus zur�ck.
    
    \step[3]
    GIANT kopiert die Knoten und Kanten der 
    Quell-Selektion in das Ziel-Anzeigefenster.
    Je nachdem, welchen Eintrag der Benutzer im PopUp Men� ausgew�hlt hat,
    geschieht mit den bereits im Ziel-Anzeigefenster vorhandenen Knoten
    der Quell-Selektion folgendes:
    \begin {enumerate}
       \item
       Ihre Position wird im Ziel-Anzeigefenster nicht ver�ndert.
       
       \item
       Ihre Position wird im Zielanzeigefenster gem�� den dem Layout der
       Quell-Selektion ver�ndert.
        
    \end {enumerate}
   
     
  \end{proc}



\end{uc}

% ==============================================================================
\begin{uc}[Knoten und Kanten einer Selektion aus Anzeigefenster l�schen]
      {UC: Knoten und Kanten einer Selektion aus Anzeigefenster l�schen}
	  
	  
Die Selektion wird gel�scht, alle Fenster-Knoten und Fenster-Kanten
der Selektion werden ebenfalls gel�scht. 
Siehe auch 
\ref{Verhalten beim Entfernen von Fenster-Knoten und Fenster-Kanten}.

 
  \begin{precond}
    \cond Ein Projekt mit mindestens einem ge�ffneten 
          Anzeigefenster ist geladen.
    
    \cond Es gibt eine Selektion.
    
  \end{precond}

  \begin{postsuccess}
    
    \cond
    Die Selektion ist aus dem Anzeigefenster gel�scht.
    
    \cond 
    Alle betroffenen Fenster-Knoten und Fenster-Kanten sind gem��
    der unter 
    \ref{Verhalten beim Entfernen von Fenster-Knoten und Fenster-Kanten}
    beschriebenen Konvention aus dem Anzeigefenster entfernt.
    
 
   
  \end{postsuccess}

  \begin{postfail}
    \cond Hat der Benutzer den UseCase an irgendeinem Punkt abgebrochen,
    kehrt das System zu dem Zustand zur�ck, in dem es vor Start des
    UseCase war.
  \end{postfail}
  
  \begin{proc}    
  
    \step[1]
    Der User f�hrt einen Rechtsklick auf die entsprechende Selektion durch
    \ref{Selektionsauswahlliste} und w�hlt im 
    entsprechenden PopUp Men� den Eintrag (Delete Selection) aus.
      
    
    \step[2]
    GIANT zeigt die Sicherheitsabfrage (siehe \ref{Sicherheitsabfrage}) 
    und fragt nach, ob es die Selektion samt ihrer Knoten und Kanten
    wirklich l�schen soll.
    (\gq{Really delete Selection xy from its window including Nodes and Edges?})
    
    \step[3]
    Der Benutzer best�tigt mit YES.
    
    \step[4]
    GIANT l�scht die Selektion samt allen zugeh�rigen Fenster-Knoten und
    Fenster-Kanten aus dem entsprechenden Anzeigefenster.
      
  
  \end{proc}

  \begin{aproc}
    \astep{2} Der Benutzer bricht den UseCase mit NO ab.
  \end{aproc}






\end{uc}

% ==============================================================================
\begin{uc}[Anzeigefensters scrollen]{UC: Anzeigefensters scrollen}
Ver�ndert die Position des sichtbaren Anzeigeinhaltes.

  \begin{precond}
    \cond 
    Ein Projekt mit mindestens einem ge�ffneten Anzeigefenster ist geladen.
       
  \end{precond}

  \begin{postsuccess}
    
    \cond
    Die Position des sichtbaren Anzeigeinhalts wurde entsprechend abge�ndert.
    
  \end{postsuccess}

  \begin{proc}    
  
    \step[1]
    \begin {enumerate}
      \item
      Der Benutzer scrollt den sichtbaren Anzeigefokus mittels der horizontalen
      und vertikalen Scrollleisten des Anzeigefensterts. Dies geschieht mittels
      der Maus gem�� der g�ngigen Konventionen aus GTK/Ada f�r Scrollleisten.
      \ref{Scrolleisten}
      
      \item
      Gem�� der g�ngigen intuitiven Konventionen kann auch 
      mittels der Cursortasten gescrollt werden.\\
      Das Dr�cken der linken Cursortaste f�hrt z.B.\ dazu, dass der sichtbare
      Anzeigeinhalt des Anzeigefensters, welches innerhalb des Windowmanagers
      den Fokus hat, nach links verschoben wird.
      
    \end {enumerate}
        
  \end{proc}

\end{uc}


% ==============================================================================
\begin{uc}[Label]{UC: Anzeigefenster zoomen}
Ver�ndert den Ma�stab der Darstellung von Knoten und Kanten (relativ)

  \begin{precond}
    \cond 
    Ein Projekt mit mindestens einem ge�ffneten Anzeigefenster ist geladen.
       
  \end{precond}

  \begin{postsuccess}
    
    \cond
    Der angezeigte Bereich des sichtbaren Anzeigeinhalts wurde entsprechend
    vergr��ert oder verkleinert.
    Der Detaillierungslevel wurde ggf. automatisch angepa�t.
    
  \end{postsuccess}

  \begin{proc}    
  
    \step[1]
    \begin {enumerate}
      \item
      Der Benutzer gibt in der Zoomkontroll-Combobox des Anzeigefensters einen
      neuen Zoomwert ein, w�hlt darin einen der vorgefertigten Werte aus oder �ndert
      den Zoomwert in festgelegten Schritten mit den Zoom+ oder Zoom- Buttons.
      
      \item
      Der Benutzer klickt auf den \gq{Display} - Button der Zoomkontrolle
      
    \end {enumerate}
        
  \end{proc}
\end{uc}

% ==============================================================================
\begin{uc}[Label]{UC: Zoomen auf eine gesamte Selektion}
W�hlt die Zoomstufe und Scrollt so, da� eine gesamte Selektion im Fenster sichtbar ist

  \begin{precond}
    \cond 
    Ein Projekt mit mindestens einem ge�ffneten Anzeigefenster ist geladen.
       
  \end{precond}

  \begin{postsuccess}
    
    \cond
    Der angezeigte Bereich des sichtbaren Anzeigeinhalts wurde so
    vergr��ert oder verkleinert, da� die gesamte ausgew�hlte Selektion sichtbar
    ist, ggf. wurde auch entsprechend gescrollt.
    Der Detaillierungslevel wurde ggf. automatisch angepa�t.
    
  \end{postsuccess}

  \begin{proc}    
  
    \step[1]
    \begin {enumerate}
      \item
      Der Benutzer klickt auf \gq{Zoom to make selection fill window} im Popup-Men�
      der Selektionsauswahlliste \ref{Selektionsauswahlliste} auf der Selektion,
      welche er sichtbar machen will.
      
    \end {enumerate}
        
  \end{proc}
\end{uc}

% ==============================================================================
\begin{uc}[Label]{UC: Zoomen auf gesamten Inhalt eines Anzeigefensters}

W�hlt die Zoomstufe und Scrollt so, da� der gesamte Fensterinhalt im Fenster sichtbar ist

  \begin{precond}
    \cond 
    Ein Projekt mit mindestens einem ge�ffneten Anzeigefenster ist geladen.
       
  \end{precond}

  \begin{postsuccess}
    
    \cond
    Der angezeigte Bereich des sichtbaren Anzeigeinhalts wurde so
    vergr��ert oder verkleinert, da� der gesamte Fensterinhalt sichtbar
    ist, ggf. wurde auch entsprechend gescrollt.
    Der Detaillierungslevel wurde ggf. automatisch angepa�t.
    
  \end{postsuccess}

  \begin{proc}    
  
    \step[1]
    \begin {enumerate}
      \item
      Der Benutzer klickt auf \gq{Fill window} in der Zoomkontrolle des
      Fensters.
      
    \end {enumerate}
        
  \end{proc}

\end{uc}

% ==============================================================================
\begin{uc}[Zoomen auf eine Kante]{UC: Zoomen auf eine Kante}


W�hlt die Zoomstufe und Scrollt so, eine bestimmte Kante komplett im Fenster sichtbar ist

  \begin{precond}
    \cond 
    Ein Projekt mit mindestens einem ge�ffneten Anzeigefenster mit mindestens einer Kante
    ist geladen.
       
  \end{precond}

  \begin{postsuccess}
    
    \cond
    Der angezeigte Bereich des sichtbaren Anzeigeinhalts wurde so
    vergr��ert oder verkleinert, da� die betreffende Kante komplett
    sichtbar ist, ggf. wurde auch entsprechend gescrollt.
    Der Detaillierungslevel wurde ggf. automatisch angepa�t.
    
  \end{postsuccess}

  \begin{proc}    
  
    \step[1]
    \begin {enumerate}
      \item
      XXXXXXX
      
    \end {enumerate}
        
  \end{proc}

\end{uc}


% ==============================================================================
\begin{uc}[Label]{UC: Verschieben von Knoten, Selektionen, Kantenknickpunkten}
Kantenknickpunkte, Knoten, Selektionen (jeweils die aktuelle Selektion),
Drag and Drop, Cut and Paste






\end{uc}


% ==============================================================================
\begin{uc}[Label]{UC: Platz schaffen}

Dieser UseCase wird ben�tigt, um Fenster-Knoten auseinander schieben zu 
k�nnen. So kann der Benutzer an einer beliebigen Stelle des Anzeigefensters
gen�gend Platz zum Einf�gen neuer Fenster-Knoten und Fenster-Kanten schaffen.
Siehe hierzu auch \ref{Auseinanderschieben von Fenster-Knoten}.


  \begin{precond}
    \cond Ein Projekt mit mindestens einem ge�ffneten 
          Anzeigefenster ist geladen.
      
  \end{precond}

  \begin{postsuccess}
    
    \cond 
    Alle Fenster-Knoten und Fenster-Kanten des Anzeigefensters sind
    um den entsprechenden Betrag vom vorgegebenen Punkt innerhalb
    des Anzeigeigeihaltes weggeschoben worden.
    An der entsprechenden Stelle im Anzeigeinhalt ist eine freie Fl�che
    ohne Knoten und Kanten geschaffen worden.
    
    \cond
    Das Layout der betroffenen Fenster-Knoten und Fenster-Kanten bleibt
    weitgehend unver�ndert.
 
   
  \end{postsuccess}

 
  \begin{proc}    
    \step[1]
    Der Benutzer startet den UseCase �ber das PopUp Men�
    \ref{Empty Vis Pane Right click} beim Klick auf eine leere
    Fl�che in der VIS\_PANE und Auswahl von Make Room.
    
    \step[2]
    GIANT zeigt in der Statuszeile im Hauptfenster \gq{Select Position in Display Window
    for Making of Room}
    Der Benutzer gibt den Punkt um den herum die Fenster-Knoten (und damit
    automatisch auch die Fenster-Kanten) auseinander geschoben werden
    sollen �ber das Fadenkreuz vor. \ref{Fadenkreuzcursor}
    Die Statuszeile im Hauptfenster schaltet auf Normalmodus zur�ck.
     
    \step[3] 
    GIANT zeigt einen Dialog an, in dem der Benutzer ausw�hlt, um welchen 
    Betrag die Fenster-Knoten auseinander geschoben werden sollen.
     \ref{Platz Schaffen-Dialog}
   
    \step[4]
    Der Benutzer w�hlt einen geeigneten Betrag aus und best�tigt mit OK.
    
    \step[5]
    GINAT schiebt die Knoten entsprechend auseinander
    (siehe \ref{Auseinanderschieben von Fenster-Knoten}).
    
  \end{proc}


\end{uc}


% ==============================================================================
\begin{uc}[Label]{UC: Pin anlegen}

Mit diesem UseCase kann ein neuer Pin erzeugt werden.

 \begin{precond}
    \cond Es gibt ein ge�ffnetes Anzeigefenster.
  \end{precond}

  \begin{postsuccess}
    \cond 
    In der Liste �ber die Pins des Anzeigefensters
     \ref{VIS-PANE-Pins} befindet sich ein neuer Pin mit dem entsprechenden Namen.

  \end{postsuccess}

  \begin{postfail}
    \cond Das System bleibt im bisherigen Zustand.
  \end{postfail}
  
  \begin{proc}    
    \step[1]
    
    Zur Durchf�hrung gibt es zwei M�glichkeiten:
    
    \begin{enumerate}
     \item
      Der Benutzer f�hrt einen Rechtklick auf den Anziegeinhalt des
      entsprechenden Anzeigefensters durch und w�hlt aus dem PopUp
      Men� \ref{Empty Vis Pane Right click} den Eintrag \gq{New Pin} aus.
     \item   
      Der Benutzer w�hlt in der PIN\_LIST (\ref{VIS-PANE-Pins}) nach
      Rechtsklick auf der Liste im Kontextmen� \gq{New Pin} aus.
    \end{enumerate}
    \step[2] 
    GIANT �ffnet den allgemeinen Texteingabedialog \ref{DIALOG-WINDOW},
    \gq{Please enter name for new pin}
      
    \step[3] 
    Der Benutzer gibt dort einen zul�ssigen Namen f�r den neuen Pin
    ein und best�tigt mit OK.\\
      
    \step[4]
    GIANT erzeugt den Pin.
  
  \end{proc}

  \begin{aproc}
    \astep{3} Der Benutzer bricht die Verarbeitung mit Cancel ab.
  \end{aproc}

\end{uc}


% ==============================================================================
\begin{uc}[Label]{UC: Pin anspringen}
Stellt eine �ber den Pin gespeicherte Position des Anzeigefokus
wieder her.

 \begin{precond}
    \cond Es gibt ein ge�ffnetes Anzeigefenster mit mindestens 
    einem Pin.
  \end{precond}

  \begin{postsuccess}
    \cond 
    Der sichtbare Anzeigefokus des Anzeigefensters ist auf die entsprechenden
    Koordinaten und die entsprechende Zoomstufe, wie sie im ausgew�hlten
    Pin hinterlegt waren, gesetzt.
    
  \end{postsuccess}
  
  \begin{proc}    
    \step[1]
    Zur Durchf�hrung gibt es zwei M�glichkeiten:
    \begin{enumerate}
    
      \item
      Der Benutzer f�hrt einen Doppelklick auf den entsprechenden Pin
      in der Liste \ref{VIS-PANE-Pins} aus.
      
      
      \item
      Der Benutzer �ffnet das entsprechende PopUp Men� durch
      rechtsklick auf den Pin bei \ref{VIS-PANE-Pins}und w�hlt den Eintrag
      \gq{Focus Pin} aus.
        
    \end {enumerate}
    
          
    \step[2]
    GIANT setzt den sichtbaren Anzeigeinhalt gem�� den im Pin gespeicherten
    Informationen.
  
  \end{proc}



\end{uc}


% ==============================================================================
\begin{uc}[Label]{UC: Pin l�schen}
L�scht einen Pin.

 \begin{precond}
    \cond Es gibt ein ge�ffnetes Anzeigefenster mit mindestens 
    einem Pin.
  \end{precond}

  \begin{postsuccess}
    \cond 
    Der Pin ist gel�scht und nicht mehr in der Liste \ref{VIS-PANE-Pins} des entsprechenden
    Anzeigefensters ausw�hlbar.

    
  \end{postsuccess}
  
  \begin{proc}    
    \step[1]
    Der Benutzer �ffnet das entsprechende PopUp Men� durch Rechtsklick
    auf den Pin in \ref{VIS-PANE-Pins} und w�hlt den Eintrag \gq{Delete Pin} aus.

     
    \step[2]
    GIANT l�scht den entsprechenden Pin.
  
  \end{proc}

\end{uc}

% ==============================================================================
%  $RCSfile: additional.tex,v $, $Revision: 1.2 $
%  $Date: 2003/02/05 14:01:12 $
%  $Author: schulzgt $
%
%  Description: Sonstige Use-Cases
%
% ==============================================================================

\begin{uc}[Label]{UC: Verfolgen von Kanten}
Wird �ber mitgelieferte Anfragen implementiert.
Dann kein eigener UC mehr.
\end{uc}

\begin{uc}[Label]{UC: Layout auf Selektion anwenden}
\end{uc}

\begin{uc}[Label]{UC: Knoteninformation �ber Tooltip anzeigen}
\end{uc}

\begin{uc}[Label]{UC: Knoteninformation im Infofenster anzeigen}
\end{uc}

\begin{uc}[Label]{UC: Anzeige des Quellcodes eines Knotes in externem Editor}
\end{uc}

\begin{uc}[Label]{UC: Verschieben des sichtbaren Bereichs �ber Minimap}
\end{uc}


\chapter{Knoten-Annotationen}
% ==============================================================================
%  $RCSfile: node_annotations.tex,v $
%  $Date: 2003/02/18 17:05:37 $
%  $Author: schwiemn $
%
%  Description: Use-Cases zum Annotieren von Knoten
%
% ==============================================================================

\begin{uc}[Fenster-Knoten annotieren]{UC: Fenster-Knoten annotieren}
  Dieser UseCase dient zum Erzeugen von Knoten-Annotationen.
  
  \begin{precond}
    \cond Das Programm ist gestartet.
    \cond Ein Projekt ist geladen.
    \cond Es gibt mindestens ein Anzeigefenster mit Fenster-Knoten.
  \end{precond}

  \begin{postsuccess}
    \cond Die neue Annotation ist noch nicht
          in der Verwaltungsdatei f�r Knoten-Annotationen eingetragen.
    \cond Der annotierte Knoten wird in allen Anzeigefenstern als annotiert
          hervorgehoben (hat nun das entsprechende Icon).
	  
  \end{postsuccess}

  \begin{postfail}
    \cond Das System und die Verwaltungsdatei f�r Knoten-Annotationen
          bleiben im bisherigen Zustand.
  \end{postfail}
  
  \begin{proc}
    \step[1] 
    Der Benutzer w�hlt den Fenster-Knoten aus, der annotiert werden soll und
    �ffnet das entsprechende POPUPMEN� -> VERWEIS ZU GUI.
    \step[2]
    GIANT zeigt nun den EINGABEDIALOG F�R ANNOTATIONEN.
    
    \step[3]
    Der Benutzer gibt dort den entsprechenden Text f�r die Knoten-Annotation 
    ein.
  
    \step[4]
    Nach Abschluss der Eingabe bet�tigt der Benutzer den \gq{OK Button}.
            
    \step[5]
    GIANT �bernimmt die vorgenommenen �nderungen. Die eingebegbene Annotation 
    muss allerdings mindestens ein Zeichen haben, ansonsten erscheint eine
    Fehlermeldung.

  \end{proc}

  \begin{aproc}
    \ageneral 
    Der Benutzer kann die Eingabe der neuen Knoten-Annotation jeder Zeit 
    mittels des \gq{Cancel Buttons} abbrechen.
    

  \end{aproc}
\end{uc}

% ==============================================================================

\begin{uc}[Fenster-Knoten annotieren]{UC: Knoten-Annotation �ndern}
  Dieser UseCase dient zum Erzeugen von Knoten-Annotationen.
  
  \begin{precond}
    \cond Das Programm ist gestartet.
    \cond Ein Projekt ist geladen.
    \cond Es gibt mindestens ein Anzeigefenster mit Fenster-Knoten der
          annotiert ist.
  \end{precond}

  \begin{postsuccess}
    \cond Die �nderung an der Annotation ist noch nicht
          in der Verwaltungsdatei f�r Knoten-Annotationen eingetragen.
    \cond Die �nderung der Annotation ist dem System bekannt und wird
          entsprechend angezeigt.
	  
  \end{postsuccess}

  \begin{postfail}
    \cond Das System und die Verwaltungsdatei f�r Knoten-Annotationen
          bleiben im bisherigen Zustand.
  \end{postfail}
  
  \begin{proc}
    \step[1] 
    Der Benutzer w�hlt den Fenster-Knoten aus, der annotiert werden soll und
    �ffnet das entsprechende POPUPMEN� -> VERWEIS ZU GUI.
    \step[2]
    GIANT zeigt nun den EINGABEDIALOG F�R ANNOTATIONEN.
    
    \step[3]
    Der Benutzer gibt dort den entsprechenden Text f�r die Knoten-Annotation 
    ein.
  
    \step[4]
    Nach Abschluss der Eingabe bet�tigt der Benutzer den \gq{OK Button}.
            
    \step[5]
    GIANT �bernimmt die vorgenommenen �nderungen. Die eingebegbene Annotation 
    muss allerdings mindestens ein Zeichen haben, ansonsten erscheint eine
    Fehlermeldung.

  \end{proc}

  \begin{aproc}
    \ageneral 
    Der Benutzer kann die Eingabe der neuen Knoten-Annotation jeder Zeit 
    mittels des \gq{Cancel Buttons} abbrechen.
    

  \end{aproc}
\end{uc}

% ==============================================================================

\begin{uc}[Label]{UC: Knoten-Annotation l�schen}
L�scht eine bestehende Knotenannotation.
\end{uc}

% ==============================================================================

\begin{uc}[Label]{UC: Knoten-Annotationen filtern}
>>>> EVENTUELL
Entfernt alle Knoten-Annotationen, deren zugehl�rige Knoten nicht mindestens
in einem Anzeigefenster visualisiert sind.
\end{uc}


\chapter{Selektionen}
Hier Text zu Selektionen
% ==============================================================================
%  $RCSfile: selection.tex,v $, $Revision: 1.8 $
%  $Date: 2003/03/19 19:03:35 $
%  $Author: stupro $
%
%  Description: Use-Cases f�r die Selektionen
%
%
% ==============================================================================


\begin{uc}[Selektion zur aktuellen Selektion machen]
          {UC: Selektion zur aktuellen Selektion machen}
	  
Dieser UseCase dient dazu, eine Selektion zur aktuellen Selektion zu machen.
Dies ist n�tig, da nur die aktuelle Selektion mittels der Maus
modifiziert werden kann. 
Siehe auch \ref{Aktuelle Selektion vs Selektionen}. 
	  
	 
  \begin{precond}
    \cond Es gibt ein Anzeigefenster mit mindestens einer Selektion.
  \end{precond}

  \begin{postsuccess}
    \cond Die vorherige aktuelle Selektion ist nicht mehr aktuell.
    \cond Die entsprechende Selektion ist nun die aktuelle Selektion und
          als solche erkennbar angezeigt und hervorgehoben, auch mittels
	  einer Hervorhebung in der Selektionsauswahlliste des Fensters.
    
  \end{postsuccess}
 
  \begin{proc}    
    \step[1]
    Der Benutzer f�hrt mit der linken Maustaste einen Doppelklick 
    auf eine nicht aktuelle Selektion durch.

    \step[2]
    GIANT macht die entsprechende Selektion zur aktuellen Selektion.  
  \end{proc}
	  
\end{uc}


% ==============================================================================
\begin{uc}[Aktuelle Selektion zur�ck stufen]
          {UC: Aktuelle Selektion zur�ck stufen}
	  
Siehe auch \ref{Aktuelle Selektion vs Selektionen}. Mit diesem UseCase kann
eine aktuelle Selektion auf den Status einer \gq{normalen} Selektion
zur�ckgestuft werden.
	  
	 
  \begin{precond}
    \cond Es gibt ein Anzeigefenster mit einer aktuellen Selektion.
  \end{precond}

  \begin{postsuccess}
    \cond Es gibt keine aktuelle Selektion mehr.
    
  \end{postsuccess}
 
  \begin{proc}    
    \step[1]
    Der Benutzer f�hrt mit der linken Maustaste einen Doppelklick 
    auf die aktuelle Selektion aus.

    \step[2]
    GIANT stuft die aktuelle Selektion auf den Status einer \gq{normalen}
    Selektion zur�ck.
  \end{proc}
    
\end{uc}


% ==============================================================================
\begin{uc}[Selektion graphisch hervorheben]
          {UC: Selektion graphisch hervorheben}
	  
Dieser UseCase dient zum Hervorheben von Selektionen innerhalb eines
Anzeigefensters.

  \begin{precond}
    \cond Es gibt ein Anzeigefenster mit mindestens einer Selektion.
   
  \end{precond}

  \begin{postsuccess}
    \cond 
    Die Selektion ist im entsprechenden Anzeigefenster hervorgehoben.
    
    \cond
    Die Selektion, welche vorher mit der gleichen Farbe hervorgehoben war,
    ist nicht mehr hervorgehoben.
 
    
  \end{postsuccess}
 
  \begin{proc}    
    \step[1]
    Der Benutzer startet den UseCase mit Rechtsklick in
    die Selektionsauswahlliste \ref{Selektionsauswahlliste}
    auf die entsprechende Selektion �ber das PopUp Men�
    \gq{Highlight Selection Color1 (2,3)}.
        
      
    \step[2]
    GIANT hebt die Selektion mit der entsprechenden Farbe hervor.
    
  \end{proc}



\end{uc}


% ==============================================================================
\begin{uc}[Graphische Hervorhebung einer Selektion aufheben]
      {UC: Graphische Hervorhebung einer Selektion aufheben}
      
Dieser UseCase dient dazu, die Hervorhebung von Selektionen innerhalb eines 
Anzeigefensters aufzuheben.
      
  \begin{precond}
    \cond 
    Es gibt ein Anzeigefenster mit mindestens einer hervorgehobenen
    Selektion.
   
  \end{precond}

  \begin{postsuccess}
    \cond 
    Die Selektion ist im entsprechenden Anzeigefenster nicht mehr 
    hervorgehoben.
 
  \end{postsuccess}

  
  \begin{proc}    
    \step[1]
    Der Benutzer startet den UseCase mit Rechtsklick in
    die Selektionsauswahlliste \ref{Selektionsauswahlliste}
    auf die entsprechende Selektion �ber den PopUp-Men�punkt
    \gq{Unhighlight Selection}.

    \step[2]
    GIANT setzt die Hervorhebung der Selektion zur�ck.
    
  \end{proc} 
      
      
\end{uc}

%===============================================================================
\begin{uc}[Neue Selektion anlegen]{UC: Neue Selektion anlegen}

Mit diesem UseCase k�nnen neue, leere Selektionen angelegt werden.

  \begin{precond}
    \cond Es gibt ein ge�ffnetes Anzeigefenster.
  \end{precond}

  \begin{postsuccess}
    \cond Eine neue Selektion mit entsprechendem Namen
         ist angelegt und erscheint in der Liste der Selektionen 
	 (SELECTION\_LIST).
    \cond Diese neue Selektion hat keinen Inhalt (selektierte Knoten und
          Kanten).

  \end{postsuccess}

  \begin{postfail}
    \cond Das System bleibt im bisherigen Zustand.
  \end{postfail}
  
  \begin{proc}    
    \step[1]
    Der Benutzer startet den UseCase mit Rechtsklick in
    die Selektionsauswahlliste \ref{Selektionsauswahlliste}
    auf die entsprechende Selektion �ber den PopUp-Men�punkt
    \gq{New Selection}.
    
    \step[2] 
    GIANT �ffnet den allgemeinen Texteingabedialog \ref{DIALOG-WINDOW},
    \gq{Enter Name for new Selection}.
      
    \step[3] 
    Der Benutzer gibt dort einen zul�ssigen Namen f�r die neue Selektion 
    ein und best�tigt mit OK.\\
    Hat bereits eine andere Selektion innerhalb des Anzeigefensters den selben
    Namen erscheint eine Fehlermeldung.
    
    \step[4]
    GIANT erzeugt die neue Selektion.
  
  \end{proc}

  \begin{aproc}
    \astep{3} Der Benutzer bricht die Verarbeitung mit Cancel ab.
  \end{aproc}

\end{uc}


%===============================================================================
\begin{uc}[Selektion kopieren]{UC: Selektion kopieren}
Dieser UseCase dient zum Kopieren von Selektionen innerhalb eines 
Anzeigefensters (nicht zum Kopieren in ein anderes Anzeigefenster).

  \begin{precond}
    \cond Es gibt ein ge�ffnetes Anzeigefenster mit einer Selektion.
  \end{precond}

  \begin{postsuccess}
    \cond 
    Eine neue Selektion mit entsprechendem Namen ist angelegt 
    und erscheint in der Liste der Selektionen (SELECTION\_LIST).
    \cond
    Die neue Selektion umfasst die gleichen Knoten und Kanten wie die
    Selektion, von der kopiert wurde.

  \end{postsuccess}

  \begin{postfail}
    \cond Das System bleibt im bisherigen Zustand.
  \end{postfail}
  
  \begin{proc}    
    \step[1]
    Der Benutzer startet den UseCase durch Rechtsklick auf die zu kopierende
    Selektion in der Selektionsauswahlliste \ref{Selektionsauswahlliste}
    (\gq{Quellselektion}) und w�hlt aus dem PopUp Men�:
    Copy Selection.
    
    \step[2] 
    GIANT �ffnet den allgemeinen Texteingabedialog \ref{DIALOG-WINDOW},
    \gq{Please enter name for copy of Selection X} .
      
    \step[3] 
    Der Benutzer gibt dort einen zul�ssigen Namen f�r die neue Selektion 
    ein und best�tigt mit OK.\\
    Hat bereits eine andere Selektion innerhalb des Anzeigefensters den selben
    Namen erscheint eine Fehlermeldung.
    
    \step[4]
    GIANT kopiert die Quellselektion und legt eine neue Selektion an.
  
  \end{proc}

  \begin{aproc}
    \astep{3} Der Benutzer bricht die Verarbeitung mit Cancel ab.
  \end{aproc}

\end{uc}
%===============================================================================
\begin{uc}[Selektion umbenennen]{UC: Selektion umbenennen}
>>>> WEG LASSEN - KANN AUCH DURCH KOPIEREN UND L�SCHEN ERLEDIGT WERDEN

\end{uc}
%===============================================================================
\begin{uc}[Selektion l�schen]{UC: Selektion l�schen}

Dieser UseCase dient zum L�schen von Selektionen innerhalb eines 
Anzeigefensters. Hierdurch bleiben die Fenster-Knoten und Fenster-Kanten
unver�ndert.

  \begin{precond}
     \cond Es gibt ein ge�ffnetes Anzeigefenster mit einer Selektion.
   \end{precond}


  \begin{postsuccess}
    \cond 
    Die entsprechende Selektion ist gel�scht.
    
    \cond 
    Die Fenster-Knoten und Fenster-Kanten, 
    die zu dieser Selektion geh�rten, werden nicht gel�scht.
	  
    \cond War die Selektion hervorgehoben, so wird die entsprechende
    Hervorhebung der Fenster-Knoten und Fenster-Kanten aufgehoben.
    
  \end{postsuccess}

  \begin{postfail}
    \cond Das System bleibt im bisherigen Zustand.
  \end{postfail}
  
  \begin{proc}    
    \step[1]
    Der Benutzer startet den UseCase durch Rechtsklick auf die zu l�schende
    Selektion in der Selektionsauswahlliste \ref{Selektionsauswahlliste}
    (\gq{Quellselektion}) und w�hlt aus dem PopUp Men�:
    Delete Selection.
  
    
    \step[2]
    GIANT l�scht die entsprechende Selektion.
  
  \end{proc}

\end{uc}
%===============================================================================
\begin{uc}[Selektion manuell modifizieren]
         {UC: Selektionen manuell modifizieren}
	 
Jeweils die aktuelle Selektion kann mittels der Maus modifiziert 
werden. 


  \begin{precond}
     \cond 
     Es gibt ein ge�ffnetes Anzeigefenster mit mindestens
     einer Selektion.
   \end{precond}


  \begin{postsuccess}
    \cond 
    Die entsprechenden �nderungen an der Selektion werden von Giant
    sofort durchgef�hrt und �bernommen.
    
  \end{postsuccess}

  
  \begin{proc}    
    \step[1]
    Falls noch nicht der Fall, macht der Benutzer die zu modifizierende 
    Selektion zur aktuellen Selektion
    (siehe \ref{Selektion zur aktuellen Selektion machen}).
    
    \step[2]
    Mittels der unter 
    \ref{Selektieren von Fenster-Knoten und Fenster-Kanten in Anzeigefenstern}
    beschriebenen M�glichkeiten f�gt der Benutzer der Selektion neue
    Fenster-Knoten und Fenster-Kanten hinzu oder entfernt bestehende
    Fenster-Knoten und Fenster-Kanten aus der Selektion.
    
  \end{proc}


\end{uc}

%===============================================================================
\begin{uc}[Selektion aus IML-Teilgraph erzeugen]
         {UC: Selektion aus IML-Teilgraph erzeugen}
	 
Leitet eine Selektion aus einem IML-Teilgraphen ab.
Siehe hierzu \ref{Selektion aus IML-Teilgraphen ableiten}.


  \begin{precond}
     \cond 
     Es gibt ein ge�ffnetes Anzeigefenster.
     
     \cond
     Es gibt mindestens einen IML-Teilgraphen.
     
   \end{precond}


  \begin{postsuccess}
    \cond 
    Im Ziel-Anzeigefenster wurde eine neue Selektion
    mit dem entsprechenden Namen erzeugt.
    
  \end{postsuccess}
  
  \begin{postfail}
    \cond Das System bleibt im bisherigen Zustand.
  \end{postfail}
  
  \begin{proc} 
     
    \step[1]
    Der Benutzer f�hrt einen Rechtsklick auf den Quell-IML-Teilgraphen 
    in der Subgraph List im Hauptfenster aus
    und w�hlt im dazugeh�rigen PopUp Men� \ref{PopUp Men� Subgraph List}
    den Eintrag \gq{Create Window Selection from IML Subgraph}.
     
    \step[2]
    GIANT zeigt den allgemeinen Texteingabediolog \ref{DIALOG-WINDOW},
    \gq{Enter name for new selection}
      	    
    \step[3]
    Der Benutzer gibt einen Namen f�r die neu zu erstellende Selektion ein
    und best�tigt mit OK.\\
    Gibt der Benutzer hier keinen Namen ein, so vergibt GIANT automatisch 
    einen Namen.
    
    \step[4]
    Die Statuszeile im Hauptfenster zeigt an \gq{Please select window for
    Insertion of new window selection}, der Mauszeiger verwandelt sich in
    ein Fadenkreuz.
    
    \step[5]
    Der Nutzer klickt auf die VIS\_PANE des Windows, in dem er die neue
    Selektion erstellen will (Ziel-Anzeigefenster).
       
    \step[6]
    Die Statuszeile im Hauptfenster wechselt wieder zur normalen Anzeige.
    GIANT erzeugt gem�� der unter \ref{Selektion aus IML-Teilgraphen ableiten}
    beschriebenen Konvention im Ziel-Anzeigefenster eine neue Selektion als 
    Ableitung aus dem Quell-IML-Teilgraphen.

  \end{proc}
  
  \begin{aproc}
    \astep{3} Der Benutzer bricht den UseCase mit Cancel ab.
  \end{aproc}


\end{uc}



%===============================================================================
\begin{uc}[Mengenoperationen auf 2 Selektionen]
{UC: Mengenoperationen auf 2 Selektionen}
Zus�tzlich zu den M�glichkeiten der Anfragesprache kann der Benutzer
die g�ngigen Mengenoperationen, wie Mengenvereinigung, Schnitt und Differenz,
auch direkt �ber einen entsprechenden Dialog ausf�hren.
Beschreibung des Dialoges siehe \ref{Common-Set-Operation-Dialog}.
Vorgehen analog zu \ref{Mengenoperationen auf 2 IML-Teilgraphen}.



  \begin{precond}
    \cond Es gibt ein ge�ffnetes Anzeigefenster mit mindestens zwei 
          Selektionen.
  \end{precond}

  \begin{postsuccess}
    \cond 
    Eine neue Selektion mit entsprechendem Namen (eingegeben unter TARGET) 
    ist angelegt 
    und erscheint in der Liste der Selektionen (SELECTION\_LIST).
       
    \cond
    Im Falle einer Mengenvereinigung umfasst, die neue Selektion TARGET alle
    Knoten und Kanten aus der LEFT\_SOURCE Selektion und der 
    RIGHT\_SOURCE Selektion.
    
    \cond
    Im Falle einer Mengendifferenz umfasst, die neue Selektion TARGET alle
    Knoten und Kanten aus der LEFT\_SOURCE Selektion, die nicht 
    Bestandteil der 
    RIGHT\_SOURCE Selektion sind.

    \cond
    Im Falle eines Mengenschnitts umfasst, die neue Selektion TARGET alle
    Knoten und Kanten, die der LEFT\_SOURCE Selektion und der 
    RIGHT\_SOURCE Selektion gemeinsam angeh�ren.

  \end{postsuccess}

  \begin{postfail}
    \cond Das System bleibt im bisherigen Zustand.
  \end{postfail}
  
  \begin{proc}    
    \step[1]
    Der Benutzer startet den UseCase �ber durch Auswahl des
    Men�punktes Selection Set Operation (Union/Difference/Intersection) 
    aus dem PopUp Men� in der Selektionsauswahlliste SELECTION\_LIST
    \label{Selektionsauswahlliste}.
    
    \step[2] 
    GIANT �ffnet den Common\_Set\_Operation\_Dialog 
    (siehe \ref{Common-Set-Operation-Dialog}).
    
    
    \step[3] 
    Der Benutzer w�hlt dort die beiden Quell-Selektionen (LEFT\_SOURCE und 
    RIGHT\_SOURCE)
    aus, bestimmt die durchzuf�hrende Mengenoperation und gibt unter 
    TARGET den Namen der neue zu erzeugenden Selektion 
    ein.\\
    Dann best�tigt er die Eingabe mit OK.
       
    \step[4]
    GIANT f�hrt die Mengenoperation aus.
  
  \end{proc}

  \begin{aproc}
    \ageneral Der Benutzer bricht die Eingabe der Daten mit Cancel ab.
  \end{aproc}



\end{uc}
%===============================================================================
\begin{uc}[Label]{UC: Mengendifferenz von 2 Selektionen}
>>> Ist mit dem oberen UseCase abgedeckt
>>> KANN gestrichen werden

\end{uc}
%===============================================================================
\begin{uc}[Label]{UC: Mengenschnitt von 2 Selektionen}
>>> Ist mit dem oberen UseCase abgedeckt
>>> KANN gestrichen werden
\end{uc}


\chapter{Filter}
Hier Text zu Filtern
% ==============================================================================
%  $RCSfile: filter.tex,v $, $Revision: 1.7 $
%  $Date: 2003/04/03 14:22:48 $
%  $Author: squig $
%
%  Description: UseCases f�r Filter
%
%  Last-Ispelled-Revision: 1.5
%
% ==============================================================================

\begin{uc}[Selektionen ausblenden]{UC: Selektionen ausblenden}
Mit diesem UseCase k�nnen Selektionen innerhalb
eines Anzeigefensters ausgeblendet werden.

  \begin{precond}
    \cond Es gibt ein ge�ffnetes Anzeigefenster mit einer Selektion.
  \end{precond}

  \begin{postsuccess}

    \cond Alle zu der Selektion geh�rende Fenster-Knoten und
          Fenster-Kanten sind ausgeblendet, d.h. sie sind
          im Anzeigefenster nicht mehr sichtbar. Dies trifft
          auch f�r Fenster-Knoten und Fenster-Kanten, die
          noch zu weiteren Selektionen, geh�ren zu.
    
    \cond Die ausgeblendeten Selektionen k�nnen nicht 
          zur aktuellen Selektion gemacht werden (siehe 
          \ref {Selektion zur aktuellen Selektion machen} und damit
          nicht mehr direkt bearbeitet werden, 
          d.h. die Menge der selektierten 
          Fenster-Knoten und Fenster-Kanten kann nicht
          mehr abge�ndert werden (siehe hierzu \ref {Selektieren von 
          Fenster-Knoten und Fenster-Kanten in Anzeigefenstern}).


    \cond Die Fenster-Knoten und Fenster-Kanten sind aber immer noch 
          Bestandteil des Anzeigefensters und k�nnen �ber den folgenden
          UseCase (siehe \ref{Selektionen einblenden}) wieder zur Anzeige
          gebracht werden.
    
  \end{postsuccess}

   
  \begin{proc}

    \step[1] 
    Der Benutzer f�hrt einen Rechtsklick mit der Maus auf
    die auszublendende Selektion in der Selektionsauswahlliste 
    (siehe \ref{Selektionsauswahlliste}) durch 
    und w�hlt im Popup-Men� den Eintrag \gq{Fade Out Selection}
    aus. Diese Funktionalit�t kann nicht auf die aktuelle
    Selektion angewendet werden 
    (siehe \ref{Aktuelle Selektion vs Selektionen}).

    \step[2]
    GIANT blendet die Selektion aus.
 
  \end{proc}

\end{uc}



\begin{uc}[Selektionen einblenden]{UC: Selektionen einblenden}
Mit diesem UseCase k�nnen ausgeblendete Selektionen
wieder eingeblendet werden.

  \begin{precond}
    \cond Es gibt ein ge�ffnetes Anzeigefenster mit einer 
          ausgeblendeten Selektion.
  \end{precond}

  \begin{postsuccess}

    \cond 
    Die Selektion ist wieder eingeblendet, alle zu ihr geh�renden 
    Fenster-Knoten und Fenster-Kanten sind wieder
    im Anzeigeinhalt sichtbar dargestellt.
    
  \end{postsuccess}

   
  \begin{proc}

    \step[1] 
    Der Benutzer f�hrt einen Rechtsklick mit der Maus auf
    eine ausgeblendete Selektion in der Selektionsauswahlliste 
    (siehe \ref{Selektionsauswahlliste}) durch 
    und w�hlt im Popup-Men� den Eintrag \gq{Fade in Selection}
    aus. 

    \step[2]
    GIANT blendet die Selektion ein.
 
  \end{proc}

\end{uc}


%\begin{uc}[Label]{UC: Detailfilter f�r ein Anzeigefenster}
%<<<<<<<Martin: BIN MIR NICHT GANZ SICHER
%ICH GLAUBE DIESE ANFORDERUNG WURDE GECANCELT>>>>>>>>>>>>>>>
%
%
%
%\end{uc}
%
%\begin{uc}[Label]{UC: Detailfilter f�r eine Selektion}
%<<<<<<<Martin: BIN MIR NICHT GANZ SICHER - MARTIN
%ICH GLAUBE DIESE ANFORDERUNG WURDE GECANCELT>>>>>>>>>>>>>>>
%
%\end{uc}


\chapter{Teilgraphen}
Hier Text zu Teilgraphen
% ==============================================================================
%  $RCSfile: subgraph.tex,v $, $Revision: 1.2 $
%  $Date: 2003/02/05 14:01:12 $
%  $Author: schulzgt $
%
%  Description: Use-Cases f�r IML-Teilgraphen
%
% ==============================================================================

\begin{uc}[Label]{UC: Graphisch hervorheben}
\end{uc}

\begin{uc}[Label]{UC: Graphische Hervorhebung aufheben}
\end{uc}

\begin{uc}[Label]{UC: IML-Teilgraph aus einer Selektion erzeugen}
\end{uc}

\begin{uc}[Label]{UC: IML-Teilgraph kopieren}
\end{uc}

\begin{uc}[Label]{UC: IML-Teilgraph umbenennen}
\end{uc}

\begin{uc}[Label]{UC: IML-Teilgraph l�schen}
\end{uc}

\begin{uc}[Label]{UC: Mengenvereinigung von 2 IML-Teilgraphen}
\end{uc}

\begin{uc}[Label]{UC: Mengendifferenz von 2 IML-Teilgraphen}
\end{uc}

\begin{uc}[Label]{UC: Mengenschnitt von 2 IML-Teilgraphen}
\end{uc}

\begin{uc}[Label]{UC: Teilgraph exportieren}
\end{uc}

\begin{uc}[Label]{UC: Teilgraph importieren}
\end{uc}


\chapter{Anfragen}
Hier Text zu Anfragen
% ==============================================================================
%  $RCSfile: query.tex,v $, $Revision: 1.8 $
%  $Date: 2003/03/31 16:29:39 $
%  $Author: squig $
%
%  Description: UseCases f�r die Anfragen
%
%  Last-Ispelled-Revision: 1.7
%
% ==============================================================================

% ==============================================================================
\begin{uc}[Anfrage ausf�hren]{UC: Neue Anfrage ausf�hren}
\index{Anfragen!ausf�hren}
Mit diesem UseCase kann eine Anfrage �ber den Anfragedialog
(siehe \ref{GUI Anfragedialog}) eingegeben werden.
Die M�glichkeiten der GQSL sind im Detail unter Kapitel
\ref {GIANT Query Skripting Language} beschrieben.


  \begin{precond}
    \cond Ein Projekt ist geladen.
  \end{precond}

  \begin{postsuccess}
    \cond Die Anfrage wurde ausgef�hrt. Alle Ergebnisse liegen vor.
      
  \end{postsuccess}

  \begin{postfail}
    \cond Das System bleibt im bisherigen Zustand.

    \cond Wurde der UseCase w�hrend der Berechnung des
          Anfrageergebnisses durch GIANT abgebrochen, so
          gehen s�mtliche bereits fertig gestellten Teilergebnisse
          verloren. 
  \end{postfail}
  
  \begin{proc}    
    \step[1]
    Der Benutzer startet den UseCase durch Auswahl des Eintrags 
    \gq{Tools -- Execute GQSL Query} im Hauptmen�  
    (siehe hierzu \ref{Main-Window-Tools}).
      
    \step[2] 
    GIANT �ffnet den Anfragedialog (siehe \ref{GUI Anfragedialog}).
      
    \step[3]
    Der Benutzer gibt dort im daf�r vorgesehenen Textfeld die GQSL Anfrage 
    (siehe auch \ref {GIANT Query Skripting Language}) ein und best�tigt 
    mit \gq {Start Query}.

    \step[4]
    GIANT pr�ft das eingegebene GQSL Skript. Sollte das Skript
    nicht den Vorgaben der Grammatik 
    (siehe \ref {GIANT Query Skripting Language} entsprechen, erscheint
    eine Fehlermeldung (siehe \ref {afa Fehlerverhalten})
    und das System kehrt zu Schritt 3 des UseCase zur�ck.
        
    \step[5]    
    GIANT berechnet die Anfrage und teilt dem Benutzer
    den Fortschritt mittels eines Progressbars (\ref{Progressbar-Modale}) mit.
    W�hrend der Abarbeitung der Anfrage ist das System nicht bedienbar.\\
    
  
  \end{proc}

  \begin{aproc}
    \astep{3} Der Benutzer bricht mit Cancel ab.

    \astep{5} Die laufende Berechnung des Anfrageergebnisses 
    kann vom Benutzer jeder Zeit durch Bet�tigen des
    Buttons \gq{Cancel Calculation} (\ref{Progressbar-Modale-Cancel}) abgebrochen werden.

  \end{aproc}

\end{uc}


% ==============================================================================
\begin{uc}[UC Anfrage laden]{UC: Anfrage laden}
\index{Anfragen!aus Datei laden}
Der Benutzer kann zus�tzlich zur manuellen Eingabe von GQSL Anfragen
(siehe \ref{Anfrage ausf�hren}) auch gespeicherte Anfragen aus 
einer Anfragedatei (siehe \ref {Config Anfrage-Dateien}) laden.

  \begin{precond}
    \cond Ein Projekt ist geladen.
  \end{precond}

  \begin{postsuccess}
    \cond Die Anfrage wurde ausgef�hrt. Alle Ergebnisse liegen vor.
      
  \end{postsuccess}

  \begin{postfail}
    \cond Das System bleibt im bisherigen Zustand.

    \cond Wurde der UseCase w�hrend der Berechnung des
          Anfrageergebnisses durch GIANT abgebrochen, so
          gehen s�mtliche bereits fertig gestellten Teilergebnisse
          verloren. 
  \end{postfail}
  
  \begin{proc}    
    \step[1]
    Der Benutzer startet den UseCase durch Auswahl des Eintrags 
    \gq{Tools -- Execute GQSL Query} im Hauptmen�  
    (siehe hierzu \ref{Main-Window-Tools}).
      
    \step[2] 
    GIANT �ffnet den Anfragedialog (siehe \ref{GUI Anfragedialog}).
      
    \step[3] 
    Der Benutzer bet�tigt im Dialog den Button \gq{Load Query}.

    \step[4]
    Daraufhin zeigt GIANT den Standard-Filechooser-Dialog 
    (siehe \ref {Standard-Filechooser-Dialog}).

    \step[5]
    Der Benutzer w�hlt die gew�nschte Anfragedatei (siehe 
    \ref {Config Anfrage-Dateien}) aus.
        
    \step[6]
    GIANT zeigt das aus der Anfragedatei geladene GQSL Skript 
    (siehe \ref {GIANT Query Skripting Language}) im Textfeld
    des Anfragedialoges an.

    \step[7]
    Falls gew�nscht kann der Benutzer das GQSL Skript im Textfeld
    noch manuell weiter modifizieren.
    
    \step[8]
    Der Benutzer startet die Berechnung der Anfrage durch Bet�tigung
    des \gq{Start Query} im Anfragedialog (siehe \ref{GUI Anfragedialog}).

    \step[9]
    GIANT pr�ft das geladene und eventuell modifizierte GQSL Skript. 
    Sollte das Skript nicht den Vorgaben der Grammatik 
    (siehe \ref {GIANT Query Skripting Language} entsprechen, erscheint
    eine Fehlermeldung (siehe \ref {afa Fehlerverhalten})
    und das System kehrt zu Schritt 7 des UseCase zur�ck.

    \step[10]
    GIANT berechnet die Anfrage und teilt dem Benutzer
    den Fortschritt mittels eines Progressbars (\ref{Progressbar-Modale}) mit.
    W�hrend der Abarbeitung der Anfrage ist das System nicht bedienbar.\\

  \end{proc}

  \begin{aproc}
    \astep{3} Der Benutzer bricht den UseCase mit Cancel ab.
    \astep{4} Der Benutzer bricht die Auswahl der Anfragedatei mit Cancel ab.
              Das System kehrt dann zu Schritt 2 bei der
              Abarbeitung des UseCase zur�ck.
    \astep{6} Der Benutzer bricht den UseCase mit Cancel ab.

    \astep{10} Die laufende Berechnung des Anfrageergebnisses 
    kann vom Benutzer jeder Zeit durch Bet�tigen des
    Buttons \gq{Cancel Calculation} (\ref{Progressbar-Modale-Cancel})abgebrochen werden.


  \end{aproc}

\end{uc}

% ==============================================================================
\begin{uc}[Label]{UC: Anfrage speichern}
\index{Anfragen!in eine Datei speichern}
Mit diesem UseCase kann der Benutzer GQSL Skripte aus dem 
Anfragedialog (siehe \ref{GUI Anfragedialog}) in Anfragedateien 
(siehe \ref {Config Anfrage-Dateien}) speichern.

  \begin{precond}
    \cond Der Anfragedialog (siehe \ref{GUI Anfragedialog}) ist
          ge�ffnet und enth�lt in dem daf�r vorgesehenen Textfeld
          entweder ein manuell eingegebenes oder ein aus einer
          Datei geladenes und eventuell modifiziertes GQSL Skript.

  \end{precond}

  \begin{postsuccess}
    \cond Eine Anfragedatei, welche das GQSL Skript enth�lt, wurde
          angelegt.
    \cond GIANT zeigt den Anfragedialog, das gespeicherte GQSL Skript
          ist weiterhin in dem Textfeld vorhanden.
      
  \end{postsuccess}

  \begin{postfail}
    \cond Das System bleibt im bisherigen Zustand.
    \cond Es wurde keine Anfragedatei erzeugt
    \cond GIANT zeigt weiterhin den  Anfragedialog 
          (siehe \ref{GUI Anfragedialog}) und alle dort get�tigten
          Eingaben (insbesondere das GQSL Skript im Textfeld des
          Dialoges) bleiben erhalten.
   
  \end{postfail}
  
  \begin{proc}    

    \step[1]
    Der Benutzer bet�tigt im Anfragedialog (siehe \ref{GUI Anfragedialog})
    den Button \gq{Save Query}.
    
    \step[2]
    GIANT pr�ft ob das GQSL Skript im Textfeld des Anfragedialoges den
    Vorgaben der Grammatik (siehe \ref {GIANT Query Skripting Language} 
    entspricht. Falls nein erscheint eine Fehlermeldung 
    (siehe \ref {afa Fehlerverhalten})
    und das System kehrt zu Schritt 1 des UseCase zur�ck.

    \step[3]
    GIANT �ffnet den Standard-Filechooser-Dialog (siehe 
    \ref {Standard-Filechooser-Dialog}).
  
    \step[4]
    Der Benutzer gibt den Pfad und die Datei, in der das GQSL Skript
    gespeichert werden soll, vor und best�tigt mit OK.

    \step[5]
    GIANT speichert das GQSL Skript in der vorgegebenen Anfragedatei.

  \end{proc}

  \begin{aproc}
    \astep{4} Der Benutzer bricht den UseCase mit Cancel ab.

  \end{aproc}

\end{uc}





%===============================================================================
% 
% Anfragesprache
%
\chapter{GIANT Query \& Skripting Language}
% =============================================================================
%  $RCSfile: language.tex,v $, $Revision: 1.8 $
%  $Date: 2003/04/18 00:50:12 $
%  $Author: keulsn $
%
%  Description:
%
%  Last-Ispelled-Revision: 1.1
%
% =============================================================================

\index{GSL!Beschreibung}
In diesem Kapitel werden die F�higkeiten der GIANT Scripting Language (GSL)
beschrieben. Syntax und Semantik der GSL sind einem separaten Dokument,
der GIANT Scripting Language Spezifikation zu entnehmen.

GSL Scripts werden von einem GSL Interpreter ausgef�hrt. Der GSL Interpreter
ist Bestandteil von GIANT.

Die GSL dient dazu, IML-Teilgraphen und
Selektionen aus einem IML-Graph anzufragen und auf diesen Aktionen
auszuf�hren. Aktionen sind:
\begin{enumerate}
\item Erzeugen eines neuen Anzeigefensters
\item Hinzuf�gen der Graph-Knoten und Graph-Kanten eines IML-Teilgraphen
in ein ge�ffnetes Anzeigefenster
\item Speichern einer neuen Selektion in einem Anzeigefenster
\item �ndern des Inhalts einer Selektion in einem Anzeigefenster
\item L�schen einer Selektion in einem Anzeigefenster
\item Speichern eines neuen IML-Teilgraphen
\item �ndern des Inhalts eines IML-Teilgraphen
\item L�schen eines IML-Teilgraphen
\end{enumerate}



%===============================================================================
% 
% Projektverwaltung
%
\chapter{GIANT Projektverwaltung}\label{GIANT Projektverwaltung}

In diesem Kapitel wird beschrieben, wie persistente Arbeitsergebnisse
von GIANT logisch und physich strukturiert und gespeichert werden.
Arbeitsergebnisse sind z.B., vom Benutzer erzeugte Anzeigefenster
mit visualisierten Knoten eines IML-Graphen.

% ==============================================================================
%  $RCSfile: project.tex,v $, $Revision: 1.17 $
%  $Date: 2003/02/25 14:33:49 $
%  $Author: squig $
%
%  Description:
%
% ==============================================================================


%===================
\section {Persistenz �ber Projekte}

GIANT speichert persistente Informationen in so genannten Projekten.
\begin {enumerate}

  \item 
  Wird w�hrend des Betriebs von GIANT ein neues Projekt angelegt, so erh�lt 
  es automatisch eine Projektdatei.
  
  \item
  Ein Projekt besteht aus einem Verweis auf eine IML-Graph-Datei, auf die sich
  die gespeicherten Informationen beziehen, sowie aus den gespeicherten 
  Informationen f�r IML-Teilgraphen, Anzeigefenster und Knoten-Annotationen.
 
  \item
  Jedes Projekt hat einen vom Benutzer definierbaren Namen, dieser Name 
  entspricht dem Namen der Projektdatei und wird innerhalb des IML-Browsers 
  angezeigt.

  \item
  Der Name eines bereits angelegten Projektes kann mit den Mitteln von 
  GIANT nicht ge�ndert werden (au�er dadurch, dass  man das Projekt unter 
  neuem Namen neu speichert). 

  \item
  Der Benutzer kann beliebig viele Projekte anlegen.
  
  \item
  In GIANT darf immer nur ein Projekt gleichzeitig ge�ffnet sein. 
  
  \item
  W�hrend der Arbeit mit GIANT kann jederzeit ein Projekt geladen oder ein
  neues Projekt angelegt werden. 
  Voraussetzung hierf�r ist allerdings, dass dies 
  in der Reflektion zum IML-Graphen unterst�tzt wird.
  
  \item
  GIANT wei� nicht, ob ein geladenes Projekt gegen�ber den f�r das Projekt
  in der Projektdatei und in den Verwaltungssdateien gespeicherten 
  Informationen modifiziert wurde oder nicht.

\end {enumerate}


  \subsection {Das Projektverzeichnis}
  \begin {enumerate}

    \item  
    S�mtliche Dateien, die die Informationen f�r ein Projekt enthalten,
    befinden sich in diesem Verzeichnis. 

    \item
    In einem Projektverzeichnis darf nur ein Projekt abgelegt werden.
    
  \end {enumerate}    


  \subsection {Die Projektdatei}
  \begin {enumerate}

    \item
    Die Projektdatei liegt als XML-Datei vor.

    \item
    Die Projektdatei befindet sich im Projektverzeichnis und enth�lt 
    Informationen, die zur Identifikation des zu einem Projekt geh�renden 
    IML-Graphen n�tig sind. 
  
    \item
    Der Name der Projektdatei entspricht dem Namen des Projektes.

    \item
    Die Projektdatei enth�lt Referenzen zu allen Dateien, die Bestandteil
    des Projektes sind (Verwaltungsdateien f�r IML-Teilgraphen und
    Anzeigefenster und die Verwaltungsdatei f�r Knoten-Annotationen).

    \item
    Der Pfad zu der Datei, die den IML-Graphen enth�lt, ist in der Projektdatei 
    gespeichert. 
  \end {enumerate}
  
 
  \subsection {Pr�fung der IML-Graph Datei}
  Die Reflektion muss f�r jede IML-Graph Datei eine m�glichst eindeutige 
  Pr�fsumme berechnen k�nnen. Beim Laden eines Projektes wird �berpr�ft, 
  ob die in der Projektdatei gespeicherte Pr�fsumme
  der Pr�fsumme der zu ladenden IML-Graph Datei entspricht. 
  Das Verhalten von GIANT f�r den Fall, dass eine IML-Graph Datei geladen wird, 
  die zwar die passende Pr�fsumme hat, aber nicht den IML-Graphen enth�lt, 
  der dem Projekt eigentlich zu Grunde liegt, ist undefiniert.


  \subsection {Verwaltungsdateien f�r Anzeigefenster}
 
  \begin {enumerate}
  
    \item
    Zu jedem Anzeigefenster gibt es eine Verwaltungsdatei. 
    
    \item
    Diese Verwaltungsdatei enth�lt alle Informationen zur kompletten 
    Rekonstruktion eines Anzeigefensters. Alle Informationen werden
    in bin�rer Form gespeichert.
 
  \end {enumerate}
  Insbesondere werden folgende Informationen gespeichert:
 
  \begin {enumerate}
 
    \item Der komplette Anzeigeinhalt 
          (alle visualisierten Knoten und Kanten mit Position).
    \item Alle Pins (gespeicherte sichtbare Anzeigeinhalte).
    \item Alle dem Anzeigefenster bekannten Selektionen.
 
  \end {enumerate}
 
  Der f�r das Anzeigefenster gew�hlte Visualisierungsstiel wird zwar mit 
  gespeichert, da die Viualisierungsstiele v�llig unabh�ngig von den 
  Projekten �ber Konfigurationsdateien realisiert werden, 
  kann nicht gerantiert werden, dass der gew�nschte Stiel beim Laden 
  des Projektes auch wieder gefunden wird,
  in diesem Fall wird dann ein \gq{Standard-Stil} verwendet.


  \subsection {Verwaltungsdateien f�r IML-Teilgraphen}
  \begin {enumerate}

   \item
    F�r jeden IML-Teilgraphen gibt es eine eigene Verwaltungsdatei.
    
    \item
    S�mtliche erzeugten IML-Teilgraphen werden in Verwaltungsdateien in 
    bin�rer Form gespeichert. 

  \end{enumerate}
  
  
  \subsection {Die Verwaltungsdatei f�r Knoten-Annotationen}
  \begin {enumerate}

   \item
   Alle Knoten-Annotationen werden in einer Verwaltungsdatei gespeichert.
   
   \item
   Diese Datei liegt als XML Datei vor.
  \end{enumerate}
  

 
 
%===================
\section {Grundlegendes Verhalten von GIANT beim Speichern von Projekten} 
  

  \subsection{\gq{Alles Speichern}}\label{Alles Speichern}
  Diese Funktionalit�t wird von entsprechenden UseCases genutzt. 
  Hierbei werden alle Anzeigefenster, IML-Teilgraphen 
  und Knoten-Annotationen in die Verwaltungsdateien geschrieben, 
  wobei alle �nderungen an noch offenen Anzeigefenstern ber�cksichtigt 
  werden.\\
 

  %===
  \subsection {Persistenz von Anzeigefenstern}
  \begin {enumerate}

    \item
    S�mtliche dem Projekt bekannten Anzeigefenster (alle Anzeigefenster zu 
    denen es eine entsprechende Verwaltungsdatei gibt) werden auf der GUI in 
    der Liste �ber die Anzeigefenster angezeigt, egal ob sie ge�ffnet sind
    oder nicht. 

    \item 
    Zu jedem Anzeigefenster eines Projektes gibt es eine Verwaltungsdatei,
    beim �ffnen eines neuen Fensters wird diese Datei automatisch mit
    erzeugt.

    \item
    Wird ein ge�ffnetes Anzeigefenster geschlossen, 
    so fragt GIANT nach, ob es eventuelle �nderungen speichern soll
    oder nicht. Falls ja, werden eventuelle �nderungen in die f�r das 
    Anzeigefenster vorhandene Verwaltungsdatei geschrieben, 
    anderenfalls bleibt der Zustand des Anzeigefensters nach der letzten
    Speicherung vorhanden (die zugeh�rige Verwaltungsdatei wird nicht 
    ver�ndert).

    \item
    Modifikationen (z.B. das Verschieben von Knoten) 
    auf einem Anzeigefenster werden nicht automatisch
    nach deren Durchf�hrung gespeichert (so kann notfalls ein
    Undo durchgef�hrt werden).

    \item
    Ein zu einem Projekt geh�rendes Anzeigefenster kann gel�scht werden, 
    hierbei werden alle Information �ber das Anzeigefenster einschlie�lich
    der Verwaltungsdatei vernichtet.
    >>>>>>>>>>>>>>>>USE CASE - L�schen persistenter Anzeigefenster
  \end {enumerate}

  %===
  \subsection {Persistenz von IML-Teilgraphen}
  \begin {enumerate}

    \item
    Alle in einem Projekt bereits vorhandenen IML-Teilgraphen werden auf der 
    GUI angezeigt (in einer entsprechenden Liste).
  
    \item
    Neu erzeugte IML-Teilgraphen und �nderungen an bestehenden 
    IML-Teilgraphen k�nnen �ber \gq{alles Speichern} (siehe oben) 
    gespeichert werden.
       
    \item
    Wird das Programm beendet, ohne das zuvor \gq{Alles Speichern} ausgef�hrt 
    worden ist, so gehen  alle nicht gespeicherten Informationen zu den 
    IML-Teilgraphen verloren (alle zwischenzeitlich ausgef�hrten Modifikationen 
    und alle zwischenzeitlich neu erzeugten IML-Teilgraphen). 
    Der Zustand des Projektes  nach dem letzten Speichern bleibt dann erhalten.

    \item
    Modifikationen an bestehenden IML-Teilgraphen werden nicht automatisch 
    gespeichert, zu neu erzeugten IML-Teilgraphen wird nicht automatisch
    eine Verwaltungsdatei erzeugt.
    \\
    N�TIG; DA SONST ANFRAGEN EVENTUELL AUSGEBREMST.
    IM GEGENSATZ ZU ANZEIGEFENSTERN KANN MAN IML-TEILGRAPHEN NICHT �FFNEN
    IM GEGENSATZ ZU ANZEIGEFENSTERN (fliegen beim Schlie�en aus dem
    speicher raus) WERDEN ALL IML-TEILGRAPHEN pauschal im
    SPEICHER GEHALTEN.

    \item
    IML-Teilgraphen k�nnen gel�scht werden. Falls vorhanden, wird dann auch
    die entsprechende Verwaltungsdatei ebenfalls sofort gel�scht.
  \end {enumerate}
  
  
  
\subsection {Persistenz von Knoten-Annotationen}
  \begin {enumerate}
  
   \item
   �nderungen bestehender oder neu erzeugte Knoten-Annotationen werden nur
   �ber die Funktionali�t unter \gq{alles Speichern} in die 
   Verwaltungsdatei geschrieben.
   
   \item
   Einmal erzeugte Knoten-Annotationen werden jedem IML-Knoten mit der 
   entsprechenden ID zugeordnet, egal in welchem Anzeigefenster diese
   visualisiert sind. Ein Knoten kann auch annotiert sein, wenn er in keinem
   Anzeigefenster visualisiert ist. Wird ein annotierter Knoten gel�scht 
   (aus einem Anzeigefenster entfernt), so wird der dazu vorhandene 
   Eintrag f�r die Annotation in der Verwaltungsdatei nicht automatisch mit gel�scht.
   
   >>>>>>>>>>>>>>>> EVENTUELL M�SSEN WIR HIER EINEN FILTER BAUEN
   DER KNOTEN-Annotationen entfernt, die nicht mehr gebraucht werden.
   
 \end {enumerate}
  
  
  
  
  
  
  
  
  



%===============================================================================
% 
% Beschreibung aller Konfigurationsdateien
%
\chapter{Konfiguration von GIANT}\label{Konfiguration von GIANT}

Hier wird beschrieben, wie benutzerdefinierbare Einstellungen von
GIANT gespeichert und verwaltet werden. In diesem Kapitel wird 
insbeondere spezfifizert, welche Einstellungen der Benutzer
in welchen Konfigurationsdateien vornehmen kann.
% ==============================================================================
%  $RCSfile: config.tex,v $, $Revision: 1.9 $
%  $Date: 2003/02/24 21:45:42 $
%  $Author: birdy $
%
%  Description:
%
% ==============================================================================


\section {Allgemeines}

S�mtliche konfigurierbaren Einstellungen werden in XML-Dateien vorgenommen.
Es gibt genau eine \gq{globale Konfigurationsdatei}. Zudem kann es beliebig 
viele weitere XML-Dateien zur Konfiguration der weiter unten beschriebenen 
Visualisierungsstiele geben.

\section {Die globale Konfigurationsdatei}
Diese Datei ist in XML verfasst und existiert genau ein mal. In ihr werden 
die anschlie�end beschriebenen Einstellungen vorgenommen.

  \subsection {Verweise auf Men�-Makros}
  Es k�nnen beliebig viele Men�-Makros definiert werden.
  Jeder Makro wird durch einen eindeutigen Namen und durch einen Verweis
  auf die Datei, welche den Makro enth�lt, spezifiziert.

  \subsection {Farbe von Hervorhebungen}
  Einstellung der Farben, mittels derer Selektionen oder IML-Teilgraphen 
  hervorgehoben werden.
  \begin {enumerate}
    \item Eine beliebige Farbe f�r die aktuelle Selektion.
    \item Beliebige (sinnvoll w�ren verschiedene) Farben f�r das
          Hervorheben weiterer Selektionen.
    \item Beliebige (sinnvoll w�ren verschiedene) Farben f�r das 
          Hervorheben von IML-Teilgraphen.
  \end {enumerate}


\section {Visualisierungsstile}
\begin {enumerate}

  \item
  F�r jeden Visualisierungsstil muss es eine entsprechende XML-Datei geben. 

  \item
  Ein Visualisierungsstil beschreibt, wie die Knoten und Kanten des 
  IML-Graphen innerhalb eines Anzeigefensters graphisch dargestellt 
  werden k�nnen.

  \item
  In einem Visualisierungsstil k�nnen klassenspezifische Einstellungen 
  vorgenommen werden, die nur f�r Knoten und Kanten gelten, 
  die zu den entsprechenden Klassen geh�ren.
  Bei jeder klassenspezifischen Einstellung f�r Knoten und Kanten kann daher 
  eine Liste der betroffenen Knoten- und Kantenklassen angegeben werden, 
  f�r die diese Einstellungen gelten sollen.

  \item
  Wird eine Kanten- oder eine Knotenklasse innerhalb eines
  Visualisierungsstieles mehreren klassenspezifischen Einstellungen zugeordnet, 
  f�hrt dies zu keinem Fehler, es bleibt aber unspezifiziert, 
  welche Einstellung tats�chlich genommen wird.

  \item
  Es gibt immer einen Default-Visualisierungsstil (von GIANT fest
  vorgegeben).

\end {enumerate}


  \subsection {Name des Visualisierungsstils}
  Jeder Visualisierungsstiel erh�lt einen Namen. Unter diesem Namen ist der 
  Visualisierungsstiel in der GUI (bei den Anzeigefenstern) ausw�hlbar.
 
  \subsection {Einstellungen innerhalb des Visualisierungsstils}
    \begin{enumerate}
      \item Die Hintergrundfarbe im Anzeigefenster.

      \item Wann Kanten im sichtbaren Anzeigeinhalt angezeigt werden sollen.
      >>>>>>>>>>>>>>>>>>>>>>>      ACHTUNG - ES KANN AUCH SCHLEIFEN GEBEN
        \subitem Nur falls Start- und Zielknoten ebenfalls im sichtbaren 
                 Anzeigeinhalt.
        \subitem Auch anzeigen falls nur Startknoten im sichtbaren 
                 Anzeigeinhalt.
        \subitem Auch anzeigen falls nur Zielknoten im sichtbaren Anzeigeinhalt.
        \subitem Auch anzeigen falls weder Start- und Zielknoten im sichtbaren 
                 Anzeigeinhalt. 
    \end{enumerate}


  \subsection {Klassenspezifische Einstellungen f�r Knoten}
  Folgende  Einstellungen k�nnen f�r Knotenklassen vorgenommen werden. 
  Es muss eine DEFAULT-Einstellung erstellt werden, die f�r alle
  Knotenklassen angewendet wird, f�r die nichts anderes definiert ist.
  \begin{enumerate}  
    \item Ein Icon f�r die Knotenklasse (Verweis auf eine entsprechende 
          Bilddatei). Das Icon muss im Pixmap Format bei 32*32 Pixel
          vorliegen.

    \item Eine Liste der Attribute der Knotenklasse, welche dirket innerhalb 
          des Anzeigefensters in dem \gq(Viereck) f�r den Knoten dargestellt 
          werden sollen.
    \item Die F�llfarbe des \gq{Vierecks} in welchem die Attribute zu dem 
          Knoten dargestellt werden.
    \item Die Farbe des Rahmens f�r das \gq{Viereck} in welchem die 
          Attribute zu dem Knoten dargestellt werden.

  \end{enumerate}

  
  \subsection {Klassenspezifische Einstellungen f�r Kanten}
  Folgende Einstellungen k�nnen f�r Kantenklassen vorgenommen werden. \\
  Es muss eine DEFAULT-Einstellung erstellt werden, die f�r alle
  Kantenklassen angewendet wird, f�r die nichts anderes definiert ist.
  \begin{enumerate}
    \item Die Farbe der Kante.
    \item Die Art der Linie der Kante (normal, gestrichelt).
    \item Ob die Kante mit ihrer Kantenklasse beschriftet werden soll oder 
          nicht.
  \end{enumerate}



\section {Anfrage-Dateien}
Eine Textdatei, die genau eine Anfrage enth�lt - also genau ein gem�� der 
Grammitk f�r die Anfragesprache zul�ssiges Wort. 
Auf diese Art und Weise k�nnen komplexe Anfragen gespeichert und 
wiederverwendet werden.\\
Erstellt werden k�nnen solche Anfrage-Dateien entweder manuell oder
automatisch aus einer im ENTSPRECHENDEN DIALOG eingegebenen Anfrage.\\
Geldaden werden k�nnen solche Dateien beim Kommandozeilenaufruf
oder im entsprechenden Dialog zur Eingabe einer Anfrage.


%===============================================================================
% 
% Beschreibung der GUI
%
\chapter{Beschreibung der GUI}

In diesem Kapitel werden die Dialoge und Men�s der grafischen
Benutzerschnittstelle von GINAT beschrieben.
Die Funktionalit�t dieser Dialoge wird im Detail bei
den jeweiligen UseCases beschrieben.
% ==============================================================================
%  $RCSfile: gui.tex,v $, $Revision: 1.1 $
%  $Date: 2003/02/13 03:15:14 $
%  $Author: schulzgt $
%
%  Description:
%
% ==============================================================================

\section{�ber die Benutzeroberfl�che}

\section{Beschreibung der GUI}    

\subsection{Das Hauptmen�}   
Kill-Button, der automatisch alle 

\subsection{Popup-Men�s}

\subsection{Hauptfenster}

\subsection{Anzeigefenster}
Zeigt zu jedem Knoten die Klasse, die ID und alle Attribute an.
Es k�nnen auch alle Knoten, die auf diesen Knoten verweisen angeziegt 
werden (DIES SOLLTE EVETUELL AUCH IN DIE ANFRAGESPRACHE). 

\subsection{Knoten-Informationsfenster}

\subsection{Fenster f�r Minimap} %Falls in extra Fenster

\subsection{Anfrage Dialog}

\subsection{Darstellung von Attributen}
Bietet �bersicht �ber alle Knotenklassen (mit zugeh�rigen Attributen), alle Attribute und
Kantenklassen. Auswahlm�glichkeit.

\subsection{Dialog f�r Layoutalgorithmen}

\subsection{Dialog f�r Mengenoperationen}

\subsection{Allgemeiner Texteingabedialog}
F�r Namen (Pins, Selektionen, Anzeigfenster ...)

\section{Dateneingabe}
Beschreibung von verschiedenen Textfeldtypen etc. zur Eingabe von Daten.

\section{Ausgabe von Fehlermeldungen}
\subsection{Allgemeiner Fehlerdialog}



%===============================================================================
% 
% Visualisierung des IML-Graphen
%
\chapter{Visualisierung des IML-Graphen}

Dieses Kapitel beschreibt, wie Knoten und Kanten des
IML-Graphen innerhalb von Anzeigefenstern als Fenster-Knoten
und Fenster-Kanten dargestellt werden. Speziefiert wird hier
jediglich das Ergebnis dieser Visualisierung, also das
Aussehen der Knoten und Kanten, nicht aber die Art und Weise,
wie GIANT diese Visualisierung erzeugt.
% ==============================================================================
%  $RCSfile: visulization.tex,v $, $Revision: 1.2 $
%  $Date: 2003/02/14 14:34:49 $
%  $Author: schwiemn $
%
%  Description:
%
% ==============================================================================

Genaue Beschreibung der Visualisierung des Graphen in einem Anzeigefenster.

\section{Visualisierungsstiele}
Der Benutzer kann �ber XML-Konfigurationsdateien beliebige viele Visualisierungsstiele
erzeugen. �ber so solch einen Visualisierungsstiel kann er definieren wie Knoten und
Kanten von Graphen in den Anzeigefenstern dargestellt werden, wobei er die 
Einstellungen f�r Knoten und Kanten differenziert f�r die verschiedenen Knoten- und
Kantenklassen des IML-Graphen vornehmen kann.\\
Zu jedem Anzeigfenster gibt es eine Liste aus der zur Laufzeit ein 
entsprechender vordefinierter Visualisierungsstiel ausgew�hlt werden kann.\\
Auf diese Art und Weise kann der Benutzer also je nach aktuellem Bedarf
die Darstellung im ANzeigefenster �nderen.

WAS KANN ALLES KONFIGURIERT WERDEN USW.:- VERWEIS AUF KONFIGURATIOS DATEIEN


\section{Visualisierung von Knoten}
Hier wird die Visualisierung eines Knotens auf dem Anziegeinhalt innerhalb eines Anzeigefensters 
beschrieben. Die Tats�chliche Darstellung h�ngt stark von der aktuellen Zoomstufe und dem gew�hlten
Visualisierungsstiel ab.

  \subsection {Grafik}
  HIER EVENTUELL SKIZZE EINF�GEN; WELCHE DAS AUSSEHEN EINES KNOTENS BESCHREIBT


  \subsection {Knoten-Rechteck}
  Grundlage der Darstellung eines Knotens ist immer ein Rechteck mit fixer Breite.
  Die H�he des Rechtecks ist proportional zur Anzahl der Attribute des Knotens, die  
  direkt im Anzeigefenster dargestellt werden.

  \subsection {Knoten-Icon}
  Jeder Knoten kann �ber ein vordefiniertes Icon verf�gen, welches im linken oberen Eck des
  Knoten-Rechtecks darhgestellt wird. Dieses Icon muss eine Gr��e 32*32 Pixel haben.


  \subsection {Klassenname und ID}
  Der Name der Knotenklasse und die eindeutige ID eines Knotens werden innerhalb des 
  Knoten-Rechtecks angezeigt.

  \subsection {Attribute des Knoten}
  Innerhalb des Knoten-Rechtecks k�nnen auch Attribute dargestellt werden (Attributname und
  Wert), da das Knoten-Rechteck eine fixe Breite hat, wird der zugeh�rige Text abgeschnitten,
  falls er zu lang ist. Dieses Abschneiden wird dem Benutzer durch das Anf�gen von \gq{...}
  an das Ende des Textes dargestellt.


\section{Visualisierung von Kanten}

Kanten sind immer gerade Linien von einem Start- zu einem Zielknoten. Kanten k�nnen sich hinsichtlich
ihrer Linienfarbe und der Art der Line (z.B. gestrichelt oder durchgezogen) unterscheiden.
Der IML-Graph kennt auch Schleifen, solche Kanten werden durch zwei Kantenknickpunkte
umgelenkt.

  \subsection{Kantenknickpunkte}
  Der Benutzer kann bei jeder Kante manuell beliebig viele Kantenknickpunkte hinzuf�gen.
  Die Kante selbst besteht dann aus mehreren geraden Linien vom Startknoten �ber beliebig
  viele Kantenknickpunkt hiwnweg zum Zielknoten.\\
  
  \begin{itemize}   
    \item Kantenknickpunkte werden als kleine Rauten dargestellt
    \item Die Farbe der Kantenknlickpunkte entspricht der Farbe f�r die Linie der Kante
  \end{itemize}

  Layoutalgorithmen, die diese Kantenknickpunkt automatisch erzeugen sind momentan nicht vorgesehen.


  \subsection{Vorhandene Kanten nicht anzeigen}



\section {Hervorheben von Knoten und Kanten}

Innerhalb eines Anzeigefensters k�nnen
\begin {itemize}
  \item die \gq{aktuelle Selektion} (kann nat�rlich auch lehr sein),
  \item bis zu drei weitere Selektionen,
  \item und bis zu vier IML-Teilgraphen
\end {itemize}
unterschiedbar hervorgehoben werden.


\subsection {Selektionen und aktuelle Selektionen}

Beim Hervorheben von Knoten und Kanten von Selektionen werden diese mittels einer 
definierbaren Farbe eingef�rbt.

  \begin {itemize} 
    \item Jeder Knoten einer Selektion wird dadurch hervorgehoben, dass das die F�llfarbe 
          und die Rahmenfarbe des Knotenvierecks auf die entsprechende Farbe gesetzt werden.

    \item Jede Kante wird dadurch hervorgehoben, dass ihre Linie und alle ihre Kanten



und jeder Kantenknickpunkt 


  \end {itemize}



Geh�rt ein Knoten oder eine Kante zu mehrern hervorgehobenen Selektionen, so wird immer die Farbe der
zuletzt zum Hervorheben gew�hlten Selektion genommen. Knoten und Kanten der aktuellen Selektion werden
immer mit der Farbe f�r aktuelle Selektionen hervorgehoben.
>>>>>>>>>>>>>>>>>>>>>CHECKEN OB SO MACHBAR UND SINNVOLL




Werden Knoten und Kanten von IML-Teilgraphen hervorgehoben 
 



Wie weiter unten beschrieben kann der Kunde �ber die Konfigurationsdateien einstellen mit welchen Farben
Selektionen und IML-Teilgraphen hervorgehoben werden sollen, und mit welchen Farben Knoten und Kanten
(je nach Klasse) generell eigef�rbt werden. Es macht nat�rlich keinen Sinn, f�r das Einf�rben von
Knoten und Kanten  hier die gleiche oder eine zu �hnliche Farbe zu w�hlen, wie f�r das hervorheben 
von Selektionen.





\section {Detailstufen beim Zoomen}
Je nach Zoomstufe erh�ht sich die Anzahl der im Anzeigefenster dargestellten Knoten und Kanten.
Beim Herauszoomen m�ssen also Details verringert werden, dies geschieht bei GIANT in mehreren
Stufen.\\
Eine exakte Spezifikation der Detailstufen, insbesondere ab welcher Zoomstufe welche 
Detailstufe gew�hlt wird, ist an dieser Stelle nicht sinnvoll. Dies wird bei
der Implementierung geschehen. Auf jeden Fall wird es aber die folgenden Detailstufen geben:

  \subsection {Volle Details}
  Falls sehr nah herangezoomt wurde.
  Das Knoten-Rechteck enth�lt alle 








\section{Minimap}


  Der Benutzer kann �ber die Konfigurationsdateien beliebig viele Visualisierungsstiele
erzeugen. Ein Visualisierungsstiel beschreibt wie Knoten und Kanten innerhalb eines
Anzeigefensters dargestellt werden (Icons, Farbe, Attribute ...). Der Benutzer
kann zu jeder Zeit in einem Anzeigefenster einene der zuvor definierten 
Visualisierungsstiele ausw�hlen
Laufzeit von GIANT kann der Benutzer innerhalb eines Anzeigefensters 


%===============================================================================
% 
% Nichtfunktionale Anforderungen
%
\chapter{Nichtfunktionale Anforderungen}

Hier werden die wesentlichen nichtfunktionalen Anforderungen an GIANT
spezifiziert. Insbeondere sind dies Anforderungen aus dem Bereich
des Softwareengineering, wie z.B. Wartbarkeit und Erweiterbarkeit.

% ==============================================================================
%  $RCSfile: nfa.tex,v $ 
%  $Date: 2003/02/13 04:23:25 $
%  $Author: schulzgt $ 
%
%  Description: Nichtfunktionale Anforderungen
%
% ==============================================================================

\section{Plattformunabh�ngigkeit}
Das Produkt soll ein hohes Ma� an Plattformunabh�ngigkeit, wie sie sich aus der eingesetzten Entwicklungsumgebung ergibt, 
verf�gen (also alle Systeme auf denen GTK/ADA 1.2.12 lauff�hig ist). Explizit w�hrend der Entwicklung getestet 
werden kann dies aber nur f�r Sun Solaris und Linux.

\section{Wartbarkeit}
Das System soll sich durch ein hohes Ma� an Wartbarkeit und Erweiterbarkeit auszeichnen.
12.3.1  Wartbarkeit und Erweiterbarkeit gehen �ber Performance
Entsprechende Kapselung und Strukturierung des Entwurfes (Informationhiding, hohe Lokalit�t, 
strikte Trennung von GUI und Funktionalit�t etc.).

\section{Robustheit gegen�ber �nderungen der IML-Graph-Spezifikation}
Das System soll so an Bauhaus angebunden werden, dass �nderungen in der Spezifikation des IML-Graphen 
m�glichst keine Wartungsarbeiten erfordern. Besonders soll dies f�r die verschiedenen Attribute, 
Knoten- und Kantenklassen des Bauhaus-IML-Graphen gelten, da diese sich oft �ndern k�nnen.
Die Klassen und Attribute der Bauhaus-IML-Graph Bibliothek werden durch den IML-Browser nur so unterschieden, 
dass neu hinzukommende Attribute automatisch mit erfasst und damit angezeigt werden k�nnen.

\section{Mengenger�st}
Ein explizites Mengenger�st, z.B. hinsichtlich der minimalen und maximalen Anzahl von darstellbaren Knoten, 
gibt es nicht. Gr��enbeschr�nkungen (z.B. Gr��e der zu ladenden IML-Graph-Datei, Anzahl Knoten in einem Anzeigefenster ...) 
sollen sich alleine aus dem verf�gbaren Arbeitsspeicher ergeben. Falls der verf�gbare Arbeitsspeicher ersch�pft ist, 
soll das System nicht abst�rzen, sondern vielmehr die entsprechende Aktion kontrolliert abbrechen und dem Benutzer 
ein Weiterarbeiten erm�glichen.

\section{Unterst�tzung paralleler Rechner}
Layoutalgorithmen werden zur Unterst�tzung paralleler Rechner in separaten Threads ausgef�hrt. 
Die einzelnen Layoutalgorithmen sind nicht parallel. In einem Anzeigefenster kann immer nur ein es k�nnen 
aber mehrere Layoutalgorithmen in verschiedenen Anzeigefenstern gleichzeitig laufen.

\section{Leistungsanforderungen}
Ein-Benutzer System.

\section{Antwortverhalten}
Layoutalgorithmen k�nnen abgebrochen werden (darauf kann falls zu schwer verzichtet werden).

\section{Sicherheit}
Eventuell Sicherheitsabfrage zur Datensicherung bei Beenden des Programms!

\section{Robustheit}
Fehlermeldungen, keine Sicherheit bei defektem IML-Graphen  

\section{Wartbarkeit}

\section{Portabilit�t}

\section{Erweiterbarkeit}



%===============================================================================
% 
% Technische Produktumgebung
%
\chapter{Technische Produktumgebung}

Hier werden die zum Betrieb von GIANT n�tige Hardware 
und Software spezifiziert.

% ==============================================================================
%  $RCSfile: technical.tex,v $, $Revision: 1.4 $
%  $Date: 2003/02/21 17:55:53 $
%  $Author: schulzgt $
%
%  Description: Hier sind die Anforderungen an die Programmumgebung 
%  spezifiziert. Dazu geh�hren die verwendete Hard- und Software, sowie
%  Schnittstellen zu anderen Produkten.
%
% ==============================================================================

\section{Software}
Das Programm stellt an die installierte Software folgende Anforderungen:
\begin{itemize}
  \item Sun Solaris oder Linux Betriebssystem
  \item Bauhaus Tools
  \item Emacs oder vi Texteditor
\end{itemize}


\section{Hardware}
Das Programm l�uft auf SPARC Workstations und x86 kompatiblen PCs.
Im Folgenden sind die minimalen Hardwareanforderungen zur Arbeit mit kleinen 
und mittleren Projekten beschrieben. Bei gro�en Projekten ist ein Speicherausbau
von 2 GB und mehr empfehlenswert.

\subsection{Hardwareanforderungen SPARC}
\begin{itemize}
  \item UltraSPARC-II 300 MHz
  \item 512 MB Hauptspeicher
  \item 8 Bit Grafik mit einer min. Aufl�sung von 1024*786
  \item Maus mit mindestens 3 Tasten
\end{itemize}

\subsection{Hardwareanforderungen x86}
\begin{itemize}
  \item Pentium III 600 MHz
  \item 512 MB Hauptspeicher
  \item 8 Bit Grafik mit einer min. Aufl�sung von 1024*786
  \item Maus mit mindestens 3 Tasten
\end{itemize}

\section{Produkt-Schnittstellen}
Das Programm soll in die Bauhaus Suite integriert werden k�nnen.
Als Schnittstelle dient dabei die Bauhaus IML Bibliothek. Weiterhin
kann die Verwendung von Kommandozeilenparametern und der GQSL zur 
Integration in das vorhandene System genutzt werden.



%===============================================================================
% 
% Anforderungen an die Entwicklungsumgebung
%
\chapter{Anforderungen an die Entwicklungsumgebung}

Hier werden die wesentlichen Anforderungen an die Entwicklungsumgebung
von GIANT spezifiziert. Dies sind Werkzeuge, Programmiersprachen und
Bibliotheken, die bei der Entwicklung eingesetzt werden m�ssen, und
Standards, die beachtet werden sollen.
Spezifiziert wird hier auch die Sprache aller zum Lieferumfang von GIANT
geh�render Dokumente.
% ==============================================================================
%  $RCSfile: development.tex,v $, $Revision: 1.7 $
%  $Date: 2003/04/06 21:14:53 $
%  $Author: squig $
%
%  Description:
%
%  Last-Ispelled-Revision: 1.5
%
% ==============================================================================

\section {Compiler und Bibliotheken} 
\index{Compiler}
\index{Bibliotheken}
\index{GTK/ADA}
\index{GNAT}
Das System soll in der Sprache Ada95 mit GNAT 3.14 entwickelt werden, 
wobei GTK/ADA 1.2.12 als graphische
GUI-Bibliothek eingesetzte werden soll. 
Das System baut auf der vom Kunden bereit gestellten IML-Graph-Bibliothek auf 
und soll des weiteren zur Unterst�tzung der Wartbarkeit m�glichst auch die 
vom Kunden zur Verf�gung gestellten 
Datenstrukturen (wie z.B. Hashtables aus Bauhaus/reuse/src) nutzen.

  
  \subsection {Lizenzrechtliches zu den Paketen des Kunden}
  \index{Pakete der Bauhaus-Reengineering GmbH}
  Folgendes gilt nicht f�r die IML-Graph-Bibliothek.\\
  Die vom Kunden zur Verf�gung gestellten Datenstrukturen 
  werden den Entwicklern von GIANT ohne lizenzrechtliche Bedingungen 
  �berlassen. Die Nutzungsrechte der Entwickler am Produkt GIANT werden 
  durch Einsatz dieser Datenstrukturen in keinster Weise ber�hrt.
 


\section {Einlesen und Schreiben von XML Dateien}
\index{XML Dateien}
Auf XML-Dateien soll mittels des DOM (Document Object Model) Parsers
aus XML/Ada 0.7.1 zugegriffen werden. XML/Ada 0.7.1 unterliegt lizenzrechtlich der \gq{GNAT Modified GNU Public
License} (GMGPL).\\
Als Alternative ist der XML-Parser aus GTK/Ada vorgesehen -- Paket Glib.XML.



\section {Sprache}
\index{Sprache der Dokumente}
Hier wird beschrieben, in welcher Sprache die einzelnen Dokumente
des Systems GIANT verfasst werden.

\section {Spezifikation}
Die Spezifikation -- dieses Dokument -- wird in deutscher Sprache verfasst.

\section {Benutzerhandbuch}
Das Benutzerhandbuch wird in deutscher Sprache verfasst.

\section {Entwurf}
Der Entwurf wird in Englisch verfasst.

\section {Sprache der internen Dokumentation}
Die interne Dokumentation von GIANT -- Kommentare im Quellcode -- 
erfolgt in englischer Sprache.

\section {Interaktionssprache mit dem Benutzer}
\index{Interaktionssprache}
Die GUI von GIANT interagiert mit dem Benutzer ausschlie�lich in
englischer Sprache.

\section {Sprache der Konfigurationsdateien}
Die Knoten und Attribute der XML-Konfigurationsdateien
werden mit englischen Begriffen benannt.






%===============================================================================
% 
% Begriffslexikon
%
\chapter{Begriffslexikon}
Das Begriffslexikon nennt Begriffe aus der \gq{realen Welt} und definiert
ihre besondere Bedeutung innerhalb dieser Spezifikation und darauf 
aufbauender Dokumente. Des weiteren wird hier die englische �bersetzung 
dieser Begriffe f�r alle englischprachigen Dokumente von GIANT vorgegeben.

\begin{nomenclature}

\term{Begriff}{Englische �bersetzung}{Erkl�rung}

\end{nomenclature}

%%% Local Variables: 
%%% TeX-master: "spec"
%%% End: 



%===============================================================================
% 
% Anhang
%
\appendix

\end{document}
