\documentclass[a4paper,titlepage,11pt,german,twoside]{scrbook}
               
\usepackage{../styles/common}
\usepackage{spec}

\makeindex

\begin{document}

\fontfamily{cmss}\selectfont

% title page
\thispagestyle{empty}
\hfill
\parbox{5cm}{Universit�t Stuttgart \\
  Studienprojekt A IML Browser}

\vspace{5cm}

\begin{center}
  \Huge
  \textsf{Spezifikation}

  \vspace{1cm}\Large\today
\end{center}
\newpage


%===============================================================================
%
% ToDo Files - Aus fertigem Dokument entferenen (Martin Schwienbacher)
%
\chapter{ToDo -- nicht in fertiger SPez.}
Hier Sachen beschreiben, die irgenwie beim Erstellen der Spez.\ beachtet werden sollen.
\input{todo_spec}
%===============================================================================


% toc
\setcounter{tocdepth}{1}
\tableofcontents

% content
\chapter{Chapter}

Hier ist der Text.

\section{Section}

\subsection{Subsection}

\subsubsection{Subsubsection}

\paragraph{Paragraph} Hier ist noch mehr Text.


%===============================================================================
%
% ToDo Files - Aus fertigem Dokument entferenen (verbrochen von Martin Schwienbacher)
%
\chapter{ToDo -- nicht in fertiger SPez.}
Hier Sachen beschreiben, die irgenwie beim Erstellen der Spez.\ beachtet werden sollen.
\input{todo_spec}


%===============================================================================
%
% Funktionale Anforderungen
%
\chapter{Teilgraphen}
Hier Text zu Teilgraphen
% ==============================================================================
%  $RCSfile: subgraph.tex,v $, $Revision: 1.2 $
%  $Date: 2003/02/05 14:01:12 $
%  $Author: schulzgt $
%
%  Description: Use-Cases f�r IML-Teilgraphen
%
% ==============================================================================

\begin{uc}[Label]{UC: Graphisch hervorheben}
\end{uc}

\begin{uc}[Label]{UC: Graphische Hervorhebung aufheben}
\end{uc}

\begin{uc}[Label]{UC: IML-Teilgraph aus einer Selektion erzeugen}
\end{uc}

\begin{uc}[Label]{UC: IML-Teilgraph kopieren}
\end{uc}

\begin{uc}[Label]{UC: IML-Teilgraph umbenennen}
\end{uc}

\begin{uc}[Label]{UC: IML-Teilgraph l�schen}
\end{uc}

\begin{uc}[Label]{UC: Mengenvereinigung von 2 IML-Teilgraphen}
\end{uc}

\begin{uc}[Label]{UC: Mengendifferenz von 2 IML-Teilgraphen}
\end{uc}

\begin{uc}[Label]{UC: Mengenschnitt von 2 IML-Teilgraphen}
\end{uc}

\begin{uc}[Label]{UC: Teilgraph exportieren}
\end{uc}

\begin{uc}[Label]{UC: Teilgraph importieren}
\end{uc}


\chapter{Selektionen}
Hier Text zu Selektionen
% ==============================================================================
%  $RCSfile: selection.tex,v $, $Revision: 1.8 $
%  $Date: 2003/03/19 19:03:35 $
%  $Author: stupro $
%
%  Description: Use-Cases f�r die Selektionen
%
%
% ==============================================================================


\begin{uc}[Selektion zur aktuellen Selektion machen]
          {UC: Selektion zur aktuellen Selektion machen}
	  
Dieser UseCase dient dazu, eine Selektion zur aktuellen Selektion zu machen.
Dies ist n�tig, da nur die aktuelle Selektion mittels der Maus
modifiziert werden kann. 
Siehe auch \ref{Aktuelle Selektion vs Selektionen}. 
	  
	 
  \begin{precond}
    \cond Es gibt ein Anzeigefenster mit mindestens einer Selektion.
  \end{precond}

  \begin{postsuccess}
    \cond Die vorherige aktuelle Selektion ist nicht mehr aktuell.
    \cond Die entsprechende Selektion ist nun die aktuelle Selektion und
          als solche erkennbar angezeigt und hervorgehoben, auch mittels
	  einer Hervorhebung in der Selektionsauswahlliste des Fensters.
    
  \end{postsuccess}
 
  \begin{proc}    
    \step[1]
    Der Benutzer f�hrt mit der linken Maustaste einen Doppelklick 
    auf eine nicht aktuelle Selektion durch.

    \step[2]
    GIANT macht die entsprechende Selektion zur aktuellen Selektion.  
  \end{proc}
	  
\end{uc}


% ==============================================================================
\begin{uc}[Aktuelle Selektion zur�ck stufen]
          {UC: Aktuelle Selektion zur�ck stufen}
	  
Siehe auch \ref{Aktuelle Selektion vs Selektionen}. Mit diesem UseCase kann
eine aktuelle Selektion auf den Status einer \gq{normalen} Selektion
zur�ckgestuft werden.
	  
	 
  \begin{precond}
    \cond Es gibt ein Anzeigefenster mit einer aktuellen Selektion.
  \end{precond}

  \begin{postsuccess}
    \cond Es gibt keine aktuelle Selektion mehr.
    
  \end{postsuccess}
 
  \begin{proc}    
    \step[1]
    Der Benutzer f�hrt mit der linken Maustaste einen Doppelklick 
    auf die aktuelle Selektion aus.

    \step[2]
    GIANT stuft die aktuelle Selektion auf den Status einer \gq{normalen}
    Selektion zur�ck.
  \end{proc}
    
\end{uc}


% ==============================================================================
\begin{uc}[Selektion graphisch hervorheben]
          {UC: Selektion graphisch hervorheben}
	  
Dieser UseCase dient zum Hervorheben von Selektionen innerhalb eines
Anzeigefensters.

  \begin{precond}
    \cond Es gibt ein Anzeigefenster mit mindestens einer Selektion.
   
  \end{precond}

  \begin{postsuccess}
    \cond 
    Die Selektion ist im entsprechenden Anzeigefenster hervorgehoben.
    
    \cond
    Die Selektion, welche vorher mit der gleichen Farbe hervorgehoben war,
    ist nicht mehr hervorgehoben.
 
    
  \end{postsuccess}
 
  \begin{proc}    
    \step[1]
    Der Benutzer startet den UseCase mit Rechtsklick in
    die Selektionsauswahlliste \ref{Selektionsauswahlliste}
    auf die entsprechende Selektion �ber das PopUp Men�
    \gq{Highlight Selection Color1 (2,3)}.
        
      
    \step[2]
    GIANT hebt die Selektion mit der entsprechenden Farbe hervor.
    
  \end{proc}



\end{uc}


% ==============================================================================
\begin{uc}[Graphische Hervorhebung einer Selektion aufheben]
      {UC: Graphische Hervorhebung einer Selektion aufheben}
      
Dieser UseCase dient dazu, die Hervorhebung von Selektionen innerhalb eines 
Anzeigefensters aufzuheben.
      
  \begin{precond}
    \cond 
    Es gibt ein Anzeigefenster mit mindestens einer hervorgehobenen
    Selektion.
   
  \end{precond}

  \begin{postsuccess}
    \cond 
    Die Selektion ist im entsprechenden Anzeigefenster nicht mehr 
    hervorgehoben.
 
  \end{postsuccess}

  
  \begin{proc}    
    \step[1]
    Der Benutzer startet den UseCase mit Rechtsklick in
    die Selektionsauswahlliste \ref{Selektionsauswahlliste}
    auf die entsprechende Selektion �ber den PopUp-Men�punkt
    \gq{Unhighlight Selection}.

    \step[2]
    GIANT setzt die Hervorhebung der Selektion zur�ck.
    
  \end{proc} 
      
      
\end{uc}

%===============================================================================
\begin{uc}[Neue Selektion anlegen]{UC: Neue Selektion anlegen}

Mit diesem UseCase k�nnen neue, leere Selektionen angelegt werden.

  \begin{precond}
    \cond Es gibt ein ge�ffnetes Anzeigefenster.
  \end{precond}

  \begin{postsuccess}
    \cond Eine neue Selektion mit entsprechendem Namen
         ist angelegt und erscheint in der Liste der Selektionen 
	 (SELECTION\_LIST).
    \cond Diese neue Selektion hat keinen Inhalt (selektierte Knoten und
          Kanten).

  \end{postsuccess}

  \begin{postfail}
    \cond Das System bleibt im bisherigen Zustand.
  \end{postfail}
  
  \begin{proc}    
    \step[1]
    Der Benutzer startet den UseCase mit Rechtsklick in
    die Selektionsauswahlliste \ref{Selektionsauswahlliste}
    auf die entsprechende Selektion �ber den PopUp-Men�punkt
    \gq{New Selection}.
    
    \step[2] 
    GIANT �ffnet den allgemeinen Texteingabedialog \ref{DIALOG-WINDOW},
    \gq{Enter Name for new Selection}.
      
    \step[3] 
    Der Benutzer gibt dort einen zul�ssigen Namen f�r die neue Selektion 
    ein und best�tigt mit OK.\\
    Hat bereits eine andere Selektion innerhalb des Anzeigefensters den selben
    Namen erscheint eine Fehlermeldung.
    
    \step[4]
    GIANT erzeugt die neue Selektion.
  
  \end{proc}

  \begin{aproc}
    \astep{3} Der Benutzer bricht die Verarbeitung mit Cancel ab.
  \end{aproc}

\end{uc}


%===============================================================================
\begin{uc}[Selektion kopieren]{UC: Selektion kopieren}
Dieser UseCase dient zum Kopieren von Selektionen innerhalb eines 
Anzeigefensters (nicht zum Kopieren in ein anderes Anzeigefenster).

  \begin{precond}
    \cond Es gibt ein ge�ffnetes Anzeigefenster mit einer Selektion.
  \end{precond}

  \begin{postsuccess}
    \cond 
    Eine neue Selektion mit entsprechendem Namen ist angelegt 
    und erscheint in der Liste der Selektionen (SELECTION\_LIST).
    \cond
    Die neue Selektion umfasst die gleichen Knoten und Kanten wie die
    Selektion, von der kopiert wurde.

  \end{postsuccess}

  \begin{postfail}
    \cond Das System bleibt im bisherigen Zustand.
  \end{postfail}
  
  \begin{proc}    
    \step[1]
    Der Benutzer startet den UseCase durch Rechtsklick auf die zu kopierende
    Selektion in der Selektionsauswahlliste \ref{Selektionsauswahlliste}
    (\gq{Quellselektion}) und w�hlt aus dem PopUp Men�:
    Copy Selection.
    
    \step[2] 
    GIANT �ffnet den allgemeinen Texteingabedialog \ref{DIALOG-WINDOW},
    \gq{Please enter name for copy of Selection X} .
      
    \step[3] 
    Der Benutzer gibt dort einen zul�ssigen Namen f�r die neue Selektion 
    ein und best�tigt mit OK.\\
    Hat bereits eine andere Selektion innerhalb des Anzeigefensters den selben
    Namen erscheint eine Fehlermeldung.
    
    \step[4]
    GIANT kopiert die Quellselektion und legt eine neue Selektion an.
  
  \end{proc}

  \begin{aproc}
    \astep{3} Der Benutzer bricht die Verarbeitung mit Cancel ab.
  \end{aproc}

\end{uc}
%===============================================================================
\begin{uc}[Selektion umbenennen]{UC: Selektion umbenennen}
>>>> WEG LASSEN - KANN AUCH DURCH KOPIEREN UND L�SCHEN ERLEDIGT WERDEN

\end{uc}
%===============================================================================
\begin{uc}[Selektion l�schen]{UC: Selektion l�schen}

Dieser UseCase dient zum L�schen von Selektionen innerhalb eines 
Anzeigefensters. Hierdurch bleiben die Fenster-Knoten und Fenster-Kanten
unver�ndert.

  \begin{precond}
     \cond Es gibt ein ge�ffnetes Anzeigefenster mit einer Selektion.
   \end{precond}


  \begin{postsuccess}
    \cond 
    Die entsprechende Selektion ist gel�scht.
    
    \cond 
    Die Fenster-Knoten und Fenster-Kanten, 
    die zu dieser Selektion geh�rten, werden nicht gel�scht.
	  
    \cond War die Selektion hervorgehoben, so wird die entsprechende
    Hervorhebung der Fenster-Knoten und Fenster-Kanten aufgehoben.
    
  \end{postsuccess}

  \begin{postfail}
    \cond Das System bleibt im bisherigen Zustand.
  \end{postfail}
  
  \begin{proc}    
    \step[1]
    Der Benutzer startet den UseCase durch Rechtsklick auf die zu l�schende
    Selektion in der Selektionsauswahlliste \ref{Selektionsauswahlliste}
    (\gq{Quellselektion}) und w�hlt aus dem PopUp Men�:
    Delete Selection.
  
    
    \step[2]
    GIANT l�scht die entsprechende Selektion.
  
  \end{proc}

\end{uc}
%===============================================================================
\begin{uc}[Selektion manuell modifizieren]
         {UC: Selektionen manuell modifizieren}
	 
Jeweils die aktuelle Selektion kann mittels der Maus modifiziert 
werden. 


  \begin{precond}
     \cond 
     Es gibt ein ge�ffnetes Anzeigefenster mit mindestens
     einer Selektion.
   \end{precond}


  \begin{postsuccess}
    \cond 
    Die entsprechenden �nderungen an der Selektion werden von Giant
    sofort durchgef�hrt und �bernommen.
    
  \end{postsuccess}

  
  \begin{proc}    
    \step[1]
    Falls noch nicht der Fall, macht der Benutzer die zu modifizierende 
    Selektion zur aktuellen Selektion
    (siehe \ref{Selektion zur aktuellen Selektion machen}).
    
    \step[2]
    Mittels der unter 
    \ref{Selektieren von Fenster-Knoten und Fenster-Kanten in Anzeigefenstern}
    beschriebenen M�glichkeiten f�gt der Benutzer der Selektion neue
    Fenster-Knoten und Fenster-Kanten hinzu oder entfernt bestehende
    Fenster-Knoten und Fenster-Kanten aus der Selektion.
    
  \end{proc}


\end{uc}

%===============================================================================
\begin{uc}[Selektion aus IML-Teilgraph erzeugen]
         {UC: Selektion aus IML-Teilgraph erzeugen}
	 
Leitet eine Selektion aus einem IML-Teilgraphen ab.
Siehe hierzu \ref{Selektion aus IML-Teilgraphen ableiten}.


  \begin{precond}
     \cond 
     Es gibt ein ge�ffnetes Anzeigefenster.
     
     \cond
     Es gibt mindestens einen IML-Teilgraphen.
     
   \end{precond}


  \begin{postsuccess}
    \cond 
    Im Ziel-Anzeigefenster wurde eine neue Selektion
    mit dem entsprechenden Namen erzeugt.
    
  \end{postsuccess}
  
  \begin{postfail}
    \cond Das System bleibt im bisherigen Zustand.
  \end{postfail}
  
  \begin{proc} 
     
    \step[1]
    Der Benutzer f�hrt einen Rechtsklick auf den Quell-IML-Teilgraphen 
    in der Subgraph List im Hauptfenster aus
    und w�hlt im dazugeh�rigen PopUp Men� \ref{PopUp Men� Subgraph List}
    den Eintrag \gq{Create Window Selection from IML Subgraph}.
     
    \step[2]
    GIANT zeigt den allgemeinen Texteingabediolog \ref{DIALOG-WINDOW},
    \gq{Enter name for new selection}
      	    
    \step[3]
    Der Benutzer gibt einen Namen f�r die neu zu erstellende Selektion ein
    und best�tigt mit OK.\\
    Gibt der Benutzer hier keinen Namen ein, so vergibt GIANT automatisch 
    einen Namen.
    
    \step[4]
    Die Statuszeile im Hauptfenster zeigt an \gq{Please select window for
    Insertion of new window selection}, der Mauszeiger verwandelt sich in
    ein Fadenkreuz.
    
    \step[5]
    Der Nutzer klickt auf die VIS\_PANE des Windows, in dem er die neue
    Selektion erstellen will (Ziel-Anzeigefenster).
       
    \step[6]
    Die Statuszeile im Hauptfenster wechselt wieder zur normalen Anzeige.
    GIANT erzeugt gem�� der unter \ref{Selektion aus IML-Teilgraphen ableiten}
    beschriebenen Konvention im Ziel-Anzeigefenster eine neue Selektion als 
    Ableitung aus dem Quell-IML-Teilgraphen.

  \end{proc}
  
  \begin{aproc}
    \astep{3} Der Benutzer bricht den UseCase mit Cancel ab.
  \end{aproc}


\end{uc}



%===============================================================================
\begin{uc}[Mengenoperationen auf 2 Selektionen]
{UC: Mengenoperationen auf 2 Selektionen}
Zus�tzlich zu den M�glichkeiten der Anfragesprache kann der Benutzer
die g�ngigen Mengenoperationen, wie Mengenvereinigung, Schnitt und Differenz,
auch direkt �ber einen entsprechenden Dialog ausf�hren.
Beschreibung des Dialoges siehe \ref{Common-Set-Operation-Dialog}.
Vorgehen analog zu \ref{Mengenoperationen auf 2 IML-Teilgraphen}.



  \begin{precond}
    \cond Es gibt ein ge�ffnetes Anzeigefenster mit mindestens zwei 
          Selektionen.
  \end{precond}

  \begin{postsuccess}
    \cond 
    Eine neue Selektion mit entsprechendem Namen (eingegeben unter TARGET) 
    ist angelegt 
    und erscheint in der Liste der Selektionen (SELECTION\_LIST).
       
    \cond
    Im Falle einer Mengenvereinigung umfasst, die neue Selektion TARGET alle
    Knoten und Kanten aus der LEFT\_SOURCE Selektion und der 
    RIGHT\_SOURCE Selektion.
    
    \cond
    Im Falle einer Mengendifferenz umfasst, die neue Selektion TARGET alle
    Knoten und Kanten aus der LEFT\_SOURCE Selektion, die nicht 
    Bestandteil der 
    RIGHT\_SOURCE Selektion sind.

    \cond
    Im Falle eines Mengenschnitts umfasst, die neue Selektion TARGET alle
    Knoten und Kanten, die der LEFT\_SOURCE Selektion und der 
    RIGHT\_SOURCE Selektion gemeinsam angeh�ren.

  \end{postsuccess}

  \begin{postfail}
    \cond Das System bleibt im bisherigen Zustand.
  \end{postfail}
  
  \begin{proc}    
    \step[1]
    Der Benutzer startet den UseCase �ber durch Auswahl des
    Men�punktes Selection Set Operation (Union/Difference/Intersection) 
    aus dem PopUp Men� in der Selektionsauswahlliste SELECTION\_LIST
    \label{Selektionsauswahlliste}.
    
    \step[2] 
    GIANT �ffnet den Common\_Set\_Operation\_Dialog 
    (siehe \ref{Common-Set-Operation-Dialog}).
    
    
    \step[3] 
    Der Benutzer w�hlt dort die beiden Quell-Selektionen (LEFT\_SOURCE und 
    RIGHT\_SOURCE)
    aus, bestimmt die durchzuf�hrende Mengenoperation und gibt unter 
    TARGET den Namen der neue zu erzeugenden Selektion 
    ein.\\
    Dann best�tigt er die Eingabe mit OK.
       
    \step[4]
    GIANT f�hrt die Mengenoperation aus.
  
  \end{proc}

  \begin{aproc}
    \ageneral Der Benutzer bricht die Eingabe der Daten mit Cancel ab.
  \end{aproc}



\end{uc}
%===============================================================================
\begin{uc}[Label]{UC: Mengendifferenz von 2 Selektionen}
>>> Ist mit dem oberen UseCase abgedeckt
>>> KANN gestrichen werden

\end{uc}
%===============================================================================
\begin{uc}[Label]{UC: Mengenschnitt von 2 Selektionen}
>>> Ist mit dem oberen UseCase abgedeckt
>>> KANN gestrichen werden
\end{uc}


\chapter{GUI: Anzeigefenster}
Hier Text zum Anzeigefenster
% ==============================================================================
%  $RCSfile: gui_window.tex,v $, $Revision: 1.17 $
%  $Date: 2003/03/19 16:40:55 $
%  $Author: stupro $
%
%  Description: Use-Cases f�r die Fensterfunktionalit�t der GUI
%
% ==============================================================================

\begin{uc}[Leeres Anzeigefenster erzeugen]{UC: Leeres Anzeigefenster erzeugen}
�ber diesen UseCase kann der Benutzer neue Anzeigefenster innerhalb eines
Projektes anlegen.


  \begin{precond}
    \cond Ein Projekt ist geladen.
  \end{precond}

  \begin{postsuccess}
    \cond 
    Das neue leere Anzeigefenster ist ge�ffnet.
    
    \cond
    Das neue Anzeigefenster ist Bestandteil des Projektes. Eine
    Verwaltungsdatei f�r das Anzeigefenster ist im Projektverzeichnis
    angelegt.

  \end{postsuccess}

  \begin{postfail}
    \cond Das System bleibt im bisherigen Zustand.
  \end{postfail}
  
  \begin{proc}    
    \step[1]
    Der Benutzer startet den UseCase �ber das PopUp Men� \ref{WINDOW-LIST-POPUP}
    durch Auswahl von \gq{New Window}
    
    \step[2] 
    GIANT �ffnet den allgemeinen Texteingabedialog \gq{Enter Name for new Window}.
    \ref{DIALOG-WINDOW}.
      
    \step[3] 
    Der Benutzer gibt dort einen zul�ssigen Namen f�r das neue Anzeigefenster
    ein und best�tigt seine Eingabe mit OK.
    
    \step[4]
    GIANT erzeugt ein neues Anzeigefenster und �ffnet dies.
  
  \end{proc}

  \begin{aproc}
    \astep{3} Der Benutzer bricht die Verarbeitung mit Cancel ab.
  \end{aproc}


\end{uc}




% ==============================================================================
\begin{uc}[Anzeigefenster l�schen]{UC: Anzeigefenster l�schen}
Mit diesem UseCase werden bestehende Anzeigefenster aus dem Projekt 
entfernt und gel�scht. Alle Informationen zu dem Anzeigefenster
gehen hierbei unwiederbringlich verloren.


  \begin{precond}
    \cond Ein Projekt mit mindestens einem Anzeigefenster ist geladen.
  \end{precond}

  \begin{postsuccess}
    \cond 
    Das gel�schte Anzeigefenster ist nicht mehr Bestandteil des Projektes.
    
    \cond
    Die Verwaltungsdatei f�r das Anzeigefenster wurde ebenfalls gel�scht.
 
  \end{postsuccess}

  \begin{postfail}
    \cond Das System bleibt im bisherigen Zustand.
  \end{postfail}
  
  \begin{proc}    
    \step[1]
    Der Benutzer startet den UseCase �ber das PopUp Men� \ref{WINDOW-LIST-POPUP}
    
    \step[2] 
    Giant zeigt die allgemeine Sicherheitsabfrage und fragt nach, ob das
    Anzeigefenster wirklich gel�scht werden soll.
    \gq{Do you really want to delete Window xy ?}
      
    \step[3] 
    Der Benutzer best�tigt mit YES.
    
    \step[4]
    GIANT entfernt das Anziegefenster aus dem Projekt und l�scht die 
    zugeh�rige Verwaltungsdatei.
  
  \end{proc}

  \begin{aproc}
    \astep{3} Der Benutzer bricht die Verarbeitung mit NO ab.
  \end{aproc}
  
\end{uc}


% ==============================================================================
\begin{uc}[Anzeigefenster �ffnen]{UC: Anzeigefenster �ffnen}
Dient zum �ffnen eines Anzeigefensters des Projektes.


  \begin{precond}
    \cond Ein Projekt mit mindestens einem Anzeigefenster ist geladen.
    \cond Es gibt mindestens ein nicht ge�ffnetes Anzeigefenster.
  \end{precond}

  \begin{postsuccess}
  
    \cond Das Anzeigefenster ist ge�ffnet.
 
  \end{postsuccess}


  \begin{proc}    

    \step[1]
    Der Benutzer f�hrt einen Rechtsklick auf ein nicht ge�ffnetes
    Anzeigefenster in der \ref{WINDOW-LIST} durch und w�hlt im
    PopUp Men� \ref{WINDOW-LIST-POPUP}  den entsprechenden Eintag aus.
   
    \step[2]
    GIANT �ffnet das entsprechende Anzeigefenster.
  
  \end{proc}
 
\end{uc}

% ==============================================================================
\begin{uc}[Anzeigefenster schliesen]{UC: Anzeigefenster schliessen}
Mit diesem UseCase wird ein ge�ffnetes Anzeigefenster geschlossen.

  \begin{precond}
    \cond Ein Projekt mit mindestens einem Anzeigefenster ist geladen.
    \cond Es gibt mindestens ein ge�ffnetes Anzeigefenster.
  \end{precond}

  \begin{postsuccess}
  
    \cond Das Anzeigefenster ist geschlossen.
    
    \cond Nach dem letzten Speichern am Anzeigefenster vorgenommene
    Modifikationen (neue Knoten eingef�gt etc.) sind in der Verwaltungsdatei
    gespeichert oder nicht.
 
  \end{postsuccess}


  \begin{proc}    

    \step[1]
    Der Benutzer schlie�t das Anzeigefenster (durch klicken auf das ''X''
    Symbol rechts oben in der Titelleiste des Anzeigefensters).\\
    Alternativ kann er das Anzeigefenster �ber das entsprechende PopUp
    Men� \ref{WINDOW-LIST-POPUP} in der Liste \ref{WINDOW-LIST} schlie�en.
         
    \step[2]
    GIANT zeigt die allgemeine Sicherheitsabfrage 
    (siehe \ref{Sicherheitsabfrage}) und fragt nach, ob es eventuelle
    �nderungen im Anziegefenster speichern soll oder nicht.
    \gq{Do you want to save changes made to Window xy before closing it?}
    
    
    \step[3]
    Best�tigt der Benutzer mit YES, werden die �nderungen in die 
    Verwaltungsdatei geschrieben. Anderenfalls gehen s�mtliche 
    nicht gespeicherten �nderungen am Anzeigefenster verloren.
    
    \step[4]
    GIANT schlie�t das Anzeigefenster.
  
  \end{proc}


\end{uc}


% ==============================================================================
\begin{uc}[IML-Teilgraph in Anzeigefenster einf�gen]
          {UC: IML-Teilgraph in Anzeigefenster einf�gen}

Mit diesem UseCase k�nnen die Graph-Kanten und Graph-Knoten 
von IML-Teilgraphen in Anzeigefenster eingef�gt werden.
Siehe hierzu auch   
\ref{Verhalten beim Einf�gen von IML-Teilgraphen und Selektionen 
in Anzeigefenster},
insbesondere \ref{Einf�gen von IML-Teilgraphen in Anzeigefenster}.
	  

  \begin{precond}
    \cond Ein Projekt mit mindestens einem ge�ffneten 
          Anzeigefenster ist geladen.
    
    \cond Es gibt mindestens einen IML-Teilgraphen.
    
  \end{precond}

  \begin{postsuccess}
    
    \cond 
    Alle Knoten und Kanten des IML-Teilgraphen sind in das Anzeigefenster
    entsprechend dem gew�hlten Layout an der vorgegebenen Position
    eingef�gt.
    
    \cond
    In dem Anzeigefenster gibt es eine Selektion, die die neu eingef�gten
    Knoten und Kanten umfasst.
   
  \end{postsuccess}

  \begin{postfail}
    \cond Hat der Benutzer den UseCase an irgendeinem Punkt abgebrochen,
    kehrt das System zu dem Zustand zur�ck, in dem es vor Start des
    UseCase war.
  \end{postfail}
  
  \begin{proc}    
    \step[1]
    Der Benutzer startet den UseCase �ber das PopUp Men� \ref{SUBGRAPH-LIST-POPUP}
    im Hauptfenster \gq{Insert IML Subgraph}.
    Hierdurch wird der einzuf�gende IML-Teilgraph bestimmt (immer
    der IML-Teilgraph, auf dem der Rechtsklick ausgef�hrt wurde).
    
    \step[2] 
    GIANT zeigt in der Statuszeile im Hauptfenster \gq{Select Position in Display Window
    for Insertion of IML Subgraph}
    Der Benutzer w�hlt das entsprechende Anzeigefenster aus und
    gibt �ber das Fadenkreuz \ref{Fadenkreuzcursor} die Position vor, an der die neuen
    Fenster-Knoten und Fenster-Kanten eingef�gt werden sollen.
    Die Statuszeile im Hauptfenster schaltet auf Normalmodus zur�ck.
    
    \step[3]
    GIANT zeigt den Dialog zur Auswahl des entsprechenden Layoutalgorithmus
    \ref{Layoutalgorithmen-Dialog}.
    
    \step[4]
    Der Benutzer w�hlt einen der vorgegebenen Layoutalgorithmen aus. Bei 
    semantischen Layouts gibt er auch die Kantenklassen vor,
    die ber�cksichtigt werden sollen.
    
    \step[5]
    Der Benutzer gibt den Namen f�r die Selektion ein, die im Zielfenster
    f�r die neu eingef�gten Fenster-Knoten und Fenster-Kanten erzeugt wird.\\
    Gibt der Benutzer keinen Namen an, vergibt Giant automatisch einen Namen.
        
    \step[6]
    Der Benutzer best�tigt mit OK.
          
    \step[7] 
    GIANT berechnet das entsprechende Layout und zeigt derweil einen
    DIALOG an, �ber den der Benutzer �ber den Fortschritt der Berechnung
    informiert wird \ref{Progressbar}\\
    W�hrend der Berechnung des Layouts kann das System GIANT nicht bedient
    werden (die gesamte GUI ist gesperrt). Zug�nglich ist nur der
    Button zum Abbruch des Algorithmus.
    
    \step[8]
    Nach Abschluss der Berechnung f�gt GIANT die Fenster-Knoten und 
    Fenster-Kanten in das entsprechende Anzeigefenster ein.
   
   
  
  \end{proc}

  \begin{aproc}
    \astep{6} Der Benutzer bricht die Eingabe mit Cancel ab.
    \astep{7} Der Benutzer bricht die Berechnung des Layouts ab.  
  \end{aproc}



\end{uc}

% ==============================================================================
\begin{uc}[Selektion in Anzeigefenster einf�gen]
          {UC: Selektion in Anzeigefenster einf�gen}
	  
Mit diesem UseCase kann eine Selektion aus einem Anzeigefenster
in ein anderes Anzeigefenster unter Beibehaltung des Layouts
der Quell-Selektion kopiert werden. 
Siehe hierzu  
\ref{Verhalten beim Einf�gen von IML-Teilgraphen und Selektionen in Anzeigefenster},
insbesondere \ref{Einf�gen von Selektionen in Anzeigefenster}.  

  \begin{precond}
    \cond Ein Projekt mit mindestens zwei ge�ffneten 
          Anzeigefenstern ist geladen.
    
    \cond Es gibt mindestens eine Selektion.
    
  \end{precond}

  \begin{postsuccess}
    
    \cond 
    Alle Fenster-Knoten und Fenster-Kanten der \gq{Quell-Selektion} 
    sind auch in der \gq{Ziel-Anzeigefenster} unter Beibehaltung des
    Layouts der Quell-Selektion vorhanden.
    
    \cond
    Das Layout von Fenster-Knoten, die vor dem Einf�gen bereits im
    Ziel-Anzeigefenster vorhanden waren, bleibt 
    je nach Wahl des Benutzers unver�ndert oder wird ebenfalls ge�ndert.
    
    \cond
    Die kopierte Quell-Selektion existiert auch im \gq{Ziel-Anzeigefenster}
    als Selektion und tr�gt dort ebenfalls den Namen der Quell-Selektion. 

   
  \end{postsuccess}

 
  \begin{proc}    
    \step[1]
    Der Benutzer startet den UseCase �ber das PopUp Men�
    \ref{PopUp Men� Subgraph List} im Hauptfenster.
    Durch Auswahl einer der Men�eintr�ge 
    Copy IML Subgraph with new layout oder Copy IML Subgraph with existing layout.

    Hierdurch wird automatisch die Quell-Selektion bestimmt (immer
    die Selektion, auf der der Rechtsklick ausgef�hrt wurde).
    
    \step[2] 
    Der Benutzer w�hlt das entsprechende Ziel-Anzeigefenster aus.
    GIANT zeigt in der Statuszeile im Hauptfenster \gq{Select Position in Display Window
    for Insertion of copied IML Subgraph}
    Der Nutzer gibt gibt �ber das Fadenkreuz die Position vor, an der die neuen
    Fenster-Knoten und Fenster-Kanten eingef�gt werden sollen.
    \ref{Fadenkreuzcursor}
    Die Statuszeile im Hauptfenster schaltet auf Normalmodus zur�ck.
    
    \step[3]
    GIANT kopiert die Knoten und Kanten der 
    Quell-Selektion in das Ziel-Anzeigefenster.
    Je nachdem, welchen Eintrag der Benutzer im PopUp Men� ausgew�hlt hat,
    geschieht mit den bereits im Ziel-Anzeigefenster vorhandenen Knoten
    der Quell-Selektion folgendes:
    \begin {enumerate}
       \item
       Ihre Position wird im Ziel-Anzeigefenster nicht ver�ndert.
       
       \item
       Ihre Position wird im Zielanzeigefenster gem�� den dem Layout der
       Quell-Selektion ver�ndert.
        
    \end {enumerate}
   
     
  \end{proc}



\end{uc}

% ==============================================================================
\begin{uc}[Knoten und Kanten einer Selektion aus Anzeigefenster l�schen]
      {UC: Knoten und Kanten einer Selektion aus Anzeigefenster l�schen}
	  
	  
Die Selektion wird gel�scht, alle Fenster-Knoten und Fenster-Kanten
der Selektion werden ebenfalls gel�scht. 
Siehe auch 
\ref{Verhalten beim Entfernen von Fenster-Knoten und Fenster-Kanten}.

 
  \begin{precond}
    \cond Ein Projekt mit mindestens einem ge�ffneten 
          Anzeigefenster ist geladen.
    
    \cond Es gibt eine Selektion.
    
  \end{precond}

  \begin{postsuccess}
    
    \cond
    Die Selektion ist aus dem Anzeigefenster gel�scht.
    
    \cond 
    Alle betroffenen Fenster-Knoten und Fenster-Kanten sind gem��
    der unter 
    \ref{Verhalten beim Entfernen von Fenster-Knoten und Fenster-Kanten}
    beschriebenen Konvention aus dem Anzeigefenster entfernt.
    
 
   
  \end{postsuccess}

  \begin{postfail}
    \cond Hat der Benutzer den UseCase an irgendeinem Punkt abgebrochen,
    kehrt das System zu dem Zustand zur�ck, in dem es vor Start des
    UseCase war.
  \end{postfail}
  
  \begin{proc}    
  
    \step[1]
    Der User f�hrt einen Rechtsklick auf die entsprechende Selektion durch
    \ref{Selektionsauswahlliste} und w�hlt im 
    entsprechenden PopUp Men� den Eintrag (Delete Selection) aus.
      
    
    \step[2]
    GIANT zeigt die Sicherheitsabfrage (siehe \ref{Sicherheitsabfrage}) 
    und fragt nach, ob es die Selektion samt ihrer Knoten und Kanten
    wirklich l�schen soll.
    (\gq{Really delete Selection xy from its window including Nodes and Edges?})
    
    \step[3]
    Der Benutzer best�tigt mit YES.
    
    \step[4]
    GIANT l�scht die Selektion samt allen zugeh�rigen Fenster-Knoten und
    Fenster-Kanten aus dem entsprechenden Anzeigefenster.
      
  
  \end{proc}

  \begin{aproc}
    \astep{2} Der Benutzer bricht den UseCase mit NO ab.
  \end{aproc}






\end{uc}

% ==============================================================================
\begin{uc}[Anzeigefensters scrollen]{UC: Anzeigefensters scrollen}
Ver�ndert die Position des sichtbaren Anzeigeinhaltes.

  \begin{precond}
    \cond 
    Ein Projekt mit mindestens einem ge�ffneten Anzeigefenster ist geladen.
       
  \end{precond}

  \begin{postsuccess}
    
    \cond
    Die Position des sichtbaren Anzeigeinhalts wurde entsprechend abge�ndert.
    
  \end{postsuccess}

  \begin{proc}    
  
    \step[1]
    \begin {enumerate}
      \item
      Der Benutzer scrollt den sichtbaren Anzeigefokus mittels der horizontalen
      und vertikalen Scrollleisten des Anzeigefensterts. Dies geschieht mittels
      der Maus gem�� der g�ngigen Konventionen aus GTK/Ada f�r Scrollleisten.
      \ref{Scrolleisten}
      
      \item
      Gem�� der g�ngigen intuitiven Konventionen kann auch 
      mittels der Cursortasten gescrollt werden.\\
      Das Dr�cken der linken Cursortaste f�hrt z.B.\ dazu, dass der sichtbare
      Anzeigeinhalt des Anzeigefensters, welches innerhalb des Windowmanagers
      den Fokus hat, nach links verschoben wird.
      
    \end {enumerate}
        
  \end{proc}

\end{uc}


% ==============================================================================
\begin{uc}[Label]{UC: Anzeigefenster zoomen}
Ver�ndert den Ma�stab der Darstellung von Knoten und Kanten (relativ)

  \begin{precond}
    \cond 
    Ein Projekt mit mindestens einem ge�ffneten Anzeigefenster ist geladen.
       
  \end{precond}

  \begin{postsuccess}
    
    \cond
    Der angezeigte Bereich des sichtbaren Anzeigeinhalts wurde entsprechend
    vergr��ert oder verkleinert.
    Der Detaillierungslevel wurde ggf. automatisch angepa�t.
    
  \end{postsuccess}

  \begin{proc}    
  
    \step[1]
    \begin {enumerate}
      \item
      Der Benutzer gibt in der Zoomkontroll-Combobox des Anzeigefensters einen
      neuen Zoomwert ein, w�hlt darin einen der vorgefertigten Werte aus oder �ndert
      den Zoomwert in festgelegten Schritten mit den Zoom+ oder Zoom- Buttons.
      
      \item
      Der Benutzer klickt auf den \gq{Display} - Button der Zoomkontrolle
      
    \end {enumerate}
        
  \end{proc}
\end{uc}

% ==============================================================================
\begin{uc}[Label]{UC: Zoomen auf eine gesamte Selektion}
W�hlt die Zoomstufe und Scrollt so, da� eine gesamte Selektion im Fenster sichtbar ist

  \begin{precond}
    \cond 
    Ein Projekt mit mindestens einem ge�ffneten Anzeigefenster ist geladen.
       
  \end{precond}

  \begin{postsuccess}
    
    \cond
    Der angezeigte Bereich des sichtbaren Anzeigeinhalts wurde so
    vergr��ert oder verkleinert, da� die gesamte ausgew�hlte Selektion sichtbar
    ist, ggf. wurde auch entsprechend gescrollt.
    Der Detaillierungslevel wurde ggf. automatisch angepa�t.
    
  \end{postsuccess}

  \begin{proc}    
  
    \step[1]
    \begin {enumerate}
      \item
      Der Benutzer klickt auf \gq{Zoom to make selection fill window} im Popup-Men�
      der Selektionsauswahlliste \ref{Selektionsauswahlliste} auf der Selektion,
      welche er sichtbar machen will.
      
    \end {enumerate}
        
  \end{proc}
\end{uc}

% ==============================================================================
\begin{uc}[Label]{UC: Zoomen auf gesamten Inhalt eines Anzeigefensters}

W�hlt die Zoomstufe und Scrollt so, da� der gesamte Fensterinhalt im Fenster sichtbar ist

  \begin{precond}
    \cond 
    Ein Projekt mit mindestens einem ge�ffneten Anzeigefenster ist geladen.
       
  \end{precond}

  \begin{postsuccess}
    
    \cond
    Der angezeigte Bereich des sichtbaren Anzeigeinhalts wurde so
    vergr��ert oder verkleinert, da� der gesamte Fensterinhalt sichtbar
    ist, ggf. wurde auch entsprechend gescrollt.
    Der Detaillierungslevel wurde ggf. automatisch angepa�t.
    
  \end{postsuccess}

  \begin{proc}    
  
    \step[1]
    \begin {enumerate}
      \item
      Der Benutzer klickt auf \gq{Fill window} in der Zoomkontrolle des
      Fensters.
      
    \end {enumerate}
        
  \end{proc}

\end{uc}

% ==============================================================================
\begin{uc}[Zoomen auf eine Kante]{UC: Zoomen auf eine Kante}


W�hlt die Zoomstufe und Scrollt so, eine bestimmte Kante komplett im Fenster sichtbar ist

  \begin{precond}
    \cond 
    Ein Projekt mit mindestens einem ge�ffneten Anzeigefenster mit mindestens einer Kante
    ist geladen.
       
  \end{precond}

  \begin{postsuccess}
    
    \cond
    Der angezeigte Bereich des sichtbaren Anzeigeinhalts wurde so
    vergr��ert oder verkleinert, da� die betreffende Kante komplett
    sichtbar ist, ggf. wurde auch entsprechend gescrollt.
    Der Detaillierungslevel wurde ggf. automatisch angepa�t.
    
  \end{postsuccess}

  \begin{proc}    
  
    \step[1]
    \begin {enumerate}
      \item
      XXXXXXX
      
    \end {enumerate}
        
  \end{proc}

\end{uc}


% ==============================================================================
\begin{uc}[Label]{UC: Verschieben von Knoten, Selektionen, Kantenknickpunkten}
Kantenknickpunkte, Knoten, Selektionen (jeweils die aktuelle Selektion),
Drag and Drop, Cut and Paste






\end{uc}


% ==============================================================================
\begin{uc}[Label]{UC: Platz schaffen}

Dieser UseCase wird ben�tigt, um Fenster-Knoten auseinander schieben zu 
k�nnen. So kann der Benutzer an einer beliebigen Stelle des Anzeigefensters
gen�gend Platz zum Einf�gen neuer Fenster-Knoten und Fenster-Kanten schaffen.
Siehe hierzu auch \ref{Auseinanderschieben von Fenster-Knoten}.


  \begin{precond}
    \cond Ein Projekt mit mindestens einem ge�ffneten 
          Anzeigefenster ist geladen.
      
  \end{precond}

  \begin{postsuccess}
    
    \cond 
    Alle Fenster-Knoten und Fenster-Kanten des Anzeigefensters sind
    um den entsprechenden Betrag vom vorgegebenen Punkt innerhalb
    des Anzeigeigeihaltes weggeschoben worden.
    An der entsprechenden Stelle im Anzeigeinhalt ist eine freie Fl�che
    ohne Knoten und Kanten geschaffen worden.
    
    \cond
    Das Layout der betroffenen Fenster-Knoten und Fenster-Kanten bleibt
    weitgehend unver�ndert.
 
   
  \end{postsuccess}

 
  \begin{proc}    
    \step[1]
    Der Benutzer startet den UseCase �ber das PopUp Men�
    \ref{Empty Vis Pane Right click} beim Klick auf eine leere
    Fl�che in der VIS\_PANE und Auswahl von Make Room.
    
    \step[2]
    GIANT zeigt in der Statuszeile im Hauptfenster \gq{Select Position in Display Window
    for Making of Room}
    Der Benutzer gibt den Punkt um den herum die Fenster-Knoten (und damit
    automatisch auch die Fenster-Kanten) auseinander geschoben werden
    sollen �ber das Fadenkreuz vor. \ref{Fadenkreuzcursor}
    Die Statuszeile im Hauptfenster schaltet auf Normalmodus zur�ck.
     
    \step[3] 
    GIANT zeigt einen Dialog an, in dem der Benutzer ausw�hlt, um welchen 
    Betrag die Fenster-Knoten auseinander geschoben werden sollen.
     \ref{Platz Schaffen-Dialog}
   
    \step[4]
    Der Benutzer w�hlt einen geeigneten Betrag aus und best�tigt mit OK.
    
    \step[5]
    GINAT schiebt die Knoten entsprechend auseinander
    (siehe \ref{Auseinanderschieben von Fenster-Knoten}).
    
  \end{proc}


\end{uc}


% ==============================================================================
\begin{uc}[Label]{UC: Pin anlegen}

Mit diesem UseCase kann ein neuer Pin erzeugt werden.

 \begin{precond}
    \cond Es gibt ein ge�ffnetes Anzeigefenster.
  \end{precond}

  \begin{postsuccess}
    \cond 
    In der Liste �ber die Pins des Anzeigefensters
     \ref{VIS-PANE-Pins} befindet sich ein neuer Pin mit dem entsprechenden Namen.

  \end{postsuccess}

  \begin{postfail}
    \cond Das System bleibt im bisherigen Zustand.
  \end{postfail}
  
  \begin{proc}    
    \step[1]
    
    Zur Durchf�hrung gibt es zwei M�glichkeiten:
    
    \begin{enumerate}
     \item
      Der Benutzer f�hrt einen Rechtklick auf den Anziegeinhalt des
      entsprechenden Anzeigefensters durch und w�hlt aus dem PopUp
      Men� \ref{Empty Vis Pane Right click} den Eintrag \gq{New Pin} aus.
     \item   
      Der Benutzer w�hlt in der PIN\_LIST (\ref{VIS-PANE-Pins}) nach
      Rechtsklick auf der Liste im Kontextmen� \gq{New Pin} aus.
    \end{enumerate}
    \step[2] 
    GIANT �ffnet den allgemeinen Texteingabedialog \ref{DIALOG-WINDOW},
    \gq{Please enter name for new pin}
      
    \step[3] 
    Der Benutzer gibt dort einen zul�ssigen Namen f�r den neuen Pin
    ein und best�tigt mit OK.\\
      
    \step[4]
    GIANT erzeugt den Pin.
  
  \end{proc}

  \begin{aproc}
    \astep{3} Der Benutzer bricht die Verarbeitung mit Cancel ab.
  \end{aproc}

\end{uc}


% ==============================================================================
\begin{uc}[Label]{UC: Pin anspringen}
Stellt eine �ber den Pin gespeicherte Position des Anzeigefokus
wieder her.

 \begin{precond}
    \cond Es gibt ein ge�ffnetes Anzeigefenster mit mindestens 
    einem Pin.
  \end{precond}

  \begin{postsuccess}
    \cond 
    Der sichtbare Anzeigefokus des Anzeigefensters ist auf die entsprechenden
    Koordinaten und die entsprechende Zoomstufe, wie sie im ausgew�hlten
    Pin hinterlegt waren, gesetzt.
    
  \end{postsuccess}
  
  \begin{proc}    
    \step[1]
    Zur Durchf�hrung gibt es zwei M�glichkeiten:
    \begin{enumerate}
    
      \item
      Der Benutzer f�hrt einen Doppelklick auf den entsprechenden Pin
      in der Liste \ref{VIS-PANE-Pins} aus.
      
      
      \item
      Der Benutzer �ffnet das entsprechende PopUp Men� durch
      rechtsklick auf den Pin bei \ref{VIS-PANE-Pins}und w�hlt den Eintrag
      \gq{Focus Pin} aus.
        
    \end {enumerate}
    
          
    \step[2]
    GIANT setzt den sichtbaren Anzeigeinhalt gem�� den im Pin gespeicherten
    Informationen.
  
  \end{proc}



\end{uc}


% ==============================================================================
\begin{uc}[Label]{UC: Pin l�schen}
L�scht einen Pin.

 \begin{precond}
    \cond Es gibt ein ge�ffnetes Anzeigefenster mit mindestens 
    einem Pin.
  \end{precond}

  \begin{postsuccess}
    \cond 
    Der Pin ist gel�scht und nicht mehr in der Liste \ref{VIS-PANE-Pins} des entsprechenden
    Anzeigefensters ausw�hlbar.

    
  \end{postsuccess}
  
  \begin{proc}    
    \step[1]
    Der Benutzer �ffnet das entsprechende PopUp Men� durch Rechtsklick
    auf den Pin in \ref{VIS-PANE-Pins} und w�hlt den Eintrag \gq{Delete Pin} aus.

     
    \step[2]
    GIANT l�scht den entsprechenden Pin.
  
  \end{proc}

\end{uc}


\chapter{Filter}
Hier Text zu Filtern
% ==============================================================================
%  $RCSfile: filter.tex,v $, $Revision: 1.7 $
%  $Date: 2003/04/03 14:22:48 $
%  $Author: squig $
%
%  Description: UseCases f�r Filter
%
%  Last-Ispelled-Revision: 1.5
%
% ==============================================================================

\begin{uc}[Selektionen ausblenden]{UC: Selektionen ausblenden}
Mit diesem UseCase k�nnen Selektionen innerhalb
eines Anzeigefensters ausgeblendet werden.

  \begin{precond}
    \cond Es gibt ein ge�ffnetes Anzeigefenster mit einer Selektion.
  \end{precond}

  \begin{postsuccess}

    \cond Alle zu der Selektion geh�rende Fenster-Knoten und
          Fenster-Kanten sind ausgeblendet, d.h. sie sind
          im Anzeigefenster nicht mehr sichtbar. Dies trifft
          auch f�r Fenster-Knoten und Fenster-Kanten, die
          noch zu weiteren Selektionen, geh�ren zu.
    
    \cond Die ausgeblendeten Selektionen k�nnen nicht 
          zur aktuellen Selektion gemacht werden (siehe 
          \ref {Selektion zur aktuellen Selektion machen} und damit
          nicht mehr direkt bearbeitet werden, 
          d.h. die Menge der selektierten 
          Fenster-Knoten und Fenster-Kanten kann nicht
          mehr abge�ndert werden (siehe hierzu \ref {Selektieren von 
          Fenster-Knoten und Fenster-Kanten in Anzeigefenstern}).


    \cond Die Fenster-Knoten und Fenster-Kanten sind aber immer noch 
          Bestandteil des Anzeigefensters und k�nnen �ber den folgenden
          UseCase (siehe \ref{Selektionen einblenden}) wieder zur Anzeige
          gebracht werden.
    
  \end{postsuccess}

   
  \begin{proc}

    \step[1] 
    Der Benutzer f�hrt einen Rechtsklick mit der Maus auf
    die auszublendende Selektion in der Selektionsauswahlliste 
    (siehe \ref{Selektionsauswahlliste}) durch 
    und w�hlt im Popup-Men� den Eintrag \gq{Fade Out Selection}
    aus. Diese Funktionalit�t kann nicht auf die aktuelle
    Selektion angewendet werden 
    (siehe \ref{Aktuelle Selektion vs Selektionen}).

    \step[2]
    GIANT blendet die Selektion aus.
 
  \end{proc}

\end{uc}



\begin{uc}[Selektionen einblenden]{UC: Selektionen einblenden}
Mit diesem UseCase k�nnen ausgeblendete Selektionen
wieder eingeblendet werden.

  \begin{precond}
    \cond Es gibt ein ge�ffnetes Anzeigefenster mit einer 
          ausgeblendeten Selektion.
  \end{precond}

  \begin{postsuccess}

    \cond 
    Die Selektion ist wieder eingeblendet, alle zu ihr geh�renden 
    Fenster-Knoten und Fenster-Kanten sind wieder
    im Anzeigeinhalt sichtbar dargestellt.
    
  \end{postsuccess}

   
  \begin{proc}

    \step[1] 
    Der Benutzer f�hrt einen Rechtsklick mit der Maus auf
    eine ausgeblendete Selektion in der Selektionsauswahlliste 
    (siehe \ref{Selektionsauswahlliste}) durch 
    und w�hlt im Popup-Men� den Eintrag \gq{Fade in Selection}
    aus. 

    \step[2]
    GIANT blendet die Selektion ein.
 
  \end{proc}

\end{uc}


%\begin{uc}[Label]{UC: Detailfilter f�r ein Anzeigefenster}
%<<<<<<<Martin: BIN MIR NICHT GANZ SICHER
%ICH GLAUBE DIESE ANFORDERUNG WURDE GECANCELT>>>>>>>>>>>>>>>
%
%
%
%\end{uc}
%
%\begin{uc}[Label]{UC: Detailfilter f�r eine Selektion}
%<<<<<<<Martin: BIN MIR NICHT GANZ SICHER - MARTIN
%ICH GLAUBE DIESE ANFORDERUNG WURDE GECANCELT>>>>>>>>>>>>>>>
%
%\end{uc}


\chapter{Anfragen}
Hier Text zu Anfragen
% ==============================================================================
%  $RCSfile: query.tex,v $, $Revision: 1.8 $
%  $Date: 2003/03/31 16:29:39 $
%  $Author: squig $
%
%  Description: UseCases f�r die Anfragen
%
%  Last-Ispelled-Revision: 1.7
%
% ==============================================================================

% ==============================================================================
\begin{uc}[Anfrage ausf�hren]{UC: Neue Anfrage ausf�hren}
\index{Anfragen!ausf�hren}
Mit diesem UseCase kann eine Anfrage �ber den Anfragedialog
(siehe \ref{GUI Anfragedialog}) eingegeben werden.
Die M�glichkeiten der GQSL sind im Detail unter Kapitel
\ref {GIANT Query Skripting Language} beschrieben.


  \begin{precond}
    \cond Ein Projekt ist geladen.
  \end{precond}

  \begin{postsuccess}
    \cond Die Anfrage wurde ausgef�hrt. Alle Ergebnisse liegen vor.
      
  \end{postsuccess}

  \begin{postfail}
    \cond Das System bleibt im bisherigen Zustand.

    \cond Wurde der UseCase w�hrend der Berechnung des
          Anfrageergebnisses durch GIANT abgebrochen, so
          gehen s�mtliche bereits fertig gestellten Teilergebnisse
          verloren. 
  \end{postfail}
  
  \begin{proc}    
    \step[1]
    Der Benutzer startet den UseCase durch Auswahl des Eintrags 
    \gq{Tools -- Execute GQSL Query} im Hauptmen�  
    (siehe hierzu \ref{Main-Window-Tools}).
      
    \step[2] 
    GIANT �ffnet den Anfragedialog (siehe \ref{GUI Anfragedialog}).
      
    \step[3]
    Der Benutzer gibt dort im daf�r vorgesehenen Textfeld die GQSL Anfrage 
    (siehe auch \ref {GIANT Query Skripting Language}) ein und best�tigt 
    mit \gq {Start Query}.

    \step[4]
    GIANT pr�ft das eingegebene GQSL Skript. Sollte das Skript
    nicht den Vorgaben der Grammatik 
    (siehe \ref {GIANT Query Skripting Language} entsprechen, erscheint
    eine Fehlermeldung (siehe \ref {afa Fehlerverhalten})
    und das System kehrt zu Schritt 3 des UseCase zur�ck.
        
    \step[5]    
    GIANT berechnet die Anfrage und teilt dem Benutzer
    den Fortschritt mittels eines Progressbars (\ref{Progressbar-Modale}) mit.
    W�hrend der Abarbeitung der Anfrage ist das System nicht bedienbar.\\
    
  
  \end{proc}

  \begin{aproc}
    \astep{3} Der Benutzer bricht mit Cancel ab.

    \astep{5} Die laufende Berechnung des Anfrageergebnisses 
    kann vom Benutzer jeder Zeit durch Bet�tigen des
    Buttons \gq{Cancel Calculation} (\ref{Progressbar-Modale-Cancel}) abgebrochen werden.

  \end{aproc}

\end{uc}


% ==============================================================================
\begin{uc}[UC Anfrage laden]{UC: Anfrage laden}
\index{Anfragen!aus Datei laden}
Der Benutzer kann zus�tzlich zur manuellen Eingabe von GQSL Anfragen
(siehe \ref{Anfrage ausf�hren}) auch gespeicherte Anfragen aus 
einer Anfragedatei (siehe \ref {Config Anfrage-Dateien}) laden.

  \begin{precond}
    \cond Ein Projekt ist geladen.
  \end{precond}

  \begin{postsuccess}
    \cond Die Anfrage wurde ausgef�hrt. Alle Ergebnisse liegen vor.
      
  \end{postsuccess}

  \begin{postfail}
    \cond Das System bleibt im bisherigen Zustand.

    \cond Wurde der UseCase w�hrend der Berechnung des
          Anfrageergebnisses durch GIANT abgebrochen, so
          gehen s�mtliche bereits fertig gestellten Teilergebnisse
          verloren. 
  \end{postfail}
  
  \begin{proc}    
    \step[1]
    Der Benutzer startet den UseCase durch Auswahl des Eintrags 
    \gq{Tools -- Execute GQSL Query} im Hauptmen�  
    (siehe hierzu \ref{Main-Window-Tools}).
      
    \step[2] 
    GIANT �ffnet den Anfragedialog (siehe \ref{GUI Anfragedialog}).
      
    \step[3] 
    Der Benutzer bet�tigt im Dialog den Button \gq{Load Query}.

    \step[4]
    Daraufhin zeigt GIANT den Standard-Filechooser-Dialog 
    (siehe \ref {Standard-Filechooser-Dialog}).

    \step[5]
    Der Benutzer w�hlt die gew�nschte Anfragedatei (siehe 
    \ref {Config Anfrage-Dateien}) aus.
        
    \step[6]
    GIANT zeigt das aus der Anfragedatei geladene GQSL Skript 
    (siehe \ref {GIANT Query Skripting Language}) im Textfeld
    des Anfragedialoges an.

    \step[7]
    Falls gew�nscht kann der Benutzer das GQSL Skript im Textfeld
    noch manuell weiter modifizieren.
    
    \step[8]
    Der Benutzer startet die Berechnung der Anfrage durch Bet�tigung
    des \gq{Start Query} im Anfragedialog (siehe \ref{GUI Anfragedialog}).

    \step[9]
    GIANT pr�ft das geladene und eventuell modifizierte GQSL Skript. 
    Sollte das Skript nicht den Vorgaben der Grammatik 
    (siehe \ref {GIANT Query Skripting Language} entsprechen, erscheint
    eine Fehlermeldung (siehe \ref {afa Fehlerverhalten})
    und das System kehrt zu Schritt 7 des UseCase zur�ck.

    \step[10]
    GIANT berechnet die Anfrage und teilt dem Benutzer
    den Fortschritt mittels eines Progressbars (\ref{Progressbar-Modale}) mit.
    W�hrend der Abarbeitung der Anfrage ist das System nicht bedienbar.\\

  \end{proc}

  \begin{aproc}
    \astep{3} Der Benutzer bricht den UseCase mit Cancel ab.
    \astep{4} Der Benutzer bricht die Auswahl der Anfragedatei mit Cancel ab.
              Das System kehrt dann zu Schritt 2 bei der
              Abarbeitung des UseCase zur�ck.
    \astep{6} Der Benutzer bricht den UseCase mit Cancel ab.

    \astep{10} Die laufende Berechnung des Anfrageergebnisses 
    kann vom Benutzer jeder Zeit durch Bet�tigen des
    Buttons \gq{Cancel Calculation} (\ref{Progressbar-Modale-Cancel})abgebrochen werden.


  \end{aproc}

\end{uc}

% ==============================================================================
\begin{uc}[Label]{UC: Anfrage speichern}
\index{Anfragen!in eine Datei speichern}
Mit diesem UseCase kann der Benutzer GQSL Skripte aus dem 
Anfragedialog (siehe \ref{GUI Anfragedialog}) in Anfragedateien 
(siehe \ref {Config Anfrage-Dateien}) speichern.

  \begin{precond}
    \cond Der Anfragedialog (siehe \ref{GUI Anfragedialog}) ist
          ge�ffnet und enth�lt in dem daf�r vorgesehenen Textfeld
          entweder ein manuell eingegebenes oder ein aus einer
          Datei geladenes und eventuell modifiziertes GQSL Skript.

  \end{precond}

  \begin{postsuccess}
    \cond Eine Anfragedatei, welche das GQSL Skript enth�lt, wurde
          angelegt.
    \cond GIANT zeigt den Anfragedialog, das gespeicherte GQSL Skript
          ist weiterhin in dem Textfeld vorhanden.
      
  \end{postsuccess}

  \begin{postfail}
    \cond Das System bleibt im bisherigen Zustand.
    \cond Es wurde keine Anfragedatei erzeugt
    \cond GIANT zeigt weiterhin den  Anfragedialog 
          (siehe \ref{GUI Anfragedialog}) und alle dort get�tigten
          Eingaben (insbesondere das GQSL Skript im Textfeld des
          Dialoges) bleiben erhalten.
   
  \end{postfail}
  
  \begin{proc}    

    \step[1]
    Der Benutzer bet�tigt im Anfragedialog (siehe \ref{GUI Anfragedialog})
    den Button \gq{Save Query}.
    
    \step[2]
    GIANT pr�ft ob das GQSL Skript im Textfeld des Anfragedialoges den
    Vorgaben der Grammatik (siehe \ref {GIANT Query Skripting Language} 
    entspricht. Falls nein erscheint eine Fehlermeldung 
    (siehe \ref {afa Fehlerverhalten})
    und das System kehrt zu Schritt 1 des UseCase zur�ck.

    \step[3]
    GIANT �ffnet den Standard-Filechooser-Dialog (siehe 
    \ref {Standard-Filechooser-Dialog}).
  
    \step[4]
    Der Benutzer gibt den Pfad und die Datei, in der das GQSL Skript
    gespeichert werden soll, vor und best�tigt mit OK.

    \step[5]
    GIANT speichert das GQSL Skript in der vorgegebenen Anfragedatei.

  \end{proc}

  \begin{aproc}
    \astep{4} Der Benutzer bricht den UseCase mit Cancel ab.

  \end{aproc}

\end{uc}



\chapter{Laden und Speichern}
Hier Text zur Lade- und Speicherfunktionalit�t
% ==============================================================================
%  $RCSfile: load_store.tex,v $, $Revision: 1.16 $
%  $Date: 2003/04/05 12:41:57 $
%  $Author: squig $
%
%  Description: UseCases f�r Lade- und Speicherfunktionalit�t
%
%  Last-Ispelled-Revision: 1.12
%
% ==============================================================================

\begin{uc}[Neues Projekt]{UC: Neues Projekt}
\index{Projekte!neues Projekt erstellen}
  Erstellt ein neues GIANT Projekt. Ein eventuell bereits ge�ffnetes
  Projekt wird dabei geschlossen, wobei �nderungen auf Nachfrage 
  vorher gespeichert werden.
  Weitere wichtige Informationen, die in engem Zusammenhang mit 
  diesem UseCase stehen, sind unter dem Abschnitt 
  \ref{Project Persistenz der Projekte} zu finden.\\
  Zul�ssige Namen f�r Projekte sind unter Abschnitt 
  \ref{afa Zulaessige Namen} spezifiziert.
  
  
  \begin{precond}
    \cond Das Programm ist gestartet.
  \end{precond}

  \begin{postsuccess}
    \cond Ein neues GIANT Projekt mit dem eingegebenen Namen 
          ist erstellt und geladen.   

    \cond Eine IML-Datei ist geladen.

    \cond Das angegebene Projektverzeichnis ist gegebenenfalls erstellt 
          worden (siehe \ref {Project Das Projektverzeichnis}).

    \cond Die Projektdatei (siehe \ref{Project Die Projektdatei})
          liegt im Projektverzeichnis.
 
    \cond Ein eventuell zuvor ge�ffnetes Projekt ist geschlossen.
    
    \cond �nderungen an einem zuvor ge�ffneten Projekt sind gespeichert
          oder verworfen.
  \end{postsuccess}

  \begin{postfail}
    \cond Das System bleibt im bisherigen Zustand.
  \end{postfail}
  
  \begin{proc}
    \step[1]
    Der Benutzer startet den UseCase �ber das Hauptmen� \gq{New Project}
    (siehe \ref{Main-Window-File}) 
  
    \step[2] Der Benutzer w�hlt im Standard-Filechooser-Dialog
    (\gq{Please Select IML File}) eine IML-Datei aus und best�tigt
    seine Eingabe mit OK (siehe \ref{Standard-Filechooser-Dialog}).
    
    \step[3] Der Benutzer gibt im Standard-Filechooser-Dialog
    den Pfad und Namen der Projektdatei (\gq{Please select Project
    Path and Project File Name}) ein. Der Name der Projektdatei ist
    automatisch auch der Name f�r das Projekt.

    \step[4]
    Dann best�tigt er seine Eingabe mit OK.\\
    Existiert die eingegebene Projektdatei bereits, 
    erscheint eine Fehlermeldung gem�� den unter Abschnitt 
    \ref{afa Fehlerverhalten} beschriebenen Konventionen.\\
    Existiert in dem Projektverzeichnis bereits eine andere Projektdatei, so 
    erscheint ebenfalls eine Fehlermeldung.\\
    Das Verzeichnis in dem die Projektdatei liegt, wird automatisch zum 
    Projektverzeichnis.
    
    \step[5] Falls bereits ein Projekt ge�ffnet ist, erscheint eine 
    Sicherheitsabfrage (siehe \ref{Sicherheitsabfrage})
    ob dieses gespeichert werden soll. Lehnt der Benutzer dies ab, gehen
    alle nicht gespeicherten Informationen verloren.\\
    Entscheidet der Benutzer sich f�r Speichern, so wird
    die unter \ref{Alles Speichern} beschriebene Funktionalit�t 
    ausgef�hrt.
    
    \step[6]
    GIANT schlie�t das aktuell ge�ffnete Projekt, erstellt 
    ein neues Projekt und �ffnet dieses.
       
  \end{proc}

  \begin{aproc}
    \astep{2} Der Benutzer bricht die Verarbeitung mit Cancel ab.  
    \astep{3} Der Benutzer bricht die Verarbeitung mit Cancel ab.
  \end{aproc}
\end{uc}

% ==============================================================================

\begin{uc}[Projekt �ffnen]{UC: Projekt �ffnen}
\index{Projekte!�ffnen}
  �ffnet ein GIANT Projekt. Ein eventuell bereits vorhandenes 
  Projekt wird dabei geschlossen, wobei �nderungen auf Nachfrage 
  vorher gespeichert werden. \\ 
  Sollte der Benutzer die XML-Dateien 
  innerhalb des Projektverzeichnisses (siehe 
  \ref{Project Das Projektverzeichnis}) manuell modifiziert haben, so
  dass diese von den durch GIANT automatisch erstellten Dateien
  abweichen, so wird keinerlei Garantie f�r das korrekte �ffnen
  des Projektes �bernommen. Das Verhalten
  bez�glich eventuell auftretender Fehler ist undefiniert.

  \begin{precond}
    \cond Das Programm ist gestartet.
  \end{precond}

  \begin{postsuccess}
    \cond Das gew�nschte GIANT Projekt ist geladen.
    \cond Die zugeh�rige IML-Datei ist geladen.
    \cond �nderungen an einem zuvor ge�ffneten Projekt sind gespeichert
    oder verworfen.
  \end{postsuccess}

  \begin{postfail}
    \cond Das System bleibt im bisherigen Zustand.
  \end{postfail}
  
  \begin{proc}    
    \step[1]
    Der Benutzer startet den UseCase �ber das Hauptmen�. 
    (siehe \ref{Main-Window-File} Men�) \gq{Load Project}
    
    \step[2] 
    Der Benutzer w�hlt aus dem Standard-Filechooser-Dialog (siehe 
    \ref {Standard-Filechooser-Dialog})
    eine vorhandene GIANT Projektdatei (siehe \ref{Project Die Projektdatei}) 
    aus und best�tigt mit OK.
    
    \step[3] Falls bereits ein Projekt ge�ffnet ist, erscheint eine 
    Sicherheitsabfrage (siehe \ref{Sicherheitsabfrage})
    ob dieses gespeichert werden soll. Lehnt der Benutzer dies ab, gehen
    alle nicht gespeicherten Informationen verloren.\\
    Entscheidet der Benutzer sich f�r Speichern, so wird
    die unter \ref{Alles Speichern} beschriebene Funktionalit�t 
    ausgef�hrt.
    
    \step[4]
    GIANT schlie�t das alte Projekt und l�dt das angegebene Projekt.
  \end{proc}

  \begin{aproc}
    \astep{2} Der Benutzer bricht die Verarbeitung mit Cancel ab.
  \end{aproc}
\end{uc}

% ==============================================================================

\begin{uc}[Projekt speichern]{UC: Projekt speichern}  
\index{Projekte!speichern}
  Speichert alle �nderungen an einem Projekt. Der Zustand der
  entsprechenden Verwaltungsdateien im Projektverzeichnis entspricht
  nach erfolgreicher Ausf�hrung dieses UseCases exakt dem aktuellen
  Zustand des ge�ffneten Projektes. Alle Konventionen zur
  Persistenz von Projekten sind im Abschnitt 
  \ref{Project Persistenz der Projekte} exakt spezifiziert.

  \begin{precond}
    \cond Das Programm ist gestartet.

    \cond Ein Projekt ist ge�ffnet. 

  \end{precond}

  \begin{postsuccess}
    \cond Die Informationen des Projekts 
    (einschlie�lich aller m�glichen �nderungen) sind persistent in die
    Verwaltungsdateien geschrieben.
 
  \end{postsuccess}

  \begin{postfail}
    \cond Das System bleibt im bisherigen Zustand.
  \end{postfail}
  
  \begin{proc}    
    \step[1]
    Der Benutzer startet den UseCase �ber das Hauptmen�
    (\ref{Main-Window-File}) \gq{Save Project}.
 
     
    \step[2]
    GIANT f�hrt die unter \ref{Alles Speichern} beschriebene Funktionalit�t 
    aus und speichert alle Informationen zu dem Projekt in der zugeh�rigen
    Projektdatei.
  \end{proc}


  \begin{aproc}

    \ageneral 
    Es sind keine alternativen Abl�ufe vorgesehen.

  \end{aproc}

\end{uc}



% ==============================================================================

\begin{uc}[Projekt speichern unter]{UC: Projekt speichern unter}
\index{Projekte!unter neuem Namen speichern}
Speichert alle Informationen zu einem Projekt in eine neue Projektdatei
(entsprechende Verwaltungsdateien werden ebenfalls dupliziert).


  \begin{precond}
    \cond Das Programm ist gestartet.
    \cond Ein Projekt ist ge�ffnet (entweder ein neu erzeugtes oder ein
    geladenes).
  \end{precond}

  \begin{postsuccess}
    \cond Eine neue Projektdatei ist erzeugt.
    \cond Das angegebene Projektverzeichnis ist gegebenenfalls erstellt 
          worden.
    \cond Die Informationen des Projekts sind persistent in die 
          neuen Verwaltungsdateien im Projektverzeichnis 
          der neuen Projektdatei geschrieben (siehe auch
          \ref{Project Persistenz der Projekte}).
        
    \cond Das Projekt bleibt in GIANT ge�ffnet, zuk�nftiges Speichern
          (siehe \ref{Projekt speichern} betrifft nur die Dateien 
          im neu erzeugten Projekt.

    \cond Die alte Projektdatei und alle Verwaltungsdateien bleiben
          unver�ndert.
    
 
  \end{postsuccess}

  \begin{postfail}
    \cond Das System bleibt im bisherigen Zustand.
  \end{postfail}
  
  \begin{proc}    
    \step[1]
    Der Benutzer startet den UseCase �ber das Hauptmen�
    (siehe \ref{Main-Window-File}) \gq{Save Project As...}.

     
    \step[2] Der Benutzer gibt im Standard-Filechooser-Dialog 
             (siehe \ref {Standard-Filechooser-Dialog}) den Pfad, 
             das neue Projektverzeichnis 
             (siehe \ref{Project Das Projektverzeichnis})
             und den Namen f�r die neue Projektdatei
             (siehe \ref{Project Die Projektdatei}) vor. \\
    Der Name der Projektdatei ist automatisch auch der Name f�r das Projekt.
    Zul�ssige Namen sind unter Abschnitt \ref{afa Zulaessige Namen}
    spezifiziert. \\
    Das Verzeichnis in dem die Projektdatei liegt, wird automatisch zum 
    Projektverzeichnis. 
  
    \step[3]  
    Der Benutzer best�tigt seine Eingabe mit OK.\\
    Existiert die eingegebene Datei schon, erscheint eine Fehlermeldung.\\
    Existiert in dem Projektverzeichnis bereits eine andere Projektdatei, so 
    erscheint eine entsprechende Fehlermeldung.\\
  
  \end{proc}


 \begin{aproc}
    \astep{2} Der Benutzer bricht die Verarbeitung mit Cancel ab.  
 \end{aproc}
\end{uc}








\chapter{Sonstige Use-Cases}
Hier Text zu sonstigen Use-Cases
% ==============================================================================
%  $RCSfile: additional.tex,v $, $Revision: 1.2 $
%  $Date: 2003/02/05 14:01:12 $
%  $Author: schulzgt $
%
%  Description: Sonstige Use-Cases
%
% ==============================================================================

\begin{uc}[Label]{UC: Verfolgen von Kanten}
Wird �ber mitgelieferte Anfragen implementiert.
Dann kein eigener UC mehr.
\end{uc}

\begin{uc}[Label]{UC: Layout auf Selektion anwenden}
\end{uc}

\begin{uc}[Label]{UC: Knoteninformation �ber Tooltip anzeigen}
\end{uc}

\begin{uc}[Label]{UC: Knoteninformation im Infofenster anzeigen}
\end{uc}

\begin{uc}[Label]{UC: Anzeige des Quellcodes eines Knotes in externem Editor}
\end{uc}

\begin{uc}[Label]{UC: Verschieben des sichtbaren Bereichs �ber Minimap}
\end{uc}


% sample usecase

\begin{uc}[Label]{Use-Case}

  \begin{actors}
    \actor{Benutzer}
  \end{actors}

  \begin{precond}
    \cond Bedingung
  \end{precond}

  \begin{proc}
    \step[Schritt] Schritt
    \begin{subproc}
      \step[Unter Schritt] Unter Schritt
    \end{subproc}
  \end{proc}

  \begin{aproc}
    \astep{Schritt} Alternativer Schritt
  \end{aproc}

  \begin{postcond}
    \cond Bedingung
  \end{postcond}
  
\end{uc}

\chapter{Begriffslexikon}
\begin{nomenclature}

\term{Begriff}{Englische �bersetzung}{Erkl�rung}

\end{nomenclature}

%%% Local Variables: 
%%% TeX-master: "spec"
%%% End: 


% appendix
\appendix
Anhang

\end{document}
