% ==============================================================================
%  $RCSfile: technical.tex,v $, $Revision: 1.3 $
%  $Date: 2003/02/13 21:17:51 $
%  $Author: schulzgt $
%
%  Description: Hier sind die Anforderungen an die Programmumgebung 
%  spezifiziert. Dazu geh�hren die verwendete Hard- und Software, sowie
%  Schnittstellen zu anderen Produkten.
%
% ==============================================================================

\section{Software}
Das Programm stellt an die installierte Software folgende Anforderungen:
\begin{itemize}
  \item Sun Solaris oder Linux Betriebssystem
  \item Bauhaus Tools
  \item Emacs oder vi Texteditor
\end{itemize}


\section{Hardware}
Das Programm l�uft auf SPARC Workstations und x86 kompatiblen PCs.
Die minimalen Hardwareanforderungen sind im Folgenden beschrieben.

\subsection{Hardwareanforderungen SPARC}
\begin{itemize}
  \item UltraSPARC-II 300 MHz
  \item 512 MB Hauptspeicher
  \item 8 Bit Grafik mit einer min. Aufl�sung von 1024*786
  \item Maus mit mindestens 3 Tasten
\end{itemize}

\subsection{Hardwareanforderungen x86}
\begin{itemize}
  \item Pentium III 600 MHz
  \item 512 MB Hauptspeicher
  \item 8 Bit Grafik mit einer min. Aufl�sung von 1024*786
  \item Maus mit mindestens 3 Tasten
\end{itemize}

\section{Produkt-Schnittstellen}
Das Programm soll in die Bauhaus Suite integriert werden k�nnen.
Als Schnittstelle dient dabei die Bauhaus IML Bibliothek. Weiterhin
kann die Verwendung von Kommandozeilenparametern und der GQSL zur 
Integration in das vorhandene System genutzt werden.
