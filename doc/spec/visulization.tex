% ==============================================================================
%  $RCSfile: visulization.tex,v $, $Revision: 1.2 $
%  $Date: 2003/02/14 14:34:49 $
%  $Author: schwiemn $
%
%  Description:
%
% ==============================================================================

Genaue Beschreibung der Visualisierung des Graphen in einem Anzeigefenster.

\section{Visualisierungsstiele}
Der Benutzer kann �ber XML-Konfigurationsdateien beliebige viele Visualisierungsstiele
erzeugen. �ber so solch einen Visualisierungsstiel kann er definieren wie Knoten und
Kanten von Graphen in den Anzeigefenstern dargestellt werden, wobei er die 
Einstellungen f�r Knoten und Kanten differenziert f�r die verschiedenen Knoten- und
Kantenklassen des IML-Graphen vornehmen kann.\\
Zu jedem Anzeigfenster gibt es eine Liste aus der zur Laufzeit ein 
entsprechender vordefinierter Visualisierungsstiel ausgew�hlt werden kann.\\
Auf diese Art und Weise kann der Benutzer also je nach aktuellem Bedarf
die Darstellung im ANzeigefenster �nderen.

WAS KANN ALLES KONFIGURIERT WERDEN USW.:- VERWEIS AUF KONFIGURATIOS DATEIEN


\section{Visualisierung von Knoten}
Hier wird die Visualisierung eines Knotens auf dem Anziegeinhalt innerhalb eines Anzeigefensters 
beschrieben. Die Tats�chliche Darstellung h�ngt stark von der aktuellen Zoomstufe und dem gew�hlten
Visualisierungsstiel ab.

  \subsection {Grafik}
  HIER EVENTUELL SKIZZE EINF�GEN; WELCHE DAS AUSSEHEN EINES KNOTENS BESCHREIBT


  \subsection {Knoten-Rechteck}
  Grundlage der Darstellung eines Knotens ist immer ein Rechteck mit fixer Breite.
  Die H�he des Rechtecks ist proportional zur Anzahl der Attribute des Knotens, die  
  direkt im Anzeigefenster dargestellt werden.

  \subsection {Knoten-Icon}
  Jeder Knoten kann �ber ein vordefiniertes Icon verf�gen, welches im linken oberen Eck des
  Knoten-Rechtecks darhgestellt wird. Dieses Icon muss eine Gr��e 32*32 Pixel haben.


  \subsection {Klassenname und ID}
  Der Name der Knotenklasse und die eindeutige ID eines Knotens werden innerhalb des 
  Knoten-Rechtecks angezeigt.

  \subsection {Attribute des Knoten}
  Innerhalb des Knoten-Rechtecks k�nnen auch Attribute dargestellt werden (Attributname und
  Wert), da das Knoten-Rechteck eine fixe Breite hat, wird der zugeh�rige Text abgeschnitten,
  falls er zu lang ist. Dieses Abschneiden wird dem Benutzer durch das Anf�gen von \gq{...}
  an das Ende des Textes dargestellt.


\section{Visualisierung von Kanten}

Kanten sind immer gerade Linien von einem Start- zu einem Zielknoten. Kanten k�nnen sich hinsichtlich
ihrer Linienfarbe und der Art der Line (z.B. gestrichelt oder durchgezogen) unterscheiden.
Der IML-Graph kennt auch Schleifen, solche Kanten werden durch zwei Kantenknickpunkte
umgelenkt.

  \subsection{Kantenknickpunkte}
  Der Benutzer kann bei jeder Kante manuell beliebig viele Kantenknickpunkte hinzuf�gen.
  Die Kante selbst besteht dann aus mehreren geraden Linien vom Startknoten �ber beliebig
  viele Kantenknickpunkt hiwnweg zum Zielknoten.\\
  
  \begin{itemize}   
    \item Kantenknickpunkte werden als kleine Rauten dargestellt
    \item Die Farbe der Kantenknlickpunkte entspricht der Farbe f�r die Linie der Kante
  \end{itemize}

  Layoutalgorithmen, die diese Kantenknickpunkt automatisch erzeugen sind momentan nicht vorgesehen.


  \subsection{Vorhandene Kanten nicht anzeigen}



\section {Hervorheben von Knoten und Kanten}

Innerhalb eines Anzeigefensters k�nnen
\begin {itemize}
  \item die \gq{aktuelle Selektion} (kann nat�rlich auch lehr sein),
  \item bis zu drei weitere Selektionen,
  \item und bis zu vier IML-Teilgraphen
\end {itemize}
unterschiedbar hervorgehoben werden.


\subsection {Selektionen und aktuelle Selektionen}

Beim Hervorheben von Knoten und Kanten von Selektionen werden diese mittels einer 
definierbaren Farbe eingef�rbt.

  \begin {itemize} 
    \item Jeder Knoten einer Selektion wird dadurch hervorgehoben, dass das die F�llfarbe 
          und die Rahmenfarbe des Knotenvierecks auf die entsprechende Farbe gesetzt werden.

    \item Jede Kante wird dadurch hervorgehoben, dass ihre Linie und alle ihre Kanten



und jeder Kantenknickpunkt 


  \end {itemize}



Geh�rt ein Knoten oder eine Kante zu mehrern hervorgehobenen Selektionen, so wird immer die Farbe der
zuletzt zum Hervorheben gew�hlten Selektion genommen. Knoten und Kanten der aktuellen Selektion werden
immer mit der Farbe f�r aktuelle Selektionen hervorgehoben.
>>>>>>>>>>>>>>>>>>>>>CHECKEN OB SO MACHBAR UND SINNVOLL




Werden Knoten und Kanten von IML-Teilgraphen hervorgehoben 
 



Wie weiter unten beschrieben kann der Kunde �ber die Konfigurationsdateien einstellen mit welchen Farben
Selektionen und IML-Teilgraphen hervorgehoben werden sollen, und mit welchen Farben Knoten und Kanten
(je nach Klasse) generell eigef�rbt werden. Es macht nat�rlich keinen Sinn, f�r das Einf�rben von
Knoten und Kanten  hier die gleiche oder eine zu �hnliche Farbe zu w�hlen, wie f�r das hervorheben 
von Selektionen.





\section {Detailstufen beim Zoomen}
Je nach Zoomstufe erh�ht sich die Anzahl der im Anzeigefenster dargestellten Knoten und Kanten.
Beim Herauszoomen m�ssen also Details verringert werden, dies geschieht bei GIANT in mehreren
Stufen.\\
Eine exakte Spezifikation der Detailstufen, insbesondere ab welcher Zoomstufe welche 
Detailstufe gew�hlt wird, ist an dieser Stelle nicht sinnvoll. Dies wird bei
der Implementierung geschehen. Auf jeden Fall wird es aber die folgenden Detailstufen geben:

  \subsection {Volle Details}
  Falls sehr nah herangezoomt wurde.
  Das Knoten-Rechteck enth�lt alle 








\section{Minimap}


  Der Benutzer kann �ber die Konfigurationsdateien beliebig viele Visualisierungsstiele
erzeugen. Ein Visualisierungsstiel beschreibt wie Knoten und Kanten innerhalb eines
Anzeigefensters dargestellt werden (Icons, Farbe, Attribute ...). Der Benutzer
kann zu jeder Zeit in einem Anzeigefenster einene der zuvor definierten 
Visualisierungsstiele ausw�hlen
Laufzeit von GIANT kann der Benutzer innerhalb eines Anzeigefensters 