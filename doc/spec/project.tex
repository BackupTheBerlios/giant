% ==============================================================================
%  $RCSfile: project.tex,v $, $Revision: 1.6 $
%  $Date: 2003/02/13 12:10:00 $
%  $Author: schwiemn $
%
%  Description:
%
% ==============================================================================


\section {Persistenz �ber Projekte}

GIANT speichert persistente Informationen in sogenannten Projekten.
Ein Projekt besteht aus einem Verweis auf eine IML-Graph-Datei, auf die sich
die gespeicherten Informationen beziehen, sowie aus gespeicherten Informationen f�r
IML-Teilgraphen und Anzeigefenstern.\\
Jedes Projekt hat einen vom Benutzer definierbaren Namen, dieser Name entspricht dem Namen der 
Projektdatei und wird innerhalb des IML-Browsers angezeigt.
Der Name eines bereits angelegten Projektes kann nicht ge�ndert werden (au�er dadurch, dass
man dass Projekt neu speichert und das alte l�scht). 
Der Benutzer kann beliebig viele Projekte erzeugen.\\
W�hrend der Arbeit mit GIANT kann jederzeit ein Projekt geladen werden, wobei immer nur ein
Projekt gleichzeitig ge�ffnet sein darf.Vorraussetzung hierf�r ist alledings, dass dies 
von Reflektion zum IML-Graphen unterst�tzt wird.

  \subsection {Das Projektverzeichnis}
  In dem Projektverzeichnis liegt die Projektdatei, der Name des Projektverzeichnisses ist beliebig.
  Im Projektverzeichnis befindet sich auch die Dateie f�r jedes gespeicherte Anzeigefenster und jeden 
  gespeicherten IML-Teilgraphen.
  In jedem Projektverzeichnis darf nur ein Projekt gespeichert werden.

  \subsection {Die Projektdatei}
  Die Projektdatei befindet sich im Projektverzeichnis und enth�lt Informationen, 
  die zur Identifikation des zu einem Projekt geh�renden IML-Graphen n�tig sind. 
  Der Name der Projektdatei entspricht dem Namen des Projektes.

  \subsection {Der zugeh�rige IML-Graph}
  Der absolute oder realtive Pfad zu der Datei, die den IML-Graphen enth�lt, ist in der
  Projektdatei gespeichert. Alternativ dazu kann der Benutzer den Pfad zur IML-Graph Datei auch
  eingeben.
 
  \subsection {Die Identifikation der IML-Graph Datei}
  Die Reflektion muss f�r jede IML-Graph Datei eine m�glichst eindeutige Pr�fsumme berechnen k�nnen.
  Bei Laden eines Projektes wird �berpr�ft ob die in der Projektdatei gespeicherte Pr�fsumme
  der der zu ladenen IML-Graph Datei entspricht. 
  Das Verhalten von GIANT f�r den Fall, dass eine IML-Graph Datei geladen wird, 
  die zwar die passende Pr�fsumme hat, aber nicht den IML-Graphen enth�lt, 
  der dem Projekt zu Grunde liegt, ist undefiniert.

  \subsection {Verwaltungsdateien f�r Anzeigefenster}
  F�r jedes gespeicherte Anzeigefenster gibt es eine Verwaltungsdatei. 
  Diese Verwaltungsdatei enth�lt alle Informationen zur kompletten Rekonstruktion 
  eines Anzeigefensters. Gespeichert werden:
 
  \begin {itemize}
 
    \item Der komplette Anzeigeinahlt (alle visualisierten Knoten und Kanten mit Position).
    \item Alle Pins (gespeicherte sichtbare Anzeigeinhalte).
    \item Alle dem Anzeigefenster bekannten Selektion.
 
  \end {itemize}
 
  Die VISUALISIERUNGSSTIELE (f�r jedes Anzeigefenster kann ein VISUALISIERUNGSSTIEL ausgew�hlt werden)
  werden aber nicht gespeichert, da sie von der Konfiguartionsdatei vorgegeben werden.

  \subsection{Verwaltungsdateien f�r IML-Teilgraphen}
  S�mtliche erzeugten IML-Teilgraphen werden ebenfalls in Verwaltungsdateien gespeichert. 
  F�r jeden IML-Teilgraphen gibt es
 


 

    


  \subsection {Verwaltungsdateien f�r IML-Teilgraphen}
  S�mtliche IML-Teilgraphen 

  \subsection{Verwaltungsdatei f�r Anfragen}
  KANN er auch SELBER mittels Paste und Copy im Emacs machen.





\section {Verhalten von GIANT beim Speichern} 


\subsection{Verwaltungsdateien f�r Anzeigefenster}
In einem Unterverzeichnis des Projektverzeichnisses werden jeweils in einzelnen Dateien.

\subsection{Verwaltungsdateien f�r IML-Teilgraphen}
S�mtliche IML-Teilgraphen

subsection{Speichern von IML-Teilgraphen}


\subsection{Grundlegendes Verhalten beim Speichern}
