% ==============================================================================
%  $RCSfile: project.tex,v $, $Revision: 1.2 $
%  $Date: 2003/02/13 07:23:31 $
%  $Author: schwiemn $
%
%  Description:
%
% ==============================================================================

\section {Das Projekte}

GIANT speichert persistente Informationen in sogenannten Projekten. Der Benutzer kann beliebig viele 
Projekte erzeugen. W�hrend der Arbeit mit GIANT kann ein einzelnes Projekt.
(vorausgesetzt die Reflektion zum IML-Graphen em�glicht es IML-Graphen w�hrend des laufenden Betriebs
aus dem Speicher zu entfernen).


\subsection {Name des Projektes}
Jedes Projekt hat einen vom Benutzer definierbaren Namen, dieser Name entspricht dem Namen der 
Projektdatei und wird innerhalb des IML-Browsers angezeigt.
Es kann mehrere Projektverzeichnisse geben, wobei der Name des Projektverzeichnisses dem Namen 
der Projektes selbst entspricht.
Keine der weiter unten beschriebenen Dateien ist zwischen Projekten austauschbar, darf also nicht in
ein anderes Projektverzeichnis kopiert werden.


   \subsection{Die Projektdatei}
   Die Projektdatei 
   
   Die Projektdatei enth�lt Informationen, die zur Identifikation der zu einem 
   Projekt geh�renden 
  

   \subsection{Die Identifikation der IML-Graph Datei}


  \subsection{Verwaltungsdateien f�r Anzeigefenster}
  In einem Unterverzeichnis des Projektverzeichnisses werden jeweils in einzelnen Dateien 
   


  \subsection{Verwaltungsdateien f�r IML-Teilgraphen}
  S�mtliche IML-Teilgraphen 

  \subsection{Verwaltungsdatei f�r Anfragen}
  KANN er auch SELBER mittels Paste und Copy im Emacs machen.





\section {Verhalten 

  \subsection {Name des Projektes}
  Jedes Projekt kann mit einem Namen versehen werden, dieser ergibt


  \subsection{Grundlegendes Verhalten beim Speichern}

