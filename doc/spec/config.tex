% ==============================================================================
%  $RCSfile: config.tex,v $, $Revision: 1.1 $
%  $Date: 2003/02/13 03:15:14 $
%  $Author: schulzgt $
%
%  Description:
%
% ==============================================================================

\section{Men�-Makros}
Es k�nnen sogenannte Men�-Makros definiert werden. 
Jeder Makro erh�lt einen eindeutigen Namen und eine ``Anweisung''.

\section{Visualisierungsstiele}
Es k�nnen verschiedene Visualisierungsstiele angelegt werden. 
Sie beschreiben die Art und Weise wie Knoten und Kanten in den Anzeigefenstern 
dargestellt werden. F�r jedes Anzeigefenster kann hierbei ein eigener 
vordefinierter Visualisierungsstiel ausgew�hlt werden.\\
Folgende Einstellungen k�nnen hierbei vorgenommen werden:

\subsection{Global f�r jeden Visualisierungsstiel}
\begin{itemize}
  \item Die Hintergrundfarbe im Anzeigefenster.
  \item Die Position der Kantenbeschriftung (immer im Sichtbaren Anzeigeinhalt,
        pauschal in der Mitte der Kante, am Startknoten oder am Zielknoten).
\end{itemize}

\subsection{Knoten}
Folgende  Einstellungen k�nnen f�r jede Knotenklasse vorgenommen werden.
Des weiteren muss eine DEFAULT-Einstellung erstellt werden, die f�r alle
Knotenklassen angewendet wird, f�r die nichts anderes definiert ist.
\begin{itemize}  
  \item Ein Icon f�r die Knotenklasse (als Verweis auf ein entsprechendes Bild).  
  \item Eine Liste der Attribute der Knotenklasse, welche dirket innerhalb des
        Anzeigefensters dargestellt werden sollen.
  \item Die Farbe des ``Vierecks'' in welchem die Attribute zu de
  \item Ob die ID eines Knotens angezeigt werden soll oder nicht.
\end{itemize}

\subsection{Kanten}
Folgende Einstellungen k�nnen f�r jede Kantenklasse vorgenommen werden. 
Des weiteren muss eine DEFAULT-Einstellung erstellt werden, die f�r alle
Kantenklassen angewendet wird, f�r die nichts anderes definiert ist.
\begin{itemize}
  \item Die Farbe der Kante.
  \item Die Art der Linie der Kante (normal, gestrichelt).
  \item Ob die Kante mit ihrer Kantenklasse beschriftet werden soll oder nicht.
\end{itemize}
