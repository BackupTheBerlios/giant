% ==============================================================================
%  $RCSfile: config.tex,v $, $Revision: 1.5 $
%  $Date: 2003/02/17 15:50:58 $
%  $Author: schwiemn $
%
%  Description:
%
% ==============================================================================


\section {Allgemeines}

S�mtliche konfigurierbaren Einstellungen werden in XML-Dateien vorgenommen.
Es gibt genau eine \gq{globale Konfigurationsdatei}. Zudem kann es beliebig 
viele weitere XML-Dateien zur Konfiguration der weiter unten beschriebenen 
Visualisierungsstiele geben.

\section {Die globale Konfigurationsdatei}
Diese Datei ist in XML verfasst und existiert genau ein mal. In ihr werden 
die anschlie�end beschriebenen Einstellungen vorgenommen.

  \subsection {Verweise auf Men�-Makros}
  Es k�nnen beliebig viele Men�-Makros definiert werden.
  Jeder Makro wird durch einen eindeutigen Namen und durch einen Verweis
  auf die Datei, welche den Makro enth�lt, spezifiziert.

  \subsection {Farbe von Hervorhebungen}
  Einstellung der Farben mittels derer Selektionen oder IML-Teilgraphen 
  hervorgehoben werden.
  \begin {itemize}
    \item Eine beliebige Farbe f�r die aktuelle Selektion.
    \item Drei beliebige (sinnvoll w�ren verschiedene) Farben f�r das
          Hervorheben weiterer Selektionen.
    \item Vier beliebige (sinnvoll w�ren verschiedene) farben f�r das 
          Hervorheben von IML-Teilgraphen.
  \end {itemize}


\section {Visualisierungsstiele}
F�r jeden Visualisierungsstiel muss es eine entsprechende XML-Datei geben. 
Ein Visualisierungsstiel beschreibt wie die Knoten und Kanten des 
IML-Graphen innerhalb eines Anzeigefensters grafisch dargestellt 
werden k�nnen.\\
In einem Visualisierungsstiel k�nnen klassenspezifische Einstellungen 
vorgenommen werden,
die nur f�r Knoten und Kanten gelten, die zu den entsprechenden Klassen geh�ren.
Bei jeder klassenspezifischen Einstellung f�r Knoten und Kanten kann daher 
eine Liste der betroffenen
Knoten- und Kantenklassen angegeben werden, f�r die diese Einstellungen 
gelten sollen.\\
Wird eine Kanten- oder eine Knotenklasse innerhalb eines Visualisierungsstieles 
mehreren 
klassenspezifischen Einstellungen zugeordnet, f�hrt dies zu keinem Fehler, 
es bleibt aber
unspezifiziert, welche Einstellung tats�chlich genommen wird.

  \subsection {Name des Visualisierungsstiels}
  Jeder Visualisierungsstiel erh�lt einen Namen. Unter diesem Namen ist der 
  Visualisierungsstiel in der GUI (bei den Anzeigefenstern) ausw�hlbar.
  Der Name muss nicht eindeutig sein oder dem Namen der Datei, 
  in der der Visualisierungssteil gespeichert ist, entsprechen.

  \subsection {Einstellungen innerhalb des Visualisierungsstiels}
    \begin{itemize}
      \item Die Hintergrundfarbe im Anzeigefenster.
      \item Die Position der Kantenbeschriftung. 
        \subitem Immer im sichtbaren Anzeigeinhalt.
        \subitem Pauschal in der Mitte der Kante. 
        \subitem Am Startknoten.
        \subitem Am Zielknoten.
      \item Wann Kanten im sichtbaren Anzeigeinhalt angezeigt werden sollen.
      >>>>>>>>>>>>>>>>>>>>>>>      ACHTUNG - ES KANN AUCH SCHLEIFEN GEBEN
        \subitem Nur falls Start- und Zielknoten ebenfalls im sichtbaren 
                 Anzeigeinhalt.
        \subitem Auch anzeigen falls nur Startknoten im sichtbaren 
                 Anzeigeinhalt.
        \subitem Auch anzeigen falls nur Zielknoten im sichtbaren Anzeigeinhalt.
        \subitem Auch anzeigen falls weder Start- und Zielknoten im sichtbaren 
                 Anzeigeinhalt. 
    \end{itemize}


  \subsection {Klassenspezifische Einstellungen f�r Knoten}
  Folgende  Einstellungen k�nnen f�r Knotenklassen vorgenommen werden.
  Des weiteren muss eine DEFAULT-Einstellung erstellt werden, die f�r alle
  Knotenklassen angewendet wird, f�r die nichts anderes definiert ist.
  \begin{itemize}  
    \item Ein Icon f�r die Knotenklasse (Verweis auf eine entsprechende 
          Bilddatei). Das Icon muss im Pixmap Format bei 32*32 Pixel
          vorliegen.

    \item Eine Liste der Attribute der Knotenklasse, welche dirket innerhalb 
          des Anzeigefensters in dem \gq(Viereck) f�r den Knoten dargestellt 
          werden sollen.
    \item Die F�llfarbe des \gq{Vierecks} in welchem die Attribute zu dem 
          Knoten dargestellt werden.
    \item Die Farbe des Rahmens f�r das \gq{Viereck} in welchem die 
          Attribute zu dem Knoten dargestellt werden.

  \end{itemize}

  
  \subsection {Klassenspezifische Einstellungen f�r Kanten}
  Folgende Einstellungen k�nnen f�r Kantenklassen vorgenommen werden. 
  Des weiteren muss eine DEFAULT-Einstellung erstellt werden, die f�r alle
  Kantenklassen angewendet wird, f�r die nichts anderes definiert ist.
  \begin{itemize}
    \item Die Farbe der Kante.
    \item Die Art der Linie der Kante (normal, gestrichelt).
    \item Ob die Kante mit ihrer Kantenklasse beschriftet werden soll oder 
          nicht.
  \end{itemize}



\section {Anfrage-Dateien}
Eine Textdatei, die genau eine Anfrage enth�lt - also genau ein gem�� der 
Grammitk f�r die Anfragesprache zul�ssiges Wort. 
Auf diese Art und Weise k�nnen komplexe Anfragen gespeichert und 
wiederverwendet werden.\\
Erstellt werden k�nnen solche Anfrage-Dateien entweder manuell oder
automatisch aus einer im ENTSPRECHENDEN DIALOG eingegebenen Anfrage.\\
Geldaden werden k�nnen solche Dateien beim Kommandozeilenaufruf
oder im entsprechenden Dialog zur Eingabe einer Anfrage.
