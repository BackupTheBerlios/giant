% ==============================================================================
%  $RCSfile: config.tex,v $, $Revision: 1.13 $
%  $Date: 2003/03/27 07:14:46 $
%  $Author: stupro $
%
%  Description:
%
% ==============================================================================


\section {Allgemeines}
\begin {enumerate}

  \item
  S�mtliche konfigurierbaren Einstellungen werden in XML-Dateien vorgenommen.
  
  \item \label{config-allgemeines-punkt2}
  Es gibt genau eine \gq{globale Konfigurationsdatei}. Zudem kann es beliebig 
  viele weitere XML-Dateien zur Konfiguration der weiter unten beschriebenen 
  Visualisierungsstile geben.\\
  Diese Dateien liegen in einem von GIANT fest vorgegebenen Verzeichnis.
  
  \item
  Die unter \ref{config-allgemeines-punkt2} beschriebenen Dateien zu
  Konfiguration k�nnen vom Benutzer auch in einem separaten 
  Unterverzeichnis seines Home-Verzeichnisses abgelegt werden. Findet
  GIANT beim Start ein derartiges Verzeichnis, so werden die
  Einstellungen aus der dort vorgefundenen Konfigurationsdatei, sowie
  die dort vorgefundenen Visualisierungstile geladen.
 
\end {enumerate}


\section {Die globale Konfigurationsdatei}
Diese Datei ist in XML verfasst. In Ihr k�nnen die anschlie�end beschriebenen
Einstellungen vorgenommen werden.

%EVENTUELL F�R JEDEN BENUTZER EIGENE KONFIG-DATEI VORSEHEN.

  \subsection {Verweise auf GQSL-Skript-Makros}
  Es k�nnen GQSL-Skript-Makros definiert werden. Jeder Makro wird durch 
  einen eindeutigen Namen und durch einen Verweis
  auf die Datei, welche das GQSL-Skript enth�lt, spezifiziert.
  Der Makro kann dann aus einem PopUp Men� heraus aufgerufen werden, wobei
  dann das in der Datei stehende GQSL Skript ausgef�hrt wird.

  \subsection {Farbe von Hervorhebungen}
  Einstellung der Farben, mittels derer Selektionen oder IML-Teilgraphen 
  hervorgehoben werden.
  \begin {enumerate}
    \item Eine beliebige Farbe f�r die aktuelle Selektion.
    \item Beliebige Farben f�r das
          Hervorheben weiterer Selektionen.
    \item Beliebige Farben f�r das 
          Hervorheben von IML-Teilgraphen.
  \end {enumerate}


\section {Visualisierungsstile}
\begin {enumerate}

  \item
  F�r jeden Visualisierungsstil muss es eine entsprechende XML-Datei geben. 

  \item
  Ein Visualisierungsstil beschreibt, wie die Knoten und Kanten des 
  IML-Graphen innerhalb eines Anzeigefensters graphisch dargestellt 
  werden k�nnen.

  \item
  In einem Visualisierungsstil k�nnen klassenspezifische Einstellungen 
  vorgenommen werden, die nur f�r Knoten und Kanten gelten, 
  die zu den entsprechenden Klassen geh�ren.
  Bei jeder klassenspezifischen Einstellung f�r Knoten und Kanten kann daher 
  eine Liste der betroffenen Knoten- und Kantenklassen angegeben werden, 
  f�r die diese Einstellungen gelten sollen.

  \item
  Wird eine Kanten- oder eine Knotenklasse innerhalb eines
  Visualisierungsstieles mehreren klassenspezifischen Einstellungen zugeordnet, 
  f�hrt dies zu keinem Fehler, es bleibt aber unspezifiziert, 
  welche Einstellung tats�chlich genommen wird.

  \item
  Es gibt immer einen Standard-Visualisierungsstil (von GIANT fest
  vorgegeben).

\end {enumerate}


  \subsection {Name des Visualisierungsstils}
  Jeder Visualisierungsstiel erh�lt einen Namen. Unter diesem Namen ist der 
  Visualisierungsstiel f�r jedes Anzeigefenster einzeln ausw�hlbar.
 
  \subsection {Einstellungen innerhalb des Visualisierungsstils}
    \begin{enumerate}
      \item Die Hintergrundfarbe im Anzeigefenster.

     % \item Wann Kanten im sichtbaren Anzeigeinhalt angezeigt werden sollen.
 
     %  \subitem Nur falls Start- und Zielknoten ebenfalls im sichtbaren 
     %           Anzeigeinhalt.
     %  \subitem Auch anzeigen falls nur Startknoten im sichtbaren 
     %           Anzeigeinhalt.
     %  \subitem Auch anzeigen falls nur Zielknoten im sichtbaren Anzeigeinhalt.
     %  \subitem Auch anzeigen falls weder Start- und Zielknoten im sichtbaren 
     %           Anzeigeinhalt. 
    \end{enumerate}


  \subsection {Klassenspezifische Einstellungen f�r Knoten}
  Folgende  Einstellungen k�nnen f�r Knotenklassen vorgenommen werden. 
  Es muss eine Standard-Einstellung erstellt werden, die f�r alle
  Knotenklassen angewendet wird, f�r die nichts anderes definiert ist.
  \begin{enumerate}  
  
    \item Ein Icon f�r die Knotenklasse (Verweis auf eine entsprechende 
          Bilddatei). Das Icon muss im Pixmap Format bei 16*16 Pixel
          vorliegen.
    \item Eine Liste der Attribute der Knotenklasse, welche dirket innerhalb 
          des Anzeigefensters in dem \gq(Rechteck) f�r den Knoten dargestellt 
          werden sollen.
    \item Die F�llfarbe des \gq{Rechtecks} in welchem die Attribute zu dem 
          Knoten dargestellt werden.
    \item Die Rahmenfarbe des \gq{Rechtecks}.
    
  \end{enumerate}

  
  \subsection {Klassenspezifische Einstellungen f�r Kanten}
  Folgende Einstellungen k�nnen f�r Kantenklassen vorgenommen werden.
  Es muss eine Standard-Einstellung erstellt werden, die f�r alle
  Kantenklassen angewendet wird, f�r die nichts anderes definiert ist.
  \begin{enumerate}
    \item Die Farbe der Kante.
    \item Die Art der Linie der Kante (normal, gestrichelt).
    \item Ob die Kante mit ihrer Kantenklasse beschriftet werden soll oder 
          nicht.
  \end{enumerate}


\section {Anfrage-Dateien}
Eine Textdatei, die genau eine Anfrage in Form eines GQSL-Skriptes enth�lt 
-- also genau ein gem�� der Grammatik f�r die Anfragesprache zul�ssiges Wort. 
Auf diese Art und Weise k�nnen komplexe Anfragen gespeichert und 
wiederverwendet werden.\\
Erstellt werden k�nnen solche Anfrage-Dateien entweder manuell oder
automatisch aus einer eingegebenen Anfrage.
Geldaden werden k�nnen solche Dateien beim Kommandozeilenaufruf
oder im entsprechenden Dialog zur Eingabe einer Anfrage.
