% ==============================================================================
%  $RCSfile: gui.tex,v $, $Revision: 1.2 $
%  $Date: 2003/02/16 13:25:39 $
%  $Author: birdy $
%
%  Description:
%
% ==============================================================================

\section{�ber die Benutzeroberfl�che}

GIANT besitzt eine graphische Benutzeroberfl�che, die per Maus zu
bedienen ist. Ausgew�hlte Funktionen k�nnen auch per Tastatur ausgel�st
werden.

\section{Beschreibung der GUI}    

Diese Beschreibung beschreibt alle Elemente der GUI. Sofern nicht anders
angegeben, sind hintereinander angegebene Elemente (z.B. Listeneintr�ge)
immer von links nach rechts oder von oben nach unten beschrieben.
Sofern nicht anders angegeben, sind alle Ausgaben immer linksb�ndig
formatiert.

\subsection{Das Hauptmen�}   
Kill-Button, der automatisch alle 

\subsection{Popup-Men�s}

\subsection{Hauptfenster}

\subsection{Anzeigefenster}
Zeigt zu jedem Knoten die Klasse, die ID und alle Attribute an.
Es k�nnen auch alle Knoten, die auf diesen Knoten verweisen angeziegt 
werden (DIES SOLLTE EVETUELL AUCH IN DIE ANFRAGESPRACHE). 

\subsection{Knoten-Informationsfenster}

In Knoten-Informationsfenstern (NODE\_INFO) k�nnen n�here Informationen
zu Knoten angezeigt werden.
Es kann beliebig viele Knoten-Informationsfenster geben, jedoch nur
eines pro Knoten.

Zuoberst im Fenster wird die Knotentyp und Knoten-ID dargestellt
(TYPE und ID), Format: "Knoten: TYPE, ID: ID) darunter befinden sich
(jeweils untereinander) drei getrennt scrollbare Listen:

Die Liste ATTRIBUTES enth�lt alle Attribute des Knotens, hat zwei Spalten
ATTRIBUTE und VALUE, in denen f�r jedes Attribut der Attributname und
der Wert des Attributes angezeigt werden.

Die Liste SUCCESSOR\_EDGES hat eine Zeile f�r jede vom Knoten abgehende
Kante mit drei Spalten: EDGE\_TYPE, NODE\_TYPE, NODE\_ID, in denen
der Typ der abgehenden Kante, der Typ des Knotens, zu dem sie f�hrt und
die ID des Knotens, zu dem sie f�hrt, dargestellt werden.

Die Liste PREDECESSOR\_NODES hat eine Zeile f�r jede am Knoten ankommende
Kante mit drei Spalten: EDGE\_TYPE, NODE\_TYPE, NODE\_ID, in denen
der Typ der ankommenden Kante, der Typ des Knotens, von dem sie kommt und
die ID des Knotens, von dem sie kommt, dargestellt werden.

Unter den drei Listen befinden sich nebeneinander mittig zwei Buttons
CLOSE und PICK.
Der CLOSE-Button schliesst durch Mausklick das Informationsfenster,
das Informationsfenster kann auch durch Druecken von ** geschlossen
werden.
Mit dem PICK-Button kann man einen anderen Knoten ausw�hlen, dessen
Information das Knoten-Informationsfenster nun anzeigen soll.
Durch Klick auf den PICK-Button verwandelt sich der Mauscursor,
wenn m�glich, in ein Fadenkreuz, es kann dann durch Klick in eines
der Graphenfenster auf deinen Knoten dieser Knoten f�r die Anzeige
ausgew�hlt werden.

\subsection{Fenster f�r Minimap} %Falls in extra Fenster

\subsection{Anfrage Dialog}

Der Anfragedialog hat zuoberst ein Textfeld QUERY\_TEXT,
in dem der Text der Anfrage eingegeben werden kann.
Durch Klicken auf die darunterliegenden Listen kann
Text in QUERY\_TEXT eingef�gt werden.

Der unter QUERY\_TEXT liegende Teil des Fensters
ist im Wesentlichen dreigeteilt.
Er besitzt vier nebeneinanderliegende scrollbare Listen,
NODE\_CLASS\_LIST , EDGE\_CLASS\_LIST , ATTRIB\_LIST,
SUBGRAPH\_LIST
genannt Knotenklassenliste, Kantenklassenliste,
Attributliste und Teilgraphenliste.

�ber den ersten drei Listen von links nach rechts befinden
sich drei Eingabefelder, genannt NODE\_WILDCARD , EDGE\_WILDCARD,
ATTRIB\_WILDCARD. Neben diesen Feldern befindet sich jeweils ein
Button
Show.

Ueber der Kantenklassen-Liste und der Knotenklassenliste befindet
sich zus�tzlich je ein Eingabefeld  nur Knotentyp (EDGE\_NODETYPE
und ATTRIB\_NODETYPE), sowie je ein danebenliegender Button 
Aus Knotenklassenliste uebernehmen (EDGE\_USE\_NODETYPE
und ATTRIB\_USE\_NODETYPE).

Unter den Listen befinden sich Buttons Anfrage Starten
(QUERY\_START), Abbruch (QUERY\_CANCEL), Anfrage laden
(QUERY\_LOAD), Anfrage speichern (QUERY\_SAVE),
eine Combobox beschriftet mit letzte 5 Anfragen (QUERY\_LATEST)
und neben dieser ein Button Laden (LOAD\_QUERY\_LATEST).

Der Nutzer kann nun in der Knotenklassenliste
links eine Knotenklasse durch einfachklicks
auswaehlen, die Darstellung kann durch einen
Eintrag im NODE\_WILDCARD-Feld auf zur Wildcard
passende Knotenklassennamen eingeschr�nkt werden.
Die Liste wird nach Klicken von Show anhand der
Wildcards neu aufgebaut.

In der Kantenklassenliste werden alle Kantenklassen
dargestellt, die Liste kann durch das EDGE\_WILDCARD
Feld anhand der Kantennamen eingeschraenkt werden.
Die Liste wird nach Klicken von Show anhand der
Wildcards neu aufgebaut.
In EDGE\_NODETYPE kann festgelegt werden, dass nur
Kanten, die als vom eingegebenen Knotentyp
abgehend existieren, angezeigt werden, hier sind
keine Wildcards m�glich.
Durch einfaches Klicken auf den Button Aus Knotenklassenliste
uebernehmen kann der in der linken Liste
ausgew�hlte Knotenname in das Feld EDGE\_NODETYPE
uebernommen werden.

In der Attributliste werden alle Attribute
dargestellt, die Liste kann durch das ATTRIB\_WILDCARD
Feld anhand der Attributsnamen eingeschraenkt werden.
Die Liste wird nach Klicken von Show anhand der
Wildcards neu aufgebaut.
In ATTRIB\_NODETYPE kann festgelegt werden, dass nur
Attribute, die als vom eingegebenen Knotentyp
abgehend existieren, angezeigt werden, hier sind
keine Wildcards m�glich.
Durch einfaches Klicken auf den Button Aus Knotenklassenliste
uebernehmen kann der in der linken Liste
ausgew�hlte Knotenname in das Feld ATTRIB\_NODETYPE
uebernommen werden.

In der Teilgraphenliste befinden sich alle z.Z. definierten
Teilgraphen, sowie zuoberst ein Eintrag IML-Graph,
welcher f�r den gesamten IML-Graphen steht.

Wenn auf Listenelemente doppelt geklickt wird, wird
der Name des betreffenden Listenelementes
am Ende des Textes im Anfragefenster angeh�ngt.

Die Anfrage wird durch Eintippen des Textes per
Tastatur in QUERY\_TEXT formuliert, wobei
die drei Listen als Eingabehilfe verwendet
werden koennen.
Die Anfrage wird durch QUERY\_START gestartet,
durch QUERY\_CANCEL wird das Fenster nach
einer positiven Sicherheitsabfrage Wollen Sie wirklich
das Anfragefenster ohne Speichern schliessen?
geschlossen.
Durch QUERY\_LOAD erhaelt man ein GTK-Dateiauswahlfenster
zum Laden einer gespeicherten Anfrage, beim Klick
auf QUERY\_SAVE geht ein GTK-Dateiauswahlfenster
auf, in dem der Name der zu speichernden Anfrage
festgelegt werden kann.
Mittels QUERY\_LATEST und LOAD\_QUERY\_LATEST kann
ggf. schneller auf die letzten 5 Anfragen
zugegriffen werden.
Vor jedem Laden einer Anfrage wird eine
Sicherheitsabfrage Alte Anfrage nicht gespeichert - Wollen
Sie wirklich eine neue Anfrage laden? gestellt,
wenn QUERY\_TEXT nicht leer ist.

\subsection{Darstellung von Attributen}
Bietet �bersicht �ber alle Knotenklassen (mit zugeh�rigen Attributen), alle Attribute und
Kantenklassen. Auswahlm�glichkeit.

\subsection{Dialog f�r Layoutalgorithmen}

\subsection{Dialog f�r Mengenoperationen}

\subsection{Allgemeiner Texteingabedialog}
F�r Namen (Pins, Selektionen, Anzeigfenster ...)

\section{Dateneingabe}
Beschreibung von verschiedenen Textfeldtypen etc. zur Eingabe von Daten.

\section{Ausgabe von Fehlermeldungen}
\subsection{Allgemeiner Fehlerdialog}
