% ==============================================================================
%  $RCSfile: gui.tex,v $, $Revision: 1.1 $
%  $Date: 2003/02/13 03:15:14 $
%  $Author: schulzgt $
%
%  Description:
%
% ==============================================================================

\section{�ber die Benutzeroberfl�che}

\section{Beschreibung der GUI}    

\subsection{Das Hauptmen�}   
Kill-Button, der automatisch alle 

\subsection{Popup-Men�s}

\subsection{Hauptfenster}

\subsection{Anzeigefenster}
Zeigt zu jedem Knoten die Klasse, die ID und alle Attribute an.
Es k�nnen auch alle Knoten, die auf diesen Knoten verweisen angeziegt 
werden (DIES SOLLTE EVETUELL AUCH IN DIE ANFRAGESPRACHE). 

\subsection{Knoten-Informationsfenster}

\subsection{Fenster f�r Minimap} %Falls in extra Fenster

\subsection{Anfrage Dialog}

\subsection{Darstellung von Attributen}
Bietet �bersicht �ber alle Knotenklassen (mit zugeh�rigen Attributen), alle Attribute und
Kantenklassen. Auswahlm�glichkeit.

\subsection{Dialog f�r Layoutalgorithmen}

\subsection{Dialog f�r Mengenoperationen}

\subsection{Allgemeiner Texteingabedialog}
F�r Namen (Pins, Selektionen, Anzeigfenster ...)

\section{Dateneingabe}
Beschreibung von verschiedenen Textfeldtypen etc. zur Eingabe von Daten.

\section{Ausgabe von Fehlermeldungen}
\subsection{Allgemeiner Fehlerdialog}
