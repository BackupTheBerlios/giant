% ==============================================================================
%  $RCSfile: intro.tex,v $, $Revision: 1.6 $
%  $Date: 2003/03/28 01:31:37 $
%  $Author: schwiemn $
%
%  Description:
%
% ==============================================================================


%========
\section{�ber dieses Dokument}

Diese Spezifikation beschreibt die funktionalen und die nicht-funktionalen 
Anforderungen an den IML-Browser GIANT, sowie die Rahmenbedingungen, unter 
denen GIANT lauff�hig sein muss. \\
Dieses Dokument ist Grundlage f�r die Entwicklung aller weiteren Dokumente 
innerhalb dieses Projektes, das hei�t im besonderen f�r das Benutzerhandbuch, 
den Entwurf, die Implementierung und den Test des Softwaresystems. 
Das Dokument richtet sich prim�r an die Mitglieder der Projektgruppe 
sowie an den Kunden und dessen technische Berater. \\
Desweiteren ist dieses Dokument Vetragsbestandteil f�r die weitere 
Entwicklung von GIANT und somit auch Grundlage f�r die Abnahme des fertigen 
Produktes.


%========
\section{�ber GIANT  Graphic IML Analysis Navigation Tool}
\index{GIANT}

GIANT ist ein Werkzeug, welches an der Universit�t Stuttgart von der 
CodeFabrik@Stuttgart im Rahmen des Studienprojektes A des Studiengangs 
Softwaretechnik entwickelt wird. \\
Ziel dieser Entwicklung ist es, den bereits bestehenden HTML-basierten Browser 
der Bauhaus Reengineering GmbH f�r IML-Graphen durch ein komfortables 
graphisches Werkzeug zu ersetzen.\\
GIANT soll es dem Kunden erm�glichen, geeignete Teile von gro�en IML-Graphen
grafisch zu visualisieren. Durch die Unterscheidung von verschiedenen 
Kanten- und Knotenklassen des IML-Graphen sollen dar�ber hinaus auch 
im IML-Graphen 
enthaltene Analyseergebnisse �bersichtlich dargestellt werden k�nnen.\\
Das Produkt soll sich durch einen hohen Grad an Wartbarkeit auszeichnen. 
Dies soll insbesondere dadurch erreicht werden, dass die Anbindung an die 
Bauhaus-IML-Graph-Bibliothek derart vorgenommen wird,
dass �nderungen in der Spezifikation des IML-Graphen, wie z.B. 
das Einbringen neuer Attribute, m�glichst keine Wartungsarbeiten an GIANT 
selbst nach sich ziehen.
