% ==============================================================================
%  $RCSfile: nfa.tex,v $ 
%  $Date: 2003/02/13 04:23:25 $
%  $Author: schulzgt $ 
%
%  Description: Nichtfunktionale Anforderungen
%
% ==============================================================================

\section{Plattformunabh�ngigkeit}
Das Produkt soll ein hohes Ma� an Plattformunabh�ngigkeit, wie sie sich aus der eingesetzten Entwicklungsumgebung ergibt, 
verf�gen (also alle Systeme auf denen GTK/ADA 1.2.12 lauff�hig ist). Explizit w�hrend der Entwicklung getestet 
werden kann dies aber nur f�r Sun Solaris und Linux.

\section{Wartbarkeit}
Das System soll sich durch ein hohes Ma� an Wartbarkeit und Erweiterbarkeit auszeichnen.
12.3.1  Wartbarkeit und Erweiterbarkeit gehen �ber Performance
Entsprechende Kapselung und Strukturierung des Entwurfes (Informationhiding, hohe Lokalit�t, 
strikte Trennung von GUI und Funktionalit�t etc.).

\section{Robustheit gegen�ber �nderungen der IML-Graph-Spezifikation}
Das System soll so an Bauhaus angebunden werden, dass �nderungen in der Spezifikation des IML-Graphen 
m�glichst keine Wartungsarbeiten erfordern. Besonders soll dies f�r die verschiedenen Attribute, 
Knoten- und Kantenklassen des Bauhaus-IML-Graphen gelten, da diese sich oft �ndern k�nnen.
Die Klassen und Attribute der Bauhaus-IML-Graph Bibliothek werden durch den IML-Browser nur so unterschieden, 
dass neu hinzukommende Attribute automatisch mit erfasst und damit angezeigt werden k�nnen.

\section{Mengenger�st}
Ein explizites Mengenger�st, z.B. hinsichtlich der minimalen und maximalen Anzahl von darstellbaren Knoten, 
gibt es nicht. Gr��enbeschr�nkungen (z.B. Gr��e der zu ladenden IML-Graph-Datei, Anzahl Knoten in einem Anzeigefenster ...) 
sollen sich alleine aus dem verf�gbaren Arbeitsspeicher ergeben. Falls der verf�gbare Arbeitsspeicher ersch�pft ist, 
soll das System nicht abst�rzen, sondern vielmehr die entsprechende Aktion kontrolliert abbrechen und dem Benutzer 
ein Weiterarbeiten erm�glichen.

\section{Unterst�tzung paralleler Rechner}
Layoutalgorithmen werden zur Unterst�tzung paralleler Rechner in separaten Threads ausgef�hrt. 
Die einzelnen Layoutalgorithmen sind nicht parallel. In einem Anzeigefenster kann immer nur ein es k�nnen 
aber mehrere Layoutalgorithmen in verschiedenen Anzeigefenstern gleichzeitig laufen.

\section{Leistungsanforderungen}
Ein-Benutzer System.

\section{Antwortverhalten}
Layoutalgorithmen k�nnen abgebrochen werden (darauf kann falls zu schwer verzichtet werden).

\section{Sicherheit}
Eventuell Sicherheitsabfrage zur Datensicherung bei Beenden des Programms!

\section{Robustheit}
Fehlermeldungen, keine Sicherheit bei defektem IML-Graphen  

\section{Wartbarkeit}

\section{Portabilit�t}

\section{Erweiterbarkeit}
