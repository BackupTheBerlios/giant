% =============================================================================
%  $RCSfile: language.tex,v $, $Revision: 1.9 $
%  $Date: 2003/04/22 01:08:35 $
%  $Author: keulsn $
%
%  Description:
%
%  Last-Ispelled-Revision: 1.1
%
% =============================================================================

\section{GQSL}

In der Version 1.0 dieses Dokuments wurde die Sprache GIANT Query and
Skripting Language (GQSL) spezifiziert. Diese Sprache bot die selbe
Funktionalit�t wie die hier beschriebene GSL. In dem Kundenreview am
14. und 15. April 2003
lehnte der Kunde GQSL jedoch ab, weil GQSL f�r Erweiterungen des Kunden einen
zu hohen Anpassungsaufwand erfordere.

Deshalb wird stattdessen die GIANT Scripting Language (GSL) durch die \company
nach den pr�zisierten W�nschen des Kunden entwickelt.
GSL wird die selbe Funktionalit�t bieten wie ihr Vorg�nger GQSL, wird jedoch
auf einer vom Kunden angeregten, deutlich vereinfachten Syntax basieren.

Das Kapitel �ber GQSL wurde aus dieser Spezifikation entfernt. GSL wird in
einem externen Dokument spezifiziert.

\section{GSL Beschreibung}
\index{GSL!Beschreibung}

GSL Scripts werden von einem GSL Interpreter ausgef�hrt. Der GSL Interpreter
ist Bestandteil von GIANT.

Die GSL dient dazu, IML-Teilgraphen und
Selektionen aus einem IML-Graph anzufragen und auf diesen Aktionen
auszuf�hren. Aktionen sind:
\begin{enumerate}
\item Erzeugen eines neuen Anzeigefensters
\item Hinzuf�gen der Graph-Knoten und Graph-Kanten eines IML-Teilgraphen
in ein ge�ffnetes Anzeigefenster mit konfigurierbarem Layout der neuen
Graph-Knoten.
\item Entfernen bestimmter Fenster-Knoten und Fenster-Kanten aus einem
Anzeigefenster
\item Erzeugen einer neuen Selektion in einem Anzeigefenster
\item �ndern des Inhalts einer Selektion in einem Anzeigefenster
\item L�schen einer Selektion in einem Anzeigefenster
\item Erzeugen eines neuen IML-Teilgraphen
\item �ndern des Inhalts eines IML-Teilgraphen
\item L�schen eines IML-Teilgraphen
\end{enumerate}

Selektionen und IML-Teilgraphen sind in GSL in bestimmten Variablen direkt
sichtbar. Durch Inspektionen an diese Variablen kann der Inhalt einer Selektion
oder eines IML-Teilgraphen ermittelt werden. Durch eine Zuweisung an eine
solche Variable kann der Inhalt ver�ndert werden.

In GSL k�nnen Vereinigung, Schnitt und Differenz von IML-Knoten- und
IML-Kantenmengen dargestellt werden. Au�erdem k�nnen durch Anfragen an die
Reflektion geziehlt bestimmte IML-Knoten oder IML-Kanten ausgew�hlt und
als neue Menge in Variablen zwischengespeichert werden. Hierbei
wird eine bestimmte Traversierungsstrategie f�r den IML-Graph vorgegeben.
Ob ein dabei erreichter IML-Knoten oder eine IML-Kante in das Ergebnis
aufgenommen wird, kann durch ein benutzerkonfiguriertes Pr�dikat entschieden
werden.

Ein solches Pr�dikat entscheidet f�r jeden IML-Knoten und jede IML-Kante ob
er/sie gew�hlt werden soll oder nicht. Daf�r stehen dem Benutzer die
M�glichkeiten zur Verf�gung:
\begin{enumerate}
\item Inspektion der Attribute eines IML-Knoten, Vergleich der ermittelten
Werte mit GSL-Variablen oder Konstanten oder Test ob der Wert in der
Sprache eines regul�ren Ausdrucks enthalten ist
\item Abfrage der Knotenklasse eines IML-Knoten, Test ob diese Knotenklasse
in einer bestimmten Klassenmenge enthalten ist
\item Abfrage der Kantenklasse einer IML-Kante, Test ob diese Kantenklasse in
einer bestimmten Klassenmenge enthalten ist
\end{enumerate}
