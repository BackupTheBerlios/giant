% ==============================================================================
%  $RCSfile: konzepteundbegriffe.tex,v $, $Revision: 1.1 $
%  $Date: 2003/06/30 15:22:06 $
%  $Author: birdy $
%
%  Description:
%
%  Last-Ispelled-Revision:
%
% ==============================================================================


\section{�ber dieses Kapitel}

Dieses Kapitel f�hrt die f�r das Verst�ndnis dieses Handbuches und die
Benutzung von GIANT wichtigen Begriffe ein.

\section{Begriffe}

\begin{enumerate}

  \item{Anfrage (Query):}{ Eine Anfrage beschreibt einen Vorgang, bei dem
    �ber geeignete Kriterien IML-Konten und IML-Kanten aus dem IML-Graphen
    oder aus IML-Teilgraphen ausgew�hlt werden.}
  
  \item{Anzeigefenster (Visualization Window):}{ Ein Fenster in dem ein
    Teilgraph des IML-Graphen nach bestimmten Kriterien visualisiert
    ist. Jedem Anzeigefenster ist ein Anzeigeinhalt zugeordnet.}
  
  \item{Anzeigeinhalt (Window Content):}{ Eine \gq{virtuelle} Oberfl�che
    auf der die Objekte des visualisierten Teilgraphen (also
    Fenster-Knoten und Fenster-Kanten) angeordnet sind, d.h. r�umliche
    Layoutinformation zu allen Objekten des entsprechenden
    Anzeigefensters. Abh�ngig von der Zoomstufe ist jeweils nur ein
    bestimmter Teil des Anzeigeinhaltes sichtbar - der sichtbare
    Anzeigeinhalt.  Die Gr��e des Anzeigeinhaltes ist theoretisch
    unbegrenzt.}
  
  \item{Fenster-Kante (Window Edge): }{Die graphische Repr�sentation einer
    IML-Kante innerhalb eines Anzeigefensters.}
  
  \item{Fenster-Knoten (Window Node): }{Die graphische Repr�sentation
    eines IML-Knoten innerhalb eines Anzeigefensters.}
  
  \item{Graph-Kante (Graph Edge): }{Eine Kante des IML-Graphen welche
    Bestandteil eines IML-Teilgraphen ist.}
  
  \item{Graph-Knoten (Graph Node): }{Ein Knoten des IML-Graphen welcher
    Bestandteil eines IML-Teilgraphen ist.}
  
  \item{IML-Graph (IML Graph): }{Der IML-Graph, wie er von der Bauhaus
    Reengineering GmbH gestellt wird.  Auf diesen Graphen wird �ber
    das sogenannte Reflection Model zugegriffen.}
  
  \item{IML-Kante (IML Edge): }{Eine Kante des IML-Graphen.}
  
  \item{IML-Knoten (IML Node): }{Ein Knoten des IML-Graphen.}
  
  \item{IML-Teilgraph (IML Subgraph): }{Eine Menge �ber Knoten und Kanten
    des IML-Graphen, die so gestaltet ist, dass sie einen Teilgraphen
    des IML-Graphen darstellt.}
  
  \item{Kantenklasse (Edge Class): }{Die Einteilung der IML-Kanten des
    IML-Graphen in verschiedene Klassen, wie sie sich aus der
    IML-Graph-Bibliothek von Bauhaus ergibt.\\
    Innerhalb dieser Spezifikation ist der Begriff Kantenklasse so
    zu verstehen, dass jede IML-Kante eindeutig zu genau einer Kantenklasse
    geh�rt, eine Vererbungshierarchie existiert nicht. Die Zuordnung
    einer IML-Kante zu einer Kantenklasse wird durch die Knotenklasse
    des Start-Knotens und den Namen des Attributes
    (aus dem Bauhaus-IML-Graphen), welches die IML-Kante
    beschreibt, festgelegt. Jede vorkommende Kombination aus der
    Knoten-Klasse eines Start-Knotens und dem Namen eines Attributes,
    welches eine Kante beschreibt, ist somit eine eigene Kantenklasse}

  \item{Klassenmenge (Class Set): }{Eine durch die IML-Graph Bibliothek
    vorgegebene Zusammenfassung von Kantenklassen und Knotenklassen.
    Es kann mehrere Klassenmengen geben. Die selben Knotenklassen und 
    Kantenklassen k�nnen gleichzeitig zu mehreren Klassenmengen geh�ren.}

  \item{Knoten-Annotationen (Node Annotation): }{Eine textuelle
    Beschreibung zu einem bestimmten Knoten des IML-Graphen.}
  
  \item{Knotenklasse (Node Class): }{Die Einteilung der IML-Knoten des
    IML-Graphen in verschiedene Klassen, wie sie sich aus der
    IML-Graph-Bibliothek von Bauhaus ergibt.\\
    Im Sinne der Verwendung dieses Begriffes innerhalb dieser Spezifikation
    liegt den Knotenklassen keine Vererbungshierarchie zu Grunde, jeder
    IML-Knoten geh�rt also eindeutig zu genau einer Knotenklasse}
  
  \item{Layout (Layout): }{Die zweidimensionale r�umliche Anordnung von
    Fenster-Knoten und Fenster-Kanten innerhalb eines Anzeigefensters
    auf dem sogenannten Anzeigeinhalt.}
 
  \item{Reflektion (Reflection Model): }{Die Schnittstelle zum Zugriff
    auf den IML-Graphen.}
  
  \item{Schleife (Loop): }{Eine Kante mit identischem Start- und Zielknoten.
    Wird oft auch als Selbstkante bezeichnet.}
  
  \item{Selektion (Selection): }{Eine Auswahl von Fenster-Knoten und Fenster-Kanten 
    eines visualisierten Teilgraphen des IML-Graphen innerhalb eines
    Anzeigefensters.}
  
  \item{Sichtbarer Anzeigeinhalt (Visible Window Content): }{ Der Bereich
    des Anzeigeinhaltes eines Anzeigefensters, der zur Zeit sichtbar
    dargestellt wird.}
  
  \item{Zoomstufe (Zoom Level): }{ Dieser Faktor beschreibt die Gr��e des
    sichtbaren Anzeigeinhaltes.  Bei einer sehr niedrigen Zoomstufe
    (auch: weit weg gezoomt) ist ein gr��erer Teil des im
    Anzeigefenster visualisierten IML-Graphen sichtbar als bei einer
    hohen Zoomstufe (auch: sehr nach heran gezoomt).}
  
  \item{hervorheben (to highlight): }{Hervorheben bedeutet, dass in einem
    Anzeigefenster visualisierte Fenster-Knoten oder Fenster-Kanten
    z.B.\ durch eine farbige Umrahmung von anderen Fenster-Knoten oder
    Fenster-Kanten unterscheidbar gemacht werden.}
    
  \item{selektieren (to select): }{Selektieren beschreibt einen Vorgang
    �ber den der Benutzer, z.B.\ durch Anklicken von Fenster-Knoten
    oder Fenster-Kanten mit der Maus, eine Selektion aufbaut.}
    
\end{enumerate}
