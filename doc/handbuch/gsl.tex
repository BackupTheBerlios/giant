Die F�higkeiten der GIANT Scripting Language (GSL) werden im Anhang 
\gq{GSL}, der Teil der GIANT Spezifikation ist, beschrieben.

\section{Parameter der Layoutalgorithmen}
Bei der Funktion \verb1insert_into_window1 auf Seite 34 der
GSL-Spezifikation ist als Parameter \verb1parameters1 f�r den
gew�hlten Layoutalgorithmus zu �bergeben. Diese werden 1:1 dem
Parameter \verb1additional_parameters1 der Routine \verb1Create1 in
\verb1src/vis/giant-layout_factory.ads1 �bergeben. Die m�glichen
Parameter sind dort dokumentiert.

F�r das Matrix-Layout sind keine Parameter vorgesehen.

F�r das Treelayout sind es folgende (Auszug aus dem Kommentar):

\begin{verbatim}
    Format:  [<Root_Node_ID>]; <List_Of_Class_Set_Names>
    Example: "5; Aber, Hallo"
             "; Ja, genau"
             It is not possible to use " or ; or , in a classsetname
             Brackets are possible.
 
    Meaning:
      Root_Node_Id    : The root-node of the tree to layout
                        If not given, the root-node is searched
                        If there's more than one possible root-node-id,
                        the result is random.
      Class_Set_Names : Names of ClassSet containing node-classes
                          and edge-classes to layout
 
      Target_Position : Position on window, where the root-node has to
                           be placed
\end{verbatim}

\section{GSL-FAQ}

Hier finden sich einige oft gestellte Fragen zur GSL.

\subsection{Wie starte ich GSL?}
Zwei M�glichkeiten:

\begin{itemize}
  \item �ber Kommandozeile
  \item �ber Query-Dialog
\end{itemize}

Dar�berhinaus gibt es noch die M�glichkeit, vorgefertige Skripte �ber
die Men�s zu starten. 

\subsection{Mein Script bei if funktioniert nicht als condition}
So ist das auch nicht spezifiziert. Abhilfe:

\begin{verbatim}
  // ...
  +start_cond;
  if
    (is_script (b),
     {() set ('start_cond, b ())},
     {() set ('start_cond, b   )});
  if
    (start_cond,
    // ...
\end{verbatim}

Wichtig ist hier, dass beide Zweige als Skripte implementiert
sind. Siehe auch Frage \ref{if_eval}.

\subsection{If f�hrt immer beide Zweige aus}
\label{if_eval}
Beispiel:

\begin{verbatim}
  if 
    (true,
     set ('a, 10),
     set ('a, 20))
\end{verbatim}

a ist jetzt 20 und nicht 10.

Der Grund liegt in der Auswertung: if ist auch ein GSL-Skript. Um ein
Skript auszuf�hren, werden zuerst die Parameter f�r dieses Skript
interpretiert und das Ergebnis dieser Interpretation dem Skript
�bergeben. Bei obigem Beispiel sind das true, 10 und 20, da die
Skripte \verb1set ('a, 10)1 und \verb1set ('a, 20)1 \gq{ausgef�hrt}
wurden. Nachdem true true ist, wird 10 ausgewertet, was zu 10 f�hrt.

Richtig ist deshalb:

\begin{verbatim}
  if 
    (true,
     {() set ('a, 10)},
     {() set ('a, 20)})
\end{verbatim}

Hier in den Zweigen nach der Auswertung und vor der \gq{Ausf�hrung}
von if Aktivierungsinformationen liegen. Und so wird nach der
Entscheidung, welcher Zweig ausgewertet werden soll, die
Aktivierungsinformationen zu Scripten ausgewertet.

\subsection{Wie erzeuge ich einen neuen Subgraphen?}

Folgendes Script erzeugt einen leeren Subgraphen NAME

\begin{verbatim}
[
  set (+target_edges, empty_edge_set ());
  set (+target_nodes, empty_node_set ());

  set 
    ('subgraph.NAME, 

     (target_nodes, 
      target_edges));
]
\end{verbatim}
